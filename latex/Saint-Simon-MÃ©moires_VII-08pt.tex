\PassOptionsToPackage{unicode=true}{hyperref} % options for packages loaded elsewhere
\PassOptionsToPackage{hyphens}{url}
%
\documentclass[oneside,8pt,french,]{extbook} % cjns1989 - 27112019 - added the oneside option: so that the text jumps left & right when reading on a tablet/ereader
\usepackage{lmodern}
\usepackage{amssymb,amsmath}
\usepackage{ifxetex,ifluatex}
\usepackage{fixltx2e} % provides \textsubscript
\ifnum 0\ifxetex 1\fi\ifluatex 1\fi=0 % if pdftex
  \usepackage[T1]{fontenc}
  \usepackage[utf8]{inputenc}
  \usepackage{textcomp} % provides euro and other symbols
\else % if luatex or xelatex
  \usepackage{unicode-math}
  \defaultfontfeatures{Ligatures=TeX,Scale=MatchLowercase}
%   \setmainfont[]{EBGaramond-Regular}
    \setmainfont[Numbers={OldStyle,Proportional}]{EBGaramond-Regular}      % cjns1989 - 20191129 - old style numbers 
\fi
% use upquote if available, for straight quotes in verbatim environments
\IfFileExists{upquote.sty}{\usepackage{upquote}}{}
% use microtype if available
\IfFileExists{microtype.sty}{%
\usepackage[]{microtype}
\UseMicrotypeSet[protrusion]{basicmath} % disable protrusion for tt fonts
}{}
\usepackage{hyperref}
\hypersetup{
            pdftitle={SAINT-SIMON},
            pdfauthor={Mémoires\_VII},
            pdfborder={0 0 0},
            breaklinks=true}
\urlstyle{same}  % don't use monospace font for urls
\usepackage[papersize={4.80 in, 6.40  in},left=.5 in,right=.5 in]{geometry}
\setlength{\emergencystretch}{3em}  % prevent overfull lines
\providecommand{\tightlist}{%
  \setlength{\itemsep}{0pt}\setlength{\parskip}{0pt}}
\setcounter{secnumdepth}{0}

% set default figure placement to htbp
\makeatletter
\def\fps@figure{htbp}
\makeatother

\usepackage{ragged2e}
\usepackage{epigraph}
\renewcommand{\textflush}{flushepinormal}

\usepackage{indentfirst}
\usepackage{relsize}

\usepackage{fancyhdr}
\pagestyle{fancy}
\fancyhf{}
\fancyhead[R]{\thepage}
\renewcommand{\headrulewidth}{0pt}
\usepackage{quoting}
\usepackage{ragged2e}

\newlength\mylen
\settowidth\mylen{...................}

\usepackage{stackengine}
\usepackage{graphicx}
\def\asterism{\par\vspace{1em}{\centering\scalebox{.9}{%
  \stackon[-0.6pt]{\bfseries*~*}{\bfseries*}}\par}\vspace{.8em}\par}

\usepackage{titlesec}
\titleformat{\chapter}[display]
  {\normalfont\bfseries\filcenter}{}{0pt}{\Large}
\titleformat{\section}[display]
  {\normalfont\bfseries\filcenter}{}{0pt}{\Large}
\titleformat{\subsection}[display]
  {\normalfont\bfseries\filcenter}{}{0pt}{\Large}

\setcounter{secnumdepth}{1}
\ifnum 0\ifxetex 1\fi\ifluatex 1\fi=0 % if pdftex
  \usepackage[shorthands=off,main=french]{babel}
\else
  % load polyglossia as late as possible as it *could* call bidi if RTL lang (e.g. Hebrew or Arabic)
%   \usepackage{polyglossia}
%   \setmainlanguage[]{french}
%   \usepackage[french]{babel} % cjns1989 - 1.43 version of polyglossia on this system does not allow disabling the autospacing feature
\fi

\title{SAINT-SIMON}
\author{Mémoires\_VII}
\date{}

\begin{document}
\maketitle

\hypertarget{chapitre-premier.}{%
\chapter{CHAPITRE PREMIER.}\label{chapitre-premier.}}

1708

~

{\textsc{Chamillart renvoyé en Flandre.}} {\textsc{- Récompenses de la
défense de Lille.}} {\textsc{- Retour de Chamillart à la cour.}}
{\textsc{- Tranchée ouverte devant la citadelle de Lille (29 octobre).}}
{\textsc{- L'Artois désolé et délivré.}} {\textsc{- Chamillart juge des
avis des généraux\,; sa partialité.}} {\textsc{- Audace de Vendôme.}}
{\textsc{- Berwick retourne de sa personne sur le Rhin, où l'armée se
sépare.}} {\textsc{- Incroyable hardiesse de Vendôme.}} {\textsc{-
Marlborough passe l'Escaut sans opposition.}} {\textsc{- Mensonge
prodigieux de Vendôme.}} {\textsc{- Fautes personnelles de Mgr le duc de
Bourgogne, dont avantages pris contre lui avec éclat.}} {\textsc{- Belle
mais difficile retraite de plusieurs détachements de l'armée, où
Hautefort se distingue sans combat, et Nangis en combattant.}}
{\textsc{- Étrange ignorance du roi, à qui le duc de La Trémoille
apprend cette action à son dîner.}} {\textsc{- Sousternon perdu.}}
{\textsc{- Saint-Guillain perdu et repris par Hautefort et Albergotti.}}
{\textsc{- Position des armées.}} {\textsc{- État de la citadelle de
Lille.}} {\textsc{- Boufflers reçoit un ordre de la main du roi de
capituler.}} {\textsc{- Ordre au prince de revenir, et à Vendôme de
séparer l'armée, et, malgré ses adroites instances, de revenir aussi.}}

~

Lille perdu, question fut d'un parti à prendre. Quoique M. de Vendôme
eût assuré que la prise de Leffinghem empêcherait les convois des
ennemis, on n'en crut pas moins la citadelle un peu plus tôt, un peu
plus tard perdue, et le roi voulut d'autant plus tôt se fixer à quelque
chose, que les ennemis faisaient divers mouvements, et n'avaient que
vingt bataillons devant cette citadelle pour en faire le siège. Cette
raison de décision, et celle d'éclaircir plusieurs choses qui s'étaient
passées depuis que Chamillart était revenu de Flandre, firent prendre le
parti subit de l'y renvoyer. Il partit donc le mardi 30 octobre, à
quatre heures du matin, de Versailles, pour aller coucher à Cambrai\,;
et Chamlay, si expert dans la connaissance des moindres lieux et des
plus petits ruisseaux de la Flandre, partit à midi du même jour pour l'y
suivre. Si la cour fut surprise de voir si près à près disparaître
Chamillart, l'armée ne le fut pas moins de le voir arriver à Tournai. Il
y porta les grâces répandues sur ceux qui venaient de sortir si
glorieusement de Lille. Surville, sorti de la citadelle de Lille avec un
coup de mousquet fort considérable, eut dix mille livres de pension.
Lée, qui était aussi à Douai pour être trépané d'un autre coup de
mousquet, eut l'expectative, les marques et la pension de grand-croix de
Saint-Louis, en attendant la première vacante. Rannes, Ravignan,
Coetquen, Permangle furent faits maréchaux de camp\,; Maillebois dès
avant la fin du siège, Belle-Ile (tous deux maintenant maréchaux de
France, et le premier duc héréditaire\,; après bien de diverses et
d'étranges fortunes), Martinville, Tourrotte et Sourzy furent faits
brigadiers, et quelques autres.

La tranchée fut ouverte devant la citadelle de Lille la nuit du 29 au 30
octobre. Ils attaquèrent l'avant-chemin couvert le 7 novembre, dont ils
furent repoussés avec assez de perte, et le 10 Chamillart arriva, et
rendit compte le soir même de son voyage au roi chez
M\textsuperscript{me} de Maintenon\,; ainsi son voyage fut de douze
jours, dont il en passa huit à l'armée, pendant lesquels son fils
travailla avec le roi, comme il avait fait pendant son précédent voyage
de Flandre. En attendant, les ennemis désolaient l'Artois, et le prince
d'Auvergne fortifiait la Bassée. Cheladet y marcha avec trente
escadrons, et à la fin leur fit quitter prise et abandonner la Bassée,
mais il en coûta bon au pays.

Le désir de la cour, dont Chamillart fut porteur, était la garde de
l'Escaut. M. de Vendôme l'en avait infatuée, séduit par l'avantage de
couper la retraite aux ennemis, et comptant pour rien la plus que très
difficile garde de quarante lieues du cours de cette rivière. Berwick,
peu ployant sous le poids de Vendôme, et peu soucieux du mépris qu'il
faisait de son sentiment, ne crut pas le devoir taire dans une occasion
si importante, où il ne voyait que de pitoyables raisonnements.
L'altercation recommença donc entre eux plus vive que jamais, et Mgr le
duc de Bourgogne, autant qu'il l'osait, était pour Berwick. Toutes ces
disputes s'écrivaient au roi, qui lui firent prendre le parti d'envoyer
Chamillart, devant lequel les généraux plaidèrent chacun leur avis. Il
tâcha vainement de les raccommoder\,; il écouta tout, il discuta toutes
les raisons de part et d'autre à diverses reprises. C'était à cet homme
de robe, de plume et de finance, à décider des mouvements de guerre les
plus savants et les plus importants, et à en décider seul\,; c'était
pour cela qu'il était envoyé, quoiqu'il n'eût jamais vu de guerre que
dans son cabinet et dans ses deux voyages de Flandre, si près à près et
si courts. Il prit un parti mitoyen, dans la confiance de l'exécution
duquel il repartit pour se rendre auprès du roi. Mais à peine était-il à
trente lieues de la frontière que Vendôme reprit son premier dessein de
la garde de l'Escaut, sans en pouvoir être détourné par personne.
Chamillart, plus enivré que jamais de Vendôme en ce voyage, y avait peu
ménagé Mgr le duc de Bourgogne, et le ménagea encore moins dans le
compte qu'il rendit au roi en arrivant. Ce compte fut rendu chez
M\textsuperscript{me} de Maintenon, en sa présence. Elle entendit tout
sans oser souffler, elle rendit tout à M\textsuperscript{me} la duchesse
de Bourgogne. On peut juger ce qu'il en résulta entre elles deux, et
quelle fut la colère de la princesse, avec le mécontentement qu'elle
avait déjà précédemment conçu contre le ministre, et l'indignation de
M\textsuperscript{me} de Maintenon, auprès de laquelle il était déjà de
longue main si mal.

Le premier effet du retour de Chamillart fut un ordre envoyé dès le
lendemain à Berwick de s'en aller prendre le commandement des troupes
demeurées sur le Rhin, où néanmoins la campagne était depuis longtemps
finie, et où on n'attendait plus que l'arrivée des quartiers d'hiver
pour se séparer. Berwick sentit tout le coup que Vendôme lui faisait
porter, l'inutilité de ce voyage, et le dégoût de le faire sans le voile
d'aucun prétexte, et n'y menant aucunes troupes, sans même avoir la
permission de passer à la cour, tant ils eurent peur qu'il n'y parlât au
roi et au monde. Il ne dit mot, et obéit. Pour achever cela de suite, il
ne fut pas quinze jours sur le Rhin qu'il y reçut les ordres pour les
quartiers d'hiver, à quoi du Bourg aurait été tout aussi bon que lui.
Mais il pesait trop à Vendôme par la force et la justesse de ses
raisonnements, et il avait fallu l'en soulager.

Dès qu'il fut parti, Vendôme écrivit au roi que maintenant il était au
large, et il ajouta en propres termes que désormais il était si sûr
d'empêcher les ennemis de passer l'Escaut qu'il lui en répondait sur sa
tête. Avec un tel garant, et si fort à la cour, le moyen de n'y pas
compter\,? Aussi y triompha-t-on d'avance\,; et sans se souvenir de
toutes les déplorables fadeurs qui avaient eu tant de cours sur
l'impossibilité aux ennemis de prendre Lille et de se retirer de devant,
sinon avec un passeport pour n'y pas périr de faim, les mêmes flatteries
recommencèrent sur la malheureuse destinée de ces conquérants qui
s'allaient trouver enfermés sans aucune ressource. On ne fut pas
longtemps amusé de ce roman. Le duc de Marlborough vint à Harlebeck et à
Vive-Saint-Éloy, le prince Eugène à Roosebeck, qui montrèrent ainsi
qu'ils en voulaient à l'Escaut. Nous avions des retranchements sur
Audenarde, gardés par Hautefort, et l'armée voulut s'en approcher\,;
mais dans ce mouvement, Marlborough passa l'Escaut sur quatre ponts, à
Gavre et à Berkem, la nuit du 26 au 27, sans opposition quelconque, et
sans trouver aucunes de nos troupes. Le roi l'apprit par un courrier de
M. de Vendôme, qui ajoutait dans sa lettre au roi, en termes formels,
qu'il le suppliait de se souvenir qu'il lui avait toujours mandé la
garde de l'Escaut impossible.

Il fallait que ce grand général n'eût aucune sorte de mémoire, ou qu'il
comptât le roi, la cour, son armée et tout le public pour bien peu de
chose. En moins de quinze jours, répondre au roi, sur sa tête, qu'il
empêchera aux ennemis de passer l'Escaut, et, dès qu'ils l'ont passé,
écrire au roi qu'il le supplie de se souvenir qu'il lui a toujours mandé
qu'il était impossible d'empêcher les ennemis de le passer, et cela sans
qu'il fût rien arrivé entre-deux qui eût fait changer ni la face des
choses ni à lui de langage, ce sont de ces vérités qui ne sont pas
vraisemblables, mais vérités toutefois qui ont eu le roi, la cour,
l'armée pour témoins, et dont M. de Vendôme, ni cette formidable cabale
qui l'appuyait avec un si incroyable succès, n'ont pas seulement tenu
compte de se disculper, mais bien d'en étouffer le bruit à force d'en
renouveler d'anciens et de nouveaux propos contre Mgr le duc de
Bourgogne. Ce nouveau vacarme ne put empêcher un contradictoire si
prompt, si net, si précis, si important, de la même bouche, et de cette
bouche prise sans cesse pour le seul oracle de la guerre, malgré les
succès. Les réflexions seraient trop au-dessous du fait pour s'y arrêter
ici. Voyons le court détail de cette affaire, dont la cabale se battit,
comme on dit, avec les pierres du clocher. Elle n'empêcha pas de trouver
et de dire que ce trait ne pouvait être méconnu pour être du même homme
qui en avait fait un tout pareil à M. le duc d'Orléans sur le passage du
Pô.

L'armée était au Saussoy, près de Tournai, dans une tranquillité
profonde, dont l'opium avait gagné jusqu'à Mgr le duc de Bourgogne,
lorsqu'il vint plusieurs avis de la marche des ennemis. M. de Vendôme
s'avança là-dessus de ce côté-là avec quelques détachements. Le soir, il
manda à Mgr le duc de Bourgogne que, sur les confirmations qu'il
recevait de toutes parts des mêmes nouvelles, il croyait qu'il devait
marcher avec toute l'armée le lendemain pour le suivre. Mgr le duc de
Bourgogne se déshabillait pour se coucher lorsqu'il reçut cette lettre,
sur laquelle ce qui se trouva auprès de lui alors raisonna
différemment\,: les uns furent d'avis de marcher à l'heure même, les
autres qu'il ne se couchât point, pour être prêt de plus grand matin\,;
enfin, le troisième sentiment fut qu'il se couchât pour prendre quelque
repos, et de marcher le matin, comme M. de Vendôme le lui conseillait.
Après avoir un peu balancé, le jeune prince prit ce dernier parti. Il se
coucha, il se leva le lendemain au jour, il déjeuna longtemps. Comme il
allait sortir de table, il apprit que l'armée entière des ennemis avait
passé l'Escaut. À chose faite il n'y a plus de remède. Il en fut outré
de déplaisir. La vérité est que quand il aurait suivi le premier et le
seul bon des trois avis, avant qu'on eût détendu, chargé, pris les
armes, monté à cheval, la nuit aurait été bien avancée, et qu'au chemin
qu'il fallait faire, on aurait trouvé les ennemis passés il y aurait eu
plus de six ou sept heures. Mais il est des messéances qu'il faut
éviter, et c'est le malheur de n'avoir personne auprès de soi qui le
sente, ou qui en avertisse, quand soi-même on n'y pense pas. Le premier
parti aurait été inutile à empêcher le passage, mais très utile au jeune
prince à marquer de la volonté et de l'ardeur.

À cette faute il en ajouta une autre, qui, sans pouvoir avoir aucun air
d'influer à la tranquillité de ce passage si important, en montra une
que toutefois Mgr le duc de Bourgogne n'avait pas, et dont il crut très
mal à propos pouvoir se dissiper innocemment. Il avait mangé, il était
fort matin, il n'y avait plus à marcher. Pour prendre un nouveau parti
sur un passage fait auquel on ne s'attendait pas, au moins si
brusquement, il fallait attendre ce qu'il plairait à M. de Vendôme. On
était tout auprès de Tournai\,; Mgr le duc de Bourgogne y alla jouer à
la paume. Cette partie subite scandalisa étrangement l'armée et
renouvela tous les mauvais discours. La cabale, qui ne put accuser la
lenteur du prince, par la raison que je viens d'expliquer, et parce que
M. de Vendôme ne lui avait pas mandé de marcher à l'heure même, mais le
lendemain matin, la cabale, dis-je, se jeta sur la longueur du déjeuner
en des circonstances pareilles, et sur une partie de paume faite si peu
à propos\,; et là-dessus toutes les chamarrures les plus indécentes et
les plus audacieuses à l'armée, à la cour, à Paris, pour noyer la réelle
importance du fait de M. de Vendôme par ce vacarme excité sur
l'indécence de ceux de Mgr le duc de Bourgogne en ces mêmes moments.

Hautefort, se voyant pris par ce passage des ennemis par sa droite et
par sa gauche, se retira sans avoir pu être entamé. Sousternon,
lieutenant général, voisin du lieu de passage, et averti de quelques
mouvements, manda à Nangis, maréchal de camp, de marcher à lui avec le
détachement qu'il avait, qui était de neuf bataillons et de quelque
cavalerie. Il obéit, et reçut en chemin avis d'un gros corps ennemi qui
le séparait du quartier d'où il sortait, par conséquent du gros des
autres quartiers. Les avis continuèrent\,; il arriva au quartier de
Sousternon et n'y trouva personne. Il prit donc un grand tour pour
retourner d'où il était venu dans l'obscurité de la nuit. Le jour venu,
il continua sa marche sur les quartiers voisins, de proche en proche,
pour essayer de joindre Hautefort. Il fut attaqué et fit une vigoureuse
défense, toujours marchant et gagnant du terrain sur une chaussée entre
des marais, et ramassant les traîneurs des autres quartiers qui filaient
devant et après. Dépêtré enfin de cette rude escarmouche, il rencontra
du canon abandonné, qu'il ne voulut pas laisser, et qu'il emmena. Ce
retardement donna lieu à une autre attaque plus vive, et qui, plus ou
moins vigoureusement poussée et repoussée, selon qu'il pouvait se
retourner dans l'incommodité de ce long défilé, dura, avec une grande
valeur et beaucoup de perte, jusqu'à ce qu'il eût joint la queue de
quelques autres quartiers qui s'arrêtèrent pour l'attendre Sousternon
était avec ceux-là. Ils furent encore suivis et toujours attaqués
jusqu'à un ruisseau, au delà duquel Hautefort s'était posté pour les
attendre, et protéger leur passage par le feu qu'il fit de derrière le
ruisseau, qu'il avait bordé d'infanterie à droite et à gauche. Là finit
ce combat désavantageux, qui fit perdre beaucoup de monde. Les quartiers
épars, ainsi rassemblés là, s'y rafraîchirent un peu, et, à quelques
jours de là, rejoignirent l'armée. Hautefort fut fort approuvé, même des
ennemis, qui louèrent fort sa retraite. Sousternon, au contraire, perdit
la tramontane et fut fort blâmé. Nangis, au contraire, aujourd'hui
maréchal de France, s'en tira avec tête et valeur.

Le roi ignora cette action plusieurs jours, et l'aurait ignorée
davantage sans le duc de La Trémoille, dont le fils unique y était et
s'y était même distingué. Dépité de ce que le roi ne lui en disait pas
un mot, il prit son temps qu'il servait le roi à son petit couvert de
parler du passage de l'Escaut, où il dit que son fils avait beaucoup
souffert avec son régiment. «\, Comment, souffert\,? dit le roi\,; il
n'y a rien eu. --- Une grosse action,\,» répondit le duc, et la raconta
tout de suite. Le roi l'écouta avec grande attention, le questionna
même, et avoua devant tout le monde qu'il n'en avait rien su. On peut
juger de sa surprise et de celle qu'il causa. Il arriva qu'un moment
après être sorti de table, Chamillart, sans être attendu, entra dans son
cabinet. Le roi expédia ce qui l'amenait, qui était court, puis lui
demanda ce que voulait dire l'action de l'Escaut, dont il ne lui avait
point parlé. Le ministre, embarrassé, répondit que ce n'était rien du
tout. Le roi continuant à le presser, à rapporter des détails, à citer
le régiment du prince de Tarente, Chamillart avoua que l'aventure du
passage était si désagréable en elle-même, et ce combat si désagréable
aussi, celui-ci peu important, l'autre sans remède, que
M\textsuperscript{me} de Maintenon, à qui il en avait rendu compte,
n'avait pas jugé à propos qu'il en fût importuné, et qu'ils étaient
convenus qu'il ne lui en serait point rendu compte. Sur cette singulière
réponse, le roi s'arrêta tout court et n'en dit plus mot. Cependant on
tomba rudement sur Sousternon. Il écrivit de longues justificatives. Le
fait est qu'il pouvait être plus vigilant, et surtout plus entendu en sa
retraite, et à donner mieux ordre à celle des autres quartiers. Mais,
avec toute la vigilance possible, il n'eût pu empêcher le passage avec
le peu de troupes qu'il avait, et en un endroit de l'Escaut où le
mousquet portait bien plus loin que le travers de la rivière. Néanmoins
il en fut la victime. Le maréchal de Villeroy alors était perdu\,; son
oncle, le P. de La Chaise, était mourant. Ainsi privé de ces deux
appuis, et ayant affaire à M. de Vendôme, par conséquent peu soutenu du
comte de Toulouse, duquel il était capitaine des gardes, il perdit sa
fortune, et n'a pas servi depuis.

Un peu avant cet événement, la garnison d'Ath nous avait surpris
Saint-Guillain, d'où un bataillon était sorti pour escorter des chariots
de fourrages pour notre armée. Cette perte fâchait d'autant plus que
nous y avions de gros magasins. Albergotti alla tâcher de le reprendre,
et Hautefort l'y alla renforcer au sortir de cette affaire que je viens
de raconter. Ils le reprirent avec six cents hommes qui étoient dedans
prisonniers de guerre, et tous nos magasins, qu'ils ne s'avisèrent pas
de brûler. L'Escaut passé, le duc de Marlborough alla passer la Dendre
et camper à Wetter, près de Gand, notre armée près de Douai, et le
prince Eugène, qui n'avait fait que s'approcher tout près de l'Escaut
pour en favoriser le passage, et qui ne le passa point, s'en retourna à
son siège.

Les ennemis, établis du 9 sur l'avant-chemin couvert, commencèrent à
faire jouer leur artillerie et à travailler à des sapes. Ils tentèrent
aussi de se rendre maîtres du chemin couvert sans succès. Le maréchal de
Boufflers fut encore légèrement blessé, le 21, d'un éclat de grenade qui
lui fit une contusion à la tête, en visitant le chemin couvert, qui ne
l'arrêta pas un moment sur rien. Mais tout lui manquait, et dans les
premiers jours de décembre il ne lui restait que vingt milliers de
poudre, et très peu d'autres munitions, encore moins de vivres. Ils
avaient mangé huit cents chevaux, tant dans la ville que dans la
citadelle\,; et Boufflers, qui ne se distinguait que par son activité et
sa prévoyance, en fit toujours servir à sa table dès que les autres
furent réduits à cette ressource, et en mangea lui-même. Il trouva
toujours des inventions de donner de ses nouvelles et d'en recevoir. Le
roi, voyant l'état des choses, lui envoya un ordre de sa main de se
rendre, qu'il garda secret, sans vouloir y obéir encore de plusieurs
jours, et il différa tant qu'il lui fut possible.

L'Escaut forcé, la citadelle de Lille sur le point d'être prise, notre
armée, poussée à bout de fatigues et plus encore de nécessité, demeura
peu ensemble, et fût bientôt séparée faute de pain, au scandale
universel, tandis qu'il n'était pas douteux que les ennemis, campés près
de Gand, n'en voulussent faire le siège. Les choses en cet état, les
princes ne pouvaient plus demeurer en Flandre avec bienséance. Ils
eurent donc ordre de revenir\,; ils insistèrent à demeurer à cause de
Gand. Une autre raison arrêtait, encore Mgr le duc de Bourgogne. M. de
Vendôme ne semblait pas avoir reçu les mêmes ordres, et faisait
publiquement toutes ses dispositions particulières, comme un homme qui
comptait de passer l'hiver sur la frontière, et d'y commander en
attendant le retour du printemps et de l'ouverture de la campagne. Mais
tandis qu'il en usait ainsi, il ne se vantait pas d'avoir reçu son
congé, et qu'il attendait la réponse aux représentations qu'il avait
faites sur la nécessité qu'il demeurât l'hiver. Il se sentait toucher au
moment de rendre compte\,; il commençait à le craindre, et à redouter de
près ce que de loin il avait si témérairement méprisé et si
audacieusement insulté.

Ses représentations ne réussirent pas. Il s'inquiéta de voir Mgr le duc
de Bourgogne différer son départ et observer le sien. Il redoubla donc
ses instances jusqu'à s'abaisser à demander comme une grâce ce qu'il
avait d'abord proposé et offert comme une chose nécessaire au service du
roi. Pendant cette lutte, les princes reçurent des ordres réitérés et
absolus. Ils partirent et se rendirent à la cour. J'y étais revenu une
quinzaine auparavant\,; je m'y étais mis au fait de tout ce qui s'était
passé pendant ma courte absence\,; et pendant tout ce que M. le duc
d'Orléans m'avait pu donner de temps dans les trois jours d'intervalle
entre son arrivée et celle des princes, je l'avais bien instruit de tout
le principal et le plus pressé à savoir de ce que la contrainte des
courriers et du chiffre m'avait empêché de lui pouvoir mander. La
jalousie des princes du sang, et un bel air de débauche, l'avait rendu
enclin à Vendôme par éloignement du prince de Conti. J'en craignis pour
lui l'écueil sur Mgr le duc de Bourgogne. Je l'avais informé exactement
et au long, quoiqu'en chiffre, des principaux événements de la campagne
et de la cour. À son retour, je lui expliquai plus de détails, et je lui
fis comprendre combien serait premièrement injuste, puis dangereux pour
lui dans les suites, de prendre le change. Il ne fut pas longtemps sans
s'applaudir d'avoir suivi mon conseil.

\hypertarget{chapitre-ii.}{%
\chapter{CHAPITRE II.}\label{chapitre-ii.}}

1708

~

{\textsc{Retour des princes à la cour.}} {\textsc{- Mécanique de chez
M\textsuperscript{me} de Maintenon et de son appartement.}} {\textsc{-
Réception du roi et de Monseigneur à Mgr le duc de Bourgogne et à M. le
duc de Berry, à qui ensuite Mgr le duc de Bourgogne parle longtemps et
bien.}} {\textsc{- Apophtegmes peu discrets de Gamaches.}} {\textsc{-
Citadelle de Lille rendue.}} {\textsc{- Honneurs infinis faits au
maréchal de Boufflers.}} {\textsc{- Retour et réception du duc de
Vendôme à la cour.}} {\textsc{- Retour et réception triomphante du
maréchal de Boufflers à la cour\,; fait pair, etc.}} {\textsc{- Extrême
honneur que je reçois de Mgr le duc de Bourgogne.}} {\textsc{- Retour du
duc de Berwick à la cour.}} {\textsc{- Beau projet de reprendre Lille.}}
{\textsc{- Boufflers renvoyé en Flandre.}} {\textsc{- Tranchée ouverte à
Gand\,; La Mothe dedans.}} {\textsc{- Soirée du roi singulière.}}

~

M\textsuperscript{me} la duchesse de Bourgogne était dans une grande
agitation de la réception que recevrait Mgr le duc de Bourgogne, et de
pouvoir avoir le temps de l'entretenir et de l'instruire avant qu'il pût
voir le roi ni personne. Je lui fis dire de lui mander d'ajuster son
voyage de façon qu'il arrivât à une ou deux heures après minuit, parce
que de la sorte, arrivant tout droit chez elle, et ne pouvant voir
qu'elle, ils auraient tout le temps de la nuit à être ensemble seuls,
les premiers {[}instants{]} du matin avec le duc de Beauvilliers, et
peut-être avec M\textsuperscript{me} de Maintenon, et l'avantage encore
que le prince saluerait le roi et Monseigneur avant que personne fût
entré chez eux, et que personne n'y serait témoin de sa réception, à
très peu de valets près, et même écartés. L'avis ne fut pas donné, ou,
s'il le fut, il ne fut pas suivi. Le jeune prince arriva le lundi 11
décembre, un peu après sept heures du soir, comme Monseigneur venait
d'entrer à la comédie, où M\textsuperscript{me} la duchesse de Bourgogne
n'était pas allée pour l'attendre. Je ne sais pourquoi il vint descendre
dans la cour des Princes, au lieu de la grande. J'étais en ce moment-là
chez la comtesse de Roucy, dont les fenêtres donnaient dessus. Je sortis
aussitôt, et, arrivant au haut du grand degré du bout de la galerie,
j'aperçus le prince qui le montait entre les ducs de Beauvilliers et de
La Rocheguyon, qui s'étaient trouvés à la descente de sa chaise. Il
avait bon visage, gai et riant, et parlait à droite et à gauche. Je lui
fis ma révérence au bord des marches. Il me fit l'honneur de
m'embrasser, mais de façon à me marquer qu'il était encore plus instruit
qu'attentif à ce qu'il devait à la dignité, et il ne parla plus qu'à moi
un assez long bout de chemin, pendant lequel il me glissa bas qu'il
n'ignorait pas comment j'avais parlé et comment j'en avais usé à son
égard. Il fut rencontré par un groupe de courtisans, à la tête desquels
était le duc de La Rochefoucauld, au milieu duquel il passa la grande
salle des gardes, au lieu d'entrer chez M\textsuperscript{me} de
Maintenon par son antichambre de jour et par les derrières, bien que son
plus court, et alla, par le palier du grand degré, entrer par la grande
porte de l'appartement de M\textsuperscript{me} de Maintenon. C'était le
jour ordinaire du travail de Pontchartrain, qui, depuis quelque temps,
avait changé avec Chamillart du mardi au lundi. Il était alors en tiers
avec le roi et M\textsuperscript{me} de Maintenon, et le soir même il me
conta cette curieuse réception, qu'il remarqua bien et dont il fut seul
témoin. Je dis en tiers, parce que M\textsuperscript{me} la duchesse de
Bourgogne allait et venait\,; mais pour le bien entendre, il faut un
moment d'ennui de mécanique.

L'appartement de M\textsuperscript{me} de Maintenon était de plain-pied
et faisant face à la salle des gardes du roi. L'antichambre était plutôt
un passage long en travers, étroit, jusqu'à une autre antichambre toute
pareille de forme, dans laquelle les seuls capitaines des gardes
entraient, puis une grande chambre très profonde. Entre la porte par où
on y entrait de cette seconde antichambre et la cheminée, était le
fauteuil du roi adossé à la muraille, une table devant lui, et un
ployant autour pour le ministre qui travaillait. De l'autre côté de la
cheminée, une niche de damas rouge et un fauteuil où se tenait
M\textsuperscript{me} de Maintenon, avec une petite table devant elle.
Plus loin, son lit dans un enfoncement. Vis-à-vis les pieds du lit, une
porte et cinq marches à monter, puis un fort grand cabinet qui donnait
dans la première antichambre de l'appartement de jour de Mgr le duc de
Bourgogne, que cette porte enfilait, et qui est aujourd'hui
l'appartement du cardinal Fleury. Cette première antichambre ayant à
droite cet appartement, et à gauche ce grand cabinet de
M\textsuperscript{me} de Maintenon, descendait, comme encore
aujourd'hui, par cinq marches dans le salon de marbre contigu au palier
du grand degré du bout des deux galeries, haute et basse, dites de
M\textsuperscript{me} la duchesse d'Orléans et des Princes. Tous les
soirs M\textsuperscript{me} la duchesse de Bourgogne jouait dans le
grand cabinet de M\textsuperscript{me} de Maintenon avec les dames à qui
on avoir donné l'entrée, qui ne laissait pas d'être assez étendue, et de
là entrait, tant et si souvent qu'elle voulait, dans la pièce joignante,
qui était la chambre de M\textsuperscript{me} de Maintenon, où elle
était avec le roi, la cheminée entre deux. Monseigneur, après la
comédie, montait dans ce grand cabinet, où le roi n'entrait point, et
M\textsuperscript{me} de Maintenon presque jamais.

Avant le souper du roi, les gens de M\textsuperscript{me} de Maintenon
lui apportaient son potage avec son couvert, et quelque autre chose
encore. Elle mangeait, ses femmes et un valet de chambre la servaient,
toujours le roi présent, et presque toujours travaillant avec un
ministre. Le souper achevé, qui était court, on emportait la table\,;
les femmes de M\textsuperscript{me} de Maintenon demeuraient, qui tout
de suite la déshabillaient en un moment et la mettaient au lit. Lorsque
le roi était averti qu'il était servi, il passait un moment dans une
garde-robe, allait après dire un mot à M\textsuperscript{me} de
Maintenon, puis sonnait une sonnette qui répondait au grand cabinet.
Alors Monseigneur, s'il y était, Mgr et M\textsuperscript{me} la
duchesse de Bourgogne, M. le duc de Berry, et les dames qui étaient à
{[}M\textsuperscript{me} la duchesse de Bourgogne{]}, entraient à la
file dans la chambre de M\textsuperscript{me} de Maintenon, ne faisaient
presque que la traverser, précédaient le roi, qui allait se mettre à
table suivi de M\textsuperscript{me} la duchesse de Bourgogne et de ses
dames. Celles qui n'étaient point à elle, ou s'en allaient, ou, si elles
étaient habillées pour aller au souper (car le privilège de ce cabinet
était d'y faire sa cour à M\textsuperscript{me} la duchesse de Bourgogne
sans l'être), elles faisaient le tour par la grande salle des gardes,
sans entrer dans la chambre de M\textsuperscript{me} de Maintenon. Nul
homme, sans exception que de ces trois princes, n'entrait dans ce grand
cabinet. Cela expliqué, venons à la réception et à tout son détail,
auquel Pontchartrain fut très attentif, et qu'il me rendit tète à tête
très exactement une demi-heure après qu'il fut revenu chez lui.

Sitôt que de chez M\textsuperscript{me} de Maintenon on entendit la
rumeur qui précède de quelques instants ces sortes d'arrivée, le roi
s'embarrassa jusqu'à changer diverses fois de visage.
M\textsuperscript{me} la duchesse de Bourgogne parut un peu tremblante,
et voltigeait par la chambre pour cacher son trouble, sous prétexte
d'incertitude par où le prince arriverait, du grand cabinet ou de
l'antichambre. M\textsuperscript{me} de Maintenon était rêveuse. Tout
d'un coup les portes s'ouvrirent\,: le jeune prince s'avança au roi,
qui, maître de soi plus que qui que ce fût, perdit à l'instant tout
embarras, fit un pas ou deux vers son petit-fils, l'embrassa avec assez
de démonstration de tendresse, lui parla de son voyage, puis, lui
montrant la princesse\,: «\,Ne lui dites-vous rien\,?» ajouta-t-il d'un
visage riant. Le prince se tourna un moment vers elle, et répondit
respectueusement, comme n'osant se détourner du roi, et sans avoir remué
de sa place. Il salua ensuite M\textsuperscript{me} de Maintenon, qui
lui fit fort bien. Ces propos de voyage, de couchées, de chemins,
durèrent ainsi, et tous debout, un demi-quart d'heure\,; puis le roi lui
dit qu'il n'était pas juste de lui retarder plus longtemps le plaisir
qu'il aurait d'être avec M\textsuperscript{me} la duchesse de Bourgogne,
et le renvoya, ajoutant qu'ils auraient loisir de se revoir. Le prince
fit sa révérence au roi, une autre à M\textsuperscript{me} de Maintenon,
passa devant le peu de dames du palais qui s'étaient enhardies de mettre
la tête dans la chambre au bas de ces cinq marches, entra dans le grand
cabinet, où il embrassa M\textsuperscript{me} la duchesse de Bourgogne,
y salua les dames qui s'y trouvèrent, c'est-à-dire les baisa, demeura
quelques moments, et passa dans son appartement, où il s'enferma avec
M\textsuperscript{me} la duchesse de Bourgogne.

Leur tête-à-tête dura deux heures et plus\,; tout à la fin
M\textsuperscript{me} d'O y fut en tiers\,; presque aussitôt après la
maréchale d'Estrées y entra, et peu de moments après
M\textsuperscript{me} la duchesse de Bourgogne sortit avec elles, et
revint dans le grand cabinet de M\textsuperscript{me} de Maintenon.
Monseigneur y vint à l'ordinaire au sortir de la comédie.
M\textsuperscript{me} la duchesse de Bourgogne, en peine de ce que Mgr
le duc de Bourgogne ne se pressait point d'y venir saluer Monseigneur,
l'alla chercher, et revint disant qu'il se poudrait\,; mais remarquant
que Monseigneur n'était pas satisfait de ce peu d'empressement, elle
envoya le hâter. Cependant la maréchale d'Estrées, folle et étourdie, et
en possession de dire tout ce qui lui passait par la tête, se mit à
attaquer Monseigneur de ce qu'il attendait si tranquillement son fils,
au lieu d'aller lui-même l'embrasser. Ce propos hasardé ne réussit
pas\,: Monseigneur répondit sèchement que ce n'était pas à lui à aller
chercher le duc de Bourgogne, mais au duc de Bourgogne à le venir
trouver. Il vint enfin. La réception fut assez bonne, mais elle n'égala
pas celle du roi à beaucoup près. Presque aussitôt le roi sonna, et on
passa pour le souper. Vers l'entremets, M. le duc de Berry arriva, et
vint saluer le roi à table. À celui-ci, tous les coeurs s'épanouirent.
Le roi l'embrassa fort tendrement. Monseigneur le regarda de même,
n'osant l'embrasser en présence du roi. Toute l'assistance le courtisa.
Il demeura debout auprès du roi le reste du souper, où il ne fut
question que de chevaux de poste, de chemins et de semblables
bagatelles. Le roi parla assez à table à Mgr le duc de Bourgogne\,; mais
ce fut d'un tout autre air à M. le duc de Berry. Au sortir de table, ils
allèrent tous dans le cabinet du roi à l'ordinaire, au sortir duquel M.
le duc de Berry trouva un souper servi dans la chambre de
M\textsuperscript{me} la duchesse de Bourgogne, qu'elle lui avait fait
tenir prêt de chez elle, et que l'empressement conjugal de Mgr le duc de
Bourgogne abrégea un peu trop. Le lendemain se passa en respects de
toute la cour. Ce lendemain mardi 11, le roi d'Angleterre arriva à
Saint-Germain, et vint voir le roi le mercredi avec la reine sa mère.

Je témoignai au duc de Beauvilliers, avec ma liberté accoutumée, que
j'avais trouvé Mgr le duc de Bourgogne bien gai au retour d'une si
triste campagne. Il n'en put disconvenir avec moi, jusque-là que je le
laissai en dessein de l'en avertir. Tout le monde en effet blâma
également une gaieté si peu à propos. Le mardi et le mercredi, occupés
les soirs par le travail des ministres, se passèrent sans
conversation\,; mais le jeudi, qui souvent était libre, Mgr le duc de
Bourgogne fut trois heures avec le roi chez M\textsuperscript{me} de
Maintenon. J'avais peur que la piété ne le retint sur M. de Vendôme,
mais j'appris qu'il avait parlé à cet égard sans ménagement, fortifié
par le conseil de M\textsuperscript{me} la duchesse de Bourgogne, et
rassuré sur sa conscience par le duc de Beauvilliers, avec qui il avait
été longtemps enfermé le mercredi. Le compte de la campagne, des
affaires, des choses, des avis, des procédés, fut rendu tout entier. Un
autre peut-être, moins vertueux, eût plus appesanti les termes\,; mais
enfin tout fut dit, et dit au delà des espérances, par rapport à celui
qui parlait et à celui qui écoutait. La conclusion fut une vive instance
pour commander une armée la campagne suivante, et la parole du roi de
lui en donner une. Il fut ensuite question d'entretenir Monseigneur\,:
cela vint plus tard de deux jours\,; mais enfin il eut une assez longue
conversation avec lui à Meudon, et avec M\textsuperscript{lle} Choin, à
laquelle il parla encore davantage tête à tête. Elle en avait bien usé
pour lui auprès de Monseigneur. M\textsuperscript{me} la duchesse de
Bourgogne la lui avait ménagée. La liaison entre cette fille et
M\textsuperscript{me} de Maintenon commençait à se serrer étroitement.
La Choin n'ignorait pas la vivacité que l'autre avait témoignée pour le
jeune prince\,; son intérêt n'était pas de se les aliéner tous, dont Mgr
le duc de Bourgogne recueillit quelque fruit en cette importante
occasion.

Gamaches et d'O avaient suivi les princes. Ce dernier, entièrement
disculpé par eux, rapproché déjà par les manèges de sa femme et par la
constante protection du duc de Beauvilliers, fut reçu comme toutes
choses non avenues. L'autre, bavard et franc Picard, eut le bon sens de
s'en aller aussitôt chez lui, pour éviter les questions importunes. Peu
capable de conseiller Mgr le duc de Bourgogne, il n'avait pu se
contraindre de reprendre en face et en public les enfantillages qui
échappaient à Mgr le duc de Bourgogne, et, sur son exemple, à M. le duc
de Berry. Il leur disait quelquefois qu'en ce genre ils auraient bientôt
un plus grand maître qu'eux, qui serait Mgr le duc de Bretagne.

Revenant une fois de la messe à la suite de Mgr le duc de Bourgogne,
dans des moments vifs où il l'aurait mieux aimé à cheval\,: «\,Vous
aurez, lui dit-il tout haut, le royaume du ciel, mais pour celui de la
terre, le prince Eugène et Marlborough s'y prennent mieux que vous.\,»

Ce qu'il dit, et tout publiquement encore, aux deux princes sur le roi
d'Angleterre, fut admirable. Ce pauvre prince vivait sous son incognito
dans le même respect avec les deux princes que s'il n'eût été qu'un
médiocre particulier. Eux aussi en abusaient avec la dernière indécence,
sans la moindre des attentions que ce qu'il était exigé d'eux, à travers
tous les voiles, jusqu'à le laisser très ordinairement attendre parmi la
foule dans les antichambres, et ne lui parlaient presque point. Le
scandale en fut d'autant plus grand qu'il dura toute la campagne, et que
le chevalier de Saint-Georges s'y était concilié l'estime et l'affection
de toute l'armée par ses manières et par toute sa conduite. Vers les
derniers temps de la campagne, Gamaches, poussé à bout d'un procédé si
constant, s'adressant aux deux princes devant tout le monde\,: «\,Est-ce
une gageure\,? leur demanda-t-i1 tout à coup\,; parlez franchement\,; si
c'en est une, vous l'avez gagnée, il n'y a rien à dire\,; mais au moins,
après cela, parlez un peu à M. le chevalier de Saint-Georges, et le
traitez un peu plus honnêtement.\,» Toutes ces saillies eussent été
bonnes tête à tête, et fort à propos, mais en public, ce zèle et ces
vérités n'en pouvaient couvrir l'indiscrétion. On était accoutumé aux
siennes, elles ne furent pas mal prises, mais elles ne servirent de
rien.

Boufflers, à bout de tout, comme je l'ai dit, ne put différer que de peu
de jours à obéir à l'ordre du roi qu'il avait reçu de capituler. Il fit
donc battre la chamade, et il obtint tout ce qu'il voulut par sa
capitulation, qui, sans dispute, fut signée le 9, de la meilleure grâce
du monde. Le prince Eugène était comblé d'honneur et de joie d'être venu
à bout d'une si difficile conquête, malgré une armée plus forte que la
leur, et commandée par l'héritier nécessaire de la couronne, et par
Vendôme, qui en discours l'avait si peu ménagé en Italie et en Flandre,
quoique enfants des deux soeurs.

Un jour avant que la garnison sortît, le prince Eugène envoya demander
au maréchal de Boufflers s'il voudrait bien recevoir sa visite, et dès
qu'il y eut consenti, Eugène la lui rendit. Elle se passa en forces
louanges et civilités de part et d'autre\,; il pria le maréchal à dîner
chez lui pour le lendemain, après que la garnison serait sortie, et il
fit rendre à Boufflers toutes sortes de respects et tous les mêmes
honneurs qu'à soi-même. Lorsque la garnison sortit, le maréchal ne
marcha point à sa tête, mais vint se mettre à côté du prince Eugène, que
le chevalier de Luxembourg et tous les officiers saluèrent. Après que
toute la garnison eut défilé, le prince Eugène fit monter le maréchal et
le chevalier de Luxembourg dans son carrosse, se mit sur le devant, et
voulut absolument que le chevalier de Luxembourg\,; qu'il avait fait
monter devant lui, se mit sur le derrière auprès du maréchal de
Boufflers, et donna toujours la main à la porte à tous les officiers
français que Boufflers mena dîner chez lui. Après dîner, il leur donna
son carrosse et beaucoup d'autres carrosses pour les mener coucher à
Douai, eux et les officiers principaux. Le prince d'Auvergne, et je
pense que ce ne fut pas sans affectation, à la tête d'un gros
détachement, lui toujours à cheval, les conduisit à Douai. Il eut ordre
du prince Eugène d'obéir en tout au maréchal, à qui il le dit, comme à
sa propre personne. Le maréchal fit coucher le prince d'Auvergne à Douai
cette nuit-là.

Le roi fut un peu choqué de ce que, parmi les trois otages que le prince
Eugène voulut retenir dans Lille, à son choix, pour le payement des
dettes faites par les François dans la ville, il exigea que Maillebois
en serait un, et ne se cacha pas qu'il le voulait comme étant le fils
aîné de Desmarets. Il lui permit de venir à la cour voir son père et d'y
passer quelques jours.

Dans l'intervalle de la capitulation et de la sortie de la garnison, et
lors de sa sortie, les ennemis ne se cachèrent point du siège de Gand
qu'ils allaient faire. Le duc de Marlborough s'était déjà campé tout
auprès, et c'est ce qui rendit la séparation de notre armée si
surprenante. Mais il n'y avait plus ni pain ni farines\,: il fallut
céder honteusement et périlleusement à la nécessité. Ils tinrent
parole\,; Gand fut investi le 11 décembre, par Marlborough, entre le
grand et le petit Escaut, et par le prince Eugène, entre la Lys et
l'Escaut, après avoir pourvu à Lille, où il laissa une grosse garnison.
Le comte de La Mothe commandait dans Gand, où il avait vingt-neuf
bataillons, plusieurs régiments de dragons, abondance de vivres,
d'artillerie, de munitions de guerre, et devant les yeux le grand
exemple du maréchal de Boufflers.

M. de Vendôme arriva à Versailles le matin du samedi 15 décembre, et
salua le roi comme il sortit de son cabinet pour venir se mettre à table
pour dîner à son petit couvert. Le roi l'embrassa avec une sorte
d'épanouissement qui fit triompher sa cabale. Il tint le de pendant tout
le dîner, où il ne fut question que de bagatelles. Le roi lui dit qu'il
l'entretiendrait le lendemain chez M\textsuperscript{me} de Maintenon.
Ce délai, qui lui était nouveau, ne lui fut pas de bon augure. Il alla
faire la révérence à Mgr le duc de Bourgogne, qui l'accueillit bien,
malgré tout ce qui s'était passé. Vendôme fut faire sa cour à
Monseigneur chez M\textsuperscript{me} la princesse de Conti, à son
retour de la chasse\,: c'était là surtout qu'il se tenait dans son fort.
Il fut reçu au mieux et fort entretenu de riens\,; il voulut en profiter
et engager un voyage d'Anet. Sa surprise fut grande et celle des
assistants, à la réponse incertaine de Monseigneur, qui fit pourtant
entendre, et sèchement, qu'il n'irait point. Vendôme parut embarrassé,
et il abrégea sa visite. Je le rencontrai dans le bout de la galerie de
l'aile neuve, comme je sortais de chez M. de Beauvilliers, qui tournait
au degré du milieu de la galerie. Il était seul, sans flambeaux ni
valets, avec Albéroni, suivi d'un homme que je ne connus point\,; je le
vis à la lueur des miens\,; nous nous saluâmes poliment de part et
d'autre\,; je n'avais aucune habitude avec lui. Il me parut l'air
chagrin et en chemin de chez M. du Maine, son conseil et son principal
appui.

Le lendemain, il ne fut pas une heure avec le roi chez
M\textsuperscript{me} de Maintenon. Il demeura huit ou dix jours à
Versailles ou à Meudon, et ne mit pas le pied chez M\textsuperscript{me}
la duchesse de Bourgogne\,: ce n'était pas pour lui une chose nouvelle.
Le mélange de grandeur et d'irrégularité qu'il avait dès longtemps
affecté l'avait, ce lui semblait, affranchi des devoirs dont on se
dispense le moins. Son abbé Albéroni se montrait à la messe du roi, en
courtisan, avec une effronterie sans pareille. Enfin ils s'en allèrent à
Anet. Dès avant que d'y aller, il s'était aperçu de quelque décadence,
puisqu'il s'abaissa jusqu'à convier le monde de l'y venir voir, lui qui,
les autres années, faisait grâce d'y recevoir, y regorgeait de tout ce
qu'il y avait de plus grand et de plus distingué, et ne s'y daignait
apercevoir du médiocre. Dès ce premier voyage, il sentit sa diminution
par celle de sa compagnie. Les uns s'en excusèrent, d'autres manquèrent
à l'engagement qu'ils avaient pris d'y aller. Chacun se mit à tâter le
pavé sur un voyage de quinze lieues, qui se mettait, les années
précédentes, pour le moins à côté de ceux de Marly. Vendôme se tint à
Anet jusqu'au premier Marly, où il vint le jour môme. Il en usa de la
sorte, toujours à Marly et à Meudon, jamais à Versailles, jusqu'au
changement dont j'aurai bientôt lieu de parler.

Le roi avait dépêché au maréchal de Boufflers, à Douai, pour le presser
de revenir. Il arriva le dimanche 15 décembre, le lendemain du duc de
Vendôme, héros factice de faveur et de cabale, sans que pas un des siens
même le crût tel\,; l'autre, héros malgré soi-même, par l'aveu public
des François et de leurs ennemis. Jamais homme ne mérita mieux le
triomphe, et n'évita avec une modestie plus attentive, mais la plus
simple, tout ce qui pouvait le sentir. Sa femme fut au-devant de lui dès
le matin, à quelques lieues de Paris, l'y amena dîner chez lui à huis
clos, et sans qu'on sût son arrivée, et de là à Versailles à la nuit,
droit à. leur appartement et sous clef.

Aussitôt il manda au duc d'Harcourt, en quartier de capitaine des
gardes, qu'il le priait de faire dire au roi qu'il était arrivé, et
qu'il attendait le moment de lui aller faire sa révérence. Le roi, qui
venait de finir l'audience de M. de Vendôme, lui fit dire sur-le-champ
de venir le voir chez M\textsuperscript{me} de Maintenon. En voyant
ouvrir la porte, le roi fut au-devant de lui, et dans la porte même
l'embrassa étroitement à deux et trois reprises, lui fit des
remerciements flatteurs et le combla de louanges. Pendant ces moments,
ils s'étaient avancés dans la chambre, la porte s'était fermée, et
M\textsuperscript{me} de Maintenon était venue féliciter le maréchal,
qui suivait le roi, lequel aussitôt, se tournant à lui, lui dit\,:
«\,Qu'ayant aussi grandement mérité de lui et de l'État qu'il venait de
faire, c'était à son choix qu'il en mettait la récompense.\,» Boufflers
s'abîma en respects, et répondit que de si grandes marques de
satisfaction le récompensaient au-dessus de ce qu'il pouvait non
seulement mériter, mais désirer. Le roi le pressa de lui demander tout
ce qu'il voudrait, et d'être sûr de l'obtenir à l'heure même\,; et le
maréchal toujours retranché dans la même modestie. Le roi insista encore
pour qu'il lui demandât, pour lui et pour sa famille, tout ce qu'il
pouvait désirer, et le maréchal persista à se trouver trop
magnifiquement payé de ses bontés et de son estime. «\,Oh bien\,!
monsieur le maréchal, lui dit enfin le roi, puisque vous ne voulez rien
demander, je vais vous dire ce que j'ai pensé, afin que j'y ajoute
encore quelque chose, si je n'ai pas assez pensé à tout ce qui peut vous
satisfaire\,: je vous fais pair, je vous donne la survivance du
gouvernement de Flandre pour votre fils, et je vous donne les entrées
des premiers gentilshommes de la chambre.\,» Son fils n'avait que dix ou
onze ans. Le maréchal se jeta aux genoux du roi, comblé de ses grâces
pardessus toutes espérances\,; il eut aussi en ce même moment la
survivance pour son fils des appointements du gouvernement particulier
de Lille. Le tout ensemble passe cent mille livres de rente.

Ces trois grâces, si bien méritées, étaient uniques alors, chacune dans
leur genre. Celle à laquelle le maréchal fut le plus sensible, quoique
touché de toutes au point où il devait l'être, fut la première.

La porte en était fermée depuis longtemps\,; le roi s'était repenti de
ces quatorze pairs qu'il avait faits en 1663\footnote{Voy., sur cette
  création de pairs, t. Ier, p.~449.}, tous ensemble, qui l'engagèrent
aux quatre qu'il ajouta en 1665. Il s'était déclaré qu'il n'en ferait
plus. De là les ducs vérifiés ou héréditaires qu'il fit depuis, que les
ignorants ont crus de son invention, et qui sont de toute ancienneté,
mais dont il n'y avait plus\footnote{Voy., sur les ducs vérifiés, t.
  Ier, p.~129, note.}. Bar n'a jamais été autre, les trois Nemours,
Longueville, Angoulême, Étampes et je ne sais combien d'autres.
L'archevêque de Paris, par sa faveur et par sa parole, et le duc de
Béthune, par le billet qu'il avait de sa main, comme je l'ai dit
ailleurs, la lui forcèrent encore, et avec nouvelle protestation qu'il
n'en ferait plus. Dégoûté aussi des survivances, par le peu de
satisfaction qu'il avait éprouvée de jeunes gens comblés avant l'âge, et
qui, n'avant plus rien de solide à prétendre, ne se souciaient plus de
rien mériter, il s'était si nettement expliqué sur cela depuis bien des
années que personne n'osait plus y songer. C'était une grâce réservée
aux seuls secrétaires d'État, parce qu'il n'en put jamais refuser à ses
ministres, et qu'il se complaisait à se servir de jeunes gens dans ces
places si importantes, pour montrer qu'il gouvernait seul et qu'il les
formait, bien loin d'être gouverné par eux, quoique jamais prince ne le
fût tant que lui.

Les grandes entrées, depuis la mort du père de La Feuillade, M. de
Lauzun était le seul homme qui les eût sans charge qui les donnât. Outre
la distinction et la commodité, cette grâce était regardée comme
principale, par la facilité qu'elle donnait de parler au roi sans
témoins et sans audiences, rares et difficiles à obtenir, et qui
toujours faisaient nouvelles, et de lui parler tous les jours et en
différentes heures avec toute liberté.

Boufflers eut la satisfaction qu'il ne se trouva qui que ce soit, parmi
une cour si envieuse et clans toute la France, qui n'applaudit à ce que
le roi fit pour lui, et qui ne trouvât également juste et séant qu'il
fût récompensé par une dignité la première du royaume, dont l'éclat
passait à sa postérité, par une privante également flatteuse par sa
familiarité et sa singularité, enfin par la conservation dans sa
famille, même sur la tête d'un enfant, d'un gouvernement qu'il avait si
dignement défendu, presque malgré le roi, et sans aucun besoin de le
faire, ni par son devoir d'y aller, ni pour sa réputation tout acquise,
ni pour sa fortune si grandement dès lors achevée.

On remarqua à sa gloire la différence de la défense de Namur, avec une
excellente garnison, mais sous la tutelle de l'ingénieur Mesgrigny,
quoique cette défense eût été fort belle, d'avec celle de Lille, qui
avait roulé sur lui seul, presque sans garnison, que de milices et de
troupes nouvelles qui ne valaient pas mieux, des munitions de guerre et
de bouche très médiocres, encore moins d'argent, et de l'avoir fait
durer plus de six semaines au delà de ce que le célèbre Vauban, qui
avait construit la place à plaisir, avait dit qu'il la pourrait
défendre, munie de tout ce qu'il aurait désiré.

Mais ce qui mit le comble à la gloire de Boufflers, et tout le monde à
ses pieds, fut cette rare et vraie modestie de laquelle rien ne le put
ébranler, et qui lui fit constamment rapporter à sa garnison toute la
réputation qui l'environnait, et à la pure bonté du roi l'éclat nouveau
dont il brillait par des grâces si distinguées et si complètes. À le
voir, on eût dit qu'il en était honteux\,; et, à travers la joie qu'il
ne cachait pas, on était saisi d'une vérité et d'une simplicité si
naturelles qui sortaient de lui et qui relevaient jusqu'à ses moindres
discours. Il le détournait toujours de ses louanges par celles de sa
garnison, et il avait toujours quelque action de quelqu'un à raconter
toute prête pour fermer la bouche sur les siennes.

Ce contraste avec Vendôme, arrivé de la veille, se fit bien remarquer\,:
l'un, élevé à force de machines et entassant les montagnes comme les
géants, appuyé du vice, du mensonge, de l'audace, d'une cabale ennemie
de l'État et de ses héritiers, un héros factice, érigé tel par volonté,
en dépit du vrai\,; l'autre, sans cabale, sans appui que de sa vertu, de
sa modestie, du soin de relever les autres et de s'éclipser derrière
eux, vit les grâces couler sur lui de source jusqu'à l'inonder, et les
applaudissements des ennemis suivis des acclamations publiques jusqu'à
changer la nature des courtisans, qui s'estimèrent comblés eux-mêmes de
ses récompenses.

N'oublions pas qu'il fit donner six mille livres d'augmentation de
pension au chevalier de Luxembourg, qui en avait déjà autant, et qui
avait été fait lieutenant général, comme je l'ai dit, pour être entré
dans Lille avec le secours et les poudres qu'il y jeta.

Peu de jours après le retour de Mgr le duc de Bourgogne, Cheverny,
sortant d'avec lui tête à tête, et qui était homme très véritable, me
fit un récit que je ne puis me refuser de mettre ici, et que toutefois
je n'y puis écrire sans confusion. Il me dit que, lui parlant avec
liberté des propos tenus sur lui pendant la campagne, le prince lui dit
qu'il savait comment et avec quelle vivacité j'en avais parlé, et qu'il
était instruit aussi de la manière dont M. le prince de Conti s'en était
expliqué, et ajouta que, lorsqu'on avait la voix de deux hommes
semblables, on avait lieu de se consoler des autres. Cheverny, qui en
était plein, me le vint raconter à l'instant. Je le fus de confusion
d'être mis à côté d'un homme plus supérieur encore à moi en ce genre
qu'il ne l'était en rang et en naissance\,; mais je sentis avec
complaisance combien M. de Beauvilliers m'avait effectivement tenu
parole lorsque je voulus aller à la Ferté.

Le duc de Berwick arriva à la cour le dimanche 23 décembre. Il ne se
contraignit ni en particulier ni en public sur M. de Vendôme, ni sur
tout ce qui s'était passé en Flandre. À son exemple, presque tout ce qui
en était revenu commença à parler. Les manèges sur le secours de Lille,
les mensonges de Pont-à-Marck et de Mons-en-Puelle, celui sur les
retranchements de Marlborough, le passage de l'Escaut, furent dévoilés
et mis au clair\,; l'ignorance où la retenue d'écrire en avait laissé le
gros du monde y causa un étonnement étrange, puis une indignation à quoi
la cabale de Vendôme ne put opposer que des verbiages entortillés et des
menaces secrètes, qui démontrèrent encore plus manifestement les vérités
si longuement suffoquées. Cette cabale commençait à être embarrassée du
succès si différent de l'arrivée de son héros, du peu de gens qui
allaient à Anet, et du bruit fort répandu que Mgr le duc de Bourgogne
servirait la campagne suivante, et n'aurait que des maréchaux de France
sous lui. L'air de disgrâce commençait à se faire sentir\,; elle ne
tarda pas à se déclarer tout entière.

Chamillart, pénétré de l'importance de la perte de Lille, amoureux du
bien de l'État et de la gloire personnelle du roi, avait conçu le
dessein de le reprendre incontinent après la séparation de l'armée des
ennemis, et le départ du prince Eugène et du duc de Marlborough de
Hollande. Son projet était fait, beau, bien conçu, bien digéré\,; il y
avait mis la dernière main à son dernier voyage en Flandre, et tous ses
arrangements faits, jusqu'à des troupes de l'armée qui avait servi en
Dauphiné et en Savoie, qu'il faisait venir en Flandre. Il voulait faire
marcher le roi pour donner vigueur aux troupes, et à lui seul l'honneur
de la conquête\,; mais comme l'argent était difficile, et que ce siège
serait cher, il avait résolu que les équipages seraient courts, et
surtout que les dames ne seraient pas du voyage, qui ne causent que
beaucoup de dépense et d'embarras à mener sur la frontière.

Pour s'en mieux assurer, il fallait cacher ce projet en entier à
M\textsuperscript{me} de Maintenon, et obtenir du roi d'y consentir et
de lui en garder le secret jusqu'au bout. Chamlay, à qui Chamillart le
confia, et avec qui il acheva de prendre les plus justes mesures,
approuva fort cet excellent projet, mais en ami il avertit Chamillart
qu'il jouait à se perdre\,; que M\textsuperscript{me} de Maintenon ne le
lui pardonnerait point\,; qu'un semblable dessein pour Mons, où Louvois
ne voulait point mener les dames, l'avait perdu sans ressource, quoique
plus ancré et plus établi que lui\,; que tout cela avait passé sous ses
yeux\,; qu'il se fît sage par un si funeste exemple, et qui avait suivi
la conquête de Mons de si près, puisque lui-même ne pouvait avoir oublié
qu'il savait par le roi même que si Louvois ne fût pas mort le jour
qu'il mourut si subitement, il était arrêté le lendemain même\,; et il
est vrai que Chamillart me l'a conté et m'a dit qu'il l'avait appris du
roi.

Chamillart sentit tout le danger, mais il était courageux, il aimait
l'État, et je puis dire le roi comme on aime une maîtresse. Il le compta
pour tout, soi pour rien, et passa outre. Tout bien mâché et bien
préparé, il communiqua son projet au roi, qui fut charmé de l'ordre, de
la facilité, de la beauté.

Là-dessus le maréchal de Boufflers, destiné à faire ce siégé sous le
roi, eut communication de tout, et fut renvoyé en Flandre sous prétexte
d'y donner divers ordres pendant une partie de l'hiver, en effet pour
disposer tout sur les lieux et y attendre le roi. Mais pour ne donner
point d'ombrage, on se contenta pour lors de laisser en Flandre les
officiers généraux nommés dès avant la fin de la campagne, pour y servir
l'hiver, sans leur rien communiquer du secret\,; on ne voulut pas même
renvoyer aucun colonel, ni aucun des officiers particuliers qui étaient
revenus.

Le roi, engoué de ce projet, et qui n'avait pas accoutumé de rien cacher
à M\textsuperscript{me} de Maintenon, importuné sans doute de ne
travailler à cela que chez lui avec Chamillart, à des heures rompues, ne
put tenir plus longtemps à se mettre au large, se promettant bien qu'il
rendrait M\textsuperscript{me} de Maintenon capable des solides et
pressantes raisons qui devaient la faire demeurer à Versailles avec
M\textsuperscript{me} la duchesse de Bourgogne et toutes les dames. Il
lui confia donc cet admirable projet. M\textsuperscript{me} de Maintenon
eut l'adresse de cacher sa surprise et la force de dissimuler
parfaitement son dépit\,; elle loua le projet, elle en parut charmée,
elle entra dans les détails, elle en parla à Chamillart, admira son
zèle, son travail, sa diligence, et surtout d'avoir conçu un si beau et
si grand exploit, et de l'avoir rendu possible.

Boufflers partit le 26 décembre, et le même jour Berwick eut une longue
audience du roi chez M\textsuperscript{me} de Maintenon, où il parla en
toute liberté, malgré toute sa timide politique.

Mais il était à bout des procédures et des procédés. Les régiments des
gardes françaises et suisses eurent ordre le même jour de se tenir prêts
à marcher le ter février. On verra dans les commencements de l'année
prochaine, le succès de ces grands préparatifs.

La tranchée fut ouverte à Gand la nuit du 24 au 25 décembre où le comte
de La Mothe avait pour deux mois de vivres, tant pour la garnison que
pour les habitants, qui étaient quatre-vingt mille\,; beaucoup de canon
et de mortiers, et quatre cent milliers de poudre. M\textsuperscript{me}
de Ventadour, qui s'obstinait à le vouloir voir maréchal de France, lui
procura encore cette défense, pour effacer le funeste succès de ce grand
convoi des ennemis qu'il voulait enlever, et qui le battit si
vilainement, par où s'acheva la perte de Lille.

La dernière soirée de cette année fut fort remarquable, parce qu'elle
n'avait point eu d'exemple. Le roi étant entré, au sortir de son souper,
dans son cabinet avec sa famille, à l'ordinaire, Chamillart y vint sans
être mandé. Il dit au roi, à l'oreille, qu'il lui apportait une grande
dépêche du maréchal de Boufflers. Aussitôt le roi donna le bonsoir à
Monseigneur et aux princesses, qui sortirent avec tout ce qui était dans
les cabinet, et le roi travailla une heure avec son ministre avant de se
coucher, tant il était épris du grand projet de la reprise de Lille.

\hypertarget{chapitre-iii.}{%
\chapter{CHAPITRE III.}\label{chapitre-iii.}}

1709

~

{\textsc{La Mothe rend Gand et est exilé.}} {\textsc{- La Boulaye,
gouverneur d'Exilles, à la Bastille, pour l'avoir rendu.}} {\textsc{- La
Junquière dégradé et prisonnier pour avoir rendu le Port-Mahon.}}
{\textsc{- Mort de M\textsuperscript{me} de Villetaneuse.}} {\textsc{-
Mort des deux neveux du maréchal de Boufflers.}} {\textsc{- Mort du
président Molé.}} {\textsc{- Mort, fortune et caractère de la maréchale
de La Mothe et de son mari.}} {\textsc{- Mort de la duchesse
d'Holstein\,; sa postérité et ses prétentions.}} {\textsc{- Mort du
prince Georges de Danemark.}} {\textsc{- Voyage oublié du prince royal
de Danemark en France, qui pensa perdre Broglio, qui lors commandait en
Languedoc, et est mort maréchal de France.}} {\textsc{- Projet de la
reprise de Lille avorté.}} {\textsc{- Froid extrême et ruineux.}}
{\textsc{- Vendôme exclu de servir.}} {\textsc{- Deux cent mille livres
de brevet de retenue au duc d'Harcourt sur sa charge de Normandie.}}
{\textsc{- Pensions de la duchesse de Ventadour.}} {\textsc{- Grâces
pécuniaires à M\textsuperscript{lle} de Mailly.}} {\textsc{- Accidents
de La Châtre\,; son caractère.}} {\textsc{- Prié plénipotentiaire, puis
ambassadeur de l'empereur à Rome\,; sa fortune, son caractère.}}
{\textsc{- Embarras et conduite de Tessé à Rome.}} {\textsc{- Mort de
Quiros\,; sa fortune\,; sa défection.}}

~

Dès en arrivant à Douai, Boufflers se mit à rassembler une armée. Il y
fut tôt après suivi des officiers généraux qu'on y envoya, et de tous
les colonels qui, à leur retour, avaient salué le roi et en avaient été
bien reçus. Boufflers, quoique tout occupé de l'exécution du grand
projet de reprendre incontinent Lille, ne laissait pas de songer à
délivrer Gand, en tombant sur les quartiers des ennemis séparés les uns
des autres par les rivières\,; mais c'est bien dit qu'il y songea, car
il n'eut pas même le temps d'y travailler. La tête tourna à La Mothe\,;
car il était entièrement incapable de lâcheté et d'infidélité, et il
n'avait qu'à mériter le bâton par une telle défense, sûr de l'obtenir.
Il se laissa empaumer par un capitaine suisse qui eut peur pour sa
compagnie et peut-être aussi pour sa peau, qui lui persuada si bien de
se rendre au bout de trois jours de tranchée ouverte qu'il capitula, et
sa garnison de vingt-neuf bataillons et de plusieurs régiments de
dragons sortit tout entière le 29 décembre, et fut conduite à Gand. Elle
y laissa quatre-vingt milliers de poudre, quatre mille mousquets de
rechange et beaucoup de canon. Il n'y eut ni sédition, ni murmure des
bourgeois, ni aucun coup de main depuis l'investiture jusqu'à la
capitulation. La Mothe surprit extrêmement les chefs des corps qu'il
assembla, non pour les consulter, mais pour leur déclarer la résolution
qu'il avait prise, et sans prendre leur avis. Capres, lieutenant général
des troupes espagnoles et qui avait le titre de gouverneur de Gand, ne
put jamais être persuadé de signer la capitulation, et cet exemple fut
suivi de beaucoup d'autres.

Gavaudan, aide de camp du comte de La Mothe, et fort attaché à M. du
Maine, à qui il fut depuis, apporta cette belle nouvelle au roi qui ne
voulut pas le voir, et qui pour réponse envoya au comte de La Mothe une
lettre de cachet qui le reléguait chez lui près de Compiègne, en un lieu
qui s'appelle Fayet. Ni la duchesse de Ventadour ni Chamillart ne purent
enfin parer ce coup après tant d'autres sottises qu'ils lui avaient
sauvées, et il y demeura plus d'un an sans être plaint de personne.

Les ennemis s'en moquèrent fort, et se trouvèrent bien heureux qu'il
n'eût pas tenu deux jours davantage. Il plut si abondamment et si
continuellement qu'ils auraient été forcés de lever le siège pour ne pas
y être noyés, et la saison devint tout de suite si rigoureuse qu'ils
n'auraient pu y revenir. La Mothe n'eut jamais d'autre excuse que celle
que la place était mauvaise, et qu'il avait voulu conserver une si belle
et nombreuse garnison\,; mais elle n'était pas meilleure quand il y
entra avec elle\,; pour tenir trois jours ce n'était pas la peine de
s'en charger. Jamais homme si inepte\,; et l'esprit de vertige et
d'aveuglement était tellement répandu sur nous depuis très longtemps que
l'ineptie était un titre de choix et de préférence en tout genre, sans
que les continuelles \emph{expé} riences en pussent désabuser.

De cette affaire-là nous évacuâmes Bruges et le fort de Plassendal qui
ne se pouvaient plus soutenir\,; ce qu'il y avait de troupes se retira à
Saint-Omer. Ces faciles conquêtes couronnèrent la belle campagne du
grincé Eugène et du duc de Marlborough. Ils séparèrent leurs armées, et
ils s'en allèrent triompher à la Haye, et y donner leurs soins aux
préparatifs de la campagne prochaine\,; ils y furent assez longtemps
tous deux. Le prince Eugène s'en alla après à Vienne. Marlborough
demeura à la Haye avec parole au prince Eugène, qu'il lui tint, de ne
point passer la mer qu'il ne fût de retour à la Haye, pour ne point
laisser leur ami Heinsius ni les États généraux sans l'un des deux.

La Boulaye, qui s'était rendu prisonnier de guerre avec sa garnison à
Exilles, dont il était gouverneur, fut échangé en ce temps-ci. Il était
chargé de choses fort fâcheuses\,; il vint demander d'être mis à la
Bastille pour y être condamné ou justifié. Il y a apparence qu'il ne fit
que prévenir ce qui était résolu. Il y fut interrogé plusieurs fois.

La Junquière, qui s'était laissé prendre si vilainement au Port-Mahon,
fut mis à Toulon au conseil de guerre, où présida Langeron, lieutenant
général des armées navales. Il fut jugé à être cassé et à garder
prison\,; ensuite le roi lui ôta ses pensions et la croix de
Saint-Louis, le fit casser et dégrader des armes, l'envoya dans un
château de Franche-Comté, et fit mettre en diverses prisons les
officiers qui étaient avec lui à Exilles\footnote{Nous reproduisons
  textuellement le manuscrit qui porte le nom d'Exilles, dont il a été
  question dans le paragraphe précédent. Il faut lire probablement
  Port-Mahon.}.

M\textsuperscript{me} de Villetaneuse, vieille bourgeoise fort riche et
sans enfants, mourut les premiers jours de cette année, et enrichit par
ses legs les enfants du duc de Brancas, fils de sa sueur, la duchesse de
Luxembourg, fille de sa cousine germaine, et la comtesse de Boufflers,
fille de Guénégaud, son cousin germain. Cette comtesse de Boufflers
était veuve du frère aîné du maréchal, avec qui elle vivait en grande
intelligence. Elle avait eu deux fils dont il prit soin. L'aîné mourut
en sortant de l'Académie \footnote{Ce mot désignait, aux XVIIe et XVIIIe
  siècles, une école d'équitation pour les jeunes nobles.
  M\textsuperscript{me} de Motteville, parlant de l'entrée des
  ambassadeurs de Pologne à Paris en 1645, appelle ces jeunes gens
  \emph{académistes\,;} «\,après eux (les ambassadeurs) venaient nos
  \emph{académistes}.\,»}\,; l'autre, fort peu après, se battit en duel
si imprudemment, que ce combat ne se put pallier, et qu'il lui fallut
aller chercher fortune hors du royaume, où il est mort assez tôt après.

Molé, président à mortier, mourut aussi fort mal dans ses affaires\,; il
avait obtenu sa survivance pour son fils fort jeune. Le roi n'avait
jamais oublié les services que lui avait rendus pendant les troubles de
sa minorité le premier président Molé, à qui il donna les sceaux.

La maréchale de La Mothe mourut le 6 janvier, dont la généalogie et la
fortune méritent d'être expliquées pour la singularité. Elle était
seconde fille de Louis de Prie, marquis de Toucy, et de Françoise, fille
de Guy de Saint-Gelais, seigneur de Lansac, et de la fille du maréchal
de Souvré, qui fut gouverneur de Louis XIII. M\textsuperscript{me} de
Lansac fut gouvernante de Louis XIV. Elle était ainsi grand'mère de la
maréchale de La Mothe, qui fut gouvernante des enfants de Louis XIV, de
ses petits-fils et de ses arrière-petits-fils. Elle eut en survivance
pour les derniers la duchesse de Ventadour, sa fille, qui ensuite a eu
en survivance la princesse de Soubise, femme de son petit-fils, après la
mort de laquelle elle a eu la duchesse de Tallard, sa petite-fille, qui
par la démission longtemps depuis de M\textsuperscript{me} de Ventadour,
est maintenant gouvernante en titre. Ainsi le maréchal de Souvré,
M\textsuperscript{me} de Lansac, la maréchale de La Mothe, la duchesse
de Ventadour, et les deux belles-soeurs petites-filles de celles-ci,
font cinq générations de gouverneurs et gouvernantes des enfants de
France, dont trois rois et plusieurs dauphins.

Le maréchal de La Mothe fut fait maréchal de France avant trente-huit
ans, en 1642, à force de grandes et de belles actions, à quantité
desquelles il avait commandé en chef. Il continua avec le même bonheur
encore deux ans, avec la vice-royauté de Catalogne. Il obtint en ce
pays-là le duché de Cardone, confisqué sur le propriétaire demeuré
fidèle à l'Espagne, et à ce titre il eut un brevet de duc, c'est-à-dire
des lettres non vérifiées. En 1644, il perdit la bataille de Lerida
contre les Espagnols, et leva le siège de Tarragone. Il fut calomnié, et
les intrigues de la cour s'en mêlèrent. C'était un homme qui n'avait
d'appui que ses actions et son mérite\,; il fut arrêté et demeura quatre
ans à Pierre-Encise. Son innocence fut prouvée au parlement de
Grenoble\,;. il épousa ensuite la maréchale de La Mothe, qui était fort
belle et qui a toujours été fort vertueuse. En 1651, il fut une seconde
fois vice-roi de Catalogne. Il y força les lignes de Barcelone, et
défendit cette place cinq mois durant. Il mourut à son retour â Paris en
1657, à cinquante-deux ans, et laissa trois filles qui ont été duchesses
d'Aumont, de Ventadour et de La Ferté, et la maréchale de La Mothe,
pauvre, à trente-quatre ans.

Elle vécut la plupart du temps à la campagne. Elle y était lorsque
M\textsuperscript{me} de Montausier, ne pouvant suffire à ses deux
charges de gouvernante de Monseigneur et de dame d'honneur de la reine,
obtint enfin d'être soulagée de la première. M. Le Tellier, et M. de
Louvois son fils, étaient lors en grand crédit, et fort attentifs à
procurer, tant qu'ils pouvaient, les principales places à des personnes
sur qui ils pussent compter, au moins à en écarter celles qu'ils
craignaient. M. de Louvois avait épousé l'héritière de Souvré, que le
maréchal de Villeroy son tuteur lui sacrifia, ou plutôt à sa faveur. La
maréchale de La Mothe était cousine germaine du père de
M\textsuperscript{me} de Louvois\,; elle était belle et d'un âge
convenable, d'une conduite qui l'était aussi. Ils furent avertis à temps
que M\textsuperscript{me} de Montausier obtenait enfin de quitter
Monseigneur. Ils bombardèrent la maréchale de La Mothe en sa place, que
personne ne connaissait à la cour, avant que qui que ce soit sût qu'elle
était enfin vacante. C'était la meilleure femme du monde, qui avait le
plus de soin des enfants de France, qui les élevait avec le plus de
dignité et de politesse, qui elle-même en avait le plus, avec une taille
majestueuse et un visage imposant, et qui avec tout cela n'eut jamais le
sens commun et ne sut de sa vie ce qu'elle disait\,; mais la routine, le
grand usage du monde la soutint. Elle passa sa vie à la cour dans la
plus grande considération, et dans une place où malgré une vie
splendide, et beaucoup de noblesse d'ailleurs, elle s'enrichit
extrêmement, et laissa encore de grands biens après avoir marié
grandement ses trois filles. Sa santé dura autant que sa vie. Elle
coucha encore dans la chambre de Mgr le duc de Bretagne la nuit du
vendredi au samedi. Elle s'affaiblit tellement le samedi, qu'elle reçut
les sacrements, et mourut le dimanche, à quatre-vingt-cinq ans.

La duchesse d'Holstein, sueur du roi de Suède, mourut de la petite
vérole à Stockholm, où elle était demeurée auprès de la reine sa
grand'mère, depuis la mort de son mari, tué en une bataille que le roi
de Suède gagna, comme je l'ai dit en son lieu. L'un et l'autre étaient
fort aimés du roi de Suède. Elle était l'aînée de la reine de Suède, qui
vient de mourir épouse du roi de Suède, landgrave de Hesse-Cassel, qui
est le même que nous avons vu prince héréditaire de Hesse-Cassel, battu
par Médavy en Lombardie dans le temps de la bataille de Turin, et battu
par le maréchal de Tallard à la bataille de Spire. Cette duchesse
d'Holstein laissa un fils bossu et médiocre sujet, qui fut gendre du
czar Pierre fer. Il mourut jeune après sa femme, et ne laissa qu'un fils
tout à fait enfant sous la tutelle de l'évêque d'Eutin, {[}son{]} oncle
paternel. Il a maintenant quatorze ans, et depuis la dernière révolution
de Russie y est allé, appelé par la czarine Élisabeth, soeur cadette de
sa mère, qui lui a fait une maison et le traite en héritier présomptif
de la Russie. Il prétend que le roi de Suède l'est à son préjudice, et
{[}qu'il doit{]} au moins lui succéder au titre de sa mère. Le roi de
Suède n'a point d'enfants et voudrait bien que son neveu, fils de son
frère, lui succédât en Suède, qui est gendre du roi d'Angleterre. La
Suède s'est déclarée élective, et il y a deux partis dans les états. Ce
duc d'Holstein prétend encore le duché d'Holstein et le comté
d'Oldenbourg, que le roi de Danemark lui retient et à ses pères, quoique
de même maison tous deux, et ces États `et tout l'apanage de ses cadets.
Voilà bien des prétentions qui, si elles avaient -toutes lieu, feraient
dans le Nord un trop formidable monarque.

Cette matière étrangère me rappelle la mort du prince Georges de
Danemark, sans enfants de la reine Anne d'Angleterre, son épouse,
arrivée dans les derniers temps de l'année qui vient de finir. Le peu de
figure qu'il a faite toute sa vie, même en Angleterre où il l'a toute
passée, m'y a fait faire moins d'attention. C'était un très bon homme,
fils de Frédéric III, roi de Danemark, et frère de Christiern V,
grand-père du roi de Danemark d'aujourd'hui. Il avait épousé en 1685 la
seconde fille du duc d'York mort à Saint-Germain roi d'Angleterre,
Jacques II. Ce prince Georges s'établit en Angleterre sans songer plus à
son pays, y vit tranquillement la révolution qu'y fit le prince d'Orange
en 1688, vécut paisible à sa cour, et ne se mêla jamais de rien, non pas
même depuis que sa femme fut reine, qui avait toujours fort bien vécu
avec lui avant et depuis. Il eut le titre de duc de Cumberland, la
Jarretière, et depuis le couronnement de sa femme le vain titre d'amiral
d'Angleterre, de généralissime de toutes les forces de la
Grande-Bretagne, et le gouvernement des Cinq-Ports, sans s'être jamais
mêlé de rien. Il avait eu plusieurs enfants, tous morts jeunes avant
lui.

Il me fait souvenir de dire que le roi de Danemark son neveu, mal avec
sa femme et sa mère, s'était mis à voyager sur la lin de l'année
précédente, et qu'il était en ce temps-ci à Venise pour y voir le
carnaval. Il était venu en France étant prince royal, et promettait fort
peu, et je m'aperçois que j'ai oublié ce voyage\,: quoique incognito, il
fut reçu partout en France avec une grande distinction\,; il s'arrêta
assez longtemps à Montpellier venant d'Italie, et y fit l'amoureux d'une
dame que Broglio aimait aussi. Il commandait en Languedoc par le crédit
de Bâville, frère de sa femme. Il s'avisa de trouver mauvais que le
prince royal tournât autour d'elle et qu'elle le reçût bien. Sa jalousie
l'emporta à manquer de respect au prince jusqu'à le menacer. Son
gouverneur à son tour le menaça de le faire jeter par les fenêtres. Sur
cela courriers à la cour. Le roi suspendit Broglio de tout commandement,
et ordonna à Bâville de le mener demander pardon en propres termes au
prince. Bâville l'exécuta et s'entremit si bien, que le prince demanda
au roi le rétablissement` de Broglio, auquel il ne laissa pas, et son
gouverneur aussi, de faire essuyer force rudes mortifications. Le roi se
fit prier et n'accorda le rétablissement de Broglio que lorsque le
prince fut sur le point de partir de Montpellier.

Il ne vit le roi et Monseigneur qu'en particulier dans leur cabinet. Le
roi le fit couvrir et demeura debout\,; Monseigneur lui donna la main et
un fauteuil, mais sans sortir de son cabinet et seuls. Il y eut un grand
bal paré, fort magnifique, dans le grand appartement du roi à
Versailles, où il fut sans rang, incognito\,; mais le roi lui vint
parler plus d'une fois, et {[}il eut{]} au rang près tous les honneurs
et les distinctions les plus marquées. M. de La Trémoille, qui par sa
mère était son cousin germain, en fit les honneurs. Il logea à Paris
dans une maison garnie. Monsieur et Madame, aussi sa cousine germaine,
eurent pour lui les plus grandes attentions. Il fut assez peu à Paris,
et s'en retourna en Danemark en voyageant.

Tandis que Boufflers achevait d'user sa santé pour les préparatifs
secrets de la reprise de Lille, M\textsuperscript{me} de Maintenon
n'oubliait rien pour en faire avorter le projet. La première vue l'avait
fait frémir, la réflexion combla la mesure de son dépit, de ses
craintes, et de sa résolution de rompre ce coup. Être séparée du roi
pendant un long siège, le laisser livré à un ministre à qui il saurait
gré de tout le succès, et pour qui son goût ne s'était pu démentir
jusqu'alors, un ministre sa créature à elle, qui avait osé mettre son
fils dans la famille de ceux qu'elle regardait comme ses ennemis, qui,
sans elle, et par cette même famille, avait eu le crédit de ramener
Desmarets sur l'eau, de vaincre la répugnance extrême du roi à son
égard, de le faire contrôleur général des finances, enfin ministre,
c'étaient déjà des démérites qui allaient jusqu'à la disgrâce. Mais sa
conduite sur Mgr le duc de Bourgogne et M. de Vendôme, et le projet fait
et résolu à son insu au siège de Lille, et sans l'y mener, lui montra un
danger si pressant qu'elle crut ne devoir rien épargner pour -le rompre
et pour se défaire après d'un ministre assez hardi pour oser se passer
d'elle, assez accrédité auprès du roi pour y réussir, et assez puissant
par ses autres liaisons pour avoir soutenu Vendôme, malgré elle, contre
Mgr et M\textsuperscript{me} la duchesse de Bourgogne. Elle alla d'abord
au plus pressé, et profita de tous les moments avec tant d'art, que le
projet de Lille ne parut plus au roi si aisé, bientôt après difficile,
ensuite trop hasardeux et ruineux\,; en sorte qu'il fut abandonné, et
que Boufflers eut ordre de tout cesser et de renvoyer tous les officiers
qu'on avait fait retourner en Flandre.

M\textsuperscript{me} de Maintenon fut heureuse d'avoir à s'avantager de
l'excès du froid. Il prit subitement la veille des Rois, et fut près de
deux mois au delà de tout souvenir. En quatre jours la Seine et toutes
les autres rivières furent prises, et, ce qu'on n'avait jamais vu, la
mer gela à porter le long des côtes. Les curieux observateurs
prétendirent qu'il alla au degré où il se fait sentir au delà de la
Suède et du Danemark. Les tribunaux en furent fermés assez longtemps. Ce
qui perdit tout et qui fit une année de famine en tout genre de
productions de la terre, c'est qu'il dégela parfaitement sept ou huit
jours, et que la gelée reprit subitement et aussi rudement qu'elle avait
été. Elle dura moins, mais jusqu'aux arbres fruitiers et plusieurs
autres fort durs, tout demeura gelé. M\textsuperscript{me} de Maintenon
sut tirer parti de cette rigueur de temps si extraordinaire, qui, en
effet, aurait causé d'étranges contretemps pour un siège. Elle y
joignait toutes les autres raisons dont elle se put aviser, et vint
ainsi à bout de ce qu'elle crut la plus importante affaire de sa vie,
avec le mérite d'avoir approuvé d'abord ce qu'elle ne parut détruire que
par les plus fortes raisons. Chamillart en fut très touché mais peu
surpris. Dès qu'il vit le secret échappé et M\textsuperscript{me} de
Maintenon instruite, il n'espéra plus que faiblement. Ce prélude put dès
lors lui faire craindre l'accomplissement personnel de ce que Chamlay
lui avait prédit.

Cependant M. de Vendôme continuait à être payé comme un général d'armée
qui sert l'hiver, et d'avoir cent places de fourrage, quoique dans Anet,
et des voyages de Marly et de Meudon. Cela avait tout à fait l'air de
servir la campagne suivante\,; personne n'osait en douter, et la cabale
en prenait de nouvelles forces. Ce petit triomphe ne fut pas long. M. de
Vendôme vint à Versailles pour la cérémonie ordinaire de l'ordre, à la
Chandeleur. Il y apprit qu'il ne servirait point, et qu'il ne serait
plus payé de général d'armée. Le camouflet fut violent, il le sentit en
entier\,; mais, en homme alors aussi mesuré qu'il l'avait été peu dans
la confiance en ses appuis, il avala la pilule de bonne grâce parce
qu'il en craignit de plus amères qu'il sentait n'avoir que trop
méritées, et auxquelles celle-ci le pouvait si naturellement conduire.
C'est ce qui le rendit pour la première fois de sa vie si endurant. Il
n'en fit pas mystère, sans néanmoins s'expliquer si c'était de son gré
ou non, s'il en était aise ou fâché, mais comme d'une nouvelle qui
aurait regardé un indifférent, et sans changer de conduite sur rien,
sinon en discours dont l'audace fût rabattue comme n'étant plus de
saison. Il fit vendre ses équipages.

Le duc d'Harcourt avait voulu vendre sa charge de lieutenant général de
Normandie. Marché fait à trois cent mille livres avec Le Bailleul\,;
capitaine aux gardes, le roi refusa l'agrément. Harcourt se plaignit
fort de l'embarras où cela le mettait, et obtint par là deux cent mille
livres de brevet de retenue sur cette charge qu'il garda.

En même temps le roi conserva à la duchesse de Ventadour douze mille
livres de pension qu'elle avait comme survivancière de sa mère, une
autre de dix mille livres qu'elle avait antérieurement, tellement que,
avec quarante-huit mille livres d'appointements de gouvernante en titre
par la mort de sa mère, elle eut du roi soixante-dix mille livres de
rente.

M\textsuperscript{lle} de Mailly, fille de la dame d'atours, eut aussi
six mille livres de pension et vingt-cinq mille écus sur l'hôtel de
ville, en récompense d'un avis que sa mère donna à Desmarets dont le roi
tira quelque chose. Cela s'appelle faire des affaires, et Desmarets
n'était pas homme, tout rébarbatif qu'il fût, à ne se pas prêter
là-dessus aux dames, surtout à celles qui tenaient à
M\textsuperscript{me} de Maintenon de si près.

Il arriva, le jeudi 17 janvier, un accident à La Châtre, à la comédie à
Versailles, qui en apprit de précédents. C'était un homme de qualité,
fort bien fait, qui ne le laissait point ignorer, fils du frère de la
maréchale d'Humières, fort honnête homme, fort brave, extrêmement
glorieux, fort dans le monde et toute sa vie amoureux et galant. On
l'appelait le \emph{beau berger}, et volontiers on se moquait de lui. Il
était lieutenant général, mais homme sans nul esprit et de nul talent à
la guerre, ni pour aucune autre chose. Ses manières étaient
naturellement impétueuses, qui redoublèrent peu à peu, et qui le
menèrent à des accès fâcheux. Ce soir-là, au milieu de la comédie, le
voilà tout d'un coup à s'imaginer voir les ennemis, à crier, à
commander, à mettre l'épée à la main, et à vouloir faire le moulinet sur
les comédiens et sur la compagnie. La Vallière, qui se trouva assez près
de lui, le prit à bras-le-corps, lui fit croire que lui-même se trouvait
mal, et le pria de l'emmener. Par cette adresse, il le fit sortir par le
théâtre, mais toujours voulant se ruer sur les ennemis. Cela fit grand
bruit en présence de Monseigneur et de toute la cour.

On en sut après bien d'autres. Un de ses premiers accès lui arriva chez
M. le prince de Conti, qui avait la goutte, à Paris, et qui était auprès
de son feu sur une chaise longue, mais assez reculée de la cheminée, et
sans pouvoir mettre les pieds à terre. Le hasard fit qu'après quelque
temps La Châtre demeura seul avec M. le prince de Conti. L'accès lui
prit, et c'était toujours les ennemis qu'il voyait et qu'il voulait
charger. Le voilà tout à coup qui s'écrie, qui met l'épée à la main et
qui attaque les chaises et le paravent. M. le prince de Conti, qui ne se
doutait de rien moins, surpris à l'excès, voulut lui parler. Lui
toujours à crier\,: «\,Les voilà\,! à moi\,! marche ici\,!» et choses
pareilles, et toujours à estocades et à ferrailler. M. le prince de
Conti à mourir de peur, qui était trop loin pour pouvoir ni sonner ni
pouvoir s'armer de pelle ou de pincettes, et qui s'attendait à tout
instant à être pris pour un ennemi et à le voir fondre sur lui. De son
aveu jamais homme ne passa un si mauvais quart d'heure\,; enfin
quelqu'un entra qui surprit La Châtre et le fit revenir. Il rengaina et
gagna la porte. M. le prince de Conti exigea le secret et le garda
fidèlement\,; mais il chargea le domestique qui était entré de ne le
laisser jamais seul avec La Châtre. Il envoya prier le lendemain le duc
d'Humières qu'il lui pût dire un mot de pressé, et qu'il savait bien
qu'il avait la goutte, et ne pouvait sortir. Il lui confia son aventure,
comme au plus proche parent, pour en avertir M\textsuperscript{me} de La
Châtre, l'assurer qu'elle demeurerait secrète et voir entre eux ce qu'il
y avait à faire. Il en eut depuis quantité d'autres avec un air toujours
égaré, empressé, turbulent, qui le faisait éviter, mais qu'il soutint,
et qui ne le séquestra point du monde ni même de la cour. On verra en
son temps ce qu'il devint.

Nous avons laissé Rome dans un cruel embarras. La ligue d'Italie n'avait
aucune exécution 5 et sa conclusion et sa publicité précoce ne fit
qu'ouvrir les yeux à la grande alliance sur le danger qu'elle courait de
perdre l'Italie, et irriter extrêmement l'empereur contre le pape, qui,
dans l'espérance d'entraîner par son exemple, avait, pris le premier les
armes contre ses troupes, comme je l'ai raconté et avec succès tant
qu'elles n'eurent affaire qu'à ce peu qui étaient demeurées éparses en
Italie et dont le gros formait toute la force de l'armée du duc de
Savoie. Mais sitôt que ce gros eut quitté cette armée, qui fit finir la
campagne de ce côté-là de meilleure heure, et qu'il eut paru en Italie,
les troupes du pape n'osèrent plus tenir la campagne, ni tenir nulle
part contre elles. Les Impériaux se mirent à ravager I'État
ecclésiastique et à y vivre à la tartare. Ils tirèrent des contributions
immenses et chassèrent de partout les troupes du pape. L'empereur,
content de sa vengeance et des insultes qu'il faisait faire au pape par
le cardinal Grimani, de Naples, où il était vice-roi par intérim, ne
voulait que le forcer à reconnaître l'archiduc comme roi d'Espagne. Le
pape était aux hauts cris, alléguait le respect dû à sa dignité, sentait
où on voulait l'amener, et ne savait que devenir. On n'était plus au
temps des excommunications, et l'empereur savait très bien séparer le
spirituel du temporel du pape.

Il avait envoyé le marquis de Prié en Italie avec le caractère de son
plénipotentiaire à Rome, où on ne voulait point le recevoir. Tessé, qui
prévit aisément quel serait le succès de ce ministre impérial s'il était
une fois admis, fit tout ce qu'il put pour l'empêcher\,; mais il n'avait
que des paroles, et point de secours à prêter d'aucune espèce. Les cris
de tout l'État du pape, et de. Rome même qui se sentait cruellement de
la ruine des campagnes, devinrent si grands, que le pape commença à en
craindre presque autant que des Impériaux, et consentit enfin à recevoir
le plénipotentiaire impérial dans Rome et à entrer en affaires avec lui.

Prié était peut-être l'homme de l'Europe le plus propre à cette
commission\,: c'était un Piémontais de fort peu de naissance, de
beaucoup d'esprit et fort orné, de beaucoup d'ambition et de talents qui
l'avaient assez rapidement élevé dans les armées et dans la cour de
Savoie, où pour la première fois l'ordre de l'Annonciade, qui constitue
seul les grands de cette cour, fut avili pour lui. Parvenu dans son pays
à tous les honneurs où il n'aurait osé prétendre, il le trouva désormais
trop étroit pour la fortune qu'il se proposait, et se servit de ce qu'il
y avait acquis pour passer au service de l'empereur avec plus de
considération. Il y parvint aux premiers grades. Son génie avantageux,
audacieux, plut à une cour aussi superbe et aussi entreprenante que fut
toujours celle de Vienne, et lui parut propre à la bien servir. Il en
obtint cet emploi de plénipotentiaire, et ne trompa point les espérances
qu'elle en avait conçues.

Arrivé à Rome, il demeura froid et tranquille en attendant qu'on vint à
lui. Le pape attendait de son côté quelles propositions il voudrait
faire puisqu'il n'était venu que pour négocier\,; mais à la fin, lassé
d'une présence muette, qui n'apportait aucun soulagement au pillage qui
l'avait fait recevoir, {[}il{]} envoya savoir de lui ce qu'il était
chargé de faire. Sa réponse fut désolante. Il répondit qu'il n'était
point venu pour parler, mais seulement pour écouter ce qu'on lui
voudrait dire\,; et sur les représentations de la nécessité urgente
d'arrêter les excès des Impériaux qui continuaient toujours, il s'en
défendit modestement sur ce qu'il n'avait aucun pouvoir de leur imposer.
On entendit de reste une réponse si dure et en même temps si méprisante.
Le pape sentit qu'il n'y avait point de paix ni de trêve à espérer de
ces cruels saccagements que par terminer tous différends avec
l'empereur. L'humiliation était extrême, mais le couteau était dans la
gorge\,; il fallut ployer.

Dans ces circonstances, Tessé se trouva dans une situation violente. Il
n'avait pu parer l'admission de Prié\,; il avait senti combien sa
présence lui serait pesante et même personnellement embarrassante, du
génie hardi dont il était, poussé par Grimani, et soutenu de l'armée
impériale qui ravageait l'État ecclésiastique. Il prit donc le parti
d'éviter au moins les inconvénients personnels, et d'être malade avant
l'arrivée de Prié à Rome. Il se plaignit d'une fistule et s'enferma chez
lui. De son cabinet, il se débattit comme il put\,; et j'ajouterai, pour
n'avoir pas à revenir à une affaire dont la suite fut longue, qu'il
écrivit trois lettres au pape. Elles sont si propres à caractériser ce
maréchal, qu'on a vu depuis 1696 surtout, dans les principaux emplois de
guerre et de paix et qu'on venait de choisir pour la plus importante de
ce règne, que j'ai cru les devoir mettre parmi les Pièces avec les
réflexions qu'elles m'ont paru mériter. Ces trois pièces serviront à
faire juger de ce qui a réussi avec tant d'avantage et de continuité à
la cour de Louis XIV, et de ce qui aussi a été si utilement employé en
ses affaires, surtout depuis la révolution d'Espagne. Tessé se complut
tellement en ces trois productions de son esprit qu'il les envoya à là
cour et à Paris, où il les fit répandre.

Don François-Bernard de Quiros mourut vieux aux {[}eaux{]}
d'Aix-la-Chapelle qu'il était allé prendre dans la rigueur du mois de
janvier. Il avait été toute sa vie dans les négociations, et il s'y
était rendu habile, toujours dans les cours étrangères ou dans lès
assemblées pour la paix. À la révolution d'Espagne, il se donna à
Philippe V qui l'employa de même\,; la bataille de Ramillies et ses
rapides suites le retournèrent vers la maison d'Autriche. Il fut
ambassadeur de l'archiduc comme roi d'Espagne, à la Haye où il avait
passé beaucoup d'années avec le même caractère que lui avait donné
Charles II. Cette défection ne lui fit pas honneur, et les intérêts de
Philippe V ne laissèrent pas d'en souffrir. Mais la passion des alliés
était telle contre les deux couronnes, et surtout en Hollande où le
pensionnaire\footnote{On appelait \emph{pensionnaire} ou \emph{grand
  pensionnaire de Hollande} le député de cette province aux États
  généraux des Provinces-Unies\,; il avait la présidence de l'assemblée.}
Heinsius gouvernait tout, que la considération de Quiros n'en fut point
altérée. Pour la naissance, elle était fort commune et bien au-dessous
des emplois et de la capacité.

\hypertarget{chapitre-iv.}{%
\chapter{CHAPITRE IV.}\label{chapitre-iv.}}

1709

~

{\textsc{Mort et caractère du P. de La Chaise.}} {\textsc{- Surprenant
aveu du roi.}} {\textsc{- Énorme avis donné au roi par le P. de La
Chaise.}} {\textsc{- P. Tellier confesseur\,; manière dont ce choix fut
fait.}} {\textsc{- Caractère du P. Tellier.}} {\textsc{- Pronostic de
Fagon sur le P. Tellier.}} {\textsc{- Avances du P. Tellier vers moi.}}
{\textsc{- Mort de M\textsuperscript{me} d'Heudicourt\,; son caractère,
et de son mari, et de son fils.}} {\textsc{- Mort du chevalier
d'Elboeuf\,; d'où dit le Trembleur.}} {\textsc{- M. d'Elboeuf ne passe
point la qualité de prince aux Bouillon, en son contrat de mariage avec
M\textsuperscript{lle} de Bouillon, en 1656.}} {\textsc{- Mort du comte
de Benavente.}} {\textsc{- Sa charge de sommelier du corps donnée au duc
d'Albe.}} {\textsc{- Fin et mort de M\textsuperscript{me} de Soubise.}}
{\textsc{- Entreprise de M. de Soubise rendue vaine.}}

~

La cour vit en ce temps-ci renouveler un ministère qui par sa longue
durée s'était usé jusque dans sa racine, et n'en était par là que plus
agréable au roi. Le P. de La Chaise mourut-le 20 janvier, aux
Grands-Jésuites de la rue Saint-Antoine. Il était petit-neveu du fameux
P. Cotton, et neveu paternel du P. d'Aix qui le fit jésuite où il se
distingua dans les emplois de professeur, et après dans ceux de recteur
de Grenoble et de Lyon, puis de provincial de cette province\,; il était
gentilhomme, et son père, qui s'était bien allié et avait bien servi,
aurait été riche pour son pays de Forez s'il n'avait pas eu une douzaine
d'enfants. Un de ceux-là, qui se connaissait parfaitement en chiens, en
chasses, et en chevaux qu'il montait très bien, fut longtemps écuyer de
l'archevêque de Lyon, frère et oncle des maréchaux de Villeroy, et
commanda son équipage de chasse pour laquelle ce prélat était passionné.
C'est le même que nous avons vu capitaine de la porte, et son fils après
lui.

Les deux frères étaient à Lyon dans les emplois que je viens de dire,
lorsque le P. de La Chaise succéda en 1675 au P. Ferrier, confesseur du
roi\,; ainsi le P. de La Chaise le fut plus de trente-deux ans. La fête
de Pâques lui causa plus d'une fois des maladies de politique pendant
l'attachement du roi pour M\textsuperscript{me} de Montespan. Une entre
autres, il lui envoya le P. Dechamps en sa place, qui bravement refusa
l'absolution. Ce jésuite a été fort connu provincial de Paris, et par la
confiance de M. le Prince le héros, dans les dernières années de sa vie.

Le P. de La Chaise était d'un esprit médiocre, mais d'un bon caractère,
juste, droit, sensé, sage, doux et modéré, fort ennemi de la délation,
de la violence et des éclats. Il avait de l'honneur, de la probité, de
l'humanité, de la bonté\,; affable, poli, modeste, même respectueux. Lui
et son frère ont toujours publiquement conservé une reconnaissance
marquée jusqu'à une sorte de dépendance pour les Villeroy\,; il était
désintéressé en tout genre quoique fort attaché à sa famille\,; il se
piquait de noblesse, et il la favorisa en tout ce qu'il put. Il était
soigneux de bons choix pour l'épiscopat, surtout pour les grandes
places, et il y fut heureux tant qu'il y eut l'entier crédit. Facile à
revenir quand il avait été trompé, et ardent à réparer le mal que la
tromperie lui avait fait faire. On en a vu en son lieu un exemple sur
l'abbé de Caudelet\,; d'ailleurs judicieux et précautionné, bon homme et
bon religieux, fort jésuite, mais sans rage et sans servitude, et les
connaissant mieux qu'il ne le montrait, mais parmi eux comme l'un
d'entre eux. Il ne voulut jamais pousser le Port-Royal des Champs
jusqu'à la destruction, ni entrer en rien contre le cardinal de
Noailles, quoique parvenu à tout sans sa participation. Le cas de
conscience et tout ce qui se fit contre lui de son temps, se fit sans la
sienne. Il ne voulut point non plus entrer trop avant dans les affaires
de la Chine, mais il favorisa toujours tant qu'il put l'archevêque de
Cambrai, et fut toujours fidèlement ami du cardinal de Bouillon, pour
lequel, en toutes sortes de temps, il rompit bien des glaces.

Il eut toujours sur sa table le \emph{Nouveau Testament} du P. Quesnel
qui a fait tant de bruit depuis, et de si terribles fracas\,; et quand
on s'étonnait de lui voir ce livre si familier à cause de l'auteur, il
répondait qu'il aimait le bon et le bien partout où il le rencontrait\,;
qu'il ne connaissait point de plus excellent livre, ni d'une instruction
plus abondante\,; qu'il y trouvait tout\,; et que, comme il avait peu de
temps à donner par jour à des lectures de piété, il préférait celle-là à
toute autre.

Il eut tout le crédit de la distribution des bénéfices pendant les
quinze ou vingt dernières années de l'archevêque de Paris, Harlay. Son
indépendance de M\textsuperscript{me} de Maintenon fut toujours entière
et sans commerce avec elle\,; aussi le haïssait-elle, tant pour cette
raison, que pour son opposition à la déclaration de son mariage, mais
sans oser jamais lui montrer les dents, parce qu'elle connaissait de la
disposition du roi à son égard. Elle se servit de Godet, évêque de
Chartres, qu'elle introduisit peu à peu dans la confiance du roi, puis
du cardinal de Noailles, après le mariage de sa nièce et à l'occasion de
l'affaire de M. de Cambrai, pour balancer la distribution des bénéfices,
et y entrer elle-même de derrière ses deux rideaux, ce qui commença à
déshonorer le clergé de France, par les ignorants et les gens de néant
que M. de Chartres et Saint-Sulpice introduisirent dans l'épiscopat, à
l'exclusion tant qu'ils purent de tous autres.

Vers quatre-vingts ans, le P. de La Chaise, dont la tête et la santé
étaient encore fermes, voulut se retirer\,: il en fit plusieurs
tentatives inutiles. La décadence de son corps et de son esprit, qu'il
sentit bientôt après, l'engagea à redoubler ses instances. Les jésuites,
qui s'en apercevaient plus que lui, et qui sentaient la diminution de
son crédit, l'exhortèrent à faire place à un autre qui eût la grâce et
le zèle de la nouveauté. Il désirait sincèrement le repos, et il pressa
le roi de le lui accorder tout aussi inutilement. Il fallut continuer à
porter le faix jusqu'au bout. Les infirmités et la décrépitude qui
l'accueillirent\footnote{Le manuscrit de Saint-Simon porte le mot
  \emph{accueillirent} qui ne paraît pas très exact et que lès
  précédents éditeurs ont remplacé par le mot \emph{assaillirent}.}
bientôt après ne purent le délivrer. Les jambes ouvertes, la mémoire
éteinte, le jugement affaissé, les connaissances brouillées,
inconvénients étranges pour un confesseur, rien ne rebuta le roi, et
jusqu'à la fin il se fit apporter le cadavre et dépêcha avec lui les
affaires accoutumées. Enfin, deux jours après un retour de Versailles,
il s'affaiblit considérablement, reçut les sacrements, et eut pourtant
le courage, plus encore que la force, d'écrire au roi une longue lettre
de sa main, à laquelle il reçut réponse du roi de la sienne tendre et
prompte\,; après quoi il ne s'appliqua plus qu'à Dieu.

Le P. Tellier, provincial, et le P. Daniel, supérieur de la maison
professe, lui demandèrent s'il avait accompli ce que sa conscience
pouvait lui demander et s'il avait pensé au bien et à l'honneur de la
compagnie. Sur le premier point, il répondit qu'il était en repos\,; sur
le second, qu'ils s'apercevraient bientôt par les effets qu'il n'avait
rien à se reprocher. Fort peu après, il mourut fort paisiblement à cinq
heures du matin.

Les deux supérieurs vinrent apporter au roi, à l'issue de son lever, les
clefs du cabinet du P. de La Chaise, qui y avait beaucoup de mémoires et
de papiers. Le roi les reçut devant tout le monde, en prince accoutumé
aux pertes, loua le P. de La Chaise surtout de sa bonté, puis souriant
aux pères\,: «\,Il était si bon, ajouta-t-il tout haut devant tous les
courtisans, que je le lui reprochais quelquefois, et il me répondait\,:
«\,Ce n'est pas moi qui suis bon, mais vous qui êtes dur.\,»
Véritablement les pères et tous les auditeurs furent surpris du récit
jusqu'à baisser la vue. Ce propos se répandit promptement, et personne
n'en put blâmer le P. de La Chaise.

Il para bien des coups en sa vie, supprima bien des friponneries et des
avis anonymes contre beaucoup de gens, en servit quantité, et ne fit
jamais de mal qu'à son corps défendant. Aussi fut-il généralement
regretté. On avait toujours compris que ce serait une perte\,; mais on
n'imagina jamais que sa mort serait une plaie universelle et profonde
comme elle la devint, et comme elle ne tarda pas à se faire sentir par
le terrible successeur du P. de La Chaise, à qui les ennemis mêmes des
jésuites furent forcés de rendre justice après, et d'avouer que c'était
un homme bien et honnêtement né, et tout fait pour remplir une telle
place.

Maréchal, premier chirurgien du roi, qui avait sa confiance, homme droit
et parfaitement vrai, que j'ai cité plus d'une fois, nous a conté, à
M\textsuperscript{me} de Saint-Simon et à moi, une anecdote bien
considérable et qui mérite de n'être pas oubliée. Il nous dit que le roi
dans l'intérieur de ses cabinets, regrettant le P. de La Chaise et le
louant de son attachement à sa personne, lui avait raconté une grande
marque qu'il lui en avait donnée\,: que peu d'années avant sa mort, il
lui avait dit qu'il se sentait vieillir, qu'il arriverait peut-être plus
tôt qu'il ne pensait, qu'il faudrait choisir un autre confesseur, que
l'attachement qu'il avait pour sa personne le déterminait uniquement à
lui demander en grâce de le prendre dans sa compagnie, qu'il la
connaissait, qu'elle était bien éloignée de mériter tout ce qui s'est
dit et écrit contre elle, mais qu'enfin il lui répétait qu'il la
connaissait, que son attachement à sa personne et à sa conservation
l'engageait à le conjurer de lui accorder ce qu'il lui demandait, que
c'était une compagnie très étendue composée de bien des sortes de gens
et d'esprit dont on ne pouvait répondre, qu'il ne fallait point mettre
au désespoir, et se mettre ainsi dans un hasard dont lui-même ne lui
pouvait répondre, et qu'un mauvais coup était bientôt fait et n'était
pas sans exemple. Maréchal pâlit à ce récit que lui fit le roi, et cacha
le mieux qu'il put le désordre où il en tomba.

Cette considération unique fit rappeler les jésuites par Henri IV, et
les fit combler de biens. La pyramide de Jean Châtel\footnote{Cette
  pyramide avait été élevée sur l'emplacement de la maison du père de
  Jean Châtel, qu'on avait rasée après l'attentat commis par son fils
  contre Henri IV le 27 décembre 1594. Voy. de Thou, \emph{Hist.
  universelle}, liv. CXI, chap.~xviii, et \emph{Mémoires de Condé}, t.
  VI, supplément, part. III, p.~132 et suiv.} les mettait au
désespoir\,; ils trouvèrent, sous Louis XIV, Fourcy, prévôt des
marchands, capable de les écouter, et en état de l'oser par le crédit de
Boucherat, chancelier de France, son beau-père, qui, appuyé du roi,
contint le parlement. Fourcy fit abattre la pyramide sans en laisser la
moindre trace\,; son fils, sortant du collège, en eut l'abbaye de
Saint-Vandrille de plus de trente-six mille livres à l'étonnement
publie, et en jouit encore. C'est même un fort honnête homme et
considéré, qui ne s'est pas soucié d'être évêque.

Le roi n'était pas supérieur à Henri IV\,; il n'eut garde d'oublier le
document du P. de La Chaise, et de se hasarder à la vengeance de sa
compagnie en choisissant hors d'elle un confesseur. Il voulait vivre et
vivre en sûreté. Il chargea les ducs de Chevreuse et de Beauvilliers
d'aller à Paris, de s'informer, avec toutes précautions qu'ils
pourraient y apporter, de qui d'entre les jésuites il pourrait prendre
pour confesseur.

M. de Chartres et le curé de Saint-Sulpice ne regardaient pas ce choix
avec indifférence\,; ils voulurent y influer. Toutefois ils n'en avaient
nulle commission, elle n'était donnée qu'aux deux ducs dont ils
n'étaient pas à portée. L'affaire de M. de Cambrai avait élevé un
puissant mur de séparation entre eux. Le malheur voulut que la mort du
P. de La Chaise arrivât dans la conjoncture où les affaires de Flandre
entre Mgr le duc de Bourgogne et M. de Vendôme avaient rapproché
M\textsuperscript{me} de Maintenon et M. de Beauvilliers jusqu'à
l'entière confidence là-dessus, et aux mesures communes, comme je l'ai
raconté. Ces affaires prenaient un cours qui répondait à leurs soins\,;
mais elles n'étaient pas finies. Le commerce, la confiance, les mesures
continuaient encore là-dessus. M\textsuperscript{me} de Maintenon
profita de la conjoncture, et, malgré tout ce qui s'était passé, elle
obtint que l'évêque de Chartres et le curé de Saint-Sulpice, qui n'était
qu'un, seraient admis par les deux ducs à conférer sur le choix. L'un et
l'autre étaient prévenus d'estime et d'affection pour Saint-Sulpice,
comme l'était M. de Cambrai. La Chétardie en était curé, il n'existait
pas lors de l'affaire de M. de Cambrai, et dans la vérité c'était un
homme de bien, mais une espèce d'imbécile. J'aurai lieu d'en parler
ailleurs. Mené par M. de Chartres\,; il appuya sur le P. Tellier. Les
jésuites avaient dressé pour lui toutes leurs batteries, les deux ducs
en furent les dupes, et bientôt après l'Église et l'État les victimes.

Le P. Tellier, lors provincial de Paris, eut l'approbation décisive des
deux ducs\,; sur leur rapport le roi le choisit, et ce choix fut
incompréhensible de ce même prince qui, pour beaucoup moins en même
genre, avait ôté le P. Le Comte à M\textsuperscript{me} la duchesse de
Bourgogne, dont il était confesseur depuis plusieurs années, et fort
goûté d'elle et de toute la cour, et le fit aller à Rome sans que les
jésuites avec tout leur art et leur crédit pussent parer le coup. La
délibération du choix d'un confesseur dura un mois, depuis le 20 janvier
que mourut le P. de La Chaise, jusqu'au 21 février que le P. Tellier fut
nommé. Il fut comme son prédécesseur confesseur aussi de Monseigneur,
contrainte bien dure à l'âge de ce prince. J'anticipe ici ce mois pour
ne pas couper une matière si curieuse.

Le P. Tellier était entièrement inconnu au roi\,; il n'en avait su le
nom que parce qu'il se trouva sur une liste de cinq ou six jésuites que
le P. de La Chaise avait faite de sujets propres à lui succéder. Il
avait passé par tous les degrés de la compagnie, professeur, théologien,
recteur, provincial, écrivain. Il avait été chargé de la défense du
culte de Confucius et des cérémonies chinoises, il en avait épousé la
querelle, il en avait fait un livre qui pensa attirer d'étranges
affaires à lui et aux siens, et qui à force d'intrigues et de crédit à
Rome, ne fut mis qu'à l'index\,; c'est en quoi j'ai dit qu'il avait fait
pire que le P. Le Comte, et qu'il est surprenant que malgré cette tare
il ait été confesseur du roi.

Il n'était pas moins ardent sur le molinisme, sur le renversement de
toute autre école, sur l'établissement en dogmes nouveaux de tous ceux
de sa compagnie sur les ruines de tous ceux qui y étaient contraires et
qui étaient reçus et enseignés de tout temps dans l'Église. Nourri dans
ces principes, admis dans tous les secrets de sa société par le génie
qu'elle lui avait reconnu, il n'avait vécu depuis qu'il y était entré
que de ces questions et de l'histoire intérieure de leur avancement, que
du désir d'y parvenir, de l'opinion que pour arriver à ce but il n'y
avait rien qui ne fût permis et qui ne se dût entreprendre. Son esprit
dur, entêté, appliqué sans relâche, dépourvu de tout autre goût, ennemi
de toute dissipation, de toute société, de tout amusement, incapable
d'en prendre avec ses propres confrères, et ne faisant cas d'aucun que
suivant la mesure de la conformité de leur passion avec celle qui
l'occupait tout entier. Cette cause dans toutes ces branches lui était
devenue la plus personnelle, et tellement son unique affaire, qu'il
n'avait jamais eu d'application ni travail que par rapport à celle-là,
infatigable dans l'un et dans l'autre. Tout ménagement, tout tempérament
là-dessus lui était odieux, il n'en souffrait que par force ou par des
raisons d'en aller plus sûrement à ses fins. Tout ce qui en ce genre
n'avait pas cet objet était un crime à ses yeux et une faiblesse
indigne.

Sa vie était dure par goût et par habitude, il ne connaissait qu'un
travail assidu et sans interruption\,; il l'exigeait pareil des autres
sans aucun égard, et ne comprenait pas qu'on en dût avoir. Sa tête et sa
santé étaient de fer, sa conduite en était aussi, son naturel cruel et
farouche. Confit dans les maximes et dans la politique de la société,
autant que la dureté de son caractère s'y pouvait ployer, il était
profondément faux, trompeur, caché sous mille plis et replis, et quand
il put se montrer et se faire craindre exigeant tout, ne donnant rien,
se moquant des paroles les plus expressément données lorsqu'il ne lui
importait plus de les tenir, et poursuivant avec fureur ceux qui les
avaient reçues. C'était un homme terrible qui n'allait à rien moins qu'à
destruction, à couvert et à découvert, et qui, parvenu à l'autorité, ne
s'en cacha plus.

Dans cet état, inaccessible même aux jésuites, excepté à quatre ou cinq
de même trempe que lui, il devint la terreur des autres\,; et ces quatre
ou cinq même n'en approchaient qu'en tremblant, et n'osaient le
contredire qu'avec de grandes mesures, et en lui montrant que, par ce
qu'il se proposait, il s'éloignait de son objet, qui était le règne
despotique de sa société, de ses dogmes, de ses maximes, et la
destruction radicale de tout ce qui y était non seulement contraire,
mais de tout ce qui n'y serait pas soumis jusqu'à l'abandon aveugle.

Le prodigieux de cette fureur jamais interrompue d'un seul instant par
rien, c'est qu'il ne se proposa jamais rien pour lui-même, qu'il n'avait
ni parents ni amis, qu'il était né malfaisant, sans être touché d'aucun
plaisir d'obliger, et qu'il était de la lie du peuple et ne s'en cachait
pas\,; violent jusqu'à faire peur aux jésuites les plus sages, et même
les plus nombreux et les plus ardents jésuites, dans la frayeur qu'il ne
les culbutât jusqu'à les faire chasser une autre fois.

Son extérieur ne promettait rien moins, et tint exactement parole\,; il
eût fait peur au coin d'un bois. Sa physionomie était ténébreuse,
fausse, terrible\,; les yeux ardents, méchants, extrêmement de
travers\,: on était frappé en le voyant.

À ce portrait exact et fidèle d'un homme qui avait consacré corps et âme
à sa compagnie, qui n'eut d'autre nourriture que ses plus profonds
mystères, qui ne connut d'autre Dieu qu'elle, et qui avait passé sa vie
enfoncé dans cette étude, du génie et de l'extraction qu'il était, on ne
peut être surpris qu'il fût sur tout le reste grossier et ignorant à
surprendre, insolent, impudent, impétueux, ne connaissant ni monde, ni
mesure, ni degrés, ni ménagements, ni qui que ce fût, et à qui tous
moyens étaient bons pour arriver à ses fins., Il avait achevé de se
perfectionner à Rome dans les maximes et la politique de sa société, qui
pour l'ardeur, de son naturel et son roide avait été obligée de le
renvoyer promptement en France, lors de l'éclat que fit à Rome son livre
mis à l'index.

La première fois qu'il vit le roi dans son cabinet, après lui avoir été
présenté, il n'y avait que Bloin et Fagon dans un coin. Fagon, tout
voûté et appuyé sur son bâton, examinait l'entrevue et la physionomie du
personnage, ses courbettes et ses propos. Le roi lui demanda s'il était
parent de MM. Le Tellier. Le père s'anéantit\,: «\,Moi, sire,
répondit-il, parent de MM. Le Tellier\,! je suis bien loin de cela\,; je
suis un pauvre paysan de basse Normandie, où mon père était un
fermier.\,» Fagon qui l'observait jusqu'à n'en rien perdre, se tourna en
dessous à Bloin, et faisant effort pour le regarder\,: «\,Monsieur, lui
dit-il en lui montrant le jésuite, quel sacré..\,!» et haussant les
épaules se remit sur son bâton. Il se trouva qu'il ne s'était pas trompé
dans un jugement si étrange d'un confesseur. Celui-ci avait fait toutes
les mines, pour ne pas dire les singeries hypocrites d'un homme qui
redoutait cette place, et qui ne s'y laissa forcer que par obéissance à
sa compagnie.

Je me suis étendu sur ce nouveau confesseur parce que de lui sont
sorties les incroyables tempêtes sous lesquelles l'Église, l'État, le
savoir, la doctrine et tant de gens de bien de toutes les sortes,
gémissent encore aujourd'hui, et parce que j'ai eu une connaissance plus
immédiate et plus particulière de ce terrible personnage qu'aucun homme
de la cour.

Mon père et ma mère me mirent entre les mains des jésuites pour me
former à la religion, et y choisirent fort heureusement\,; car, quelque
chose qu'il se publie d'eux, il ne faut pas croire qu'il ne s'y trouve
par-ci par-là des gens fort saints et fort éclairés. Je demeurai donc où
on m'avait mis, mais sans commerce avec d'autres qu'avec celui à qui je
m'adressais\,; celui-là avait le soin en premier des retraites qu'ils
donnaient à leur noviciat à des séculiers plusieurs fois l'année. Il
s'appelait le P. Sanadon, et son emploi le mettait en relations
nécessaires avec les supérieurs, par conséquent avec le P. Tellier,
provincial, lorsqu'il fut choisi pour être confesseur. Ce P. Tellier, de
son goût et de son habitude farouche, ne voulut voir que ce qui lui fut
impossible d'éviter. À son goût se joignit aussi la politique, pour se
montrer au roi plus isolé, en effet pour être plus indépendant et se
dérober mieux aux égards et aux sollicitations.

Je fus fort surpris que quinze jours ou trois semaines après qu'il fut
dans ce ministère, car c'en était un très réel, fort séparé des autres,
le P. Sanadon me vint dire qu'il voulait m'être présenté, ce furent ses
termes et ceux du P. Tellier lorsqu'il me l'amena le lendemain. Je ne
l'avais jamais vu, et je n'avais été, ni {[}n'avais{]} envoyé lui faire
compliment\,; il m'en accabla, et conclut par me demander la permission
de me venir voir quelquefois, et la grâce de vouloir bien le recevoir
avec bonté. En deux mots, c'était qu'il voulait lier avec moi\,; et moi
qui m'en défiais, et qui n'en avais que faire par la situation de ma
famille où personne n'était dans l'Église, j'eus beau m'écarter
poliment, je fus violé. Il redoubla ses visites, me parla d'affaires, me
consulta, et pour le dire, me désola par le danger de le rebuter d'une
manière grossière, et celui d'entrer en affaires avec lui. Cette liaison
forcée, à laquelle je ne répondis que passivement, dura jusqu'à la mort
du roi\,; elle m'apprit bien des choses qui se trouveront chacune en
leur temps.

Il fallait qu'il se fût informé de moi au P. Sanadon qui apparemment lui
apprit mes intimes liaisons avec les ducs de Chevreuse et de
Beauvilliers, peut-être celle que j'avais avec Mgr le duc de Bourgogne
qui était alors profondément cachée, et avec M. le duc d'Orléans. Il
était vrai que dès lors je pointais fort, mais c'était sous cloche, et
quoique j'entrasse depuis longtemps en beaucoup de choses importantes,
le gros du monde ne s'en apercevait pas encore parfaitement.

La cour fut délivrée d'une manière de démon domestique en la personne de
M\textsuperscript{me} d'Heudicourt, qui mourut sur les huit heures du
matin, à Versailles, le jeudi 24 janvier. J'ai parlé suffisamment d'elle
(t. Ier, p.~367), de sa fortune, de son mariage par l'hôtel d'Albret, et
de l'intime liaison qu'elle y fit avec M\textsuperscript{me} de
Maintenon qui dura toute leur vie, et de tout ce qui s'en est suivi.
Elle était devenue vieille et hideuse\,; on ne pouvait avoir plus
d'esprit ni plus agréable, ni savoir plus de choses, ni être plus
plaisante, plus amusante, plus divertissante, sans vouloir l'être. On ne
pouvait aussi être plus gratuitement, plus continuellement, plus
désespérément méchante, par conséquent, plus dangereuse, dans la
privance la plus familière dans laquelle elle passait sa vie avec
M\textsuperscript{me} de Maintenon, avec le roi\,; tout aussi, faveur,
grandeur, places, ministres, enfants du roi, même bâtards, tout
fléchissait le genou devant cette mauvaise fée, qui ne savait que nuire
et jamais servir. M\textsuperscript{me} la Duchesse était fort bien avec
elle et sut toujours s'en servir. Son appartement était un sanctuaire où
n'était pas admis qui voulait. M\textsuperscript{me} de Maintenon, qui
ne la quitta point durant sa maladie, et qui la vit mourir, en fut
extrêmement affligée\,; elle et le roi y perdirent beaucoup de plaisir,
et le monde, aux dépens de qui elle le donnait, y gagna beaucoup, car
c'était une créature sans âme.

Son mari en tirait parti le bâton haut, sans presque vivre avec elle,
mais il s'en était fait craindre. C'était un vieux vilain, fort débauché
et horrible, qui était souffert à cause d'elle, et {[}ils{]} ne
laissaient pas de se tourmenter l'un l'autre. Il était gros joueur, le
plus fâcheux et le plus emporté, et toujours piqué et furieux. C'était
un plaisir de le voir couper à Marly, au lansquenet, et faire de
brusques reculades de son tabouret à renverser ce qui l'importunait
derrière, et leur casser les jambes\,; d'autres fois cracher derrière
lui au nez de qui l'attrapait.

Sa femme, avec tout son esprit, craignait les esprits jusqu'à avoir des
femmes à gages pour la veiller toutes les nuits. Cette folie alla au
point de mourir de peur d'un vieux perroquet qu'elle perdit après
l'avoir gardé vingt ans. Elle en redoubla d'\emph{occupées}, c'était le
nom qu'elle donnait à ses veilleuses. Son fils, qui n'était point
poltron, avait la même manie, jusqu'à ne pouvoir être jamais seul le
soir ni la nuit dans sa chambre.

C'était une manière de chèvre-pied\footnote{Espèce de satyre que l'on
  représente avec des pieds de chèvre.} aussi méchant et plus laid
encore que {[}son{]} père\,; très commode aux dames, et par là dans
toutes les histoires de la cour, ivrogne à l'excès, il y a de lui mille
contes plaisants de ses frayeurs des esprits et de ses ivrogneries. Il
faisait les plus jolies chansons du monde, où il excellait à peindre les
gens avec naïveté, et leurs ridicules avec le sel le plus fin. Le grand
prévôt et sa famille, honnêtes gens d'ailleurs, en étaient farcis et
n'étaient mêlés à la cour avec personne. Heudicourt s'avisa de faire une
chanson sur eux, si naturelle et si ridiculement plaisante, qu'on en
riait aux larmes. Le maréchal de Boufflers, en quartier de capitaine des
gardes, étant derrière le roi à la messe, où le silence et la décence
étaient extrêmes, vit parler et rire autour de lui. Il voulut imposer.
Quelqu'un lui dit la chanson à l'oreille. À l'instant voilà cet homme si
sage, si grave, si sérieux, si courtisan, qui s'épouffe de rire, et qui,
à force de vouloir se retenir, éclate. Le roi se tourne une fois, puis
une seconde, le tout pour néant. Les rires continuèrent aux larmes. Le
roi, dans la plus grande surprise de voir le maréchal de Boufflers en
cet état, et derrière lui, et à la messe, lui demanda, en sortant de la
chapelle, et assez sévèrement à qui il en avait eu. Le maréchal à rire
de nouveau qui lui répondit comme il put, que cela ne pouvait lui être
conté que dans son cabinet. Dès qu'il y fut entré, le roi reprit la
question. Le maréchal la satisfit par la chanson, et voilà le roi aux
éclats à l'entendre de sa chambre. Il fut plusieurs jours sans pouvoir
regarder aucun de ces Montsoreau sans éclater, toute la cour de même.
Ils furent réduits à disparaître pour quelque temps\footnote{Cette
  anecdote se trouve déjà dans le t. V, p.~120\,; mais le récit présente
  beaucoup de variantes, qui ont déterminé à le conserver.}.

À force de boire, Heudicourt s'abrutit tout à fait, mais fort longtemps
depuis la mort du roi, et s'est enfin cassé la tête sur un escalier de
Versailles, dont il mourut le lendemain. Sa mère, qui mettait les gens
en, pièces, en sérieux ou en ridicule, et qui avait toujours quelques
\emph{mais} accablants quand elle entendait dire du bien de quelqu'un
devant le roi ou M\textsuperscript{me} de Maintenon, ne fut regrettée
que d'elle. Je disais d'elle et de M\textsuperscript{me} de Dangeau qui,
dans les mêmes privantes, en était la contrepartie parfaite qu'elles
étaient le mauvais, ange et le bon ange de M\textsuperscript{me} de
Maintenon.

La mort du chevalier d'Elboeuf, arrivée sept ou huit jours après, fit
moins de bruit dans le monde. Il était fils aîné du duc d'Elboeuf et de
sa première femme, qui n'eut que lui et M\textsuperscript{me} de
Vaudemont. Elle était fille unique du comte de Lannoy, chevalier de
l'ordre en 1633, premier maître d'hôtel du roi, et gouverneur de
Montreuil, mort en 1649. Elle épousa en 1643 le comte de La Rocheguyon,
premier gentilhomme de la chambre du roi en survivance de son père. Il
était fils unique des célèbres M. et M\textsuperscript{me} de Liancourt,
et fut tué au siège de Mardick en 1646, ne laissant qu'une fille unique,
qui épousa M. de La Rochefoucauld, le grand maître de la garde-robe, le
grand veneur, et si bien toute sa vie avec le roi. Sa veuve, épousa M.
d'Elboeuf, avec qui elle ne fut pas heureuse. Ce fut en 1648, il en eut
le gouvernement de Montreuil, qu'il joignit à celui de Picardie qu'il
avait eu de son père. Il s'emporta si étrangement contre sa femme qui
était grosse, qu'il la prit entre ses bras pour la jeter par la fenêtre.
La frayeur qu'elle en eut la saisit à tel point, que le fils dont elle
accoucha naquit tremblant de tout son corps, et ne cessa de trembler
toute sa vie. Elle mourut à Amiens en 1654, à 28 ans.

Deux ans après, M. d'Elboeuf se remaria à M\textsuperscript{lle} de
Bouillon, à qui non plus qu'à ses parents, il ne voulut jamais passer la
qualité de prince dans le contrat de mariage, parmi tout le lustre dont
brillait alors M. de Turenne. Il en eut le duc d'Elboeuf d'aujourd'hui
et le prince Emmanuel son frère. L'état de l'aîné leur fit prendre le
parti de l'engager aux voeux de Malte, à se contenter de ce qu'il en put
tirer, et à lui faire tout céder à son cadet du second lit. Il choisit
on ne sait pourquoi le Mans pour sa demeure, où il vit toujours la
meilleure compagnie du pays. Il n'était pas ignorant, avait de l'esprit
et de la politesse, même de la dignité, et ne laissait pas d'être
considéré dans sa famille.

Il n'était point mal fait et avait cinquante-neuf ans. Lui et
M\textsuperscript{me} de Vaudemont étaient frère et soeur de mère, de la
mère du duc de La Rocheguyon et de M. de Liancourt qui furent leurs
héritiers. Ils en eurent la terre de Brunoy, et fort peu de choses
d'ailleurs, et je crois rien de M\textsuperscript{me} de Vaudemont
lorsqu'elle mourut.

Le comte de Benavente, de la maison de Pimentel, grand d'Espagne de la
première classe, chevalier du Saint-Esprit, et sommelier du corps,
mourut à Madrid dans une grande considération. Il a été ci-devant assez
parlé de lui, à propos du testament de Charles II et de l'avènement de
Philippe V à la couronne d'Espagne, pour n'avoir rien à y ajouter. Il
laissa un fils, savant, obscur, toujours hors de Madrid et fort des
jésuites. Le roi d'Espagne manda au duc d'Albe, son ambassadeur en
France, par un courrier exprès, qu'il lui donnait la charge de sommelier
du corps, qui est une des trois grandes et de laquelle je parlerai en
son lieu c'est notre grand chambellan, mais tel qu'il était autrefois.

M\textsuperscript{me} de Soubise touchait enfin au bout de sa brillante
et solide carrière. Sa beauté lui coûta la vie. Soutenue de son ambition
et de l'usage qu'elle avait fait de l'une et de l'autre, je ne sais si
elle fut fort occupée d'autres pensées prête à voir des choses bien
différentes. Elle avait passé sa vie dans le régime le plus austère pour
conserver l'éclat et la fraîcheur de son teint. Du veau et des poulets
ou des poulardes rôties ou bouillies, des salades, des fruits, quelque
laitage, furent sa nourriture constante, qu'elle n'abandonna jamais,
sans aucun autre mélange, avec de l'eau quelquefois rougie\,; et jamais
elle ne fut troussée comme les autres femmes, de peur de s'échauffer les
reins et de se rougir le nez. Elle avait eu beaucoup d'enfants dont
quelques-uns étaient morts des écrouelles, malgré le miracle qu'on
prétend attaché à l'attouchement de nos rois. La vérité est que, quand
ils touchent les malades, c'est au sortir de la communion.
M\textsuperscript{me} de Soubise, qui ne demandait pas la même
préparation, s'en trouva enfin attaquée elle-même quand l'âge commença à
ne se plus accommoder d'une nourriture si rafraîchissante. Elle s'en
cacha et alla tant qu'elle put\,; mais il fallut demeurer chez elle les
deux dernières années de sa vie, à pourrir sur les meubles les plus
précieux, au fond de ce vaste et superbe hôtel de Guise qui, d'achat ou
d'embellissements et d'augmentations, leur revient à plusieurs millions.

De là, plus que jamais occupée de faveur, et d'ambition, elle
entretenait son commerce de lettres avec le roi et M\textsuperscript{me}
de Maintenon, et se soutint dans sa même considération à la cour et dans
son même crédit. On a vu avec quelle attention elle suivit la promotion
de son fils, à propos de ce que j'ai raconté du chapeau demandé par
l'empereur pour le prince de Lorraine, évêque d'Olmutz. Elle avait
souvent dit que, quelque rang que les maisons eussent acquis, il n'y
avait de solide que la dignité de duc et pair, et c'était aussi à quoi
elle avait toujours tendu. Je ne sais par quelle fatalité son crédit,
qui emporta tant de choses si étranges, ne put obtenir celle-là. Elle se
trouvait à la portée d'autres gens considérables dont le roi craignit
peut-être les cris et l'entraînement contre son goût, à l'occasion de
cette grâce accordée à M\textsuperscript{me} de Soubise. Quoi qu'il en
soit, elle n'y put parvenir\,; ce devait être un des miracles de la
constitution \emph{Unigenitus}, comme on le verra dans la suite.

Cependant M\textsuperscript{me} de Soubise, hors d'espérance d'y arriver
de plein saut, cherchait à s'y échafauder. La mort de
M\textsuperscript{me} de Nemours lui parut ouvrir une porte, non pas
telle qu'elle la voulait, mais pour bien marier une fille du prince de
Rohan pour rien. Matignon, parvenu par son ami Chamillart au comble des
richesses, cherchait partout un mariage pour son fils qui pût le faire
duc. Il comptait d'avoir le duché d'Estouteville de la succession de
M\textsuperscript{me} de Nemours\,; il espéra du crédit de
M\textsuperscript{me} de Soubise, joint à celui de Chamillart, y
réussir. Il convint de prendre pour rien une fille du prince de Rohan,
et d'en reconnaître trois cent mille livres de dot, moyennant cette
grâce. M\textsuperscript{me} de Soubise y mit les derniers efforts de
son crédit\,; mais elle était mourante, la grâce d'ailleurs impossible
au point qu'il eût été plus aisé, d'obtenir franchement une érection,
comme on le verra parmi-les Pièces, et l'affaire avorta.
M\textsuperscript{me} de Soubise n'eut donc pas le plaisir de voir son
fils duc, ni sa petite-fille en faire un. Elle ne vécut pas assez pour
avoir la joie de voir la calotte rouge sur la tête de son second fils,
par les délais de la promotion des couronnes.

Elle mourut à soixante et un ans, le dimanche matin, 3 février, laissant
la maison de la cour la plus riche et la plus grandement établie,
ouvrage dû tout entier à sa beauté et à l'usage qu'elle en avait su
tirer. Malgré de tels succès, elle fut peu regrettée dans sa famille.
Son mari ne perdit pas le jugement\,; la douleur ne l'empêcha pas de
chercher à tirer parti de la mort de sa femme et du local de sa maison
pour faire un acte de prince, non même étranger, mais du sang.

La Merci est vis-à-vis l'hôtel de Guise, et le portail de l'église
vis-à-vis la porte de cette maison, le travers étroit de la rue
entre-deux. Il s'y était fait accommoder une chapelle. De longue main il
prévoyait la mort de sa femme, et il résolut de l'y faire enterrer. La
fin de ce projet était, sous prétexte d'un si proche voisinage, de l'y
faire porter tout droit sans la faire mener à la paroisse, distinction
qui n'est que pour les princes et les princesses du sang, qu'on ne porte
point aux leurs, mais tout droit au lieu de leur sépulture. Sa femme
morte, il brusqua un superbe enterrement, embabouina le curé, qui ne, se
douta jamais de la cause réelle, et qui se rendit en dupe à la commodité
de la proximité, tellement que M\textsuperscript{me} de Soubise fut
portée droit de chez elle à la Merci, et plus tôt enterrée qu'on ne se
fût aperçu de l'entreprise. La chose faite, le cardinal de Noailles la
trouva mauvaise, gronda le curé, et ce fut tout. Il était des amis de
M\textsuperscript{me} de Soubise. Mais le monde, réveillé par ce peu de
bruit, mit incontinent le doigt sur la lettre. On en parla beaucoup et
tant et si bien que les mesures furent prises contre les récidives. En
effet, M. de Soubise étant mort en 1712, il fut porté à sa paroisse et
de là à la Merci. J'ai voulu ne pas omettre cette bagatelle qui montre
de plus en plus ces entreprises en toutes occasions, et par quels
artifices les rangs et les distinctions de ce qu'on appelle princes
étrangers, de naissance ou de grâce, se sont peu à peu formés.

\hypertarget{chapitre-v.}{%
\chapter{CHAPITRE V.}\label{chapitre-v.}}

1709

~

{\textsc{Étrange histoire du duc de Mortemart avec moi.}} {\textsc{-
Mort, maison, famille et caractère de M\textsuperscript{me} de
Maubuisson.}} {\textsc{- Mort, emplois et caractère de d'Avaux.}}
{\textsc{- Étrange et singulier motif de Louvois, qui causa la guerre de
1688.}} {\textsc{- Mort et caractère de M\textsuperscript{me} de
Vivonne.}} {\textsc{- Mort et caractère de Boisseuil.}} {\textsc{-
Retraite sainte de Janson.}}

~

Peu de jours avant la mort de M\textsuperscript{me} de Soubise, il
m'arriva une de ces aventures auxquelles ma vie a été sujette, qui sont
de ces bombes qui tombent sur la tête sans qu'on puisse les prévoir ni
même les imaginer. Je finissais d'ordinaire mes journées par aller,
entre onze heures et minuit, causer chez les filles de Chamillart, où
j'apprenais souvent quelques choses, et à ces heures-là il n'y avait
plus personne. Causant un soir avec elles trois sur leur mère, les ducs
de Mortemart et de La Feuillade s'y trouvèrent, et M\textsuperscript{me}
de Cani depuis le mariage de laquelle son frère était admis à toutes
heures. C'était une manière de fou sauvage, extrêmement ivrogne, que son
mariage rapprivoisait au monde sans que le monde se rapprivoisât à lui,
et il n'avait ouï parler chez lui que de l'esprit des Mortemart. Voulant
se mettre dans le monde, il crut qu'au nom qu'il portait il en fallait
avoir comme eux. Ne s'en donne pas qui veut, ni tel qu'on le désire. Ses
efforts n'aboutirent qu'à une maussade copie de Roquelaure, assez
mauvais original lui-même. Je ne le connaissais comme point\,; je ne le
rencontrais que chez MM. de Chevreuse et de Beauvilliers, et encore fort
rarement aux heures familières où j'y allais\,; il y était sérieux,
silencieux, emprunté, et y demeurait le moins qu'il lui était possible.
La solitude, la mauvaise compagnie, le vin surnageaient toujours au
reste de sa conduite, et M. et M\textsuperscript{me} de Beauvilliers,
quelquefois aussi M. et M\textsuperscript{me} de Chevreuse, malgré leurs
extrêmes mesures pour tout ce qui regardait leur famille, m'en contaient
leur peine et leur douleur.

Ce soir-là, n'y ayant qui que ce soit que cette compagnie et aucuns
domestiques, la conversation se tourna sur le bruit répandu d'une
promotion de l'ordre à la Chandeleur et qui ne se fit point. Ces
messieurs là-dessus me firent quelques questions sur le rang que les
princes étrangers y ont obtenu aux diverses promotions, excepté à la
première, et sur ce que MM. de Rohan et de Bouillon ne sont point
chevaliers de l'ordre. J'expliquai simplement et froidement les faits
qui m'étaient demandés, sentant bien à qui j'avais affaire\,; et en
effet M. de Mortemart se mit à faire des plaisanteries là-dessus fort
déplacées. Il s'en engoua, croyant dire merveilles\,; elles me jetèrent
dans un silence profond. La Feuillade et les dames, qui voulaient
savoir, tachèrent inutilement de m'en tirer, et M. de Mortemart à
pousser de plus belle. Quoique ses plaisanteries ne me regardassent
point et ne tombassent que sur les rangs, auxquels pourtant il n'avait
pas moins d'intérêt que moi et tous les autres, je sentis assez
d'impatience pour faire une sage retraite. Je voulus m'en aller, on me
retint malgré moi, et je ne voulus pas forcer les barricades de leurs
bras. M. de Mortemart cependant disait toujours et ne tarissait pas. À
la fin je lui dis je ne sais quoi de très mesuré, en deux mots, sur des
plaisanteries si déplacées dans sa bouche, et pour cette fois je m'en
allai. Je fus quelques jours sans y retourner. La famille s'en inquiéta.
Ils craignirent avec amitié que je ne fusse fâché\,; ils en parlèrent à
M\textsuperscript{me} de Saint-Simon. J'y retournai\,; ils m'en
parlèrent aussi. Je glissai là-dessus, mais résolu à laisser désormais
le champ libre au duc de Mortemart quand je l'y trouverais.

Cette année, il n'y eut point de bals à la cour, et de l'hiver il n'y
eut, contre la coutume du roi, qu'un seul voyage de Marly. On y alla
quatre jours après ce que je viens de rapporter. Depuis quatre ans
M\textsuperscript{me} de Saint-Simon et moi n'en manquions aucun voyage.
Nous fûmes éconduits de celui-ci. Le voyage fini et moi encore à Paris,
la comtesse de Roucy, qui en avait été, vint à Paris où elle m'avertit
que M\textsuperscript{me} de Lislebonne et M\textsuperscript{me}
d'Espinoy avaient fait des plaintes amères à M\textsuperscript{me}
d'Urfé et à Pontchartrain, comme à mes amis et pour me le dire, de ce
que j'avais dit que je voudrais qu'elles fussent mortes et toute leur
maison éteinte, bien aise au reste d'être défait de
M\textsuperscript{me} de Soubise qui n'avait que trop vécu.

Si M\textsuperscript{me} de Roucy m'eût appris que j'étais accusé
d'avoir tramé contre l'État, elle ne m'eût pas surpris davantage, ni mis
dans une plus ardente colère. Bien que mon coeur ni mon esprit ne me
reprochassent point des sentiments si misérables, je repassai tout ce
qui pouvait m'être échappé depuis quelque temps, j'eus beau m'y épuiser,
mes réflexions et mes recherches furent inutiles. Je m'en allai à
Versailles débarquer chez Pontchartrain, qui me confirma ce que sa
belle-soeur m'avait appris, et qui ajouta que M\textsuperscript{lle} de
Lislebonne et M\textsuperscript{me} d'Espinoy lui avaient dit qu'elles
le tenaient du duc de Mortemart, qui le leur avait dit à Marly. Alors je
contai à Pontchartrain la soirée dont je viens de parler, à quel point
mon silence et ma retenue avaient été poussés\,; combien de si honteuses
échappées et si éloignées de moi l'avaient été de mes propos tenus, avec
combien de réserve je m'étais borné aux réponses les plus courtes et les
plus simples\,; et je le priai et le chargeai de le dire de ma part aux
deux soeurs. Au partir de là je m'en allai trouver M\textsuperscript{me}
d'Urfé, qui m'ayant confirmé les mêmes choses et sur le duc de
Mortemart, je la priai et chargeai de dire le soir même à ces mêmes cinq
soeurs que je réputerais à injure extrême d'être accusé de penser si
indignement\,; que j'avais cette confiance que personne ne me
reconnaîtrait à de tels sentiments, de la lâcheté desquels j'étais trop
incapable pour croire avoir besoin de m'en justifier\,; que néanmoins,
outre les deux dames et le duc de La Feuillade, témoins uniques de ce
qui s'était passé, qu'elles en pouvaient interroger, je m'offrais de
donner en leur présence, et en celle de quiconque elles voudraient
nommer le démenti au duc de Mortemart en face, et le démenti net et
entier sur elles, sur leur maison\,; sur M\textsuperscript{me} de
Soubise, et sur tout ce qui directement ou indirectement pouvait avoir
trait, ou faire entendre rien de semblable. J'ajoutai, et toujours avec
charge de le leur dire, que je ne désavouais pas l'impatience avec
laquelle je supportais beaucoup de choses sur leur rang contre le nôtre,
mais que dans mes désirs, ni si j'étais homme à faire des châteaux en
Espagne, je ne serais pas content de revoir l'ordre et la règle rétablis
sur les rangs, tels qu'ils le devaient être dans un royaume conduit par
les lois de la sagesse et de la justice si elles et leur maison
n'existaient plus.

Ma commission, et tout entière, fut faite le soir même.
M\textsuperscript{lle} de Lislebonne y répondit à merveille et avec cet
air de franchise qu'elle avait assez souvent\,; sa soeur aussi, mais
avec moins d'esprit, en quoi elle était aussi fort inférieure à son
aînée. Toutes deux chargèrent M\textsuperscript{me} d'Urfé de m'assurer
qu'elles avaient été si étonnées qu'elles n'avaient point de peine à se
persuader que je n'avais rien de semblable dans le coeur ni dans la
bouche, ce qu'elles accompagnèrent de toutes sortes de marques d'estime,
de discours obligeants et de compliments pour moi. Elles tinrent le même
langage à Pontchartrain lorsqu'il leur parla.

M\textsuperscript{me} la duchesse de Ventadour, le prince de Rohan, son
gendre, et M. de Strasbourg n'avaient appris cela que par
M\textsuperscript{lle} de Lislebonne et M\textsuperscript{me} d'Espinoy.
Je ne leur fis rien dire, non plus qu'eux ne m'avaient point fait parler
comme avaient fait les deux soeurs. M\textsuperscript{me} de Ventadour
en fut apparemment piquée. Elle continua ses plaintes, et moi, content
de ce que j'avais fait, je les laissai tomber.

Cette, noirceur ne prit pas, mais ne laissa pas de faire quelque bruit.
J'étais outré contre le duc de Mortemart\,; et tout gendre qu'il fût de
M. de Beauvilliers, qui était pour moi toutes choses et en tout genre,
je crus pousser toute considération à bout de ne pas l'aller chercher,
mais bien résolu à l'insulter la première fois que je le rencontrerais.
Il était à Paris depuis Marly, et je l'attendais au retour avec
impatience. M\textsuperscript{me} de Saint-Simon, à qui, ni à personne,
je m'étais bien gardé d'en laisser rien entendre, ne laissait pas d'être
inquiète. Elle la fut encore plus de ce qu'elle remarqua que, pressé par
le duc de Charost, intimement de nos amis, je n'avais pas voulu lui
conter cette histoire qui n'avait pas été tout entière jusqu'à lui. Elle
se hâta de la lui conter en mon absence, et lui de l'aller dire à M. de
Beauvilliers qui accourut aussitôt chez moi. Il n'est pas possible
d'exprimer tout ce qu'il sentit, et dit en cette occasion, jusqu'à
déclarer qu'entre son gendre et moi il abandonnerait son gendre. Il
l'envoya chercher à Paris, qui ne trouvant ni M. ni
M\textsuperscript{me} de Beauvilliers chez eux, monta chez M. de
Chevreuse, où il crut les rencontrer. Il ne trouva que
M\textsuperscript{me} de Chevreuse qui renvoya sa compagnie, et ne
retint que M\textsuperscript{me} de Lévi sa fille, devant qui, sans rien
apprendre au duc de Mortemart, elle lui demanda seulement ce qui s'était
passé entre lui et moi chez M\textsuperscript{me} Chamillart. Il lui en
fit le récit tel que je l'ai rapporté. M\textsuperscript{me} de
Chevreuse le questionna fort, et, voyant qu'elle n'en tirait rien de
plus, elle lui conta tout le fait. Le duc de Mortemart, à son tour,
entra dans une grande surprise et parut fort en colère, nia nettement et
absolument qu'il eût rien dit d'approchant de ce qu'il apprenait là
qu'on lui imputait d'avoir dit, se récria sur la noirceur d'une chose
qu'il faudrait qu'il eût inventée, puisqu'il ne m'avait jamais entendu
rien dire qui en pût approcher. Il en dit autant après à M. de
Beauvilliers, et s'offrit de le soutenir à M\textsuperscript{lle} de
Lislebonne, à M\textsuperscript{me} d'Espinoy, à M\textsuperscript{me}
d'Urfé et à Pontchartrain. MM. de Chevreuse et de Beauvilliers me le
dirent de sa part, et me prièrent de trouver bon qu'ils me l'amenassent
pour me le dire lui-même. Je ne tardai pas à instruire Pontchartrain et
M\textsuperscript{me} d'Urfé de cette négative entière, et de la faire
porter par eux à M\textsuperscript{lle} de Lislebonne et à
M\textsuperscript{me} d'Espinoy.

Cependant nulle exécution de sa part, et les deux soeurs fermes à
maintenir son rapport. Personne ne devait être plus pressé que lui de se
tirer par ce démenti éclatant du personnage de délateur infâme (quand il
aurait été vrai que j'eusse dit ce qu'on m'imputait), ou d'imposteur
exécrable, et dans toutes les circonstances qui accompagnaient une telle
imposture. De cette façon je demeurai dans l'incertitude si le duc de
Mortemart, leur parlant de ce qui s'était passé, chose en soi
inexcusable, ne s'était point échauffé de discours en discours assez
pour leur laisser croire ce qu'elles me firent dire, et, en bons
rejetons des Guise, me commettre contre le gendre de M. de Beauvilliers.

Quoi qu'il en soit, les choses en demeurèrent là, sans que le duc de
Mortemart m'en ait jamais parlé, d'où je jugeai son cas fort sale. Sa
famille répandit son désaveu partout, et de mon côté je ne m'y épargnai
pas, et à publier le démenti que j'avais offert, dont les témoins
n'étaient pas récusables, et qui fut avoué partout de
M\textsuperscript{lle} de Lislebonne et de M\textsuperscript{me}
d'Espinoy. Je ne sais comment le duc de Mortemart s'en tira avec elles.
L'affaire demeura nette à mon égard, très sale au sien. Je demeurai
froid et fort dédaigneux avec lui lorsque je le rencontrais, lui fort
embarrassé avec moi. M. de Beauvilliers, sans que je lui en parlasse,
peiné de nous voir de la sorte, et blessé de ce que son gendre n'était
point venu chez moi, comme lui et le duc de Chevreuse l'y avaient voulu
mener, et que même il ne m'avait pas dit un mot sur cette affaire,
quelque temps après lui défendit de se trouver chez lui quand j'y
serais\,; M. et M\textsuperscript{me} de Chevreuse de, même\,; tellement
qu'il n'y entra plus lorsque j'y étais, et qu'il en sortait à l'instant
que j'y arrivais. Cela dura ainsi plusieurs années sans que j'en aie été
moins intimement avec sa propre mère et tout le reste de sa famille. Ce
n'est pas la dernière fois que j'aurai à parler du duc de Mortemart\,;
mais je dois le témoignage à La Feuillade qu'il rendit, sans que je lui
en parlasse, justice à la vérité, et partout et hautement, quoique nous
ne fussions en aucune mesure d'amitié ni de commerce.

M\textsuperscript{me} de Maubuisson mourut, à quatre-vingt-six ans, dans
son abbaye près Pontoise, plus considérée encore pour son rare savoir,
pour son esprit et pour son éminente piété, que parce qu'elle était née
et environnée. Elle était fille de Frédéric V, électeur palatin, élu roi
de Bohème en 1619, défait, dépouillé et proscrit en 1621, et ses États
avec sa dignité électorale donnés au duc de Bavière, mort en Hollande en
ce triste état, en 1632, à trente-huit ans, laissant de la fille du roi
Jacques Ier, roi de la Grande-Bretagne, un grand nombre d'enfants sans
patrimoine. L'aîné, Charles-Louis, fut rétabli dans ses États du Rhin
par la paix de Munster, en 1648, avec un nouvel et dernier électorat
créé en sa faveur, le haut Palatinat et la dignité de premier électeur
étant conservés à l'électeur de Bavière. Ce Charles-Louis n'eut qu'un
fils et une fille, qui fut seconde femme de Monsieur et mère de M. le
duc d'Orléans et de la duchesse de Lorraine. Le fils fut le dernier
électeur de cette branche, et mourut sans enfants en 1706. Son électorat
et ses États passèrent au duc de Neubourg, beau-père de l'empereur
Léopold, etc. M\textsuperscript{me} de Maubuisson eut trois autres
frères qui parurent dans le monde\,: le prince Robert, qui s'établit en
Angleterre, et qui y parut avec réputation dans le parti du malheureux
roi Charles Ier pendant les guerres civiles qui conduisirent ce monarque
sur l'échafaud, à la honte éternelle des Anglais\,; le prince Maurice,
qui, comme Robert, ne se maria point, et qui périt en mer à trente-trois
ans, en 1654, allant tenter un établissement en Amérique\,; Édouard,
qu'on appelait le prince palatin, se fit catholique, passa longtemps en
France, y épousa Anne Gonzague, soeur de la reine de Pologne, et fille
de Charles, duc de Mantoue et de Nevers, qui dut, son État à Louis XIII
en tant de façons, à la valeur personnelle de ce grand roi au pas de
Suse si célèbre, dont j'ai parlé ailleurs, et au mépris qu'il fit de la
peste qui infectait alors les Alpes et les lieux où il passa.

Cette Anne Gonzague, belle-soeur de M\textsuperscript{me} de Maubuisson,
est la même qui, sous le nom de princesse palatine, figura si habilement
dans la minorité de Louis XIV, opéra la sortie des princes du Havre, et
se lia d'une si grande amitié avec M. le Prince que, à son retour après
la paix des Pyrénées, ils marièrent leurs enfants en 1663, quelques mois
après la mort d'Édouard, qui mourut catholique à Paris. Elle eut deux
autres filles\,: la princesse de Salm, dont le mari fut gouverneur de
l'empereur Joseph\,; et la duchesse d'Hanovre, de qui j'ai parlé plus
d'une fois, qui n'eut que deux filles\,: l'une mère du duc de Modène
d'aujourd'hui, l'autre que son oncle le prince de Salm persuada à
l'empereur Léopold de faire épouser à Joseph, son fils, empereur après
lui, qui n'en a laissé que la reine de Pologne, électrice de Saxe, et
l'électrice de Bavière, aujourd'hui impératrice.

Ce prince Édouard et la princesse palatine sa femme avaient avec eux
Louise Hollandine, soeur d'Édouard, née en 1622, qui se fit catholique à
Port-Royal, où elle fut élevée, et dont elle prit parfaitement l'esprit.
Elle suivit un détachement qui se fit de ce célèbre monastère, qui alla
réformer celui de Maubuisson\,; elle s'y fit religieuse et en fut nommée
abbesse en 1644. Elle était soeur aînée de Sophie, née en 1630, mariée,
en 1658, à Ernest-Auguste, duc d'Hanovre, créé neuvième électeur par
l'empereur Léopold le 19 décembre 1692. C'est cette Sophie que Madame
aimait tant, à qui elle écrivait sans cesse et beaucoup trop, comme on
l'a vu à la mort de Monsieur. Ce fut elle que le parlement d'Angleterre
déclara, le 23 mars 1701, la première à succéder à la couronne
d'Angleterre, après le roi Guillaume, prince d'Orange, et Anne, sa
belle-soeur, princesse de Danemark, et leur postérité, au préjudice de
cinquante-deux héritiers plus proches, mais tous catholiques. Sophie,
entre plusieurs enfants, laissa, en mourant veuve en 1714, son fils aîné
Georges-Louis, duc et électeur d'Hanovre\,; qui succéda à la reine Anne
d'Angleterre, --- père du roi --- d'Angleterre d'aujourd'hui.

Ainsi M\textsuperscript{me} de Maubuisson était soeur du père de Madame
et du père de M\textsuperscript{me} la Princesse et de ses soeurs\,; de
la mère de l'électeur d'Hanovre, roi d'Angleterre\,; fille de la soeur
du roi d'Angleterre Charles Ier\,; tante des deux rois d'Angleterre, ses
fils\,; et grand'tante de l'impératrice Amélie, femme de l'empereur
Joseph. Tant d'éclat fut absorbé sous son voile. Elle ne fut
principalement que religieuse et seulement abbesse pour éclairer et
conduire sa communauté, dont elle ne souffrit jamais d'être distinguée
en rien. Elle ne connut que sa cellule, le réfectoire, la portion
commune. Elle ne manqua à aucun office ni à aucun exercice de la
communauté, écarta les visites, la première à tout et la plus régulière,
ardente à servir ses religieuses avec un esprit en tout supérieur et un
grand talent de gouvernement, dont la charité, la douceur, la
prévenance, la tendresse pour ses filles était l'âme, et desquelles
aussi elle fut continuellement adorée\,: aussi n'était-elle contente
qu'avec elles, et ne sortit jamais de sa maison. Les autres se
souvenaient d'autant plus de ce qu'elle était qu'elle semblait l'avoir
entièrement oublié, avec une simplicité parfaite et naturelle. Son
humilité avait banni toutes les différences que les moindres abbesses
affectent dans leurs maisons, et tout air de savoir les moindres choses,
encore qu'elle égalât beaucoup de vrais savants. Elle avait infiniment
d'esprit, aisé, naturel, sans songer jamais qu'elle en eût, non plus que
de science.

Madame, M\textsuperscript{me} la Princesse, le roi et la reine
d'Angleterre, l'allaient voir toujours plus souvent qu'elle ne voulait.
Madame et M\textsuperscript{me} la Princesse lui étaient extrêmement
attachées. La feue reine, M\textsuperscript{me} la dauphine de Bavière,
l'avaient été voir plusieurs fois\,; la maison de Condé souvent,
Monsieur aussi, et sa belle-soeur la princesse palatine, très souvent
tant qu'elle vécut. Pour peu qu'elle n'eût pas `été attentive à rompre
et à éviter les commerces, les visites les plus considérables et les
lettres n'auraient pas cessé\,; nais elle ne voulait pas retrouver le
monde dans le lieu qu'elle avait pris pour asile contre lui.

Elle conserva sa tête, sa santé, sa régularité entières jusqu'à la mort,
et laissa sa maison inconsolable. Quoique peu au goût de la cour, par
celui de terroir qu'elle avait apporté de Port-Royal, et qu'elle
conserva chèrement dans sa maison et dans elle-même, sans s'en cacher,
elle ne laissa pas d'avoir une grande considération toute sa vie, qui
fut sans cesse le modèle des plus excellentes religieuses et des plus
parfaites abbesses, auquel très peu ou point ont pu atteindre.
M\textsuperscript{me} la duchesse de Bourgogne était sa petite-nièce.
Toute la famille royale, excepté le roi, en prit le deuil pour sept ou
huit jours. Celui de Madame et de M\textsuperscript{me} la Princesse
dura le temps ordinaire aux nièces.

En même temps mourut M. d'Avaux. Son grand-père, son père, son frère
aîné et le fils de ce frère, furent tous quatre successivement
présidents â mortier, et le dernier est mort premier président. M. de
Mesmes, frère de d'Avaux, avait eu de La Basinière, son beau-père, la
charge de prévôt et grand maître des cérémonies de l'ordre, dont d'Avaux
eut la survivance pendant sa première ambassade en Hollande, que son
neveu eut ensuite. D'Avaux et son frère étaient neveux paternels du
président de Mesmes, et de M. d'Avaux, surintendant des finances,
célèbre par sa captivité et le nombre de ses importantes ambassades.
Tous deux étaient aînés du père du président de Mesmes et de d'Avaux
duquel je parle ici. D'Avaux l'oncle mourut sans alliance en 1650\,; et
son frère aîné, mort la même année, ne laissa que M\textsuperscript{me}
de Vivonne et une religieuse naine à la Visitation de Chaillot, soeur de
mère de la duchesse de Créqui, qui a été dame d'honneur de la reine.

D'Avaux le neveu avait été conseiller au parlement, maître des requêtes,
enfin conseiller d'État. C'était un fort bel homme et bien fait, galant
aussi, et qui avait de l'honneur, fort l'esprit du grand monde, de la
grâce, de la noblesse, et beaucoup de politesse. Il alla d'abord
ambassadeur à Venise, ensuite plénipotentiaire à Nimègue, où, en grand
courtisan qu'il était, il s'attacha à Croissy, qui l'était avec lui et
frère de Colbert, lequel le fit secrétaire d'État des affaires
étrangères à la disgrâce de Pomponne. D'Avaux, quelque temps après la
paix de Nimègue, fut ambassadeur en Hollande. Le nom qu'il portait lui
servit fort pour tous ces emplois, et le persuada qu'il en était aussi
capable que son oncle. Il faut pourtant avouer qu'il en avait des
talents, de l'adresse, de l'insinuation, de la douceur, et qu'il fut
toujours partout parfaitement averti. Il s'acquit en Hollande une amitié
et une considération si générale et jusque des peuples, et sut si bien
se ménager avec le prince d'Orange, parmi les ordres positifs et
réitérés qu'il avait de chercher à lui faire de la peine en tout jusque
dans les choses inutiles, qu'il aurait fait tout ce qu'il aurait voulu
pour le roi, sans cette aversion que le prince d'Orange ne put jamais
vaincre, et dont j'ai expliqué en son lieu la funeste origine, qui le
jeta dans le parti opposé à la France, de laquelle il devint enfin le
plus grand ennemi.

D'Avaux fut informé, dès les premiers temps, et longtemps encore les
plus secrets, du projet de la révolution d'Angleterre, et en avertit le
roi. On se moqua de lui, et on aima mieux croire Barillon, ambassadeur
du roi en Angleterre, qui, trompé par Sunderland et les autres ministres
confidents du roi Jacques, mais perfides et qui trempaient eux-mêmes
dans la conjuration, abusé par le roi d'Angleterre même dupe de ses
ministres, rassura toujours notre cour, et lui persuada que les soupçons
qu'on y donnait n'étaient que des chimères.

Ils devinrent pourtant si forts, et d'Avaux marquait tant de
circonstances et de personnes, qu'il ne tint qu'à nous de n'être pas les
dupes, en laissant le siège de Maestricht qui déconcertait toutes les
mesures, au lieu de celui de Philippsbourg qui n'en rompit aucunes. Mais
Louvois voulait la guerre, et se garda bien de l'arrêter tout court.
Outre sa raison générale d'être plus maître de tout par son département
de la guerre\,; il en eut une particulière très pressante, que j'ai sue
longtemps depuis bien certainement, et qui est trop curieuse pour
l'omettre, puisque l'occasion s'en présente si naturellement ici.

Le roi, qui aimait à bâtir, et qui n'avait plus de maîtresses, avait
abattu le petit Trianon de porcelaine qu'il avait pour
M\textsuperscript{me} de Montespan, et le rebâtissait pour le mettre en
l'état où on le voit encore. Louvois était surintendant des bâtiments.
Le roi, qui avait le coup d'oeil de la plus fine justesse, s'aperçut
d'une fenêtre de quelque peu\,: plus étroite que les autres, les
trémeaux ne faisaient encore que de s'élever, et n'étaient pas joints
par le haut. Il la montra à Louvois pour la réformer, ce qui était alors
très aisé. Louvois soutint que la fenêtre était bien. Le roi insista, et
le lendemain encore, sans que Louvois, qui était entier, brutal et enflé
de son autorité, voulût céder.

Le lendemain le roi vit Le Nôtre dans la galerie. Quoique son métier ne
fût guère que les jardins, où il excellait, le roi ne laissait pas de le
consulter sur ses bâtiments. Il lui demanda s'il avait été à Trianon. Le
Nôtre répondit que non. Le roi lui ordonna d'y aller. Le lendemain il le
vit encore\,; même question, même réponse. Le roi comprit à quoi il
tenait, tellement qu'un peu fâché, il lui commanda de s'y trouver
l'après-dînée même, à l'heure qu'il y serait avec Louvois. Pour cette
fois Le Nôtre n'osa y manquer. Le roi arrivé et Louvois présent, il fut
question de la fenêtre que Louvois opiniâtra toujours de largeur égale
aux autres. Le roi voulut que Le Nôtre l'allât mesurer, parce qu'il
était droit et vrai, et qu'il dirait librement ce qu'il aurait trouvé.
Louvois piqué s'emporta. Le roi, qui ne le fut pas moins le laissait
dire, et cependant Le Nôtre, qui aurait bien voulu n'être pas là, ne
bougeait. Enfin le roi le fit aller, et cependant Louvois toujours à
gronder, et à maintenir l'égalité de la fenêtre, avec audace et peu de
mesure. Le Nôtre trouva et dit que le roi avait raison de quelques
pouces. Louvois voulut imposer, mais le roi à la fin trop impatienté le
fit taire, lui commanda de faire défaire la fenêtre à l'heure même, et,
contre sa modération ordinaire, le malmena fort durement.

Ce qui outra le plus Louvois, c'est que la scène se passa non seulement
devant les gens des bâtiments, mais en présence de tout ce qui suivait
le roi en ses promenades, seigneurs, courtisans, officiers des gardes et
autres, et même de tous les valets, parce qu'on ne faisait presque que
sortir le bâtiment de terre, qu'on était de plain-pied à la cour, à
quelques marches près, que tout était ouvert, et que tout suivait
partout. La vesperie fut forte et dura assez longtemps, avec les
réflexions des conséquences de la faute de cette fenêtre, qui, remarquée
plus tard, aurait gâté toute cette façade et aurait engagé à l'abattre.

Louvois, qui n'avait pas, accoutumé d'être traité de la sorte, revint,
chez lui en furie et comme un homme au désespoir. Saint-Pouange, les
Tilladet et ce peu de familiers de toutes ses heures, en furent
effrayés, et, dans leur inquiétude, tournèrent pour tâcher de savoir ce
qui était arrivé. À la fin, il le leur conta, dit qu'il était perdu, et
que, pour quelques pouces, le roi oubliait tous ses services qui lui
avaient valu tant de conquêtes\,; mais qu'il y mettrait ordre, et qu'il
lui susciterait une guerre, telle qu'il lui ferait avoir besoin de lui,
et laisser là la truelle, et de là s'emporta en reproches et en fureurs.

Il ne mit guère à tenir parole. Il enfourna la guerre par l'affaire de
la double élection de Cologne, du prince de Bavière et du cardinal de
Fürstemberg\,; il la confirma en portant des flammes dans le Palatinat,
et en laissant toute liberté au projet d'Angleterre\,; il y mit le
dernier sceau pour la rendre générale, et s'il eût pu éternelle, en
désespérant le duc de Savoie, qui ne voulait que la paix, et qu'à l'insu
du roi il traita si indignement qu'il le força à se jeter entre les bras
de ses ennemis, et à devenir après, par la position de son pays, notre
partie la plus difficile et la plus ruineuse. Tout cela a été mis bien
au net depuis.

Pour en revenir à d'Avaux, de retour de Hollande par la rupture, il
passa en Irlande avec le roi d'Angleterre, en qualité d'ambassadeur du
roi auprès de lui, avec entrée dans son conseil. Il n'avait garde de
réussir auprès d'un prince avec lequel il ne fut jamais d'accord, qui
fut trompé sans cesse, qui s'opiniâtra, malgré les expériences et tout
ce que d'Avaux lui put représenter, à donner dans tous les pièges qui
lui étaient tendus. Les événements montrèrent sans cesse combien d'Avaux
avait raison\,; mais une lourde méprise le perdit pour un temps, et ce
fut par un bonheur qu'il ne pouvait guère espérer que ce ne fut pas
perdu pour toujours. Il rendait compte des affaires aux deux ministres
de la guerre et des affaires étrangères\,: des troupes, des munitions,
des mouvements et des projets de guerre à Louvois\,; des négociations du
cabinet et de la conduite du roi d'Angleterre, de l'intérieur de
l'Irlande et des intelligences d'Angleterre à Croissy, son ancien
camarade de Nimègue, et depuis cette époque son ami. Il s'était de plus
en plus attaché à lui par son ambassade de Hollande. Le fond de son
emploi dépendait de lui, le reste, qui allait à Louvois, n'était que par
accident\,; ainsi l'intérêt et le coeur étaient d'accord en faveur de
Croissy. Celui-ci était ennemi de Louvois qui le malmenait fort, et
d'Avaux lui écrivait conformément à sa passion contre Louvois.
Malheureusement le secrétaire de d'Avaux se méprit aux enveloppes. Il
adressa la lettre pour Louvois, à Croissy, et celle pour Croissy à
Louvois, qui, à sa lecture, entra dans une si furieuse colère que
Croissy lui-même s'en trouva fort embarrassé. D'Avaux en fut perdu. Il
n'eut d'autre parti à prendre que de demander à revenir. Il l'obtint.
Son bonheur voulut que Louvois, perdu lui-même auprès de
M\textsuperscript{me} de Maintenon (ce qui n'est pas de mon sujet, mais
qui se retrouvera peut-être ailleurs), ne fit plus que déchoir et allait
être arrêté, comme je l'ai déjà dit plus haut à propos du projet de
reprendre Lille, lorsqu'il mourut. Ce fut pour d'Avaux une belle
délivrance.

On l'envoya ambassadeur en Suède. Le comte d'Avaux, orné du cordon bleu,
plut infiniment en ce pays-là. Il y renouvela les traités et y servit
fort bien. Il arriva dans ce même temps que quelque indiscret ou malin
se moqua de la crédulité de la cour de Stockholm, et y révéla que ce
seigneur n'était qu'un homme de, robe, nullement chevalier du
Saint-Esprit, mais revêtu d'un cordon bleu vénal, dont aucun homme, non
seulement de qualité mais d'épée, ne voudrait depuis MM. de Rhodes, dont
l'histoire fut éclaircie. Les Suédois sont fiers, ils se crurent
dédaignés. D'Avaux, dont les manières leur avaient jusque-là beaucoup
plu, ne leur fut plus agréable. Il essuya des dégoûts qui le pressèrent
de hâter son retour.

En 1701, sur le point de la rupture des Hollandais qu'on désirait avec
passion d'éviter, il fut renvoyé à la Haye comme un homme qui leur était
personnellement agréable et qui y avait beaucoup d'amis. En effet il y
fut parfaitement bien reçu et retenu même à diverses reprises\,; mais
tout fut personnel pour lui, et pour amuser en attendant leurs dernières
mesures bien prises. Leur parti était décidé. Le roi Guillaume régnait
chez eux, et tous les charmes de d'Avaux ne purent empêcher la rupture.
Il se fit tailler peu après son retour. Les incommodités qui lui en
demeurèrent ne l'empêchèrent pas de vouloir encore être employé,
quoiqu'en effet elles l'en rendissent incapable.

C'était un homme d'un très aimable commerce, mais qui par goût, par
opinion de soi, par habitude, voulait être, se mêler et surtout être
compté. Parmi tant de bonnes choses, une misère le rendit ridicule. Il
était, comme on l'a dit, de robe, avait passé par les différentes
magistratures jusqu'à être conseiller d'État de robe aussi. Mais
accoutumé à porter l'épée et à être le comte d'Avaux en pays étranger,
où ses ambassades l'avaient tenu bien des années à reprises, il ne put
se résoudre à se défaire, en ses retours ici, ni de son épée, ni de sa
qualité de comte, ni à reprendre l'habit de son état. Il était donc à
son regret vêtu de noir, n'osant hasarder l'or ni le gris, mais avec la
cravate et le petit canif à garde d'argent au côté\,; et le cordon bleu
qu'il portait pardessus en écharpe lui contentait l'imagination, en le
faisant passer pour un chevalier de l'ordre en deuil au peuple et à ceux
qui ne le connaissaient pas. Il n'allait jamais à aucun des bureaux du
conseil, non plus que les conseillers d'État d'épée. La douleur était
qu'il fallait pourtant, aller au conseil, y être en robe de conseiller
d'État comme les autres, et porter l'ordre au cou, y voir cependant les
conseillers d'État en justaucorps gris ou d'autre couleur, en un mot, en
épée et avec leurs habits ordinaires.

Cela faisait un fâcheux contraste avec Courtin et Amelot, conseillers
d'État de robe, et longtemps ambassadeurs comme lui, et qui toujours à
leur retour avaient repris tout aussitôt leur habit, et toutes leurs
fonctions du conseil sans en manquer aucune. Le chancelier de
Pontchartrain ne pouvait digérer cela de d'Avaux\,; il mourait d'envie
de lui en parler, mais le roi le voyait, en riait tout bas, et avait la
bonté de le laisser faire. Cela arrêtait le chancelier et les
conseillers d'État, qui en douceur le trouvaient très mauvais. La pierre
lui revint, et il mourut de la seconde taille, assez pauvre, sans avoir
été marié. Il avait vendu au président de Mesmes, son neveu, sa charge
de l'ordre, avec permission de continuer de le porter. Avec tout cela il
eut toujours des amis et de la considération.

Un mois après il fut suivi par sa cousine germaine, veuve du
maréchal-duc de Vivonne. C'était une femme de beaucoup d'esprit, dont la
singularité était digne de s'allier aux Mortemart. Elle était
extrêmement riche, et ces messieurs-là, qui régulièrement se ruinaient
de père en fils, trouvaient aussi à se remplumer par de riches mariages.
Pour ces deux-ci ils n'eurent rien à se reprocher, et se ruinèrent à qui
mieux mieux chacun de leur côté. C'étaient des farces, à ce que j'ai ouï
dire aux contemporains, que de les voir ensemble\,; mais ils n'y étaient
pas souvent, et ne s'en devaient guère à faire peu de cas l'un de
l'autre.

M. de Vivonne était brouillé avec le duc de Mortemart, son fils, que
j'ai vu regretter comme un grand sujet et un fort honnête homme aux ducs
de Chevreuse et de Beauvilliers, ses beaux-frères, et à qui le roi donna
des millions avec la troisième fille de Colbert, dont
M\textsuperscript{me} de Montespan fit le mariage. À l'extrémité du duc
de Mortemart, M. de Seignelay fit tant qu'il lui amena M. de Vivonne. Il
le trouva mourant, et sans en approcher se mit tranquillement à le
considérer, le cul appuyé contre une table. Toute la famille était là
désolée. M. de Vivonne, après un long silence, se prit tout d'un coup à
dire\,: «\,Ce pauvre homme-là n'en reviendra pas, j'ai vu mourir tout
comme cela son pauvre père.\,» On peut juger quel scandale cela fit (ce
prétendu père était un écuyer de M. de Vivonne ). Il ne s'en embarrassa
pas le moins du monde, et après un peu de silence il s'en alla. C'était
l'homme le plus naturellement plaisant, et avec le plus d'esprit et de
sel et le plus continuellement, dont j'ai ouï faire au feu roi cent
contes meilleurs les uns que les autres qu'il se plaisait à raconter.

M\textsuperscript{me} de Vivonne avait été de tous les particuliers du
roi qui ne pouvait s'en passer\,; mais il s'en fallait bien qu'il l'eût
tant ni quand il voulait. Elle était haute, libre et capricieuse, ne se
souciait de faveur ni de privante et ne voulait que son amusement.
M\textsuperscript{me} de Montespan et M\textsuperscript{me} de Thianges
la ménageaient, et elle les ménageait fort peu. C'était souvent entre
elles des disputes et des scènes, excellentes. Elle aimait fort le jeu
et y était furieuse même les dernières années de sa vie qu'elle fut
dévote tant qu'elle put, et réduite, après avoir tout fricassé elle et
son mari, mort dès 1688, à n'avoir presque rien qu'une grosse pension du
roi, et à loger chez sort intendant avec un train fort court, où elle
jouait peu et aux riens, et conserva toujours de la considération, mais
laissa peu de regrets.

Boisseuil mourut en peu de temps. C'était un gentilhomme grand et gros,
fort bien fait en son temps, excellent homme de cheval, grand
connaisseur, qui dressait tous ceux du roi, et qui commandait la grande
écurie, parce que Lyonne\footnote{Le manuscrit porte Lyonne\,; mais il
  faut probablement lire Brionne comme on le voit par la fiai du
  paragraphe.}, qui en était premier écuyer, ne fit jamais sa charge.
Boisseuil s'était mis par là fort au goût du roi, qui le traita toujours
avec distinction. C'était un honnête homme et fort brave, qui voulait
être à sa place et respectueux, mais qui était gâté de la confiance
entière de M. le Grand et de M\textsuperscript{me} d'Armagnac qu'il
conserva toute sa vie. Il était parvenu à les subjuguer et à être
tellement maître de tout à la grande écurie, excepté du pécuniaire, que
M\textsuperscript{me} d'Armagnac s'était réservé et qu'elle fit
étrangement valoir, qu'il y était compté pour tout, et le comte de
Brionne pour rien.

Boisseuil était fort brutal, gros joueur et fort emporté, qui traitait
souvent M. le Grand et M\textsuperscript{me} d'Armagnac, tout hauts
qu'ils étaient, à faire honte à la compagnie, qui faisait des sorties,
et qui jurait dans le salon de Marly comme il eût pu faire dans un
tripot. On le craignait, et il disait aux femmes tout ce qu'il lui
venait en fantaisie quand la fureur d'un coupe-gorge le saisissait.

À un voyage du roi, où la cour séjourna quelque temps à Nancy, il se mit
un soir à jouer je ne sais plus chez qui de la cour. Un joueur s'y
trouva qui jouait le plus gros jeu du monde. Boisseuil perdait gros et
était fort fâché. Il crut s'apercevoir que ce joueur trompait, qui
n'était connu et souffert que par son jeu. Il le suivit et s'assura par
ses yeux si bien, que tout à coup il s'élança sur la table, et lui
saisit la main qu'il tenait sur la table avec les cartes dont il allait
donner. Le joueur, fort étonné, voulut tirer sa main et se fâcher.
Boisseuil, plus fort que lui, lui dit qu'il était un fripon, et à la
compagnie qu'elle allait le voir\,; et tout de suite, lui secouant la
main de furie, mit en évidence la tromperie. Le joueur, confondu, se
leva et s'en alla. Le jeu dura encore du temps et assez avant dans la
nuit. Lorsqu'il finit Boisseuil s'en alla. Comme il sortait la porte
pour se retirer à pied, il trouva un homme collé contre la muraille, qui
lui proposa de lui faire raison de l'affront qu'il lui avait fait\,:
c'était le même joueur qui l'avait attendu là. Boisseuil lui répondit
qu'il n'avait point de raison à lui faire et qu'il était un fripon.
«\,Cela peut être, lui répliqua le joueur\,; mais je n'aime pas qu'on me
le dise.\,» Ils s'allèrent battre sur-le-champ. Boisseuil y remboursa
deux coups d'épée, de l'un desquels il pensa mourir. Le joueur s'évada
sans blessure et se battit fort bien, à ce que dit Boisseuil. Personne
n'ignora cette aventure, que le roi qui la sut des premiers, et qui, par
bonté pour Boisseuil, la voulut toujours ignorer, prit sa blessure pour
une maladie ordinaire.

Il n'était ni marié ni riche, mais à son aise. Sa physionomie, toujours
furibonde en son temps, faisait peur, avec de gros yeux rouges qui lui
sortaient de la tête.

Janson se retira en ce temps-ci. Il était fils du frère du cardinal de
Janson, et frère de l'archevêque d'Arles. C'était un homme fort bien
fait, qui avait servi avec réputation, et qui était maréchal de camp,
sous-lieutenant de la première compagnie des mousquetaires, gouverneur
d'Antibes, estimé, bien traité, et fort à son aise. Il était veuf depuis
cinq ou six ans, et avait des enfants. Il était depuis longtemps dans
une grande piété. Vers quarante-trois ou quarante-quatre ans, il se
retira en Provence, bâtit au bout de son parc un couvent de minimes, se
retira parmi eux, vivant en tout comme eux. Il éprouva leur ingratitude
sans en vouloir sortir, pour ajouter cette dure sorte de pénitence à ses
autres austérités. Il vécut dans une grande solitude tout occupé de
prières et de bonnes oeuvres, après avoir donné ordre à sa famille,
vécut saintement près de vingt ans de la sorte, et mourut fort
saintement aussi.

\hypertarget{chapitre-vi.}{%
\chapter{CHAPITRE VI.}\label{chapitre-vi.}}

1709

~

{\textsc{Mort et caractère de M. le prince de Conti.}} {\textsc{-
Pensions à la princesse et au prince de Conti.}} {\textsc{- Deuil du roi
et ses visites.}} {\textsc{- Eau bénite du prince de Conti.}} {\textsc{-
Friponnerie débitée sur moi, bien démentie.}} {\textsc{- Adresse trop
orgueilleuse de M. le Duc, découverte et vaine.}} {\textsc{- Entreprises
inutiles de M. le Duc, forcé d'avouer et de donner des fauteuils aux
ducs pareils au sien, au service du prince de Conti, où les évêques n'en
purent obtenir.}}

~

M. le prince de Conti mourut, le jeudi 21 février, sur les neuf heures
du matin, après une longue maladie qui finit par l'hydropisie. La goutte
l'avait réduit au lait pour toute nourriture, qui lui avait réussi
longtemps. Son estomac s'en lassa\,; son médecin s'y opiniâtra et le
tua. Quand il n'en fut plus temps, il demanda et obtint de faire venir
de Suisse un excellent médecin français réfugié, nommé Trouillon, qui le
condamna dès en arrivant. Il n'avait pas encore quarante-cinq ans.

Sa figure avait été charmante. Jusqu'aux défauts de son corps et de son
esprit avaient des grâces infinies. Des épaules trop hautes, la tête un
peu penchée de côté, un rire qui eût tenu du braire dans un autre, enfin
une distraction étrange. Galant avec toutes les femmes, amoureux de
plusieurs, bien traité de beaucoup, il était encore coquet avec tous les
hommes. Il prenait à tâche de plaire au cordonnier, au laquais, au
porteur de chaise, comme au ministre d'État, au grand seigneur, au
général d'armée, et si naturellement, que le succès en était certain. Il
fut aussi les constantes délices du monde, de la cour, des armées, la
divinité du peuple, l'idole des soldats, le héros des officiers,
l'espérance de ce qu'il y avait de plus distingué, l'amour du parlement,
l'ami avec discernement des savants, et souvent l'admiration de la
Sorbonne, des jurisconsultes, des astronomes et des mathématiciens les
plus profonds. C'était un très bel esprit, lumineux, juste, exact,
vaste, étendu, d'une lecture infinie, qui n'oubliait rien, qui possédait
les histoires générales et particulières, qui connaissait les
généalogies, leurs chimères et leurs réalités, qui savait où il avait
appris chaque chose et chaque fait, qui en discernait les sources, et
qui retenait et jugeait de même tout ce {[}que{]} la conversation lui
avait appris, sans confusion, sans mélange, sans méprise, avec une
singulière netteté.

M. de Montausier et M. de Meaux, qui l'avaient vu élever auprès de
Monseigneur, l'avaient toujours aimé avec tendresse, et lui eux avec
confiance. Il était de même avec les ducs de Chevreuse et de
Beauvilliers, et avec l'archevêque de Cambrai, et les cardinaux
d'Estrées et de Janson. M. le Prince le héros ne se cachait pas d'une
prédilection pour lui au-dessus de ses enfants\,; il fut la consolation
de ses dernières années. Il s'instruisit dans son exil et sa retraite
auprès de lui\,; il écrivit sous lui beaucoup de choses curieuses. Il
fut le coeur et le confident de M. de Luxembourg dans ses dernières
années.

Chez lui, l'utile et le futile, l'agréable et le savant, tout était
distinct et en sa place. Il avait des amis\,; il savait les choisir, les
cultiver, les visiter, vivre avec eux, se mettre à leur niveau sans
hauteur et sans bassesse. Il avait aussi des amies indépendamment
d'amour. Il en fut accusé de plus d'une sorte, et c'était un de ses
prétendus rapports avec César.

Doux jusqu'à être complaisant dans le commerce, extrêmement poli, mais
d'une politesse distinguée selon le rang, l'âge, le mérite, et mesuré
avec tous. Il ne dérobait rien à personne. Il rendait tout ce que les
princes du sang doivent, et qu'ils ne rendent plus\,; il s'en expliquait
même et sur leurs usurpations et sur l'histoire des usages et de leurs
altérations. L'histoire des livres et des conversations lui fournissait
de quoi placer, avec un art imperceptible, ce qu'il pouvait de plus
obligeant sur la naissance, les emplois, les actions. Son esprit était
naturel, brillant, vif\,; ses reparties, promptes, plaisantes, jamais
blessantes\,; le gracieux répandu partout sans affectation\,; avec toute
la futilité du monde, de la cour, des femmes, et leur langage avec
elles, l'esprit solide et infiniment sensé\,; il en donnait à tout le
monde\,; il se mettait sans cesse et merveilleusement à la portée et au
niveau de tous, et parlait le langage de chacun avec une facilité
nonpareille. Tout en lui prenait un air aisé. Il avait la valeur des
héros, leur maintien à la guerre, leur simplicité partout, qui toutefois
cachait beaucoup d'art. Les marques de leur talent pourraient passer
pour le dernier coup de pinceau de son portrait, mais comme tous les
hommes il avait sa contrepartie.

Cet homme si aimable, si charmant, si délicieux, n'aimait rien. Il avait
et voulait des amis, comme on veut et comme on a des meubles. Encore
qu'il se respectât, il était bas courtisan, il ménageait tout et
montrait trop combien il sentait ses besoins en tout genre de choses et
d'hommes\,; avare, avide de biens, ardent, injuste. Le contraste de ses
voyages de Pologne et de Neuchâtel ne lui fit pas d'honneur. Ses procès
contre M\textsuperscript{me} de Nemours, et ses manières de les suivre,
ne lui en firent pas davantage, bien moins encore sa basse complaisance
pour la personne et le rang des bâtards qu'il ne pouvait souffrir, et
pour tous ceux dont il pouvait avoir besoin, toutefois avec plus de
réserve, sans comparaison, que M. le Prince.

Le roi était vraiment peiné de la considération qu'il ne pouvait lui
refuser, et qu'il était exact à n'outrepasser pas d'une ligne. Il ne lui
avait jamais pardonné son voyage de Hongrie. Les lettres interceptées
qui lui avaient été écrites et qui avaient perdu les écrivains, quoique
fils de favoris, avaient allumé une haine dans M\textsuperscript{me} de
Maintenon, et une indignation dans le roi, que rien n'avait pu effacer.
Les vertus, les talents, les agréments, la grande réputation que ce
prince s'était acquise, l'amour général qu'il s'était concilié, lui
étaient tournés en crimes. Le contraste de M. du Maine excitait un dépit
journalier dans sa gouvernante et dans son tendre père, qui leur
échappait malgré eux. Enfin la pureté de son sang, le seul qui ne fût
point mêlé avec la bâtardise, était un autre démérite qui se faisait
sentir à tous moments. Jusqu'à ses amis étaient odieux, et le sentaient.

Toutefois, malgré la crainte servile, les courtisans même aimaient à
s'approcher de ce prince. On était flatté d'un accès familier auprès de
lui\,; le monde le plus important, le plus choisi, le courait. Jusque
dans le salon de Marly il était environné du plus exquis. Il y tenait
des conversations charmantes sur tout ce qui se présentait
indifféremment\,; jeunes et vieux y trouvaient leur instruction et leur
plaisir, par l'agrément avec lequel il s'énonçait sur toutes matières,
par la netteté de sa mémoire, par son abondance sans être parleur. Ce
n'est point une figure, c'est une vérité cent fois éprouvée, qu'on y
oubliait l'heure des repas. Le roi le savait, il en était piqué,
quelquefois môme il n'était pas fâché qu'on pût s'en apercevoir. Avec
tout cela on ne pouvait s'en déprendre\,; la servitude si régnante
jusque sur les moindres choses y échoua toujours.

Jamais homme n'eut tant d'art caché sous une simplicité si naïve, sans
quoi que ce soit d'affecté en rien. Tout en lui coulait de source\,;
jamais rien de tiré, de recherché\,; rien ne lui coûtait. On n'ignorait
pas qu'il n'aimait rien ni ses autres défauts. On les lui passait tous,
et on l'aimait véritablement, quelquefois jusqu'à se le reprocher,
toujours sans s'en corriger.

Monseigneur, auprès duquel il avait été élevé, conservait pour lui
autant de distinction qu'il en était capable, mais il n'en avait pas
moins pour M. de Vendôme, et l'intérieur de sa cour était partagé entre
eux. Le roi porta toujours en tout M. de Vendôme. La rivalité était donc
grande entre eux. On a vu quelques éclats de l'insolence du grand
prieur. Son aîné, plus sage, travaillait mieux en dessous. Son élévation
rapide, à l'aide de sa bâtardise et de M. du Maine, surtout la
préférence au commandement des armées, mit le comble entre eux, sans
toutefois rompre les bienséances.

Mgr le duc de Bourgogne, élevé de mains favorables au prince de Conti,
était au dehors fort, mesuré avec lui\,; mais la liaison intérieure
d'estime et d'amitié était intime et solidement établie. Ils avaient
l'un et l'autre mêmes amis, mêmes jaloux, mêmes ennemis, et sans un
extérieur très uni l'union était parfaite.

M. le duc d'Orléans et M. le prince de Conti n'avaient jamais pu
compatir ensemble\,; l'extrême supériorité de rang avait blessé par trop
les princes du sang. M. le prince de Conti s'était laissé entraîner par
les deux autres. Lui et M. le Duc l'avaient traité un peu trop en petit
garçon à sa première campagne, et à la seconde, avec trop peu de
déférence et de ménagement. La jalousie d'esprit, de savoir, de valeur
les écarta encore davantage. M. le duc d'Orléans, qui ne sut jamais se
rassembler le monde, ne se put défaire du dépit de le voir bourdonner
sans cesse autour du prince de Conti. Un amour domestique acheva de
l'outrer. Conti charma {[}une personne{]} qui, sans être cruelle, ne fut
jamais prise que pour lui. C'est ce qui le tenait sur la Pologne, et cet
amour ne finit qu'avec lui. Il dura même longtemps après dans l'objet
qui l'avait fait naître, et peut-être y duret-il encore après tant
d'années, au fond d'un coeur qui n'a pas laissé de s'abandonner
ailleurs. M. le Prince ne pouvait s'empêcher d'aimer son gendre, qui lui
rendait de grands devoirs. Malgré de grandes raisons domestiques, son
goût et son penchant l'entraînaient vers lui. Ce n'était pas sans
nuages. L'estime venait au secours du goût, et presque toujours ils
triomphaient du dépit. Ce gendre était le coeur et toute la consolation
de M\textsuperscript{me} la Princesse.

Il vivait avec une considération infinie pour sa femme, même avec
amitié, non sans être souvent importuné de ses humeurs, de ses caprices,
de ses jalousies. Il glissait sur tout cela et n'était guère avec elle.
Pour son fils, tout jeune qu'il était, il ne pouvait le souffrir, et le
marquait trop dans son domestique. Son discernement le lui présentait
par avance tel qu'il devait paraître un jour. Il eût mieux aimé n'en
avoir point, et le temps fit voir qu'il n'avait pas tort, sinon pour
continuer la branche. Sa fille, morte duchesse de Bourbon, était toute
sa tendresse\,; l'autre, il se contentait de la bien traiter.

Pour M. le Duc, et lui, ils furent toujours le fléau l'un de l'autre et
d'autant plus fléau réciproque que la parité de l'âge et du rang, la
proximité la plus étroite redoublée, tout avait contribué à les faire
vivre ensemble à l'armée, à la cour, presque toujours dans les mêmes
lieux, quelquefois encore à Paris. Outre les causes les plus intimes,
jamais deux hommes ne furent plus opposés. La jalousie dont M. le Duc
fut transporté toute sa vie était une sorte de rage qu'il ne pouvait
cacher, de tous les genres d'applaudissements qui environnaient son
beau-frère. Il en était d'autant plus piqué que le prince de Conti
coulait tout avec lui, et l'accablait de devoirs et de prévenances. Il y
avait vingt ans qu'il n'avait mis le pied chez M\textsuperscript{me} la
Duchesse, lorsqu'il mourut. Elle-même n'osa jamais envoyer savoir de ses
nouvelles ni en demander devant le monde pendant sa longue maladie. Elle
n'en apprit qu'en cachette, le plus souvent par M\textsuperscript{me} la
princesse de Conti sa soeur. Sa grossesse et sa couche de M. le comte de
Clermont lui vinrent fort à propos pour cacher ce qu'elle aurait eu trop
de peine à retenir. Cette princesse de Conti et son beau-frère vécurent
toujours avec union, amitié et confiance. Elle entendit raison sur la
Choin, que le prince de Conti courtisa comme les autres, et qu'il n'y
avait pas moyen de négliger.

Avec M. du Maine, il n'y avait que la plus indispensable bienséance\,;
pareillement avec la duchesse du Maine, peu de crainte d'ailleurs. M. le
prince de Conti en savait et en sentait trop là-dessus pour ne pas
s'accorder quelque liberté, qui lui était d'autant plus douce qu'elle
était applaudie.

Quelque courtisan qu'il fût, il lui était difficile de se refuser
toujours de toucher par l'endroit sensible, et qu'on n'osait guère
relever, le roi, qu'il n'avait jamais pu se réconcilier, quelque soin,
quelque humiliation, quelque art, quelque persévérance qu'il y eût si
constamment employés, et c'est de cette haine si implacable qu'il mourut
à la fin, désespéré de ne pouvoir atteindre à quoi que ce fût, moins
encore au commandement des armées, et {[}d'être{]} le seul prince sans
charge, sans gouvernement, même sans régiment, tandis que les autres, et
plus encore les bâtards, en étaient accablés.

À bout de tout il chercha à noyer ses déplaisirs dans le vin et dans
d'autres amusements qui n'étaient plus de son âge et pour lesquels son
corps était trop faible, et que les plaisirs de sa jeunesse avaient déjà
altéré. La goutte l'accabla. Ainsi, privé des plaisirs et livré aux
douleurs du corps et de l'esprit, il se mina, et, pour comble
d'amertume, il ne vit un retour glorieux et certain que pour le
regretter.

On a vu qu'il fut choisi pour commander en chef toutes les diverses
troupes de la ligue d'Italie. Ce projet, qui ne fut jamais bien cimenté
ici, n'y subsista pas même longtemps en idée. Chamillart, qui, trop
gouverné, trop entêté avec des lumières trop courtes, avait le coeur
droit et français, allait toujours au bien autant qu'il le voyait,
sentait le désordre des affaires, les besoins pressants de la Flandre,
et se servit de ce premier retour forcé vers le prince de Conti sur
l'Italie, pour porter M\textsuperscript{me} de Maintenon et le roi par
elle à sentir la nécessité de relever l'état si fâcheux de cette
frontière et de l'armée qui la défendait, par ce même prince dont la
naissance même cédait à la réputation. Il l'emporta enfin, et il eut la
permission de l'avertir qu'il était choisi pour commander l'armée de
Flandre.

Conti en tressaillit de joie\,; il n'avait jamais trop compté sur
l'exécution de la ligue d'Italie, il en avait vu le projet s'évanouir
peu à peu. Il ne comptait plus d'être de rien, il se laissa donc aller
aux plus agréables espérances. Mais il n'était plus temps\,: sa santé
était désespérée\,; il le sentit bientôt\,; et ce tardif retour vers lui
ne servit qu'à lui faire regretter la vie davantage. Il périt lentement
dans les regrets d'avoir été conduit à la mort par la disgrâce, et de ne
pouvoir être ramené à la vie par ce retour inespéré du roi et par
l'ouverture d'une brillante carrière.

Il avait été, contre l'ordinaire de ceux de son rang, extrêmement bien
élevé, il était fort instruit. Les désordres de sa vie n'avaient fait
qu'offusquer ses connaissances sans les éteindre\,; il n'avait pas
laissé même de lire souvent de quoi les réveiller.

Il choisit le P. de La Tour, général de l'Oratoire, pour le préparer et
lui aider à bien mourir. Il tenait tant à la vie et venait encore d'y
être si fortement rattaché, qu'il eut besoin du plus grand courage\,;
trois mois durant, la foule remplit toute sa maison, et celle du peuple
la place qui est devant. Les églises retentissaient des voeux de tous,
des plus obscurs comme des plus connus, et il est arrivé plusieurs fois
aux gens des princesses sa femme et ses filles d'aller d'église en
église de leur part, pour faire dire des messes, et de les trouver
toutes retenues pour lui. Rien de si flatteur n'est arrivé à personne\,:
à la cour, à la ville, on s'informait sans cesse de sa santé. Les
passants s'en demandaient dans les rues. Ils étaient arrêtés, aux portes
et aux boutiques, où on en demandait à tous venants.

Un mieux fit plutôt respirer que rendre l'espérance tandis qu'il dura,
on l'amusa de toutes les curiosités qu'on put\,; il laissait faire, mais
il ne cessait pas de voir le P. de La Tour et de penser à lui. Mgr le
duc de Bourgogne l'alla voir et le vit seul longtemps. Il y fut fort
sensible. Cependant le mal redoubla et devint pressant. Il reçut plus
d'une fois les sacrements avec les plus grands sentiments.

Il arriva que Monseigneur, allant à l'Opéra, passa d'un côté de la
rivière le long du Louvre, en même temps que le saint sacrement était
porté, vis-à-vis, sur l'autre quai, au prince de Conti.
M\textsuperscript{me} la duchesse de Bourgogne sentit le contraste\,:
elle en fut outrée, et, en entrant dans la loge, le dit à la duchesse du
Lude. Paris et la cour en furent indignés. M\textsuperscript{lle} de
Melun, que M\textsuperscript{me} la princesse de Conti d'abord, puis
M\textsuperscript{me} la Duchesse avaient mise dans la familiarité de
Monseigneur, aidée de M\textsuperscript{me} d'Espinoy, sa belle-soeur,
fut la seule qui osa lui rendre le service de lui apprendre le mauvais
effet d'un Opéra si déplacé, et de lui conseiller d'en réparer le
scandale par une visite à ce prince, chez qui il n'avait pas encore
imaginé d'aller. Il la crut, sa visite fut courte.

Elle fut suivie d'une autre de Mgrs ses fils. M\textsuperscript{me} la
Princesse y passait les nuits depuis longtemps. M. le Prince n'était pas
en état de le voir\,; M. le Duc garda quelque sorte de bienséance,
surtout les derniers jours\,; M. du Maine fort peu\,; M. le prince de
Conti avait toujours vu quelques amis, et les soirs, touché de
l'affection publique, se faisait fendre compte de tout ce qui était
venu.

Sur la fin, il ne voulut plus voir personne, même les princesses, et ne
souffrit que le plus étroit nécessaire pour son service, le P. de La
Tour, M. Fleury, qui avait été son précepteur, depuis sous-précepteur
des enfants de France, qui s'est immortalisé par son admirable
\emph{Histoire ecclésiastique}, et deux ou trois autres gens de bien. Il
conserva toute sa présence d'esprit jusqu'au dernier moment, et en
profita. Il mourut au milieu d'eux, dans son fauteuil, dans les plus
grands sentiments de piété, dont j'ai ouï raconter au P. de La Tour des
choses admirables.

Les regrets en furent amers et universels. Sa mémoire est encore chère.
Mais disons tout\,: peut-être gagna-t-il par sa disgrâce. La fermeté de
l'esprit, cédait en lui à celle du coeur\,; il fut très grand par
l'espérance\,; peut-être eût-il été timide à la tête d'une armée, plus
apparemment encore dans le conseil du roi, s'il y fût entré.

Le roi se sentit fort soulagé, M\textsuperscript{me} de Maintenon aussi,
M. le Duc infiniment davantage\,; pour. M. du Maine, ce fut une
délivrance, et pour M. de Vendôme, un soulagement à l'état où il
commençait à s'apercevoir que sa chute était possible\,; Monseigneur
apprit sa mort à Meudon, partant pour la chasse. Il ne parut pas en lui
la moindre altération.

Son fils, qui avait déjà une pension du roi de quarante mille livres, en
eut une augmentation de trente mille livres, et M\textsuperscript{me} la
princesse de Conti en eut une de soixante mille livres. Le testament
parut fort sage\,; le domestique médiocrement récompensé. Ces pensions
furent données le lendemain de la mort.

Le surlendemain le roi alla chez M\textsuperscript{me} la princesse de
Conti et chez M\textsuperscript{me} du Maine, toutes deux belles-soeurs,
et M\textsuperscript{me} la duchesse de Bourgogne ensuite, et prit le
deuil en noir le jour suivant pour quinze jours. Il envoya Seignelay,
maître de sa garde-robe, faire les compliments de sa part à l'hôtel de
Conti, et à M. le Prince et à M\textsuperscript{me} la Princesse. M. le
Prince, depuis longtemps malade et renfermé dans sa chambre, reçut le
message\,; il chargea Seignelay de son très humble remercîment, et
surtout de dire au roi de sa part qu'en tout temps il aurait fait une
grande perte, que lui-même en tout temps en aurait été fort touché, mais
qu'en ce temps-ci il l'était doublement, où ce prince eût été d'une si
grande ressource s'il eût plu à Sa Majesté de se servir de lui\,;
liberté fort nouvelle pour M. le Prince, si mesuré courtisan. Il ne
l'eût pas apparemment prise, s'il n'avait pas été instruit de ce qui
s'était passé là-dessus.

M. le prince de Conti avait choisi sa sépulture à Saint-André des Arcs,
auprès de sa vertueuse mère, pour laquelle il avait toujours conservé
beaucoup de respect et de tendresse. Il avait aussi défendu toute la
pompe dont il serait possible de se passer. Je me doutai que l'orgueil
de M. le Duc ne se renfermerait pas dans des bornes si étroites\,; je
priai donc Desgranges, maître des cérémonies, Dreux, grand maître, étant
absent, de faire en sorte que je ne fusse de rien de tout ce qui se
ferait en cette occasion\,: je ne me trompai pas.

M. le duc obtint l'eau bénite en la forme réservée au seul premier
prince du sang, qui l'est aussi pour ce qui est au-dessus et non pour
aucun autre prince du sang\,: ainsi le mercredi 27 février, M. le duc
d'Enghien, vêtu en pointe avec le bonnet carré nommé pour représenter la
personne du roi, et le duc de La Trémoille, nommé par le roi comme duc,
et averti de sa part par Desgranges pour accompagner le représentant, se
rendirent, chacun de leur côté, dans la grande cour des Tuileries, où
ils trouvèrent un carrosse du roi, de ses pages et de ses valets de
pied, douze gardes du corps et quelques-uns, des Cent-Suisses avec
quelques-uns de leurs officiers. M. de La Trémoille, en long manteau, se
mit sur le derrière du carrosse du roi, à côté du prince représentant\,;
Desgranges sur le devant, servant en l'absence du grand maître des
cérémonies, les pages du roi montés devant et derrière le carrosse, qui
n'était point drapé et seulement à deux chevaux\,; environné des Suisses
à pied avec leurs hallebardes, et des valets de pied du roi, aussi à
pied aux portières, suivi du carrosse du duc d'Enghien, son gouverneur
et ses gentilshommes dedans, et de celui du duc de La Trémoille avec les
siens. Le marquis d'Hautefort, en manteau long, nommé par le roi pour
porter la queue du prince représentant, était aussi dans le carrosse du
roi sur le devant\,; les gardes du corps à cheval marchaient
immédiatement devant et derrière. Ils arrivèrent ainsi à l'hôtel de
Conti, tout tendu de deuil. M. le Duc et le nouveau prince de Conti,
accompagnés des ducs de Luxembourg et de Duras, qu'ils avaient invités
comme parents, tous quatre en manteaux longs, tous quatre de front, tous
quatre leur queue portée chacun par un gentilhomme en long manteau,
reçurent le prince représentant à sa portière, lequel reçut les mêmes
honneurs qu'on eût faits à la personne même du roi\,; la queue du
manteau du duc de La Trémoille toujours portée par un gentilhomme en
manteau long. L'abbé de Maulevrier, aumônier du roi, en rochet, et lors
en quartier, présenta le goupillon au prince représentant\,; un autre,
mais le même, le présenta à M. le Duc, à M. le prince de Conti, et aux
ducs de La Trémoille, de Luxembourg et de Duras. Les prières achevées,
la conduite se fit comme la réception, le retour comme on était venu. M.
de La Trémoille et M. d'Hautefort prirent congé de M. le duc d'Enghien
dans la cour des Tuileries, d'où chacun reprit son carrosse et s'en alla
chez soi. J'oublie de dire que, pendant cette eau bénite, d'autres
gardes du corps et Cent-Suisses avec leurs officiers gardèrent et
garnirent l'hôtel de Conti, comme il se pratique dans les maisons où le
roi va.

Le même jour huit archevêques ou évêques en rochet et camail, députés
par tous les prélats qui se trouvèrent à Paris, allèrent donner de l'eau
bénite après que tous les gardes furent retirés. Le lendemain M. le Duc,
M. le duc d'Enghien, M. le duc du Maine et M. le comte de Toulouse
allèrent donner l'eau bénite, reçus par M. le prince de Conti, tous en
long manteau, et quelques heures après le parlement y fut aussi et les
autres cours supérieures. M. le duc d'Orléans et les fils de France n'y
furent point, comme n'étant pas de même rang\,; mais le cardinal de
Noailles y fut à la tête du chapitre de Notre-Dame.

Deux jours après cette eau bénite, je sus qu'il s'était débité que
j'avais trouvé mauvais de n'avoir pas été nommé au lieu du duc de La
Trémoille, et dit qu'il y ferait quelque sottise faute de savoir\,; que
ce propos avait été tenu chez M. de Bouillon, à Versailles, en présence
de M. de La Trémoille, qui sourit et s'en moqua, et qui, sur ce qu'on le
lui soutint, tira quatre pistoles de sa poche, et fit taire en offrant
le pari que personne ne voulut accepter\,; il leur demanda si eux-mêmes
me l'avaient ouï dire et les confondit\,: cette justice et cette marque
d'amitié me fut très sensible. J'étais en effet très éloigné de
soupçonner M. de La Trémoille de se mal conduire, plus encore de le
dire, et hors de portée de trouver mauvais que mon ancien m'eût été
préféré, quand même j'aurais eu envie de faire cette fonction, et je me
sus bon gré de ma précaution avec Desgranges, que je répandis et fis
répandre par lui. Je ne pus savoir qui l'avait dit, mais en général je
m'expliquai durement sur quiconque\,; personne n'osa s'en fâcher.

Le corps de M. le prince de Conti demeura quelques jours exposé chez
lui, en attendant que tout fût prêt à Saint-André des Arcs. M. le Duc,
ardent à empiéter d'adresse où il ne pouvait de vive force, fit
cependant insinuer par ses principaux domestiques et par ceux de l'hôtel
de Conti, aux amis du feu prince et aux siens qui étaient ducs, que bien
des gens allaient donner de l'eau bénite et prier Dieu quelque temps
près du corps\,; que cette piété était une marque d'amitié qu'on
s'étonnait qu'ils n'eussent pas encore rendue et que le manteau long
était l'habit le plus décent pour ce devoir funèbre. Rien de si aisé à
attraper que les ducs, ni de si hors de garde en tout et pour tout,
malgré les expériences. Le duc de Sully et le duc de Villeroy donnèrent
dans ce panneau, le maréchal de Choiseul aussi et d'autres.
Saintrailles, premier écuyer de M. le Duc, homme fort du grand monde et
ami du duc de Villeroy, l'avait tonnelé, et allégué l'exemple du duc de
Sully. Il me le conta, et que son père, piqué au vif, ne verrait jamais
Saintrailles. La juste confiance en la facilité des ducs avait fait
commencer par eux, pour venir après, aux princes étrangers sur cet
exemple\,; mais le bruit que fit le maréchal de Villeroy éventa la mèche
et arrêta tout tout court. M. le Duc n'osa se fâcher, parce qu'au
murmure se joignit le ridicule d'avoir tenté par là de vouloir faire
garder le corps de M. le prince de Conti.

Il y avait un temps infini qu'il n'était mort de prince du sang. Le
dernier prince de Conti était mort à Fontainebleau, de la petite vérole
qu'il avait gagnée de M\textsuperscript{me} sa femme en 1685, 9
novembre, à vingt-cinq ans, sans postérité\,; M. son père, à Pézenas, en
1666, 11 février, à trente-sept ans\,; M. le Prince, 11 décembre 1686, à
soixante-cinq ans, à Fontainebleau, où il était allé de Chantilly sur la
petite vérole de M\textsuperscript{me} la duchesse de Bourbon. Cette
garde en effet avait été l'objet de M. le Duc. Il se souvenait que la
reine, les filles et les petites-filles de France étaient gardées par
des duchesses et des princesses étrangères alternativement, et par des
dames de qualité avec les unes et les autres ou toutes se relevaient\,;
il se souvenait aussi qu'à la mort de M\textsuperscript{lle} de Condé sa
soeur, en 1700, ils avaient essayé de la faire garder par des dames non
titrées dont presque aucunes n'avaient voulu tâter, et qu'ils n'avaient
osé le proposer aux titrées\,; mais il ignorait ou il avait oublié que
cette garde n'est que pour les princesses, et non pour les princes, pas
même pour les rois, près du corps desquels il ne reste que leurs
principaux officiers. On se moqua, donc du peu de dupes qui s'étaient
laissé persuader, qui crièrent fort haut, et la chose en demeura là.

Mais M. le Duc n'en fut pas moins ardent à tenter des entreprises. Il
imagina de faire porter le corps en carrosse là-dessus force
discussions. Il n'y eut pas moyen d'y réussir\,; il s'en tira par la
défense que le prince défunt avait faite de toutes les cérémonies qui se
pouvaient supprimer. C'était à quoi il aurait dû penser plus tôt.

Lorsqu'il vit qu'il fallait se réduire à l'usage ordinaire, il proposa
nettement aux ducs qui seraient invités au convoi d'y être en manteau
long. MM\hspace{0pt}. de Luxembourg et de La Rocheguyon\,; amis intimes
de feu M. le prince de Conti, et fort bien avec les princes du sang, le
refusèrent encore plus net, dont M. le Duc s'aigrit jusqu'à s'emporter
avec menaces. Dépité de la sorte, et déjà un peu brouillé avec
M\textsuperscript{me} sa soeur, il prit prétexte de se dispenser du
convoi sur ce qu'un rhume empêchait M. le prince de Conti de s'y
trouver, et il envoya M. le duc d'Enghien en long manteau. Personne ne
fut invité. Qui voulut, ducs et autres, se trouvèrent à l'arrivée du
corps à Saint-André, mais en deuil, sans manteau. Achevons tout de suite
cette triste matière pour n'avoir pas à y revenir.

On fit dans la même église un superbe service, où les évêques et les
parents seuls furent invités par la famille, mais où tout abonda. Un
prélat officia, le P. Massillon de l'Oratoire, depuis l'évêque de
Clermont, fit une admirable oraison funèbre. M. le Duc, M. le duc
d'Enghien et M. le prince de Conti firent le deuil. Les évêques se
formalisèrent de n'avoir point de fauteuils. Ils se fondaient sur ce
qu'ils étaient dans l'église, ils ne se voulaient point souvenir des
exemples de la même prétention dans les derniers temps qui n'a pas été
admise, si ce n'est pour les évêques-pairs, mais hors de rang d'avec le
clergé et à part. Néanmoins après quelques mouvements les évêques
demeurèrent sur leurs formes\footnote{Stalles de choeur.}. La règle est
constante que personne en ces cérémonies n'a que le même traitement
qu'il aurait chez le prince dont on fait les obsèques s'il était vivant.

Par cela même les ducs y devaient avoir des fauteuils, en tout pareils à
ceux des princes du sang. M. le Duc, toujours entreprenant, les avait
tous supprimés. Il ne s'en trouva que trois pour les trois princes du
deuil, et une forme joignant le dernier fauteuil et plusieurs autres
formes de suite. Les premiers arrivés s'en aperçurent et s'en
plaignirent tout haut. M. le Duc fit la sourde oreille. Bientôt après
MM. de Luxembourg, La Meilleraye et La Rocheguyon arrivèrent, ils lui en
parlèrent\,; il s'excusa sur ce qu'il n'y avait point de fauteuils et
qu'il ne savait où en prendre. Sur quoi ces trois ducs lui déclarèrent
qu'ils allaient donc sortir avec tous les autres. Cette prompte
résolution étonna M. le Duc. Il ne s'y était pas attendu. Il voulait
faire un exemple par adresse, mais de refuser les fauteuils, il le
sentit insoutenable\,; il protesta qu'il n'avait jamais imaginé de ne
leur pas donner des fauteuils, qu'il ne savait comment faire\,; puis
voyant que ces messieurs lui faisaient déjà la révérence pour se
retirer, il les arrêta, et dit qu'il fallait pourtant trouver moyen de
les satisfaire. Alors la ruse parut tout entière. Sur-le-champ il vint
des fauteuils par derrière. M. le Duc fit excuse de ce qu'il ne s'en
trouvait pas assez pour tous les ducs, et par composition on en mit un
joignant celui de M. le prince de Conti, tout pareil au sien, et sur
même ligne, et quatre ou cinq autres de suite, puis tant qu'il y en eut
d'espace en espace, et un pour le dernier duc, afin que tout ce qui
était entre deux fût réputé fauteuil et tous les ducs y être assis. On
vit ainsi qu'il y en avait en réserve pour une dernière nécessité, dont
outre l'entreprise manquée M. le Duc fut outré.

Qui que ce soit n'eut là de manteaux longs que les princes du deuil et
leur maison\,; aussi n'osèrent-ils le proposer à personne après ce qui
s'était passé là-dessus lors du convoi. Les princes étrangers se tinrent
adroitement à l'écart pour ne rien perdre et ne se point commettre. Je
me suis étendu sur ces obsèques pour faire voir que quelque grand solide
et juste que soit le rang des princes du sang, ils en veulent encore
davantage, et n'épargnent ni ruses ni violences pour usurper, en quoi
ils ont réussi, et depuis sans cesse à se faire des droits de leurs
usurpations.

\hypertarget{chapitre-vii.}{%
\chapter{CHAPITRE VII.}\label{chapitre-vii.}}

1709

~

{\textsc{Rencontre en même pensée fort singulière entre le duc de
Chevreuse et moi\,; origine des conseils mal imités à la mort de Louis
XIV.}} {\textsc{- Péril secret du duc de Beauvilliers.}} {\textsc{-
Harcourt manque à coup près d'entrer au conseil.}} {\textsc{- Mort et
deuil d'un enfant de l'électeur de Bavière.}} {\textsc{- Mariage du
marquis de Nesle avec la fille du duc Mazarin.}} {\textsc{- Mariage du
marquis d'Ancenis avec la fille de Georges d'Entragues.}} {\textsc{-
Retour de Flandre du maréchal de Boufflers, hors d'état de servir.}}
{\textsc{- Villars, sous Monseigneur, général en Flandre.}} {\textsc{-
Harcourt, sous Mgr le duc de Bourgogne, général sur le Rhin.}}
{\textsc{- Berwick en Dauphiné\,; le duc de Noailles en Roussillon\,; M.
le duc d'Orléans en Espagne.}} {\textsc{- Les princes ne sortent point
de la cour.}} {\textsc{- Comte d'Évreux ne sert plus, que
M\textsuperscript{me} la duchesse de Bourgogne empêche de se rapprocher
de Mgr le duc de Bourgogne.}} {\textsc{- Roucy admis, La Feuillade
refusé de suivre Monseigneur {[}comme{]} volontaires.}} {\textsc{-
Rouillé en Hollande.}} {\textsc{- Caractère de Rouillé.}} {\textsc{-
Conduite de Chamillart à l'égard des autres ministres, dont il emblait
le ministère.}} {\textsc{- Il s'en désiste à l'égard de Torcy, et en
signe un écrit.}} {\textsc{- Affaire fort poussée entre Chamillart et
Desmarets, dont le dernier eut l'avantage.}}

~

Cependant tout périssait peu à peu ou plutôt à vue d'oeil\,; le royaume
entièrement épuisé, les troupes point payées, et rebutées d'être
toujours mal conduites, et par conséquent toujours malheureuses\,; les
finances sans ressource, nulle dans la capacité des généraux ni des
ministres\,; aucun choix que par goût et par intrigue\,; rien de puni,
rien d'examiné ni de pesé\,; impuissance égale de soutenir la guerre et
de parvenir à la paix\,; tout en silence, en souffrance\,; qui que ce
soit qui osât porter la main à cette arche chancelante et prête à
tomber.

Je m'étais souvent échappé sur tous ces désordres entre les ducs de
Chevreuse et de Beauvilliers, et encore plus sur leurs causes. Leur
prudence, leur piété rabattait mes plaintes sans pourtant les détruire.
Accoutumés au genre de gouvernement qu'ils avaient toujours vu, et
auquel ils avaient part, je mettais des bornes à ma confiance sur les
remèdes que je pensais depuis longtemps. J'en étais si rempli qu'il y
avait des années que je les avoir jetés sur le papier, plutôt pour mon
soulagement et pour me prouver à moi-même leur utilité et leur
possibilité, que dans l'espérance qu'il en pût jamais rien réussir. Ils
n'avaient jamais vu le jour, et j e ne m'en étais laissé entendre à
personne, lorsqu'une après-dînée, le duc de Chevreuse vint chez moi dans
l'appartement du feu M. le maréchal de Larges que j'occupais, et monta
tout de suite dans un petit entresol à cheminée dont je faisais mon
cabinet, et qu'il connaissait fort. Il était plein de la situation
présente, il m'en parla avec amertume, il me proposa de chercher des
remèdes.

À mon tour je l'en pressai, je lui demandai s'il en croyait de
possibles, non que je tinsse les choses désespérées, mais bien les
obstacles invincibles. C'était un homme qui espérait toujours et qui
voulait toujours marcher en conséquence, je dis marcher, mais à part
soi. Cette manière satisfaisait son amour du raisonnement, et ne faisait
pas violence à sa prudence si à sa politique\,: c'était cela même qui me
dégoûtait. Je haïssais les châteaux en Espagne, et les raisonnements qui
ne pouvaient aboutir à rien. Je voyais manifestement l'impossibilité
d'un gouvernement sage et heureux tant que le système présent
durerait\,; je sentais toute celle d'aucun changement là-dessus, par
l'habitude du roi et l'opinion qu'il avait prise que la puissance des
secrétaires d'État était la sienne, ainsi que du contrôleur général, par
conséquent impossibilité de la borner, ni de la partager, ni de lui
persuader qu'il pût sûrement admettre dans son conseil personne qui ne
fît preuves complètes de roture\footnote{Louis XIV dit lui-même dans ses
  Mémoires (t. Ier, p.~32, 33)\,: «\,Il n'était pas de mon intérêt de
  prendre}, et de nouveauté même, excepté le seul chef du conseil des
finances, parce que rien ne dépendait de lui. Ce que j'avais donc fait
là-dessus autrefois, pour ma satisfaction seule, je l'avais condamné aux
ténèbres, et regardé comme la république de Platon.

Ma surprise fut donc grande, lorsque M. de Chevreuse, s'ouvrant de plus
en plus avec moi, se mit à déployer les mêmes idées que j'avais eues. Il
aimait â parler et il parlait bien, avec justesse, précision et choix.
On aimait aussi fort à l'entendre. Je l'écoutais donc avec toute
l'attention de voir en lui mes pensées, mon dessein, mon projet, dont je
l'avais toujours cru lui et M. de Beauvilliers si éloignés, que je
m'étais bien gardé de m'en expliquer avec eux quelle que fût ma
confiance en eux sans réserve, et la leur en moi, parce que je comptais
sur l'inutilité de heurter de front leur habitude tournée en persuasion,
et de plus avec l'impossibilité de s'en jamais pouvoir promettre quoi
que ce fût avec le roi. M. de Chevreuse parla longtemps, développa son
projet, et me récita tout le mien à si peu de choses près, et si peu
considérables que j'en demeurai stupéfait.

À la fin, il s'aperçut de mon extrême surprise\,; il voulut me faire
parler à mon tour sur ce qu'il proposait\,; et je ne répondais que
monosyllabes, absorbé que j'étais dans la singularité que j'éprouvais. À
son tour la surprise le saisit\,; il était accoutumé à ma franchise, à
m'entendre répandre avec lui, et se voir, si je l'ose dire avec tant de
différences entre nous, louer, approuver ou disputer et reprendre, car
lis deux beaux-frères me souffraient tout cela. Il me voyait morne,
silencieux, concentré. «\,Mais parlez-moi donc, me dit-il enfin\,; à qui
en avez-vous donc aujourd'hui\,? franchement, est-ce que je dis des
sottises\,?» Alors je n'y pus plus tenir, et sans répondre une parole je
tire une clef de ma poche, je me lève, j'ouvre une armoire qui était
derrière moi, j'en tire trois fort petits cahiers écrits de ma main, et
en les lui présentant\,: «\,Tenez, monsieur, lui dis-je, voyez d'où
vient ma surprise et mon silence\,;» il lut, puis parcourut et trouva
tout son plan\,; jamais je ne vis homme si étonné, ou plutôt jamais deux
hommes ne le furent l'un après l'autre davantage.

Il vit toute la substance de la forme de gouvernement qu'il venait de me
proposer\,; il vit les places des conseils remplies de noms dont
quelques-uns étaient morts depuis\,; il vit toute l'harmonie de leurs
différents ressorts, et celle des ministres de chacun des conseils\,; il
vit jusqu'au détail des appointements avec la comparaison de ceux des
ministres effectifs du roi. J'avais formé les conseils de ceux que j'y
avais cru les plus propres, pour me répondre à moi-même à l'objection
des sujets, et j'avais mis les appointements pour me répondre à celle de
la dépense, et la comparer à celle du roi pour le sien. Ces précautions
ravirent M. de Chevreuse. Les choix lui plurent presque tous, et la
balance aussi des appointements.

Lui et moi fûmes longtemps à nous remettre, de notre surprise
réciproque\,; après nous raisonnâmes, et plus nous raisonnâmes, plus
nous nous trouvâmes parfaitement d'accord, si ce n'est que j'avais plus
approfondi et dressé plus exactement toutes les parties du même plan. Il
me conjura de le lui prêter pour quelques jours\,; il voulait l'examiner
à son loisir. Huit ou dix jours après, il me le rendit. Lui et M. de
Beauvilliers en avaient fort raisonné ensemble\,; ils n'y trouvèrent
presque rien à changer, et encore des bagatelles, mais la difficulté
était l'exécution. Ils la jugèrent impossible avec le roi, ainsi que
j'avais toujours cru. Ils me prièrent instamment de le conserver avec
soin, pour des temps auxquels on pourrait s'en servir, qui étaient ceux
de Mgr le duc de Bourgogne.

On verra dans la suite que ce projet fut la source d'où sortirent les
conseils, mais très informes et mal digérés, lors de la mort du roi,
comme ayant été trouvés dans la cassette de Mgr le duc de Bourgogne à sa
mort. Toutes ces choses s'expliqueront en leur temps. On trouvera parmi
les Pièces ces mêmes conseils tels que je les montrai à M. de Chevreuse,
que M. de Beauvilliers vit avec lui, car parler à l'un c'était parler à
l'autre, et qui avec le temps allèrent jusqu'à Mgr le duc de Bourgogne.
S'il eût été question de les exécuter j'y aurais changé différentes
choses, mais rien pour le fond et l'essentiel, et cette exécution aurait
eu lieu, si ce prince avait régné, ainsi que plusieurs autres.

Tandis que nous raisonnions de la sorte, le duc de Beauvilliers courait
un grand et imminent danger. Il n'en avait pas le plus léger soupçon. Ce
fut merveille comme je l'appris et comment il fut paré si à propos qu'il
n'y avait pas une heure à perdre.

M\textsuperscript{me} de Maintenon s'était enfin vengée d'avoir vu son
crédit obscurci, et le duc de Vendôme triompher d'elle, en triomphant de
Mgr le duc de Bourgogne, qu'elle avait entrepris vainement alors de
soutenir. Peu à peu elle avait repris le dessus\,; elle avait fait
reprendre M\textsuperscript{me} la duchesse de Bourgogne, et par
conséquent Mgr le duc de Bourgogne. Elle avait éreinté Vendôme\,; elle
avait fait qu'il ne servirait plus, et l'avait fait déclarer. Dès lors
tous ses particuliers avec le duc de Beauvilliers avaient cessé. La
matière était tarie\,: il n'y avait plus à se consulter et à prendre des
mesures de concert.

J'ai remarqué que ce rapprochement n'avait jamais été que sur ce seul
point et par la seule nécessité\,; que la rancune subsistait dans le
coeur de la fée, qui ne pouvait pardonner au duc de s'être maintenu
malgré elle, et qu'elle voulut toujours depuis regarder en ennemi,
toujours attentive aux moyens de le perdre. J'ai aussi remarqué que,
dans ces mêmes temps, Harcourt, un peu refroidi avec elle, était revenu
de Normandie à Fontainebleau, et avait trouvé les moyens que j'ai
expliqués de se raccrocher avec elle plus confidemment que jamais\,: il
sut en profiter.

M\textsuperscript{me} de Maintenon reprit ses anciennes idées\,: elle
travailla de nouveau à faire entrer Harcourt dans le conseil. C'était y
mettre sa créature, et elle n'y en avait plus depuis qu'elle regardait
Chamillart comme un homme qui lui avait manqué en tout par le mariage de
son fils, par le retour de Desmarets, par sa partialité pour Vendôme,
enfin par ce projet si avancé de la reprise de Lille par le roi en
personne et sans elle. Elle le voulait perdre, et Harcourt dans le
conseil serait bien plus fort à l'y servir. Elle voulait se défaire du
duc de Beauvilliers, et Harcourt dans le conseil n'avait qu'à lui
succéder de plain-pied, et avait double intérêt à le détruire.
M\textsuperscript{me} de Maintenon n'attendit pas ce secours\,: elle
travailla en même temps à chasser Beauvilliers et à placer Harcourt. Son
labeur fut heureux. Je n'ai pas su si la chute de l'un fut promise, et
je ne veux donner pour certain que ce qui l'est, quoique ce qui arriva
me l'ait fait croire\,; mais l'entrée du conseil pour Harcourt, le roi
en donna sa parole\,: ce ne fut pas pans peine. La même raison de
l'exemple et des concurrents qui l'avait déjà empêché une fois s'y
opposait encore celle-ci, quoique avec la considération de M. de La
Rochefoucauld de moins, de la situation duquel je parlerai bientôt.

La parole donnée, ou plutôt arrachée, le comment embarrassa le roi, qui,
par la même raison des concurrents, ne voulut pas faire Harcourt
ministre en le déclarant, et aima mieux le contour et le masque du
hasard. Pour cela, il fut convenu que, pendant le premier conseil
d'État, Harcourt, averti par M\textsuperscript{me} de Maintenon, se
trouverait comme fortuitement dans les antichambres du roi\,; qu'à
propos des choses d'Espagne, le roi proposerait de consulter Harcourt,
et tout de suite ferait regarder si par hasard il n'était point quelque
part dans les pièces voisines\,; que, s'y trouvant, il le ferait
appeler\,; qu'il lui dirait tout haut un mot sur ce qui le faisait
mander, et tout de suite lui commanderait de s'asseoir, ce qui était le
faire ministre d'État, le retenir en ce conseil et l'y faire toujours
entrer après.

On a vu, à l'occasion de la disgrâce du maréchal de Villeroy, en quelle
intimé liaison j'étais avec son fils et sa belle-fille. On a vu ailleurs
sur quel tour d'intimité le duc de Villeroy était avec
M\textsuperscript{me} de Caylus, de l'exil de laquelle il avait été
cause, son retour, l'affection tendre pour elle de M\textsuperscript{me}
de Maintenon, et la liaison intime d'Harcourt avec M\textsuperscript{me}
de Caylus, sa cousine germaine, et qui entra et servit en tant de choses
Harcourt auprès de M\textsuperscript{me} de Maintenon. Le secret de
l'entrée d'Harcourt au conseil était extrême, et infiniment recommandé
par le roi. Soit imprudence, confiance, jalousie pour son père,
quoiqu'en disgrâce, quoi que ce fût, je le sus sur le point de
l'exécution, et la manière dont elle se devait faire. J'ouïs en même
temps quelques mots louches sur le duc de Beauvilliers, dont le duc de
Villeroy n'ignorait pas avec toute la cour que je ne fusse comme le
fils.

Je ne perdis par un instant, les moments étaient chers. Je quittai le
duc et la duchesse de Villeroy le plus tôt qu'il me fut possible, sans
leur rien montrer. Je gagnai ma chambré, et sur-le-champ j'envoyai un
ancien valet de chambre, que tout le monde me connaissait et qui était
entendu, chercher M. de Beauvilliers partout où il pourrait être (et il
n'allait guère), le prier de venir sur-le-champ chez moi, et que je lui
dirais ce qui m'empêchait d'aller chez lui\,: c'est que je ne voulais
pas y aller au sortir de chez ceux d'avec qui je sortais, et que, sans
grande précaution, tout se sait dans les cours.

En moins de demi-heure M. de Beauvilliers arriva, assez inquiet de mon
message. Je lui demandai s'il ne savait rien, je le tournai, moins pour
le pomper, car je n'en avais pas besoin avec lui, que pour lui faire
honte de son ignorance, qui si souvent l'avait jeté dans des panneaux et
des périls, et pour le persuader mieux après de ce que je voulais qu'il
fît. Quand je l'eus bien promené sur son ignorance, je lui appris ce que
je venais de savoir.

Mon homme fut interdit. Il ne s'attendait à rien moins\,; je n'eus pas
peine à lui faire entendre que, quand bien même son expulsion ne serait
pas résolue, l'intrusion d'Harcourt en était le cousin germain, et le
préparatif certain, qui, appuyé de M\textsuperscript{me} de Maintenon,
sans mesure et mal avec Torcy, lié au chancelier, dominerait sur les
choses de la guerre, sur celles d'Espagne, et de là sur les autres
affaires étrangères, et sur celles des finances avec la grâce de la
nouveauté, l'audace qui lui était naturelle\,; et le poids que lui
donnaient sa naissance, ses établissements, et les emplois par lesquels
il avait passé.

Après force raisonnements il fallut venir au remède, et le temps
pressait, à vingt-quatre heures près au moins. Il n'en trouvait qu'à
attendre, à se résigner, à se tenir en la main de Dieu, à se conduire au
jour le jour, puisqu'il n'y avait pas de temps assez pour parer cette
entrée, qu'il conçut pourtant fort bien être sa sortie, ou en être au
moins le signal. Il m'avoua que depuis quelques jours il trouvait le roi
froid et embarrassé avec lui, à quoi jusqu'alors il m'avoua aussi qu'il
avait donné peu d'attention, mais dont alors la cause lui fut claire.

Je pris la liberté de le gronder de sa profonde ignorance de tout ce qui
se passait à la cour, et de cette charité malentendue qui tenait ses
yeux et ses oreilles de si court, et lui si renfermé dans une bouteille.
Je lui rappelai ce que je lui avais dit et pronostiqué, dans les bas des
jardins de Marly, sur la campagne de Mgr le duc de Bourgogne, la colère
où il s'en était mis, et les événements si conformes à mes pronostics.
Enfin, j'osai lui dire qu'il s'était mis en tel état avec le roi, par ne
vouloir s'avantager de rien, qu'il ne tenait plus à lui que par
l'habitude de ses entrées comme un garçon bleu, mais que, puisqu'il y
tenait encore par là, il fallait du moins qu'il en tirât les avantages
dans la situation pressante où il se trouvait. Il me laissa tout dire,
ne se fâcha point, rêva un peu quand j'eus fini, puis sourit et me dit
avec confiance\,: «\,Eh bien\,! que pensez-vous donc qu'il y eût à
faire\,?» C'était où je le voulais. Alors je lui répondis que je ne
voyais qu'une chose unique à faire, laquelle était entre ses mains, et
du succès de laquelle je répondrais bien, au moins pour lui, s'il
voulait prendre sur lui de la bien faire, si même elle n'empêchait
Harcourt d'entrer au conseil.

Alors je lui proposai d'user de la commodité de ses entrées, de prendre
le roi, le lendemain matin, seul dans son cabinet, et là de lui dire
qu'il était informé que M. d'Harcourt devait entrer au conseil, et la
façon dont il y devait être appelé\,; qu'il n'entrait point dans les
raisons du roi là-dessus\,; qu'il n'en craignait que son importunité par
le mépris public que M. d'Harcourt faisait de ses ministres, qui n'était
pas ignoré de Sa Majesté, l'ascendant qu'il voudrait prendre sur tous et
qu'aucun n'aimerait à endurer, et l'embarras sur les affaires étrangères
par sa rupture particulière avec Torcy\,; qu'il croyait être obligé de
dire cela à Sa Majesté, mais pour son regard à soi avec une entière
indifférence\,; qu'en même temps il n'en pouvait avoir sur une chose
qu'il remarquait depuis quelques jours, et dont il ne pouvait s'empêcher
d'ouvrir son coeur avec toute la soumission, le respect et l'attachement
qu'il avait pour sa personne\,; et là à lui dire ce qu'il remarquait de
lui à son égard\,; de lui parler un peu pathétiquement et dignement,
mais avec un air d'affection\,; puis d'ajouter qu'il ne tenait qu'à son
estime et à ses bonnes grâces, point à aucunes places\,; lui parler
encore avec la même affection et reconnaissance de ce qu'il les -lui
avait toutes données sans qu'il eût jamais songé à pas une\,; qu'il
était également prêt à les lui remettre pour peu qu'il le désirât\,; et
sur, cela triompher de respect, de soumission, de désintéressement,
d'affection et de reconnaissance.

M. de Beauvilliers prit plaisir à m'entendre, il n'eut pas de peine à se
rendre à cet avis. Il m'embrassa étroitement. Il me promit de le suivre
e et de me rendre comment cela se serait passé.

J'allai chez lui sur la fin de la matinée du lendemain, où j'appris de
lui qu'il était parfaitement rassuré sur ses pieds. Il avait parlé de
point en point comme je lui avais dit que je croyais qu'il le devait
faire. Le roi parut étonné, et, à ce qui lui échappa muettement, piqué
du secret de l'entrée d'Harcourt au conseil découvert\,; et si
entièrement, et c'était aussi ce que je m'étais proposé. Il parut fort
attentif à la courte réflexion sur l'effet de cette entrée par rapport
aux ministres, et à l'embarras qui en naîtrait. Il parut embarrassé de
ce que M. de Beauvilliers lui dit sur lui-même\,; puis ouvert
l'interrompant, pour l'assurer de son estime, de sa confiance et de son
amitié. À la proposition de retraite, il s'y opposa, fit beaucoup
d'amitiés à M. de Beauvilliers, lui dit beaucoup de choses obligeantes,
et parut renouer avec lui plus que jamais. Je sus de lui que la suite y
avait depuis toujours répondu. En un mot ce fut un coup de partie. M. de
Beauvilliers m'embrassa encore bien tendrement, à plus d'une reprise. De
savoir si sans cela il était chassé ou non, c'est ce que je n'ai pu
découvrir\,; mais par le peu qui me fut dit, et par le froid et
l'embarras du roi lorsque M. de. Beauvilliers l'aborda, et qui dura
pendant les premiers temps de son discours, et qui de son aveu avait
précédé et qui fut son thème, j'en suis presque persuadé.

Harcourt, sûr de sou fait et contenant à peine sa joie sur le point
immédiat du succès, arriva au rendez-vous. Le temps se prolongea.
Pendant le conseil, il n'y a que des plus subalternes dans ces
appartements du roi, et quelques courtisans qui passent par là, pour
aller d'une aile à l'autre. Chacun de ces subalternes s'empressait de
lui demander ce qu'il voulait, s'il désirait quelque chose, et
l'importunaient étrangement. Il fallait demeurer là, il n'en avait point
de prétexte. Il allait et venait boitant sur son bâton, et ne savait que
répondre, ni aux demeurants, ni aux passants, dont il était remarqué. À
la fin, après une longue attente, fort mal à son aise, il s'en alla
comme il était venu\,; fort inquiet de n'avoir point été appelé. Il le
manda à M\textsuperscript{me} de Maintenon qui à sort tour en fut
d'autant plus en peine que le soir le roi ne lui en dit pas un mot, et
qu'elle aussi n'osa lui en parler. Elle consola Harcourt\,; elle voulut
espérer que l'occasion ne s'était pas trouvée à ce conseil de lui faire
de question sur les affaires d'Espagne, et voulut qu'il se trouvât
encore au même rendez-vous au premier conseil d'État. Harcourt y fit le
même manège, et avec aussi peu de succès. Il s'en alla fort chagrin, et
comprit son affaire rompue.

M\textsuperscript{me} de Maintenon voulut enfin en avoir le coeur net.
Elle avait assez attendu pour ne pas marquer d'impatience\,; elle en
parla au roi, supposant oubli ou faute de matière, et que la chose était
toujours sur le même pied. Le roi, embarrassé, lui répondit qu'il avait
fait des réflexions, qu'Harcourt était mal avec presque tous ses
ministres, qu'il montrait un mépris pour eux qui ferait des querelles
dans le conseil, que ces disputes l'embarrasseraient\,; que, tout bien
considéré, il aimait mieux s'en tenir où il en était, n'avoir point la
bouderie de gens qu'il considérait, et qui seraient piqués de cette
préférence, dès qu'il admettrait quelqu'un de nouveau et de leur sorte
dans le conseil\,; qu'il estimait fort la capacité d'Harcourt, et qu'il
le consulterait en particulier sur les choses dont il voudrait avoir son
avis. Cela fut dit de façon qu'elle ne crut pas avoir à répliquer\,;
elle se tint pour battue, et Harcourt fut au désespoir. Ce coup manqué
pour la deuxième fois, il n'espéra plus y revenir que par des
changements également incertains et éloignés.

J'avais été cependant comme à l'affût de ce qui arriverait de cette
entrée, sans dire mot à personne, et je fus fort aise quand le délai si
long me fit comprendre qu'elle était échouée. Le roi n'en dit pas un mot
à M. de Beauvilliers, mais il était redevenu libre avec lui et à son
ordinaire. Je demandai après doucement au duc de Villeroy à quoi tenait
donc cette entrée, et je sus ce que je viens de raconter, et qu'il n'en,
était plus question. Je ne parus y prendre nulle part. J'étais en mesure
avec Harcourt, qui même m'avait fait des avances à reprises. J'étais
content au dernier point que les choses se fussent aussi heureusement
conduites, mais je ne m'en gaudis qu'entre les ducs de Chevreuse et de
Beauvilliers, qui l'avaient échappé belle.

Monastrol, sans être en grand deuil, donna part au roi de la mort d'un
fils de l'électeur de Bavière, parce qu'en Allemagne on n'en porte aucun
des enfants au-dessous de sept ans comme était ce dernier cadet.
Néanmoins le roi le prit pour quinze jours. Voilà où conduisit le deuil
d'un maillot de M. du Maine\,: à porter le deuil d'un enfant que sa
propre cour ne porte pas, après n'en avoir point porté ici d'aucuns des
enfants de la reine morts avant sept ans.

Le marquis de Nesle épousa la fille unique du duc de Mazarin, qui
n'avait qu'un frère. La comtesse de Mailly avait fort espéré ce mariage
pour sa dernière fille, et y avait fait de son mieux, un peu aidée des
cajoleries de M\textsuperscript{me} de Maintenon\,; mais la vieille
Mailly, qui savait par expérience combien elles étaient vaines, et qui
avait, à force de travaux, fait une très puissante maison, voulut pour
son petit-fils de grandes espérances. Les biens étaient immenses si le
frère venait à manquer, et de plus l'espérance de la dignité de duc et
pair, parce que celle de Mazarin était femelle. La beauté de cette
mariée fit grand bruit dans les suites, et celle des filles quelle
laissa encore plus dans le règne suivant, jusqu'à devoir y tenir quelque
place dans l'histoire.

Le duc de Charost fut attrapé par une M\textsuperscript{me} Martel,
vieille bourgeoise de Paris, qui était un esprit, et qui voyait assez
bonne compagnie. Avec un empire fort ridicule à considérer, elle lui fit
accroire des trésors pour son deuxième fils qui n'avait rien alors, et
qui par l'événement a succédé aux dignités et aux charges de son père.
Je ne dirai pas aux biens pour le peu qu'ils valaient. Bref, Charost se
laissa embarquer, et maria le marquis d'Ancenis à la fille d'Entragues,
qui avait été petit commis, et bien pis auparavant, chez M. de Frémont,
beau-père de M. le maréchal de Lorges, et grand-père de
M\textsuperscript{me} de Saint-Simon, qui lui avait commencé une fortune
qu'il poussa fort loin, et qui lui fit épouser pour rien la fille de
Valencey et d'une soeur du maréchal de Luxembourg et de la duchesse de
Meckelbourg. Charost avait eu le gouvernement de Dourlens de
Baule-Lamet, père de sa seconde femme, dont il ne lui restait point
d'enfants, que le roi voulut bien sur sa démission donner à son fils en
faveur de ce mariage. Il fut récompensé. autant qu'il pouvait l'être par
le mérite, de la personne, sa vertu et sa conduite, qui plut fort dans
sa famille, et qui réussit fort à la cour et dans le monde.

Le maréchal de Boufflers ayant reçu en Flandre, où il était allé tout
préparer pour la reprise de Lille par le roi en personne, et qui en
avait reçu les contre-ordres, s'était mis ensuite à faire la tournée de
toutes les places de son gouvernement, accompagné de quelques officiers
généraux pour y donner les meilleurs ordres que l'extrême défaut
d'argent et de toutes choses pourrait permettre. Dans ce voyage, mal
rétabli des fatigues incroyables qu'il avait souffertes à Lille, il
tomba malade à l'extrémité. Il guérit et se rétablit à grand'peine, mais
non assez pour oser entreprendre une campagne. Il revint à Paris le 1e,
mars, et eut le lendemain deux audiences du roi, avant et après sa
messe, dans lesquelles il lui rendit compte de son gouvernement, et lui
déclara son impuissance de servir pour cette année.

Le roi, qui s'en était bien douté, fit appeler le maréchal de Villars
ensuite, après quoi il fut public qu'il commanderait l'armée de Flandre
sous Monseigneur, dans laquelle le roi d'Angleterre sous l'incognito de
l'année précédente, et M. le duc de Berry, serviraient volontaires\,; le
maréchal d'Harcourt sur le Rhin, sous Mgr le duc de Bourgogne\,; M. le
duc d'Orléans en Espagne\,; le maréchal de Berwick en Dauphiné\,; et le
duc de Noailles en Roussillon, à l'ordinaire. On verra bientôt que ces
généraux d'armée allèrent à leur destination, mais qu'aucun des princes
ne sortit de la cour.

M. le comte de Toulouse eut charge du roi de dire au comte d'Évreux
qu'il ne servirait point, lequel n'a pas servi depuis. Ce coup de foudre
lui fut adouci de la sorte, moins par égard pour son père que parce
qu'il porta sur M. de Vendôme pour le moins autant que sur lui. Ce n'est
pas que depuis son retour il n'eût essayé à se faire un protecteur du
prince qu'il avait si fort offensé, et qu'il n'y eût presque réussi\,;
mais M\textsuperscript{me} la duchesse en fit tant de honte à son époux,
et se montra si irritée, que le comte d'Évreux ne put réussir. Toute la
cabale en fut étrangement étourdie, et cruellement mortifiée de cette
nouvelle atteinte, qui montrait que ses attentats n'étaient point
pardonnés, nonobstant le châtiment de Vendôme, qu'on ne voyait plus qu'à
Marly et à Meudon, sur un ton fort différent de ce qu'il avait été, et
qui ne servait plus.

Le comte de Roucy, qui n'avait pas servi depuis la bataille d'Hochstedt,
et La Feuillade, noyé depuis celle de Turin, étaient fort de la cour de
Monseigneur. Ils virent bientôt après cette déclaration nommer les
officiers généraux pour chaque armée. Ils n'avaient pas lieu d'espérer
d'être de leur nombre\,; ils crurent se raccrocher en suivant
Monseigneur, et toucher le roi par cette conduite. Ils en demandèrent
donc la permission au roi, qui l'accorda au comte de Roucy et la refusa
à La Feuillade. Ce fut un dégoût très marqué pour lui\,; mais, dans le
fond, la fortune des deux fut pareille. Monseigneur n'alla point, par
conséquent le comte de Roucy, qui n'a jamais servi depuis non plus que
La Feuillade, mais qui n'a pas eu le temps de se faire faire maréchal de
France aussi scandaleusement et aussi inutilement que lui vingt-cinq ans
après.

Harcourt, qui, en Normand habile, savait tirer sur le temps, et que le
commandement d'une armée ne consolait point du ministère, obtint du roi
quatre-vingt mille livres comptant pour faire son équipage, et, dans un
temps aussi pressé que celui où on était, bouda encore de n'en obtenir
pas davantage. L'électeur de Bavière demeura oisif.

Rouillé partit les premiers jours de mars pour aller traiter secrètement
la paix en Hollande\,; à force de besoins on s'en flattait. Bergheyck
était venu quelque temps auparavant passer deux jours chez Chamillart\,;
il avait vu le roi, il croyait les Hollandais portés à la paix. On leur
demanda des passeports, qu'ils accordèrent en grand secret et de fort
mauvaise grâce. Je ne m'étendrai pas davantage là-dessus, non plus que
sur le voyage de Torcy, qu'il y alla furtivement faire quelque temps
après. J'en userai de même sur le voyage que firent l'année suivante le
maréchal d'Huxelles et l'abbé de Polignac, tôt après cardinal, à
Gertruydemberg\,; et pareillement sur tout ce qui amena et fit la paix
d'Utrecht. Torcy, dont la plume et la mémoire ne sont pas moins justes,
bonnes, exactes, que les lumières et la capacité, a écrit toutes ces
trois négociations. Il a bien voulu me communiquer son manuscrit
lui-même\,; je le trouvai si curieux et si important que je le copiai
moi-même\,; il ferait en trois morceaux mis ici en leur temps de trop
longues parenthèses\footnote{Voy. la seconde, la troisième et la
  quatrième partie des Mémoires de Torcy. Elles contiennent probablement
  les trois morceaux dont parle Saint-Simon.}. Ils sont plus agréables
et plus instructifs à voir tous trois de suite, et c'est ainsi qu'ils se
trouveront dans les Pièces\footnote{Voir aux Pièces toute la négociation
  de Rouillé à Bodgrave, de Torcy et de lui à la Haye, et du maréchal
  d'Huxelles et de l'abbé de Polignac à Gertruydemberg, et sur la paix
  d'Utrecht. (\emph{Note de Saint-Simon}.)}.

Il suffira donc ici de faire connaître Rouillé. Il était président en la
cour des aides, et frère de Rouillé qui, de procureur général de la
chambre des comptes, devint directeur des finances, puis conseiller
d'État, dont la brutalité et les débauches, à travers beaucoup
d'érudition et de quelque esprit, firent tant parler de lui, surtout
dans la régence de M. le duc d'Orléans. Celui-ci, qui était le cadet,
avait un esprit délicat et poli, aussi sobre et mesuré que son aîné
l'était peu, et il avait passé une partie de sa vie en diverses
négociations, et en dernier lieu ambassadeur en Portugal. On avait
toujours été content de lui, on verra qu'on ne le fut pas moins malgré
le triste succès de son voyage en Hollande.

Je ne puis mieux placer une double anecdote que fort peu de gens ont
sue, et qui ne précéda que de fort peu les dernières choses que je viens
d'écrire, mais que j'ai réservée pour mieux accompagner Rouillé en
Hollande. Chamillart avait ouï dire et vu, depuis que le billard l'avait
introduit à la cour, et qu'une charge d'intendant des finances l'en
avait approché, que M. de Louvois faisait les charges de tout le monde
et surtout de ses confrères tant qu'il pouvait, et souvent de haute
lutte. Successeur de sa charge et de celle de Colbert, et plus avant que
ni l'un ni l'autre ne furent jamais dans le goût et l'affection du roi,
il s'imagina que l'Imitation de Louvois en ces entreprises était un
droit de sa place ou de sa faveur, et il n'omit rien pour en user de
même. Ç'avait été une des causes principales et des plus continuelles
qui l'avaient tenu toujours si brouillé avec Pontchartrain. Il essaya
plus d'une fois d'embler aussi la besogne du chancelier, qui, lui étant
plus étrangère qu'aucune et appartenant à un homme plus affermi et plus
relevé, l'avait forcé autant de fois à lâcher prise. Je ne me suis pas
amusé à rendre tous ces détails trop longs et trop fréquents\,; il
suffit de les marquer en gros.

À l'égard de Torcy, il s'était mis dans la tête de lui ôter les
négociations de la paix, dont toutefois Torcy était le seul ministre, et
privativement à tout autre par son département. Chamillart, du su du
roi, tenait des gens en Hollande, et partout ailleurs, qui faisaient des
ouvertures et des propositions, et qui surtout décriaient ceux que Torcy
y employait à même fin, le disaient un homme de paille par qui rien ne
réussirait. Ceux de Torcy, et lui-même ne s'épargnaient pas à lui rendre
la pareille et à ses employés, tellement qu'on eût dit que ces gens
servaient dans les pays étrangers des ministres de différents maîtres
dont les intérêts étaient tout opposés. Ces manières de se croiser
donnaient dans ce pays-là un spectacle tout à fait ridicule, et encore
plus nuisible aux affaires\,; une opinion sinistre de la cour et de
notre gouvernement\,: enfin aux personnages à qui ces gens-là étaient
adressés, ou auprès de qui ils s'insinuaient, un grand embarras à
traiter pour ceux qui l'auraient voulu sincèrement\,; et pour les
autres, un prétexte très plausible de n'entrer en rien avec des gens si
peu d'accord entre eux. Tout en était donc non seulement suspendu, mais
dangereusement éventé, et tout se rompait avant même d'avancer.

Chamillart tomba dans un grand ridicule public par deux voyages qu'il
fit faire à Helvétius en Hollande, sous prétexte d'aller voir son père,
mais en effet pour négocier, dont personne ni là ni ici ne fut la dupe.
Helvétius était Hollandais et médecin fort habile pour plusieurs sortes
de maladies, mais qui, pour n'être pas savant à la manière des médecins
ni de leurs Facultés, en était traité d'empirique. C'est à lui qu'on
doit l'usage de l'ipécacuanha, si spécifique pour la guérison des
dysenteries, qui lui donna une grande réputation et lui attira la plus
cruelle envie des médecins, qui ne consultaient point avec lui. Il ne
laissait pas de l'être de quantité de personnes et même considérables\,;
d'ailleurs un bon et honnête homme, charitable, patient, aumônier,
droit, et qui ne manquait ni d'esprit ni de sens, et dont le fils
{[}est{]} maintenant premier médecin de la reine, avec la plus juste et
la plus grande réputation, et qui avec infiniment d'esprit et de bénie
de cour, aurait son tour dans ces Mémoires s'ils s'étendaient jusqu'au
temps où il s'est fait considérer à la cour. Son père, occupé comme il
l'était dans Paris, n'en pouvait disparaître sans bruit, ni le temps de
son absence être obscur, beaucoup moins répétée après un intervalle de
quelques mois. Il n'était rien moins qu'intrigant, il n'était pas même
intéressé. Il ne parlait même jamais de nouvelles, à la différence de
tous les médecins. Il n'était occupé que de son métier, et tous les
jours\,; à la fin de sa matinée, voyait chez lui tous les pauvres qui
voulaient y venir, les écoutait, leur donnait des remèdes, à manger,
souvent de l'argent, et ne refusait jamais d'aller chez aucun. Ainsi
grands et petits surent et souffrirent de son absence et ne s'en turent
pas. Il était le médecin de Chamillart de tout temps. Personne ne
l'accusa d'avoir brigué ces voyages\,; ils portèrent tous sur le
ministre. On peut juger de toutes les plaisanteries amères qui se
débitèrent partout, dedans et dehors le royaume, sur une négociation
d'un médecin, et d'un empirique, et de toutes les piquantes gentillesses
qui coururent là-dessus, et toutefois le roi, à qui Torcy et Chamillart
rendaient compte chacun en particulier, les laissait faire. Ainsi chacun
allait son train à part, et faisait sûrement échouer son confrère.

Torcy qui sentait le tort que cette conduite apportait aux affaires, et
qui n'était rien moins qu'insensible à celui que lui-même en souffrait,
se sentait faible contre la faveur si déclarée de Chamillart, et se
bornait aux plaintes et aux représentations qu'il lui en faisait faire
par le duc de Beauvilliers, mais rarement reçues et toujours éludées.
Sur le déclin de l'administration des finances par Chamillart, ce
ministre, accablé d'affaires et alors de langueur, avait promis de ne
plus traverser Torcy, ensuite de le laisser faire\,; mais tôt après, les
mains lui démangeant, il besogna tout de nouveau, et tout de nouveau
remit Torcy aux champs. Celui-ci, le voyant défait des finances, entre
les mains de son cousin germain et de son ami de tout temps., et son
fils marié, à la fille de la duchesse de Mortemart, son autre cousine
germaine, espéra tout de ces nouvelles considérations. Il attendit donc
encore. Il fit redoubler les représentations, et il eut encore fort
longtemps une patience inutile. À la fin elle lui échappa.

Convaincu qu'il n'obtiendrait rien par douceur, il déclara au duc de
Beauvilliers, qui comme lui voyait le préjudice que ce procédé apportait
aux affaires, que las enfin d'éprouver les continuelles entreprises de
Chamillart, quoi qu'il eût pu faire et employer pour les faire cesser,
il était résolu de faire décider par le roi qui des deux devait se mêler
des affaires étrangères. Beauvilliers parla fort sérieusement à
Chamillart qui, sentant son autorité affaiblie et combien peu il avait
fait de progrès dans ses négociations au dehors, comprit enfin qu'une
pareille décision portée devant le roi ne pourrait lui être favorable,
et protesta au duc de Beauvilliers qu'il ne se mêlerait plus d'aucune
affaire étrangère.

Torcy y avait été attrapé trop souvent pour tâter encore de pareilles
assurances. Il voulut un traité préliminaire, nécessaire selon lui pour
parvenir à celui de la paix. Il se fit donc un écrit, par lequel
Chamillart s'engagea à n'entretenir plus personne pour s'ingérer de la
paix, ni d'aucune affaire étrangère, et promit de plus de renvoyer de
bonne foi à Torcy ceux qui en ce genre pourraient s'adresser à lui
désormais. Il signa cet écrit en présence de M. de Beauvilliers, qui le
remit à Torcy. Celui-ci, content enfin et libre, se raccommoda avec
Chamillart. Il n'eut plus d'inquiétude, et Chamillart depuis ne lui en
donna plus la moindre occasion. M. de Beauvilliers, si lié à ces deux
hommes, acheva cette bonne oeuvre. J'étais trop intimement uni à lui et
à Chamillart pour l'ignorer\,; pour Torcy, notre liaison ne se fit que
depuis la mort du roi. Venons à l'autre anecdote.

Chamillart, tel qu'on vient de le voir à l'égard des autres
départements, démis des finances, en discourait plus que lorsqu'elles
étaient entre ses mains, et, libre de ce fardeau, en oublia bientôt le
poids. Il ne pensait qu'à soutenir celui dont il était demeuré chargé,
et demandait sans cesse de l'argent à son successeur, en homme qui ne
s'inquiétait plus des moyens d'en trouver. Desmarets, toujours
embarrassé, fit ce qu'il put. À la fin, piqué de n'y pouvoir suffire, il
répondit quelquefois vivement, et comme surpris de trouver si peu de
ménagement dans un homme qui ne pouvait avoir oublié l'épuisement où il
avait laissé les finances et le crédit. Enflé par ses places de
contrôleur général, et encore plus de ministre, de se sentir égal à
celui auquel il devait un si grand retour de fortune, et moins sensible
au bienfait qu'à l'importunité continuelle de lui fournir ce qu'il ne
pouvait trouver, il se lâcha quelquefois en reproches sur le mauvais
état auquel il avait trouvé les finances, dont le délabrement ne lui
pouvait être imputé, et dont le temps et la guerre générale, si
malheureuse depuis longtemps, ne lui avaient pu mettre la réparation.

Il m'en fit souvent des plaintes\,; je lui remis souvent la cause de son
retour devant les yeux\,; souvent je l'y trouvai docile, souvent aussi
je ne pouvais m'empêcher de sentir qu'il avait raison. Peu à peu je
commençai à craindre que ces deux hommes\,:rie pussent demeurer
longtemps amis. Les ducs de Chevreuse et de Beauvilliers encore plus,
étançonnaient leur amitié fugitive, et se portaient, continuellement
pour modérateurs entre eux.

L'un, pressé des besoins de la guerre, affermi par sa confiance en
l'amitié du roi, grossissant son autorité sur l'autre par ce qu'il avait
fait pour lui, ne pouvait se défaire d'en exiger durement. Desmarets
devenu son égal, impatient du joug, à bout d'industrie à suppléer aux
manquements, s'échappait aux considérations, et rétorquait les arguments
par accuser l'autre d'avoir ruiné les finances, tellement que tous deux
se trouvant aigris, et à bout de moyens, Chamillart porta ses plaintes
au roi de se trouver court de fonds. Le roi, qui ne voulait ni
accoutumer ses ministres ni s'accoutumer lui-même à ce langage,
quoiqu'il commençât à devenir fréquent, parla fortement à Desmarets,
qui, forcé à la justificative, ne put être retenu par les deux ducs
modérateurs, et saisit, sous des apparences en effet honnêtes
puisqu'elles paraissaient nécessaires, l'occasion d'éclater. Il rapporta
au roi l'état des sommes qu'il avait fournies à Chamillart, expliqua
quelles en argent, quelles en billets, et comment payées, en déduisit
les fonds et les destinations, tout cela par pièces justificatives, et
montra que Chamillart était plus que rempli. Le roi le dit à Chamillart,
qui, bien étonné, soutint toujours son dire, et avec sa confiance
accoutumée offrit d'en faire convenir Desmarets.

Il fut chez lui où, vérification faite, il -se trouva court et rempli.
La chose fut rejetée sur ses commis\,; mais Desmarets, résolu de n'avoir
pas le démenti, voulut que les commis fussent appelés, et, bien que
Chamillart se radoucit, il ne put sortir de chez le contrôleur général
que le commis des payements du bureau de la guerre, qui s'appelait de
Soye, ne fût mandé par Chamillart et ne fût venu avec ses registres. La
somme en débat s'y trouva reçue au temps et en la manière que Desmarets
l'avait soutenu. Alors le {[}débat{]} fut entre Chamillart et son
commis, mais il ne dura guère, parce que, Chamillart ayant voulu se
fâcher, de Soye, à l'instant même et en présence de Desmarets, lui en
montra l'emploi, qui était différent de celui auquel le roi l'avait
destinée, quoiqu'en chose effectivement du bien du service, mais
entièrement différente. Alors Chamillart, honteux de son oubli et du
mécompte, et Desmarets, radouci par l'issue d'une si forte dispute, se
séparèrent honnêtement, et de concert étouffèrent la chose tant qu'ils
purent\,; mais elle demeura d'autant moins secrète qu'il fallut bien que
le dénouement en fût porté au roi. Il l'apprit et le reçut avec une
extrême bonté pour Chamillart, sauvé par la multiplicité de ses
affaires, que sa mauvaise santé et ses voyages en Flandre avaient
arriérées et brouillées dans sa tête. Le public n'en jugea pas si
favorablement.

Chamillart, peu après être entré dans l'administration entière des
finances, avait pris en affection un financier appelé La Cour des
Chiens, auquel il avait donné les meilleures affaires. Ce La Cour s'y
était prodigieusement enrichi\,; il était habile, intelligent, plein de
ressources, et avait utilement servi en ce genre\,; d'ailleurs bon
homme, obligeant, éloigné de l'insolence si ordinaire à ces sortes de
gens. Mais son opulence et sa prodigalité en toutes sortes de délices
avait irrité le public. Il avait fait un bijou d'un vilain lieu et d'une
méchante maison que Chamillart lui avait donnée dans son parc de
l'Étang, et qu'avec sa permission il vendit à Desmarets lorsqu'il eut
les finances. Il venait de bâtir un hôtel superbe joignant l'hôtel de
Lorges, depuis {[}celui de la princesse{]} de Conti, fille du roi, et
Chamillart ne se cachait pas que c'était pour lui\,; mais sa fortune ne
dura pas jusque-là, et d'Antin l'acheta, qui en fit une demeure
somptueuse. La jalousie des gens d'affaires contre La Cour se joignit à
l'aversion que le public avait prise contre ses richesses, qui ramassa
mille mauvais discours que ces financiers semèrent de Chamillart et de
lui.

Dans les nécessités pressantes d'argent pour les vivres, il était
échappé au zèle de Chamillart de répondre en son nom de diverses grosses
fournitures. Sûr de sa probité et de la confiance du roi, il n'avait
rien appréhendé\,; et La Cour, assuré aussi de toute la protection du
tout-puissant ministre, était entré en des engagements prodigieux. Ils
étaient donc tels, qu'il n'y pouvait suffire que très difficilement,
surtout ne s'en contraignant pas davantage sur les dépenses prodigues
que lui coûtaient ses plaisirs et ses parents. Tout cela ensemble, sous
un autre ministère, donna prise sur la conduite de Chamillart et sur la
bourse de La Cour, et bien qu'on ne reprochât rien de honteux à
Chamillart, on l'accusa d'avoir employé ces sommes contestées, avec
plusieurs autres, à payer les parties auxquelles il s'était imprudemment
engagé en son nom, et à se tirer ainsi d'affaires, préférablement à des
choses plus pressées pour le service de la guerre, et plus présentes.

Je ne suis pas éloigné de croire qu'il en était bien quelque chose, et
que ce ministre, désormais hors du maniement des finances, craignant de
ne pas trouver toujours les moyens de sortir de ses engagements
indiscrets, y employa des sommes dont la destination était tout à fait
différente. De crime ni de faute, il n'y en avait pas l'ombre, puisqu'il
n'en détourna jamais une pistole à ses usages particuliers, et il eut
cet avantage que le soupçon n'en entra jamais dans la tête de personne.
Mais cet exemple doit faire sages et ministres et autres de ne s'engager
jamais si avant qu'on n'ait entre les mains de quoi en bien sortir.

Depuis cette explication, il n'y eut plus entre ces deux ministres que
des dehors et de grandes mesures d'honnêteté. Je l'avoir prévu dès les
commencements ainsi que je l'ai rapporté. Tous deux m'étaient chers
encore, et j'en fus aussi touché que MM. de Chevreuse et de
Beauvilliers.

\hypertarget{chapitre-viii.}{%
\chapter{CHAPITRE VIII.}\label{chapitre-viii.}}

1709

~

{\textsc{Hiver terrible\,; effroyable misère.}} {\textsc{- Cruel manége
sur les blés.}} {\textsc{- Courage de Maréchal à parler au roi,
inutile.}} {\textsc{- Grande mortification au parlement de Paris sur les
blés, et pareillement au parlement de Bourgogne.}} {\textsc{- Étranges
inventions perpétuées.}} {\textsc{- Manége des blés imité plus d'une
fois depuis.}} {\textsc{- Refonte et rehaussement de la monnaie.}}
{\textsc{- Banqueroute de Samuel Bernard.}} {\textsc{- Ma liaison intime
avec le maréchal de Boufflers.}} {\textsc{- Sa réception au parlement.}}
{\textsc{- Belsunce évêque de Marseille.}}

~

L'hiver, comme je l'ai déjà remarqué, avait été terrible, et tel, que de
mémoire d'homme on ne se souvenait d'aucun qui en eût approché. Une
gelée, qui dura près de deux mois de la même force, avait dès ses
premiers jours rendu les rivières solides jusqu'à leur embouchure, et
les bords de la mer capables de porter des charrettes qui y voituraient
les plus grands fardeaux. Un faux dégel fondit les neiges qui avaient
couvert la terre pendant ce temps-là\,; il fut suivi d'un subit
renouvellement de gelée aussi forte que la précédente, trois autres
semaines durant. La violence de toutes les deux fut telle que l'eau de
la reine de Hongrie, les élixirs les plus forts, et les liqueurs les
plus spiritueuses cassèrent leurs bouteilles dans les armoires de
chambres à feu, et environnées de tuyaux de cheminée, dans plusieurs
appartements du château de Versailles, où j'en vis plusieurs, et soupant
chez le duc de Villeroy, dans sa petite chambre à coucher, les
bouteilles sur le manteau de la cheminée, sortant de sa très petite
cuisine où il y avait grand feu et qui était de plain-pied à sa chambre,
une très petite antichambre entre-deux, les glaçons tombaient dans nos
verres. C'est le même appartement qu'a aujourd'hui son fils.

Cette seconde gelée perdit tout. Les arbres fruitiers périrent, il ne
resta plus ni noyers, ni oliviers, ni pommiers, ni vignes, à si peu près
que ce n'est pas la peine d'en parler. Les autres arbres moururent en
très grand nombre, les jardins périrent et tous les grains dans la
terre. On ne peut comprendre la désolation de cette ruine générale.
Chacun resserra son vieux grain. Le pain enchérit à proportion du
désespoir de la récolte. Les plus avisés ressemèrent des orges dans les
terres où il y avait eu du blé, et furent imités de la plupart. Ils
furent les plus heureux, et ce fut le salut, mais la police s'avisa de
le défendre, et s'en repentit trop tard. Il se publia divers édits sur
les blés\,; on fit des recherches, des amas\,; on envoya des
commissaires par les provinces trois mois après les avoir annoncés, et
toute cette conduite acheva de porter au comble l'indigence et la
cherté, dans le temps qu'il était évident par les supputations qu'il y
avait pour deux années entières de blés en France, pour la nourrir tout
entière, indépendamment d'aucune moisson.

Beaucoup de gens crurent donc que messieurs des finances avaient saisi
cette occasion de s'emparer des blés par des émissaires répandus clans
tous les marchés du royaume, pour le vendre ensuite au prix qu'ils y
voudraient mettre, au profit du roi, sans oublier le leur. Une quantité
fort considérable de bateaux de blés se gâtèrent sur la Loire, qu'on fut
obligé de jeter à l'eau, et que le roi avait achetés, ne diminuèrent pas
cette opinion, parce qu'on ne put cacher l'accident. Il est certain que
le prix du blé était égal dans tous les marchés du -royaume\,; qu'à
Paris des commissaires y mettaient le prix à main-forte, et obligeaient
souvent les vendeurs à le hausser malgré eux\,; que sur les cris, du
peuple combien cette cherté durerait, il échappa à quelques-uns des
commissaires, et dans un marché à deux pas de chez moi, près
Saint-Germain des Prés, cette réponse assez claire, \emph{Tant qu'il
vous plaira}, comme faisant entendre, poussés de compassion et
d'indignation tout ensemble, tant que le peuple souffrirait qu'il
n'entrât de blé dans Paris que sur les billets de d'Argenson, et il n'y
en entrait pas autrement. D'Argenson, que la régence a vu tenir les
sceaux, était alors lieutenant de police, et fut fait en ce même temps
conseiller d'État, sans quitter la police. La rigueur de la contrainte
fut poussée à bout sur les boulangers, et ce que je raconte fut uniforme
par toute la France. Les intendants faisaient dans leurs
généralités\footnote{On appelait généralités des circonscriptions
  financières de l'ancienne France\,; leur nom venait de ce qu'il y
  avait primitivement dans chacune de ces circonscriptions un ou
  plusieurs \emph{généraux des finances\,;} c'était ainsi qu'on appelait
  alors les receveurs généraux et les trésoriers de France. Voy., sur
  les intendants, t. III, p.~442.} ce que d'Argenson faisait à Paris\,;
et par tous les marchés, le blé qui ne se trouvait pas vendu au prix
fixé, à l'heure marquée pour finir le marché, se remportait forcément,
et ceux à qui la pitié le faisait donner à un moindre prix étaient punis
avec cruauté.

Maréchal, premier chirurgien du roi, de qui j'ai parlé plus d'une fois,
eut le courage et la probité de dire tout cela au roi, et d'y ajouter
l'opinion sinistre qu'en concevait le public, les gens hors du commun,
et même les meilleures têtes. Le roi parut touché, n'en sut pas mauvais
gré à Maréchal\,; mais il n'en fut autre chose.

Il se fit en plusieurs endroits des amas prodigieux, et avec le plus
grand secret qu'il fut possible. Rien n'était plus sévèrement défendu
par les édits aux particuliers, et les délations également prescrites.
Un pauvre homme s'étant avisé d'en faire une à Desmarets en fut rudement
châtié. Le parlement s'assembla par chambres à ces désordres, ensuite
dans la grand'chambre, par députés des autres chambres. La résolution y
fut prise d'envoyer offrir au roi que des conseillers allassent par
l'étendue du ressort, et à leurs dépens, faire la visite des blés, y
mettre la police, punir les contrevenants aux édits\,; et de joindre une
liste de ceux des conseillers qui s'offraient à ces tournées, par
départements séparés. Le roi, informé de la chose par le premier
président, s'irrita d'une façon étrange, voulut envoyer une dure
réprimande au parlement et lui commander de ne se mêler que de juger des
procès. Le chancelier n'osa représenter au roi combien ce que le
parlement voulait faire était convenable, et combien cette matière était
de son district, mais il appuya sur l'affection et le respect avec
lequel le parlement s'y présentait, et il lui fit voir combien il était
maître d'accepter ou de refuser ses offres. Ce ne fut pas sans débat
qu'il parvint à calmer le roi, assez pour sauver la réprimande\,; mais
il voulut absolument que le parlement fût au moins averti de sa part
qu'il lui défendait de se mêler des blés. La scène se passa en plein
conseil, où le chancelier parla seul, tous les autres ministres gardant
un profond silence\,; ils savaient apparemment bien qu'en penser, et se
gardèrent bien de rien dire sur une affaire qui regardait le ministère
particulier du chancelier. Quelque accoutumé que fût le parlement ainsi
que tous les autres corps aux humiliations, celle-ci lui fut très
sensible. Il y obéit en gémissant.

Le public n'en fut pas moins touché, il n'y eut personne qui ne sentît
que si les finances avaient été nettes de tous ces cruels manèges, la
démarche du parlement ne pouvait qu'être agréable au roi et utile, en
mettant cette compagnie entre lui et son peuple, et montrant ainsi qu'on
n'y entendait point finesse, et cela sans qu'il en eût rien coûté de
solide, ni même d'apparent à cette autorité absolue et sans bornes dont
il était si vilement jaloux.

Le parlement de Bourgogne, voyant la province dans la plus extrême
nécessité, écrivit à l'intendant, qui ne s'en émut pas le moins du
monde. Dans ce danger si pressant d'une faim meurtrière, la compagnie
s'assembla pour y pourvoir. Le premier président n'osa assister à la
délibération, il en devinait apparemment plus que les autres\,; l'ancien
des présidents à mortier y présida. Il n'y fut rien traité que de
nécessaire à la chose, et encore avec des ménagements infinis\,;
cependant le roi n'en fut pas plutôt informé qu'il s'irrita extrêmement.
Il envoya à ce parlement une réprimande sévère, défense de se plus mêler
de cette police, quoique si naturellement de son ressort, et ordre au
président à mortier qui avait présidé à la délibération de venir, à la
suite de la cour, rendre compte de sa conduite. Il partit aussitôt. Il
ne s'agissait de rien moins que de le priver de sa charge. Néanmoins M.
le Duc, gouverneur de la province, en survivance de M. le Prince fort
malade, s'unit au chancelier pour protéger ce magistrat, dont la
conduite était irréprochable\,; ils le sauvèrent moyennant une forte
vesperie de la part du roi, qui permit après qu'il lui fit la révérence.
Ainsi au bout de quelques semaines il retourna à Dijon, où on avait
résolu de lui faire une entrée et de le recevoir en triomphe. En homme
sage et trop expérimenté, il en redouta les suites. Il craignit même de
n'obtenir pas d'être dispensé de recevoir cet honneur\,; mais il l'évita
en mesurant son voyage de façon qu'il arriva à Dijon à cinq heures du
matin, prit aussitôt sa robe, et s'en alla au parlement rendre compte de
son voyage, et remercier de ce qu'on avait résolu de faire pour lui.

Les autres parlements, sur ces deux exemples, se laissèrent en tremblant
sous la tutelle des intendants et dans la main de leurs émissaires. Ce
fut pour lors qu'on choisit ces commissaires dont j'ai parlé, tirés tous
de sièges subalternes, qui, chargés de la visite chacun d'un -certain
canton, devaient juger des délits avec les présidiaux voisins, sous les
yeux de l'intendant, et sans dépendance aucune des parlements.

Mais pour donner une amulette plutôt qu'une vaine consolation à celui de
Paris, il fut composé un tribunal tiré de toutes ses chambres, à la tête
duquel Maisons, président à mortier, fut mis, auquel devaient ressortir
les appellations des sentences de ces commissaires dans les provinces.
Ils ne partirent que trois mois après leur établissement. Ils firent des
courses vaines, et pas un d'eux n'eut jamais aucune connaissance de
cette police. Ainsi ils ne trouvèrent rien parce qu'on s'était mis en
état qu'ils ne pussent rien rencontrer, par conséquent ni jugement ni
appel, faute de matière. Cette ténébreuse besogne demeura ainsi entre
les mains d'Argenson et des seuls intendants, dont on se garda bien de
la laisser sortir ni éclairer, et elle continua d'être administrée avec
la même dureté.

Sans porter de jugement plus précis sur qui l'inventa et en profita, il
se peut dire qu'il n'y a guère de siècle qui ait produit un ouvrage plus
obscur, plus hardi, mieux tissu, d'une oppression plus constante, plus
sûre, plus cruelle. Les sommes qu'il produisit sont innombrables, et
innombrable le peuple qui en mourut de faim réelle et à la lettre, et de
ce qu'il en périt après des maladies causées par l'extrémité de la
misère, le nombre infini de familles ruinées, et les cascades de maux de
toute espèce qui en dérivèrent.

Avec cela néanmoins les payements les plus inviolables commencèrent à
s'altérer. Ceux de la douane, ceux des diverses caisses d'emprunts, les
rentes de l'hôtel de ville, en tous temps si sacrées, tout fut suspendu,
ces dernières seulement continuées, mais avec des délais, puis des
retranchements, qui désolèrent presque toutes les familles de Paris et
bien d'autres. En même temps les impôts haussés, multipliés, exigés avec
les plus extrêmes rigueurs, achevèrent de dévaster la France. Tout
renchérit au delà du croyable, tandis qu'il ne restait plus de quoi
acheter au meilleur marché\,; et quoique la plupart des bestiaux eussent
péri faute de nourriture, et par la misère de ceux qui en avaient dans
les campagnes, on mit dessus un nouveau monopole. Grand nombre de gens
qui les années précédentes soulageaient les pauvres se trouvèrent
réduits à subsister à grand'peine, et beaucoup de ceux-là à recevoir
l'aumône en secret. Il ne se peut dire combien d'autres briguèrent les
hôpitaux, naguère la honte et le supplice des pauvres, combien
d'hôpitaux ruinés revomissant leurs pauvres à la charge publique,
c'est-à-dire alors à mourir effectivement de faim, et combien d'honnêtes
familles expirantes dans les greniers.

Il ne se peut dire aussi combien tant de misère échauffa le zèle et la
charité, ni combien immenses furent les aumônes. Mais les besoins
croissant à chaque instant, une charité indiscrète et tyrannique imagina
des taxes et un impôt pour les pauvres. Elles s'étendirent avec si peu
de mesure, en sus de tant d'autres, que ce surcroît mit une infinité de
gens plus qu'à l'étroit au delà de ce qu'ils y étaient déjà, en
dépitèrent un grand nombre, dont elles tarirent les aumônes volontaires,
en sorte qu'outre l'emploi de ces taxes peut-être mal gérées, les
pauvres en furent beaucoup moins soulagés. Ce qui a été depuis de plus
étrange, pour en parler sagement, c'est que ces taxes en faveur des
pauvres, un peu modérées, mais perpétuées, le roi se les est
appropriées, en sorte que les gens des finances les touchent
publiquement jusqu'à aujourd'hui, comme une branche des revenus du roi,
jusqu'avec la franchise de ne lui avoir pas fait changer de nom.

Il en est de même de l'imposition qui se fait tous les ans dans chaque
généralité pour les grands chemins, les finances se la sont appropriée
encore avec la même franchise, sans lui faire changer de nom. La plupart
des ponts sont rompus partout le royaume, et les plus grands chemins
étaient devenus impraticables. Le commerce, qui en souffre infiniment, a
réveillé. Lescalopier, intendant de Champagne, imagina de les faire
accommoder par corvées, sans même donner du pain. On l'a imité partout,
et il en a été fait conseiller d'État. Le monopole des employés à ces
ouvrages les a enrichis, le peuple en est mort de faim et de misère à
tas, à la fin la chose n'a plus été soutenable et a été abandonnée et
les chemins aussi. Mais l'imposition pour les faire et les entretenir
n'en a pas mains subsisté pendant ces corvées et depuis, et pas moins
touchée comme une branche des revenus du roi.

Ce manége des blés a paru une si bonne ressource, et si conforme à
l'humanité et aux lumières de M. le Duc et des Pâris, maîtres du royaume
sous son ministère, et maintenant que j'écris, au contrôleur général
Orry, le plus ignorant et le plus barbare qui administra jamais les
finances, que l'un et l'autre ont saisi la même ressource, mais plus
grossièrement, comme eux-mêmes, et avec le même succès de famine factice
qui a dévasté le royaume.

Mais pour revenir à l'année 1709, où nous en sommes, on ne cessait de
s'étonner de ce que pouvait devenir tout l'argent du royaume. Personne
ne pouvait plus payer, parce que personne ne l'était soi-même\,; les
gens de la campagne, à bout d'exactions et de non-valeurs, étaient
devenus insolvables. Le commerce tari ne rendait plus rien, la bonne foi
et la confiance abolies. Ainsi le roi n'avait plus de ressource que la
terreur et l'usage de sa puissance sans bornes, qui, tout illimitée
qu'elle fût, manquait aussi, faute d'avoir sur quoi prendre et
s'exercer. Plus de circulation, plus de voies de la rétablir. Le roi ne
payait plus même ses troupes, sans qu'on pût imaginer ce que devenaient
tant de millions qui entraient dans ses coffres.

C'est l'état affreux où tout se trouvait réduit lorsque Rouillé, et tôt
après lui Torcy, furent envoyés en Hollande. Ce tableau est exact\,;
fidèle et point chargé. Il était nécessaire de le présenter au naturel,
pour faire comprendre l'extrémité dernière où on était réduit,
l'énormité des relâchements où le roi se laissa porter pour obtenir la
paix, et le miracle visible de celui qui met des bornes à la mer, et qui
appelle ce qui n'est pas comme ce qui est, par lequel il tira la France
des mains de toute l'Europe résolue et prête à la faire périr, et l'en
tira avec les plus grands avantages vu l'état où elle se trouvait
réduite, et le succès le moins possible à espérer.

En attendant, la refonte de la monnaie et son rehaussement d'un tiers
plus que sa valeur intrinsèque apporta du profit au roi, mais une ruine
aux particuliers et un désordre dans le commerce qui acheva de
l'anéantir.

Samuel Bernard culbuta Lyon par sa prodigieuse banqueroute dont la
cascade fit de terribles effets. Desmarets le secourut autant qu'il lui
fut possible. Les billets de monnaie et leur discrédit en furent cause.
Ce célèbre banquier en fit voir pour vingt millions. Il en devait
presque autant à Lyon. On lui en donna quatorze en bonnes
assignations\footnote{On appelait assignations dans l'ancienne
  organisation financière de la France les mandements ou ordonnances aux
  trésoriers pour payer une dette sur un fonds déterminé, comme les
  gabelles, les tailles, etc. Les bonnes assignations étaient celles
  dont les fonds étaient disponibles et qui pouvaient être payées
  sur-le-champ.}, pour tâcher de le tirer d'affaires avec ce qu'il
pourrait faire de ses billets de monnaie. On a prétendu depuis qu'il
avait trouvé moyen de gagner beaucoup, à cette banqueroute\,; mais il
est vrai que, encore qu'aucun particulier de cette espèce n'eût jamais
tant dépensé ni laissé, et n'ait jamais eu, à beaucoup près, un si grand
crédit par toute l'Europe, jusqu'à sa mort arrivée trente-cinq ans
depuis, il en faut excepter Lyon et la partie de l'Italie qui en est
voisine, où il n'a jamais pu se rétablir.

Le pape enfin, poussé à bout par les exécutions militaires qui
désolaient l'État ecclésiastique, le blocus de Ferrare et du fort Urbin,
céda à toutes les volontés de l'empereur, et reconnut l'archiduc roi
d'Espagne, sur, quoi Philippe V fit défendre au nonce, qui était à
Madrid, de se présenter devant lui, fit fermer la nonciature et rompit
tout commerce avec Rome, ce qui y tarit une grande source d'argent. Son
ambassadeur sortit de Rome et des États du pape, et cependant les
Impériaux ravageaient toujours les terres de l'Église, sans que le
marquis de Prié daignât les arrêter. Il donna une comédie et un bal dans
son palais, contre les plus expresses défenses du pape, qui, dans cette
calamité, avait interdit tous les spectacles et les plaisirs dans Rome.
Il envoya faire des remontrances à Prié. sur la fête qu'il voulait
donner. Il n'en eut d'autre réponse, sinon qu'il s'y était engagé aux
dames, à qui il ne pouvait manquer de parole. Le rare est qu'après ce
mépris si publie, les neveux du pape y allèrent, et qu'il eut la
faiblesse de le souffrir.

Tessé, voyant venir cet orage qu'il ne pouvait détourner, même par ces
belles lettres qui se trouveront dans les Pièces, crut ne pouvoir pas
mieux prendre son temps pour se faire une opération au derrière, pour
vérifier la raison qui, politiquement, l'avait tenu depuis très
longtemps chez lui pour ne se point commettre, et pour y demeurer tant
qu'il le jugerait à propos sans être obligé de voir qui il ne voudrait
pas, ni de sortir de chez lui. Le pape, éperdu, avait fait tout ce qu'il
avoir pu pour retenir l'ambassadeur d'Espagne, et n'oubliait rien pour
empêcher Tessé de partir. Toutefois la partie n'étant plus tenable, et
ne faisant plus qu'un personnage inutile et honteux, il partit et s'en
revint fort lentement.

En débarquant en Provence, il apprit la mort de sa femme dans sa
province dont elle n'était jamais sortie, et qui s'appelait Auber, fille
unique du baron d'Aunay, près de Caen, et dont il paraissait qu'il ne
tenait pas grand compte. À son retour il ne laissa pas d'avoir une
longue audience du roi, quoique sur un voyage dont le succès avait été
fort désagréable et les affaires étaient vieillies. Telle fut celui de
cette ligue d'Italie si bien pensée, mais qui échoua avant d'être
formée, comme je l'ai raconté.

Malgré tant de différence d'âge et d'emplois, et de liaisons encore qui
n'étaient pas les mêmes, j'étais ami intime du maréchal de Boufflers. Je
fus donc ravi de sa gloire et de ses récompenses. Il n'ignorait pas
combien j'étais blessé de la multiplication des ducs et pairs, et
j'oserai dire qu'il se trouva flatté de ma joie de le voir revêtu de la
pairie. Il crut aussi, par ce qu'il s'était passé en diverses choses de
cette dignité, que j'y entendais quelque chose, tellement qu'en
retournant en Flandre pour ce projet de reprendre Lille, qui n'eut pas
lieu, il me pria en son absence de voir ses lettres d'érection, qu'il
avait chargé le président Lamoignon de projeter, et me demanda avec
confiance, et comme un vrai service, de vouloir bien travailler à les
dresser avec La Vrillière, secrétaire d'État en mois\footnote{Les quatre
  secrétaires d'État étaient chargés par mois, à tour de rôle, de tenir
  la plume au conseil et d'en expédier les ordonnances.}, qui les devait
expédier, qui était mon ami particulier, et qui voudrait bien m'en
croire.

Nous les dressâmes donc La Vrillière et moi le plus avantageusement et
fortement qu'il fut possible, sans outrepasser en rien dans les clauses
ce que le roi avait bien voulu accorder, mais que nous exprimâmes avec
toute la netteté et la clarté qui s'y put répandre.

Dès que le maréchal fut de retour, je lui conseillai de faire un effort
sur sa santé pour se faire recevoir au parlement le jour même que ses
lettres y seraient enregistrées, parce qu'il s'épargnerait une double
fatigue de visites, et que, après le péril où il avait été dans sa
maladie en Flandre, il n'était pas sage de différer un enregistrement
dont dépendait la réalité de sa dignité, ni sa réception qui fixait son
rang et des siens pour toujours. Il me crut et me pria de le conduire
sur l'une et sur l'autre, et d'être aussi le premier de ses quatre
témoins.

Je fus très sensible à cet honneur\,; ainsi je ne voulus pas me
contenter de l'usage ordinaire qui est que le greffier vous apporte chez
vous un témoignage, tout dressé et qu'on signe, ce qui est une manière
de formule un peu diversifiée pour varier les quatre témoignages que le
rapporteur lit tout haut en rapportant. J'en pris occasion de rendre
public ce que je pensais d'un si vertueux personnage que sa dernière
action venait de combler d'honneur. Je le dictai donc au greffier
lorsqu'il vint chez moi, je le signai et j'en envoyai un double signé
aussi au maréchal de Boufflers, dont il fut fort touché. Les trois
autres témoins furent le duc d'Aumont, parce qu'il faut cieux pairs, et
deux autres qui furent M. de Choiseul, doyen des maréchaux de France
alors, et Beringhen, premier écuyer, tous deux chevaliers de l'ordre.

La vérification ou enregistrement des lettres d'érection et la réception
du maréchal se fit tout de suite le mardi matin 19 mars. Comme il
s'agissait de l'une et de l'autre à la fois, tout le parlement fut
assemblé, en sorte qu'avec les pairs, les conseillers d'honneur et
honoraires, et les quatre maîtres des requêtes qui s'y peuvent trouver
ensemble, nous étions près de trois cents sur les fleurs de lis. Tout ce
qui put s'y trouver de pairs y assista, et jamais tant de seigneurs, de
gens de toutes sortes de qualités ni une telle affluence d'officiers,
surtout de ceux qui sortaient de Lille.

M. le Duc prit cette occasion de mener pour la première fois M. le duc
d'Enghien son fils au parlement, comme font toujours les princes du sang
à l'occasion d'une réception de pair, auxquelles toutes tous se trouvent
toujours. Pelletier, premier président, en fit un petit mot de
compliment à M. le Duc, et y mêla fort à propos quelque chose sur M. le
prince de Conti, qui venait de mourir. M. le Duc répondit si bas que
personne ne put l'entendre.

Comme on s'assemblait et qu'on prenait place, arriva le nouveau pair
fort accompagné, qui, outre tout ce que j'ai dit, qui vint là l'honorer,
trouva par les rues et dans le palais, sur tout son passage, une si
grande foule de peuple criant et applaudissant en manière de triomphes
que je ne vis jamais spectacle si beau, ni si satisfaisant, ni homme si
modeste que celui qui le reçut au milieu de toute cette pompe.

Tous étant en place, Le Nain, lors sous-doyen du parlement, et magistrat
très vénérable (le doyen étant hors de combat), fit lecture des lettres,
puis commença le rapport. Aussitôt je me levai et sortis, comme fit
aussi le duc d'Aumont, et avec nous le duc de Guiche et les autres
pairs, parents au degré de l'ordonnance. Les deux présidents de
Lamoignon, père et fils, l'un honoraire, l'autre titulaire, sortirent
après nous et aussitôt, par la petite vanité de montrer qu'ils avaient
travaillé aux lettres, car ils n'avaient aucune parenté. La foule était
si grande que les huissiers eurent peine à nous faire faire place. Les
deux présidents se retirèrent à la cheminée, et nous vers les fenêtres,
autour de notre nouveau confrère qui y était assis, et s'était un peu
trouvé mal. Sitôt que l'arrêt de réception fut prononcé, les huissiers
nous vinrent avertir. Les présidents de Lamoignon rentrèrent en place un
moment après nous. Après que nous y fûmes tous remis, les huissiers
vinrent chercher le maréchal, qui prêta son serment à la manière
accoutumée et prit après sa place.

La séance se trouva de manière que son serment se fit derrière moi. Un
moment après qu'il fut en place, le premier président lui fit un
compliment auquel le maréchal répondit fort modestement, mais fort
intelligiblement. Mon témoignage et ces deux pièces ne sont pas assez
longues pour ne tenir pas place ici\,; j'ai cru ne devoir rien omettre
de la brillante réception d'un homme si illustre. Voici le témoignage
que je rendis, et que Le Nain lut tout haut le premier des quatre\,:

«\,Messire Louis, duc de Saint-Simon, pair de France, etc. a dit que M.
le duc de Boufflers, dont la très ancienne maison est alliée aux plus
grandes du royaume, paraît encore plus illustre par le trophée de
dignités et de charges les plus éclatantes que sa vertu a ramassées sur
sa tête, sans qu'il en ait jamais recherché aucune, et pour ainsi dire
malgré son rare désintéressement et sa modestie singulière\,: c'est ce
qu'a toujours montré sa conduite si uniforme dans les divers
commandements des provinces et des armées qu'il a si dignement exercés,
et dans lesquels il est si exactement vrai de dire qu'il a bien mérité
du roi, de l'État et de chaque particulier, ainsi que dans les emplois
de la cour les plus distingués par leur élévation et par leur confiance.
Il s'est aussi rendu considérable dans les négociations les plus
importantes\,; et partout il a fait également voir une probité, un
attachement au roi, un amour pour l'État, qui l'ont continuellement
emporté chez lui sur les considérations les plus chères aux hommes. Mais
son dernier exploit est tel dans toutes ses circonstances que, s'il a
mérité l'admiration effective de toute l'Europe, l'étonnement, les
éloges et les honneurs inouïs des ennemis mêmes, les coeurs de tout ce
qui a été plus particulièrement témoin de tous ses travaux et de sa
gloire, il est bien juste que, puisqu'il se peut dire qu'il fait honneur
à. sa nation, il reçoive de l'équité du roi le comble des honneurs de
cette même nation, et que ceux qui en sont revêtus le reçoivent parmi
eux avec joie et reconnaissance. C'est donc avec une grande vérité et un
plaisir sensible que je le reconnais parfaitement digne de la pairie
dont il a plu au roi de l'honorer.\,»

Le premier président lui dit\,:

«\,Monsieur, la cour m'a chargé de vous marquer la joie sensible qu'elle
a de voir récompenser en votre personne, par la dignité éminente de duc
et pair de France, les grands services que vous avez rendus depuis si
longtemps au roi et à l'État, et notamment celui que vous venez de lui
rendre par la longue, brave et vigoureuse défense que vous avez faite
dans la ville et dans la citadelle de Lille. Vous avez fait paraître par
votre prudence, vôtre activité inconcevable et votre intrépidité, tout
ce qu'on pouvait attendre d'un général aussi consommé, d'un sujet aussi
reconnaissant, d'un citoyen aussi affectionné que vous l'êtes.\,»

Le maréchal lui répondit\,:

«\,Monsieur, je n'ai point de termes assez forts pour exprimer la vive
et sensible reconnaissance de l'honneur que la cour me fait. Je voudrais
être digne des grâces que le roi vient de répandre sur moi, des éloges
que vous me donnez, et des marques de bonté que la cour me donne en
cette occasion. Si quelque chose pouvait me les faire mériter, ce ne
pourrait être que mon extrême zèle et dévouement pour le service du roi
et de l'État, et la parfaite vénération que j'ai pour cette auguste
compagnie, et en particulier pour votre personne.\,»

Je ne sais comment il m'était échappé de n'avertir pas le maréchal du
compliment qu'il recevrait, et de celui qu'il aurait à faire\,; mais il
ne le fut que le matin même. En arrivant dans la grand'chambre, il nous
montra et nous consulta sa réponse à M. de Chevreuse et à moi, dont il
eut à peine le temps, et que nous louâmes comme elle le méritait.
Aussitôt qu'il l'eut achevée, la cour se leva sans appeler de cause
selon la coutume, parce que la longueur de la vérification avait emporté
tout le temps. Tous les princes du sang et presque nous ` tous
demeurâmes à la grande audience.

En sortant, le maréchal, s'adressant à ce grand nombre de gens de guerre
qui se trouvèrent là, ou qui l'y avaient accompagné, surtout à ceux qui
avaient été dans Lille, leur dit de fort bonne grâce\,: «\,Messieurs,
tous les honneurs qu'on me fait ici, et toutes les grâces que je reçois
du roi, c'est à vous que je crois les devoir, c'est votre mérité, c'est
votre valeur qui me les ont attirés. Je ne dois me louer que d'avoir été
à la tête de tant de braves gens qui ont fait valoir mes bonnes
intentions.\,»

Il ne donna point de repas comme plusieurs font en cette occasion\,; sa
santé ne lui permit pas de joindre cette fatigue à toute celle qu'il
venait d'essuyer.

Il dut être bien content des applaudissements universels, et encore plus
de lui-même, surtout de la modestie et de la simplicité qu'il y montra
d'une façon si naturelle, et qui achevèrent de le faire estimer digne de
l'éclat qu'il savait si bien supporter.

Il fut remarquable que le propre jour du triomphe du défenseur de Lille
fut celui même de l'éclair qui précéda la foudre lancée contre celui qui
n'avait pas voulu le secourir\,; car ce fut le soir du jour de la
réception au parlement du maréchal de Boufflers, que le comte de
Toulouse dit, par ordre du roi, au comte d'Évreux qu'il ne servirait
plus.

Le roi, après avoir fait ses pâques le samedi saint, à son ordinaire, se
trouva surpris d'une forte colique, en travaillant l'après-dînée avec,
le P. Tellier à la distribution des bénéfices. Il ne put entendre qu'une
messe basse le jour de Pâques, et fut cinq ou six jours à ne voir
presque personne, au bout desquels il n'y parut plus.

Marseille vaquait, dont le frère du comte de Luc avait été évêque
longtemps, qui a voit passé à Aix, d'où il est enfin venu à Paris où il
a succédé immédiatement au cardinal de Noailles, sans en rien retracer,
aussi était-ce pour tout le contraire qu'il y fut mis. À Marseille le
roi nomma l'abbé de Belsunce, fils d'une soeur de M. de Lauzun. C'était
un saint prêtre, nourrisson favori du P. Tellier, qui avait été
longtemps jésuite, et que les jésuites mirent hors de chez eux dans
l'espérance de s'en servir plus utilement, en quoi ils ne se trompèrent
pas. Il était trop saint et trop borné, trop ignorant et trop incapable
d'apprendre pour leur faire le moindre honneur, ni le plus léger profit.
Évêque, il imposa avec raison par la pureté de ses moeurs, par son zèle,
par sa résidence et son application à son diocèse, et y devint illustre
par les prodiges qu'il y fit dans le temps de la peste, et après par le
refus de l'évêché de Laon, pour ne pas quitter sa première épouse.

Son aveuglement pour les jésuites, et son ignorance qui parut profonde à
surprendre, le livra avec fureur à la constitution\footnote{Le mot
  \emph{constitution} signifie, dans ce passage, une décision des papes.
  Il s'agit de la constitution \emph{Unigenitus}, sur laquelle
  Saint-Simon revient souvent et avec de longs détails.} dont il pensa
être cardinal. Mais au fait et au prendre, il fallait aux Romains et aux
Jésuites un homme dans cette dignité dont ils pussent faire un autre
usage que de dire ce qu'ils lui auraient soufflé à mesure, et de signer
avec abandon tout ce qu'ils lui auraient présenté. Si un homme aussi pur
d'intention et aussi distingué par tout ce que je viens de dire, avait
pu se déshonorer, il l'aurait été par son fanatisme sur la constitution,
par les écrits étranges en tout sens qu'il adopta et signa comme siens,
et surtout par le personnage indigne en lui, infâme en tout autre, qu'il
fit en ce brigandage d'Embrun\footnote{Le concile d'Embrun, présidé par
  l'archevêque d'Embrun, Guérin de Tencin, condamna Soanen, évêque de
  Sénez, le 29 septembre 1727. On peut consulter sur cette question
  \emph{l'Histoire du concile d'Embrun}, publiée en 1728 et composée par
  un défenseur de Soanen, et le \emph{Journal du même concile}, publié
  en 1727 par un partisan de Tencin.}.

M. de Lauzun fut aussi aise de l'épiscopat de son neveu que l'aurait pu
être le plus petit bourgeois, tant les plus petites choses qui avaient
l'air de grâces lui étaient sensibles.

\hypertarget{chapitre-ix.}{%
\chapter{CHAPITRE IX.}\label{chapitre-ix.}}

1709

~

{\textsc{Mort de M. le Prince\,; son caractère.}} {\textsc{-
M\textsuperscript{lle} de Tours chassée de chez M\textsuperscript{me} la
princesse de Conti, fille de M. le Prince, par ordre du roi, obtenu par
le P. Tellier.}} {\textsc{- Ducs et princes et leurs femmes font leurs
visites sur la mort de M. le Prince en manteaux et en mantes, par ordre
du roi, et l'exécutent d'une manière ridicule.}} {\textsc{- Eau bénite
de M. le Prince.}} {\textsc{- Époque de l'entrée des domestiques des
princes du sang dans le carrosse du roi.}} {\textsc{- Suite de cette
usurpation.}} {\textsc{- Autre entreprise.}} {\textsc{- Autre
nouveauté.}} {\textsc{- Grand dégoût au duc de Bouillon.}} {\textsc{- Le
corps de M. le Prince conduit à Valery par M. de Fréjus, depuis cardinal
de Fleury, et reçu par l'archevêque de Sens, en présence de M. le Duc et
de ses seuls domestiques.}} {\textsc{- Service à Notre-Dame en présence
des cours supérieures.}} {\textsc{- Ducs parents invités.}} {\textsc{-
Cardinal de Noailles officiant, se retire à la sacristie après
l'évangile, parce que la parole fut adressée à M. le Duc à l'oraison
funèbre.}} {\textsc{- Méchanceté atroce de M. le Duc sur moi absent.}}
{\textsc{- Le roi ni les fils de France ne visitent
M\textsuperscript{me} la princesse de Conti ni M\textsuperscript{me} la
Princesse qu'à Versailles.}} {\textsc{- Progression des biens de la
maison de Condé.}} {\textsc{- M. le Duc ne change point de nom.}}

~

M. le Prince, qui depuis plus de deux ans ne paraissait plus à la cour,
mourut à Paris un peu après minuit, la nuit du dimanche de Pâques au
lundi, dernier mars et 1er avril, en sa soixante-sixième année.

C'était un petit homme très mince et très maigre, dont le visage d'assez
petite mine ne laissait pas d'imposer par le feu et l'audace de ses
yeux, en un composé des plus rares qui se soit guère rencontré. Personne
n'a eu plus d'esprit et de toutes sortes d'esprit, ni rarement tant de
savoir en presque tous les genres, et pour la plupart à fond, jusqu'aux
arts et aux mécaniques, avec un goût exquis et universel. Jamais, encore
une valeur plus franche et plus naturelle, ni une plus grande envie de
faire\,; et quand il voulait plaire, jamais tant de discernement, de
grâces, de gentillesse, de politesse, de noblesse, tant d'art caché
coulant comme de source. Personne aussi n'a jamais porté si loin
l'invention, l'exécution, l'industrie, les agréments ni la magnificence
des fêtes, dont il savait surprendre et enchanter, et dans toutes les
espèces imaginables.

Jamais aussi tant de talents inutiles, tant de génie sans usage, tant et
si continuelle et, si vive imagination, uniquement propre à être son
bourreau et le fléau des autres\,; jamais tant d'épines et de danger
dans le commerce, tant et de si sordide avarice, et de manéges bas et
honteux, d'injustices, de rapines, de violences\,; jamais encore tant de
hauteur, de prétentions sourdes, nouvelles, adroitement conduites, de
subtilités d'usages, d'artifices à les introduire imperceptiblement,
puis de s'en avantager, d'entreprises hardies et inouïes, de conquêtes à
force ouverte\,; jamais en même temps une si vile bassesse, bassesse
sans mesure aux plus petits besoins, ou possibilité d'en avoir\,; de là
dette cour rampante aux gens de robe et des finances, aux commis et aux
valets principaux, cette attention servile aux ministres, ce raffinement
abject de courtisan auprès du roi, de là encore ses hauts et bas
continuels avec tout le reste. Fils dénaturé, cruel père, mari terrible,
maître détestable, pernicieux voisin, sans amitié, sans amis, incapable
d'en avoir, jaloux, soupçonneux, inquiet sans aucun relâche, plein de
manèges et d'artifices à découvrir et à scruter tout, à quoi il était
occupé sans cesse aidé d'une vivacité extrême et d'une pénétration
surprenante, colère et d'un emportement à se porter aux derniers excès
même sur des bagatelles, difficile en tout à l'excès, jamais d'accord
avec lui-même, et tenant tout chez lui dans le tremblement\,; à tout
prendre, la fougue, et l'avarice étaient ses maîtres qui le
gourmandaient toujours. Avec cela un homme dont on avait peine à se
défendre quand il avait entrepris d'obtenir par les grâces, le tour, la
délicatesse de l'insinuation et de la flatterie, l'éloquence naturelle
qu'il employait, mais parfaitement ingrat des plus grands services, si
la reconnaissance ne lui était utile à mieux.

On a vu (t. III, p.~60), sur Rose, ce qu'il savait faire à ses voisins
dont il voulait les terres, et la gentillesse du tour des renards.
L'étendue qu'il sut donner à Chantilly et à ses autres terres, par de
semblables voies, est incroyable, aux dépens de gens qui n'avaient ni
l'audace de Rose ni sa familiarité avec le roi\,; et la tyrannie qu'il y
exerçait était affreuse. Il déroba pour rien, à force de caresses et de
souplesses, la capitainerie de Senlis et de la forêt d'Hallastre, dans
laquelle Chantilly est compris, à mon oncle et à la marquise de
Saint-Simon, alors fort vieux, qui, en premières noces, était, comme je
l'ai dit ailleurs, veuve de son grand-oncle, frère de la connétable de
Montmorency, sa grand'mère. Il leur fit accroire que le roi allait
supprimer ces capitaineries éloignées des maisons royales\,; qu'ils
perdraient celles-là qui, entre ses mains, serait conservée. Ils
donnèrent dans le panneau et la lui cédèrent. Le roi n'avait pas pensé à
en supprimer pas une. M. le Prince leur fit urne galanterie de deux
cents pistoles\,: et se moqua de leur crédulité\,; mais, à la vérité,
tant qu'ils vécurent, il les laissa, et même leurs gens, maîtres de la
chasse, comme ils l'étaient auparavant. Dès qu'elle fut entre ses mains
il ne cessa de l'étendre de ruse et de force, et de réduire au dernier
esclavage tout ce qui y était compris, et ce fut un pays
immense\footnote{Passage supprimé dans les précédentes éditions depuis
  \emph{Il leur fit accroire}.}.

Il n'eut les entrées chez le roi, et encore non les plus grandes,
qu'avec les survivances de sa charge et de son gouvernement pour son
fils, en le mariant à la bâtarde du roi\,; et tandis que, à ce titre de
gendre et de belle-fille, son fils et sa fille étaient, entre le souper
du roi et son coucher, dans son cabinet avec lui, les autres légitimités
et la famille royale, il dormait le plus souvent sur un tabouret au coin
de la porte, où je l'ai maintes fois vu ainsi attendant avec tous les
courtisans que le roi vint se déshabiller.

La duchesse du Maine le tenait en respect\,; il courtisait M. du Maine
qui lui rendait peu de devoirs, et qui le méprisait.
M\textsuperscript{me} la Duchesse le mettait au désespoir, entre le
courtisan et le père, sur lequel le courtisan l'emportait presque
toujours.

Sa fille mariée avait doucement secoué le joug. Celles qui ne l'étaient
pas le portaient dans toute sa pesanteur\,; elles regrettaient la
condition des esclaves. M\textsuperscript{lle} de Condé en mourut, de
l'esprit, de la vertu et du mérite de laquelle on disait merveilles.

M\textsuperscript{lle} d'Enghien, laide jusqu'au dégoût, et qui n'avait
rien du mérite de M\textsuperscript{lle} de Condé, lorgna longtemps,
faute de mieux, le mariage de M. de Vendôme, aux risques de sa santé et
de bien d'autres considérations. M. et M\textsuperscript{me} du Maine,
de pitié, et aussi par intérêt de bâtardise, se mirent en tête de le
faire réussir. M. le Prince le regardait avec indignation. Il sentait la
honte du double mariage de ses enfants avec ceux du roi, mais il en
avait tiré les avantages. Celui-ci ne l'approchait point du roi, et ne
pouvait lui rien produire d'agréable. Il n'osait aussi le dédaigner, à
titre de bâtardise, beaucoup moins résister au roi, si poussé par M. du
Maine, il se le mettait en gré, tellement qu'il prit le parti de la
fuite, et de faire le malade près de quinze mois avant qu'il le devint
de la maladie dont il mourut, et ne remit jamais depuis les pieds à la
cour, faisant toujours semblant d'y vouloir aller, pour s'y faire
attendre, et cependant gagner du temps, et n'être pas pressé.

M. le prince de Conti, qui lui rendait bien plus de devoirs que M. le
Duc, et dont l'esprit était si aimable, réussissait auprès de lui mieux
que nul autre, mais il n'y réussissait pas toujours. Pour M. le Duc ce
n'était que bienséance. Ils se craignaient tous deux\,: le fils, un père
fort difficile et plein d'humeur et de caprices\,; le père, un gendre du
roi\,; mais souvent le pied ne laissait pas de glisser au père, et ses
sorties sur son fils étaient furieuses.

M\textsuperscript{me} la Princesse était sa continuelle victime. Elle
était également laide, vertueuse et sotte\,; elle était un peu bossue,
et avec cela un gousset fin qui se faisait suivre à la piste, même de
loin. Toutes ces choses n'empêchèrent pas M. le prince d'en être jaloux
jusqu'à la fureur, et jusqu'à sa mort. La piété, l'attention infatigable
de M\textsuperscript{me} la Princesse, sa douceur, sa soumission de
novice, ne la purent garantir ni des injures fréquentes ni des coups de
pied et de poing qui n'étaient pas rares. Elle n'était pas maîtresse des
plus petites choses\,; elle n'en osait demander ni proposer aucune. Il
la faisait partir à l'instant que la fantaisie lui en prenait pour aller
d'un lieu à un autre. Souvent montée en carrosse, il l'en faisait
descendre, ou revenir du bout de la rue, puis recommençait l'après-dînée
ou le lendemain. Cela dura une fois quinze jours de suite pour un voyage
de Fontainebleau. D'autres fois, il l'envoyait chercher à l'église, lui
faisait quitter la grand'messe, et quelquefois la mandait au moment
qu'elle allait communier\,; et il fallait revenir à l'instant, et
remettre sa communion à une autre fois. Ce n'était pas qu'il eût besoin
d'elle, ni qu'elle osât faire la moindre démarche, ni celle-là même sans
sa permission\,; mais les fantaisies étaient continuelles.

Lui-même était toujours incertain. Il avait tous les jours quatre dîners
prêts\,: un à Paris, un à Écouen, un à Chantilly, un où la cour était.
Mais la dépense n'en était pas forte\,: c'était un potage, et la moitié
d'une poule rôtie sur une croûte de pain, dont l'autre moitié servait
pour le lendemain.

Il travaillait tout le jour à ses affaires, et courait Paris pour la
plus petite. Sa maxime était de prêter et d'emprunter tant qu'il pouvait
aux gens du parlement pour les intéresser eux-mêmes dans ses affaires,
et avoir occasion de se les dévouer par ses procédés avec eux\,; aussi
était-il bien rare qu'il ne réussît dans toutes celles qu'il
entreprenait, pour lesquelles il n'oubliait ni soins ni sollicitations.

Toujours enfermé chez lui, et presque point visible à la cour comme
ailleurs, hors les temps de voir le roi ou les ministres, s'il avait à
parler à ceux-ci, qu'il désespérait alors par ses visites allongées et
redoublées. Il ne donnait presque jamais à manger et ne recevait
personne à Chantilly, où son domestique et quelques jésuites savants lui
tenaient compagnie, très rarement d'autres gens\,; mais quand il faisait
tant que d'y en convier, il était charmant. Personne au monde n'a jamais
si parfaitement fait les honneurs de chez soi\,; jusqu'au moindre
particulier ne pouvait être si attentif. Aussi cette contrainte, qui
pourtant ne paraissait point, car toute sa politesse et ses soins
avaient un air d'aisance et de liberté merveilleuse, faisait qu'il n'y
voulait personne.

Chantilly était ses délices. Il s'y promenait toujours suivi de
plusieurs secrétaires avec leur écritoire et du papier, qui écrivaient à
mesure ce qui lui passait par l'esprit pour raccommoder et embellir. Il
y dépensa des sommes prodigieuses, mais qui ont été des bagatelles en
comparaison des trésors que son petit-fils y a enterrés et {[}des{]}
merveilles qu'il y a faites.

Il s'amusait assez aux ouvrages d'esprit et de science, il en lisait
volontiers et en savait juger avec beaucoup de goût, de profondeur et de
discernement. Il se divertissait aussi quelquefois à des choses d'arts
et de mécaniques auxquelles il se connaissait très bien.

Autrefois il avait été amoureux de plusieurs dames de la cour, alors
rien ne lui coûtait. C'était les grâces, la magnificence, la galanterie
même, un Jupiter transformé en pluie d'or. Tantôt il se travestissait en
laquais, une autre fois en revendeuse à la toilette, tantôt d'une autre
façon. C'était l'homme du monde le plus ingénieux. Il donna une fois une
fête au roi, qu'il cabala pour se la faire demander, uniquement pour
retarder un voyage en Italie d'une grande dame qu'il aimait et avec
laquelle il était bien, et dont il amusa le mari à faire les vers. Il
perça tout un côté d'une rue près de Saint-Sulpice par les maisons,
l'une dans l'autre, qu'il loua toutes et les meubla pour cacher ses
rendez-vous.

Jaloux aussi et cruellement de ses maîtresses, il eut entre autres la
marquise de Richelieu, que je nomme parce qu'elle ne vaut pas la peine
d'être tue. Il en était éperdument amoureux, et dépensait des millions
pour elle et pour être instruit de ses déportements. Il sut que le comté
de Roucy partageait ses faveurs (et c'est elle à qui ce spirituel comte
proposait bien sérieusement de faire mettre du fumier à sa porte pour la
garantir du bruit des cloches dont elle se plaignait). M. le Prince
reprocha le comte de Roucy à la marquise de Richelieu qui s'en défendit
fort. Cela dura quelque temps. Enfin, M. le Prince, outré d'amour,
d'avis certains et de dépit, redoubla ses reproches, et les prouva si
bien qu'elle se trouva prise. La frayeur de perdre un amant si riche et
si prodigue lui fournit sur-le-champ un excellent moyen de lui mettre
l'esprit en repos. Elle lui proposa de donner, de concert avec lui, un
rendez-vous chez elle au comte de Roucy, où M. le Prince aurait des gens
apostés pour s'en défaire. Au lieu du succès qu'elle se promettait d'une
proposition si humaine et si ingénieuse, M. le Prince en fut tellement
saisi d'horreur qu'il en avertit le comte de Roucy, et ne la revit de sa
vie.

Ce qui ne se peut comprendre, c'est qu'avec tant d'esprit, d'activité,
de pénétration, de valeur et d'envie de faire et d'être, un aussi grand
homme à la guerre que l'était M. son père n'ait jamais pu lui faire
comprendre les premiers éléments de ce grand art. Il en fit longtemps
son étude et son application principale\,; le fils y répondit par la
sienne, sans que jamais il ait pu acquérir la moindre aptitude à aucune
des parties de la guerre, sur laquelle M. son père ne lui cachait rien,
et lui expliquait tout à la tête des armées. Il l'y eut toujours avec
lui, voulut essayer de le mettre en chef, y demeurant néanmoins pour lui
servir de conseil, quelquefois dans les places voisines, et à portée,
avec la permission du roi, sous prétexte de ses infirmités. Cette
manière de l'instruire ne lui réussit pas mieux que les autres. Il
désespéra d'un fils doué pourtant de si grands talents, et il cessa
enfin d'y travailler, avec toute la douleur qu'il est aisé d'imaginer.
Il le connaissait et le connut de plus en plus\,; mais la sagesse
contint le père, et le fils était en respect devant cet éclat de gloire,
qui environnait le grand Condé.

Les quinze ou vingt dernières années de la vie de celui dont on parle
ici furent accusées de quelque chose de plus que d'emportement et de
vivacité. On crut y remarquer des égarements, qui ne demeurèrent pas
tous renfermés dans sa maison. Entrant un matin chez la maréchale de
Noailles, dans son appartement de quartier, qui me l'a conté, comme on
faisait son lit et qu'il n'y avait plus que la courte pointe à y mettre,
il s'arrêta un moment à la porte, où s'écriant avec transport\,:
«\,Ah\,! le bon lit, le bon lit\,!» prit sa course, sauta dessus, se
roula dessus sept ou huit tours en tous les sens, puis descendit et fit
excuse à la maréchale, et lui dit que son lit était si propre et si bien
fait, qu'il n'y avait pas moyen de s'en empêcher, et cela sans qu'il y
eût jamais rien eu entre eux, et dans un âge où la maréchale, qui avait
toute sa vie été hors de soupçon, n'en pouvait laisser naître aucun. Ses
gens demeurèrent stupéfaits, et elle bien autant qu'eux. Elle en sortit
adroitement par un grand éclat de rire et par plaisanter.

On disait tout bas qu'il y avait des temps où tantôt il se croyait
chien, tantôt quelque autre bête dont alors il imitait les façons\,; et
j'ai vu des gens très dignes de foi qui m'ont assuré l'avoir vu au
coucher du roi pendant le prier-Dieu, et lui cependant près du fauteuil,
jeter la tête en l'air subitement plusieurs fois de suite, et ouvrir la
bouche toute grande comme un chien qui aboie, mais sans faire de bruit.
Il est certain qu'on était des temps considérables sans le voir, même
ses plus familiers domestiques, hors un seul vieux valet de chambre qui
avait pris, empire sur lui, et qui ne s'en contraignait pas.

Dans les derniers temps de sa vie, et même la dernière année, il n'entra
et ne sortit rien de son corps qu'il ne le vît peser lui-même, et qu'il
n'en écrivit la balance, d'où il résultait des dissertations qui
désolaient ses médecins.

La fièvre et la goutte l'attaquèrent à reprises. Il augmenta son mal par
son régime trop austère, par une solitude où il ne voulait voir
personne, même le plus souvent de sa plus intime famille, par une
inquiétude et des précisions qui le jetèrent dans des transports de
fureur.

Finot, son médecin, et le nôtre de tout, temps et de plus notre ami, ne
savait que devenir avec lui. Ce qui l'embarrassa le plus, à ce qu'il
nous a confié plus d'une fois, fut que M. le Prince ne voulut plus rien
prendre, dit qu'il était mort, et pour toute raison que les morts ne
mangeaient point. Si fallait-il pourtant qu'il prît quelque nourriture
ou qu'il mourût véritablement. Jamais on ne put lui persuader qu'il
vivait, et que, par conséquent, il fallait qu'il mangeât. Enfin, Finot
et un autre médecin qui le voyait le plus ordinairement avec lui,
s'avisèrent de convenir, qu'il était mort, mais de lui soutenir qu'il y
avait dés morts qui mangeaient. Ils offrirent de lui en produire, et en
effet ils lui amenèrent quelques gens sûrs et bien recordés qu'il ne
connaissait point et qui firent les morts tout comme lui, mais qui
mangeaient. Cette adresse le détermina, mais il ne voulait manger
qu'avec eux et avec Finot. Moyennant cela, il mangea très bien, et cette
fantaisie dura assez longtemps, dont l'assiduité désespérait Finot, qui
toutefois mourait de rire en nous racontant ce qui se passait, et les
propos de l'autre monde qui se tenaient à ces repas. Il vécut encore
longtemps après.

Sa maladie augmentant, M\textsuperscript{me} la Princesse se hasarda de
lui demander s'il ne voulait point penser à sa conscience et voir
quelqu'un\,; il se divertit assez longtemps à la rebuter. Il y avait
déjà quelques mois qu'il voyait le P. de La Tour en cachette, le même
général de l'Oratoire qui avait assisté M\textsuperscript{lle} de Condé
et M. le prince de Conti. Il avait envoyé proposer à ce père de le venir
voir en bonne fortune, la nuit et travesti. Le messager fut un
sous-secrétaire, confident unique de ce secret. Le P. de La Tour,
surpris au dernier point d'une proposition si sauvage, répondit que le
respect qu'il devait à M. le Prince l'en gagerait à le voir avec toutes
les précautions qu'il voudrait lui imposer, mais que, quelque justice
qu'il eût droit d'attendre de sa maison, il ne pouvait dans son état et
dans sa place consentir à se travestir, ni à quitter le frère qui
l'accompagnait toujours, mais qu'avec son habit et ce frère tout lui
serait bon, pourvu encore qu'il rentrât à l'Oratoire avant qu'on y fût
retiré. M. le Prince passa ces conditions. Quand il le voulait voir, ce
sous-secrétaire allait à l'Oratoire, s'y mettait dans un carrosse de
remise avec le général et son compagnon, les menait à une petite porte
ronde d'une maison qui répondait à l'hôtel de Condé, et par de longs et
d'obscurs détours, souvent la lanterne à la main et une clef dans une
autre, qui ouvrait et fermait sur eux un grand nombre de portes, le
conduisait jusque dans la chambre de M. le Prince. Là, tête à tête avec
lui, {[}le P. de La Tour{]} quelquefois le confessait, le plus souvent
l'entretenait. Quand M. le Prince en avait pris sa suffisance ou que
l'heure pressait, car il le retenait souvent longtemps, le même homme
rentrait dans la chambre, et le remenait par les mêmes détours jusqu'au
carrosse où le frère les attendait, et de là à l'Oratoire de
Saint-Honoré.

C'est le P. de La Tour qui me l'a conté depuis, et la surprise et la
joie de M\textsuperscript{me} la Princesse, quand M. le Prince lui
apprit enfin qu'il le voyait ainsi depuis quelques mois. Alors il n'y
eut plus de mystère\,; le P. de La Tour fut mandé à découvert, et se
rendit assidu pendant le peu de semaines que M. le Prince vécut depuis.

Les jésuites y furent cruellement trompés. Ils se croyaient en
possession bien assurée d'un prince élevé chez eux, qui leur avait donné
son fils unique dans leur collège, qui n'avait qu'eux à Chantilly et
toujours pour compagnie, qui vivait avec eux en entière familiarité.
Leur P. Lucas, homme dur, rude, grossier, quoique souvent supérieur dans
leurs maisons, était son confesseur en titre, qui véritablement ne
l'occupait guère, mais qu'il envoya chercher dans une chaise de poste,
jusqu'à Rouen, tous les ans, à Pâques, où il était recteur. Ce père y
apprit son extrémité, arriva là-dessus par les voitures publiques, et ne
put ni le voir ni se faire payer son voyage. L'affront leur parut
sanglant. M. le Prince pratiqua ainsi ce que j'ai rapporté que le
premier président Harlay dit un jour aux jésuites et aux pères de
l'Oratoire en face, qui étaient ensemble chez lui pour une affaire, en
les reconduisant devant tout le monde\,: «\,Qu'il est bon,\,» se
tournant aux jésuites, «\,de vivre avec vous, mes pères\,!» et tout de
suite se tournant aux pères de l'Oratoire\,: «\,et de mourir avec vous,
mes pères\,!»

Cependant la maladie augmenta rapidement et devint extrême. Les médecins
le trouvèrent si mal la nuit de Pâques qu'ils lui proposèrent les
sacrements pour le lendemain. Il disputa contre eux, puis leur dit qu'il
les voulait donc recevoir tout à l'heure, que ce serait chose faite, et
qui le délivrerait du spectacle qu'il craignait. À leur tour, les
médecins disputèrent sur l'heure indue, et que rien ne pressait si fort.
À la fin, de peur de l'aigrir, ils consentirent. On envoya à l'Oratoire
et à la paroisse, et il reçut ainsi brusquement les derniers sacrements.
Fort peu après, il appela M. le Duc qui pleurait, régla tout avec lui et
avec M\textsuperscript{me} la Princesse, la congédia avec des marques
d'estime et d'amitié, et lui dit où était son testament. Il retint M. le
Duc, avec qui il ne s'entretint plus que des honneurs qu'il voulait à
ses obsèques, des choses omises à celles de M. son père qu'il ne fallait
pis oublier aux siennes, et même y prendre bien garde\,; répéta
plusieurs fois qu'il ne craignait point la mort, parce qu'il avait
pratiqué la maxime de M. son père que pour n'appréhender point les
périls de près, il fallait s'y accoutumer de loin\,; consola son fils,
ensuite l'entretint des beautés de Chantilly, des augmentations qu'il y
avait projetées, des bâtiments qu'il y avait commencés exprès pour
obliger à les achever après lui, d'une grande somme d'argent comptant
destinée à ces dépenses et du lieu où elle était\,; et persévéra dans
ces sortes d'entretiens jusqu'à ce que la tête vînt à se brouiller. Le
P. de La Tour et Pinot étaient cependant retirés à un coin de la
chambre, de qui j'ai appris ce détail. Ce prince laissa une grande idée
de sa fermeté, et une bien triste de l'emploi de ses dernières heures.

Finissons par un trait de Verrillon, que tout le monde a tant connu, et
qui était demeuré avec lui après avoir été à M. son père sur un pied
d'estime et de considération. Pressé un jour à Chantilly d'acheter une
maison qui en était fort proche\,: «\,Tant que j'aurai l'honneur de vos
bonnes grâces, dit-il à M. le Prince, je ne saurais être trop près de
vous ainsi je préfère ma chambre ici à un petit château au voisinage\,;
et si j'avais le malheur de les perdre, je ne saurais être trop loin de
vous\,: ainsi, la terre d'ici près, m'est fort inutile.\,»

Qui que ce soit, ni domestiques, ni parents, ni autres ne regretta M. le
Prince, que M. le Duc que le spectacle toucha un moment, et qui se
trouva bien affranchi, et M\textsuperscript{me} la Princesse, qui eut
honte de ses larmes jusqu'à en faire excuse dans son particulier.
Quoique ses obsèques aient duré longtemps, achevons-les tout de suite
pour n'avoir plus à y revenir\,: l'extrême singularité d'un homme si
marqué m'a paru digne d'être rapportée\,; mais n'oublions pas la
vengeance des jésuites qui fut le coup d'essai du P. Tellier.

Ils venaient de manquer M\textsuperscript{me} de Condé, tout
nouvellement M. le prince de Conti\,; et M. le Prince, après avoir
toujours été à eux lorsqu'il s'était confessé, leur échappait à la mort.
Ne pouvant se prendre aux princes ni aux princesses du sang, et
toutefois voulant un éclat qui intimidât les familles, ils se ruèrent
sur M\textsuperscript{lle} de Tours\,; c'était une demoiselle d'Auvergne
sans aucun bien\,; qui avait beaucoup de mérite, d'esprit et de piété.
Elle avait vécu chez M\textsuperscript{me} de Montgon jusqu'à sa mort,
parce qu'elle était parente de son mari\,; elle s'y était fait connaître
et considérer de beaucoup de dames de la cour\,; elle espérait même
obtenir de quoi vivre par M\textsuperscript{me} de Maintenon lorsqu'elle
perdit M\textsuperscript{me} de Montgon. Elle fit alors pitié à tout le
monde, on en parla à M\textsuperscript{me} la princesse de Conti, fille
de M. le Prince qui la retira auprès d'elle. Sa vertu la rendit suspecte
aux jésuites, à qui l'hôtel de Conti l'était déjà de tout temps, à cause
de l'ancien chrême du vieux hôtel de Conti, qui en effet s'était un peu
communiqué à celui-ci, même à celui de la fille du roi.
M\textsuperscript{lle} de Tours fut donc accusée d'avoir introduit le P.
de La Tour auprès du prince de Conti, et ensuite par
M\textsuperscript{me} la Princesse et M\textsuperscript{me} la princesse
sa fille auprès de M. le Prince. Bien que justifiée avec chaleur par
M\textsuperscript{me} la princesse de Conti sur ces deux points, rien ne
la put garantir. M\textsuperscript{me} la princesse de Conti eut ordre
précis de la mettre hors de chez elle. La pauvre fille, outre tout ce
qu'elle y perdait, ne savait où se retirer. Pas un couvent dans Paris
qui osât la recevoir, point d'amie qui crût s'y pouvoir commettre. La
province, où et comment\,? Au bout de quelques jours les jésuites,
impatients de lavoir encore à l'hôtel de Conti, et plus encore du bruit
que cette violence faisait, eurent un ordre de la recevoir pour le
couvent qu'elle choisirait. M\textsuperscript{me} la princesse de Conti
lui continua la pension qu'elle lui avait donnée, et au bout de quelques
années obtint la permission de la reprendre chez elle, où elle est
demeurée jusqu'à sa mort. Outre qu'il n'y avait aucun prétexte à ce
traitement, les jésuites ne prirent seulement pas la peiné d'en
chercher, et voulurent que le crime imputé d'avoir introduit le P. de La
Tour pour assister ces princes fût la matière connue et seule de la
punition.

Dès que M. le Prince fut mort, Espinac, capitaine des gardes de M. le
Duc comme gouverneur de Bourgogne, le fut dire au roi de sa part, qui le
même jour envoya le duc de Tresmes faire compliment de sa part à la
famille, sur ce que Villequier, depuis duc d'Aumont et premier
gentilhomme de la chambre aussi, y avait été envoyé à la mort de feu M.
le Prince, père de celui-ci. Le jeudi, 4 avril, M. le Duc vint à
Versailles.

On se souviendra de la prétention nouvelle des princes du sang de
s'égaler aux fils et petits-fils de France pour les visites en manteau
long aux occasions de grands deuils de famille, et qu'à la mort de
M\textsuperscript{me} d'Armagnac, l'année précédente, comme je l'ai
rapporté alors, ils firent par les bâtards associés en tout à leur rang,
que M. le Grand eût commandement du roi que ses enfants le visitassent
en manteau long, ce qu'ils furent obligés de subir. M. le Grand
n'échappa pour sa personne que parce que les maris veufs ne vont point
que chez le roi. À la mort de M. le prince de Conti, M. le Duc prétendit
la même chose, interprétant l'ordre du roi des deuils actifs et
passifs\,; mais personne, ducs, princes ni autres, ne voulut prendre de
manteau, et le roi, qui sentait la nouveauté de la prétention, et qui ne
voulut pourtant pas décider contre les princes du sang, les lassa sans
rien ordonner, tellement que M. le Duc qui s'en aperçut déclara que M.
le prince de Conti était incommodée et fort fatigué\,;
M\textsuperscript{me} la princesse de Conti, trop affligée,
M\textsuperscript{lle}s ses filles trop assidûment auprès d'elle pour
recevoir personne, et qu'ils ne verraient qui que ce soit.

Six semaines après la mort de M. le Prince, prévue et arrivée, il n'y
eut pas lieu à tergiverser davantage. M. le duc, arrivant à Versailles
trois jours après, fit publier qu'ils recevraient le lendemain les
visites, mais personne sans manteau\,; ce fut afficher en vain\,; il
attendit tout le vendredi, ainsi que le prince de Conti et M. du Maine,
chacun dans leur appartement, sans que personne s'y présentât, sinon
deux ou trois hommes non titrés qui furent refusés, parce qu'ils étaient
sans manteau. M. le Duc s'était trop commis pour reculer. Il fit par M.
du Maine qui en partageait l'honneur avec lui, que le roi envoyât sur la
fin de cette journée M. le comte de Toulouse chez eux en grand manteau,
après quoi il compta que cela irait tout de suite, mais il fallut encore
un ordre qui fut négocié le soir, et que le roi donna le lendemain à M.
de Beauvilliers pour les dues, et à M. le Grand pour les princes,
ajoutant que M. le comte de Toulouse y ayant été en manteau, il n'y
avait plus de difficulté. La réponse était bien aisée, qui est le
réciproque, mais les fils de France et M. le duc d'Orléans, qui y
perdaient cette distinction d'avec les princes du sang, n'osant souffler
de peur des bâtards, ducs et princes n'eurent qu'à se taire.

Tous y allèrent donc le samedi après midi, mais tous comme de concert,
hommes et femmes, d'une manière si indécente qu'elle tint fort de
l'insulte. On affecta généralement des cravates de dentelles au lieu de
rabats de deuil et des collerettes de même sous les mantes, et des
rubans de couleur dans la tête\,; les hommes, des bas de couleur blancs
ou rouges, peu même de bruns, des perruques nouées et poudrées blanc, et
les deux sexes des gants blancs, et les dames bordés de couleur\,: en un
mot, une franche mascarade. La manière d'entrer et de sortir fut tout
aussi ridicule, à peine faisait-on la révérence en entrant, on ne disait
mot, on se regardait les uns les autres en riant\,; un moment après on
sortait\,; ducs et princes se laissaient conduire jusqu'à la galerie par
les princes du sang, sans leur dire une parole\,; leurs femmes de même
par les princesses jusqu'à l'antichambre\,; souvent on jetait son
manteau avant qu'ils fussent hors de vue, et ces manteaux qu'on ne
prenait qu'en entrant, on les mettait tout de travers\,; les princes du
sang le sentirent vivement, mais, contents de leur victoire, n'osèrent
rien dire en cette introduction\,; ils eurent même tant de peur qu'on ne
s'excusât faute de manteaux qu'il y en avait des piles à leur porte,
qu'on présentait et qu'on reprenait avec toutes sortes de respect et
sans rien demander. Personne n'y alla ensemble\,; en un mot on fit du
pis qu'on put.

M. le duc d'Enghien était chez M. le Duc, qui crut montrer par là un
grand ménagement, pour ne pas faire aller chez lui à la ville. Les
princes du sang étaient en grand manteau et en rabat, dans tout
l'appareil lugubre, et les princesses du sang en mantes, tant que les
visites durèrent.

Le dimanche suivant le roi les alla voir, et M\textsuperscript{me} la
duchesse de Bourgogne ensuite, mais elle ne fut point chez les princes
ni aucunes dames. M\textsuperscript{me} la Duchesse, grosse de sept
mois, reçut toutes ses visites au lit, ayant M\textsuperscript{lle}s de
Bourbon et de Charolais dans sa chambre, en mantes, qui faisaient les
honneurs et qui ne reçurent point de visites chez elles.

M. le prince de Conti, sa queue portée par Pompadour et accompagné du
duc de Tresmes comme due, fut, le mardi 9 avril, donner l'eau bénite de
la part du roi, dont la cérémonie fut pareille à celle de feu M. le
prince de Conti que j'ai rapportée\,; il s'y vit deux nouvelles
usurpations, dont la première se hasarda à celle de M. le prince de
Conti et se confirma en celle-ci\,; c'est que La Noue, gouverneur de M.
le duc d'Enghien, monta dans le carrosse du roi et, s'y mit à la
portière. En celle-ci, celui de M. le prince de Conti en fit autant. On
a vu (t. Ier, p.~365) les différences des principaux domestiques des
fils et petits-fils de France d'avec ceux des princes du sang bien
expliquées et bien prouvées, et par faits, dont deux principales sont
que ces derniers n'entrent point dans les carrosses et ne mangent point
avec le roi, etc. Il en fut en cette occasion comme de la visite en
manteau. L'association des bâtards aux mêmes distinctions, rangs et
honneurs des princes du sang, empêcha les fils de France et M. le duc
d'Orléans de se plaindre. Les bâtards qui eurent à Marly, à table et
dans les carrosses leurs dames d'honneur, et à Marly chacun leurs
principaux domestiques, sans que les princes du sang, même gendres et
petits-fils, aient pu l'obtenir pour les leurs, ni M\textsuperscript{me}
la Princesse et M\textsuperscript{me} la princesse de Conti sa fille
pour leurs dames d'honneur, les bâtards, dis-je, n'osèrent rien dire en
cette occasion, la première où jamais domestique de prince du sang, même
chevalier de l'ordre, ait mis le pied dans les carrosses. Le roi, qui
sentit ce qu'il faisait pour ses enfants à cet égard, ne voulut rien
dire à chose faite, qui passa à la faveur de la jeunesse de ces princes
qu'on ne pouvait guère séparer de leur gouverneur. Mais cette
entreprise, qui ne fut pas répétée du vivant du roi, se déborda dans
tous les excès, lorsque après lui M. le Duc fut le maître, d'où il
résulta qu'il n'y eut plus de distinction, de bornes ni de mesures à
manger avec le roi et à entrer dans ses carrosses, une des grandes
sources de la confusion d'aujourd'hui.

L'autre entreprise, toute neuve à cette eau bénite et qui n'avait pas
été à la précédente, ni à pas une, fut que le prince de Conti, au lieu
de retourner dans le carrosse du roi reprendre le sien dans la cour des
Tuileries, où il l'avait quitté, se fit ramener dans le carrosse du roi
de l'hôtel de Condé droit chez lui. C'est ainsi qu'à chaque occasion,
entreprises nouvelles que le roi passait par divers égards, tous
réversibles à ses bâtards, sans que par cette même considération
personne, à commencer par les fils de France, osât représenter son
droit, son intérêt, l'usage continuel et la raison.

M. le Duc, piqué des manteaux contre les ducs, à qui il aima mieux s'en
prendre, n'en pria aucun pour l'accompagner, comme ses parents à
recevoir M. le prince de Conti à l'eau bénite\,; il invita les princes
de Tarente et de Rohan, le comte de Roucy et Blansac son frère et
Lassai, gendre bâtard de M. le Prince, dont les quatre premiers ne
furent pas contents. Apparemment que M. de Bouillon en avait été informé
d'avance, car il défendit au duc d'Albret, invité aussi, de s'y trouver,
qui envoya s'excuser sur cette défense. M. le Duc le prit avec tant de
hauteur qu'il obtint du roi un ordre à M. de Bouillon de lui aller faire
excuse.

M. de Fréjus, aujourd'hui cardinal Fleury et maître du royaume, dit les
oraisons à l'eau bénite, ce qui ne fut pas à M. le prince de Conti,
parce qu'il n'était pas premier prince du sang.

Tout ce qui avait été donner de l'eau bénite à M. le prince de Conti y
fut aussi à M. le Prince, et de plus le nonce à la tête de tous les
ambassadeurs, lesquels tous ensemble, et en manteaux longs, visitèrent
M. le Duc et M. le duc d'Enghien qui se trouva avec lui. Ces princes se
trouvèrent accompagnés de parents invités non ducs, comme à l'eau
bénite.

Il en usa de même au transport du coeur fait par l'évêque de Fréjus aux
jésuites de la rue Saint-Antoine, qui fut mis auprès de ceux des deux
derniers princes de Condé. Il crut apparemment de sa grandeur d'y avoir
des ducs et se ravisa. M. le Duc, qui alla l'y attendre, n'invita de
parents pour s'y trouver sans les y mener que les ducs de Ventadour, de
La Trémoille et de Luxembourg\,; il n'y eut rien de rangé aux jésuites,
et M. le Duc y évita tout lieu de préséance, parce qu'il y invita aussi
le prince Charles, fils de M. le Grand, le prince, de Montbazon, le
prince de Rohan, les comtes de Roucy et de Blansac avec Lassai.

Ainsi, la mort de M. le Prince est la première époque de l'invitation
des princes étrangers comme parents, avec des ducs qui, parents aussi,
l'avaient toujours été et jamais ces princes. Comme ce n'est pas le roi
qui nomme cet accompagnement, les ducs furent peu touchés d'une
préférence et d'une concurrence insipide qui ne touche en rien leur
naissance ni leur rang.

Le corps fut porté de l'hôtel de Condé droit à Valery, terre et
sépulture des derniers princes de Condé, auprès de Fontainebleau, en
grande pompe, où l'évêque de Fréjus le présenta à l'archevêque de Sens,
diocésain. Il ne s'y trouva que M. le Duc avec M. le duc d'Enghien et
leurs domestiques.

Qui eût dit alors à ces princes que M. le duc d'Enghien serait un jour
premier ministre les aurait bien surpris\,; qui les aurait assurés qu'il
en serait uniquement redevable à ce même évêque de Fréjus les aurait
étonnés bien davantage\,; qui leur aurait prédit qu'il serait chassé,
exilé et demeurerait le reste de sa vie écarté par ce même évêque, qui
prendrait sa place et la tiendrait avec toute-puissance, tout autrement
que lui, et que tout cela se ferait sans le plus léger obstacle, je
pense qu'à la fin ils se seraient moqués du prophète.

Tout se termina par un superbe service à Notre-Dame aux dépens du roi,
en présence des cours supérieures comme premier prince du sang. Le
cardinal de Noailles y officia, et le P. Gaillard, jésuite, fit
l'oraison funèbre, qui fut très mauvaise à ce que tout le monde
trouva\,; il y eut dispute à qui, du cardinal officiant ou de M. le Duc,
il adresserait la parole. À la fin le roi décida que ce servit à M. le
Duc, mais qu'aussitôt après l'évangile, le cardinal se retirerait à la
sacristie comme pour se reposer et ne reviendrait que l'oraison funèbre
achevée\,; les stalles de Notre-Dame firent qu'il ne s'agit de fauteuils
pour personne. M. le Duc envoya un gentilhomme en manteau long inviter
parents et qui il lui plut. Plusieurs ducs le furent.

Je le fus aussi\,: j'étais à la Ferté. M. le chancelier m'avait forcé,
moins par raison que par me le demander comme une marque d'amitié,
d'aller chez M. le Duc et M\textsuperscript{me} la Duchesse à la mort de
M. le prince de Conti\,: ainsi, en mon absence, M\textsuperscript{me} de
Saint-Simon fit sa visite à M\textsuperscript{me} la Duchesse, qui se
surpassa à la bien recevoir et les excuses de mon absence tant pour elle
que pour M. le Duc. J'étais à la Ferté à la mort de M. le Prince\,; je
me doutai bien qu'elle causerait des prétentions et du bruit, et je m'en
tins éloigné chez moi jusqu'à ce que tout fût fini, et même qu'on n'en
parlât plus pour n'être mêlé en rien. Ces précautions me furent
inutiles. J'appris à mon retour que M. le Duc, parlant au roi sur les
manteaux, avait eu la bonté de lui dire que c'était dommage de mon
absence, et que j'en ferais de bonnes là-dessus, si j'étais à la cour, à
quoi je sus aussi que le roi n'avait rien répondu. La vérité est que
j'en dis mon avis au chancelier sur la visite qu'il m'avait forcé de
faire, et du los que j'en recevais. Je m'en dépiquai tôt après.
M\textsuperscript{me} la Duchesse accoucha de M. le comte de Clermont.
Je ne fus ni chez M. le Duc ni chez elle, M\textsuperscript{me} de
Saint-Simon non plus, et je ne me contraignis pas de dire que je ne le
verrais de ma vie. En effet, je l'ai tenu très hautement.

Le roi ne voulut point aller à Paris, ni que les fils de France y
fussent voir M\textsuperscript{me} la princesse de Conti, ni
M\textsuperscript{me} la Princesse. M. le Duc y fit tous ses efforts et
y échoua. Le roi tint ferme, tellement qu'il fallût enfin qu'elles
vinssent à Versailles, où le roi les visita. Cette différence de Paris à
Versailles fut nouvelle pour les princes du sang, et les mortifia
beaucoup. Autrefois elle n'était pas même pour les duchesses, que la
reine, femme du roi, y allait voir de Saint-Germain à toutes les
occasions jusqu'à la mort du duc de Lesdiguières que la reine cessa
d'aller, et peu à peu les filles de France à son exemple, comme je l'ai
expliqué.

Le testament de M. le Prince brouilla son fils avec ses filles, et eut
de grandes suites qui se verront en leur temps. M. son grand-père
n'avait en tout de bien que douze mille livres de rente, lorsqu'il
épousa la fille du dernier connétable de Montmorency. Il sut en amasser
et profiter lestement de l'immense confiscation des biens du dernier duc
de Montmorency, exécuté à Toulouse en 1632. M. le Prince, son fils, et
père de celui dont nous parlons, ne gâta pas ses affaires, malgré les
dépenses des troubles qu'il excita, et de sa longue retraite en Flandre,
et il recueillit toute la riche succession de l'a maison de Maillé, par
la mort sans alliance du duc de Brézé son beau-frère, amiral de France
sous un autre nom\footnote{Armand, duc de Brézé-Fronsac, fut amiral de
  France sous le titre de \emph{surintendant général de la navigation},
  après la mort de son oncle le cardinal de Richelieu.}, tué devant
Orbitelle, en 1646, à vingt-sept ans. M. le Prince, son fils, avait
épousé une des plus riches héritières de l'Europe, et avait passé à
s'enrichir toute sa vie qu'on vient de voir finir. Outre les pierreries
et les meubles dont il laissa pour plusieurs millions, les augmentations
infinies de l'hôtel de Condé et de Chantilly, il jouissait avec
M\textsuperscript{me} la Princesse de un million huit cent mille livres
de rentes, y compris sa pension de cent cinquante mille livres de
premier prince du sang, sa charge de grand maître et son gouvernement.
M. le Duc, son fils, n'eut le temps de gâter ni d'augmenter.

M. le Duc, que nous avons vu premier ministre, puis remercié, et comme
retiré à Chantilly, où il est mort, et qui n'a rien eu de ses deux
femmes, a laissé deux millions quatre cent mille livres de rente, sans
le portefeuille qui est demeuré ignoré, et un amas prodigieux de raretés
de toute espèce, avec une très grande augmentation de pierreries\,; sa
dépense a été toujours plus que royale en tout genre, en maison, en
chasses, en table, en monde à Chantilly, en meubles somptueux, en
bâtiments et en ajustements immenses. Il n'avait pas plus du roi que M.
son grand-père il avait fallu prendre sur son bien les reprises et le
douaire de M\textsuperscript{me} sa mère qui le survit encore, et les
dots et partages de M\textsuperscript{me} la princesse de Conti, de
M\textsuperscript{me} du Maine et de M\textsuperscript{me} de Vendôme
ses tantes, de M\textsuperscript{me} la princesse de Conti, de
M\textsuperscript{me}s, de Saint-Antoine et de Beaumont, de
M\textsuperscript{lle}s de Charolais, de Clermont et de Sens ses soeurs,
et de MM. les comtes de Charolais et de Clermont ses frères. Il avait
dix-huit ans à la mort de M. son père, trente et un lorsqu'il fut
premier ministre, il ne l'a pas été tout à fait deux ans et demi, et il
est mort à Chantilly, son continuel séjour depuis, le 27 janvier 1740, à
quarante-huit ans. Il n'a rien conservé en se retirant à Chantilly de ce
qu'il avait eu comme premier ministre, ni des choses y jointes, qui
passèrent en même temps à M. de Fréjus\,; d'où on peut juger quels biens
il a amassés.

M. le Prince fut le dernier de cette branche qui ait porté ce nom\,; il
n'était premier prince du sang que de grâce, comme je l'ai dit lors de
la mort de Monsieur. M. le Duc conserva ce nom, et ne prit point celui
de M. son père\,; le roi le régla ainsi. À cette occasion il n'est
peut-être pas mal à propos de dire un mot de curiosité sur les noms
singuliers de M. le Prince, M. le Duc et M. le Comte, même de
Monseigneur, Monsieur, Mademoiselle.

\hypertarget{chapitre-x.}{%
\chapter{CHAPITRE X.}\label{chapitre-x.}}

1709

~

{\textsc{Digression sur les noms singuliers, leur origine, etc.\,: M. le
Prince, M. le Comte, M. le Duc.}} {\textsc{- Succession dernière du
comté de Soissons.}} {\textsc{- Comte de Toulouse.}} {\textsc{-
Extinction du nom tout court de M. le Prince.}} {\textsc{- Chimère
avortée d'arrière-petit-fils de France.}} {\textsc{- Extinctions du nom
de M. le Duc tout court.}} {\textsc{- Enfants d'Henri II.}} {\textsc{-
Monsieur.}} {\textsc{- Filles de France de tout temps tout court Madame,
et pourquoi.}} {\textsc{- Mademoiselle.}} {\textsc{- Brevet accordé à
M\textsuperscript{lle} de Charolais pour être appelée tout court
Mademoiselle.}} {\textsc{- Monseigneur.}} {\textsc{- Adroit et
insensible établissement de l'usage de dire Monseigneur aux princes du
sang et bâtards, puis de ne plus dire autrement parlant à eux.}}
{\textsc{- M. de Vendôme se fait appeler Monseigneur à l'armée, et le
maréchal de Montrevel en Guyenne.}} {\textsc{- Altesse simple,
sérénissime.}}

~

Jamais on n'avait ouï parler d'aucun de ces noms avant que les menées de
la maison de Lorraine contre le sang royal eussent fait prendre les
armes aux huguenots. Le prince de Condé, frère du roi de Navarre, et
oncle paternel d'Henri IV, se fit leur chef. Il était le seul du sang
royal dans ce parti qui s'accoutuma, en parlant de lui à ne le nommer
que M. le Prince\,: il était comme le leur\,; aucun du parti
n'approchait de lui en naissance ni en autorité\,; son nom était leur
honneur, leur grandeur et en partie leur force. Cet usage prévalut et si
bien, tant une fois établis ils ont de force sur la multitude, qu'après
la bataille de Jarnac où ce prince mourut (1569), son fils, succédant au
nom de prince de Condé, ne fut appelé dans le parti que M. le Prince,
quoiqu'il ne pût passer alors pour le chef du parti. Le roi de Navarre,
frère aîné du premier prince de Condé, était mort (1562, 1er novembre)
des blessures qu'il avait reçues devant Rouen. Jeanne d'Albret,
princesse de Béarn et reine titulaire et héritière de Navarre, était
huguenote\,: elle avait élevé le prince de Béarn, son fils, qui fut
depuis notre Henri IV, dans cette religion. Il avait un peu plus de
quinze ans à la mort du prince de Condé son oncle, et un an moins que le
prince de Condé, son cousin germain. Celui-ci ne pouvait lui rien
disputer\,; aussi n'y songea-t-il pas, et le prince de Béarn, titre
qu'il porta tant que la reine sa mère vécut, fut unanimement déclaré,
proclamé et reconnu chef du parti huguenot, tandis que, par le jeune âge
de ces deux princes, l'amiral de Coligny l'était en effet\,; néanmoins
le prince de Navarre porta toujours ce nom dans le parti huguenot,
tandis que le prince de Condé, son cousin, y fut toujours constamment
appelé tout court M. le Prince. Le commerce que les guerres civiles ne
détruisent jamais dans les différents partis, et celui que les divers
intervalles de guerre y multiplièrent sous le nom de paix, introduisit
dans le parti catholique l'habitude de l'autre sur ce nom de M. le
Prince tout court, en parlant du prince de Condé, qui s'établit ainsi
par toute la France et jusqu'à Paris et à la cour.

Ce second prince de Condé mourut à Saint-Jean d'Angély, 5 mars 1588, à
trente-six ans, et laissa un fils posthume, qui fut le troisième prince
de Condé, père du héros et grand-père de celui dont on vient de
rapporter la mort. Avec le nom de son père, il hérita de l'habitude
générale, et fut comme lui appelé M. le Prince tout court. Henri IV,
étant monté sur le trône, le voulut dérober aux huguenots, qui n'avaient
que lui de prince du sang, mais en trop bas âge pour être leur chef que
de nom. Il était premier prince du sang, fils du cousin germain d'Henri
IV, et personne alors entre la couronne et lui. Henri IV le fit venir à
Saint-Germain, et prit grand soin de son éducation\,: il n'avait alors
que huit ans, et c'était à la fin de 1595. Arrivé dans cet usage, qui
avait si généralement prévalu d'être appelé tout court M. le Prince, et
n'ayant au-dessus de lui que le roi, ce même usage se continua qui a
duré toute sa vie, et qui a passé à son fils, et de celui-là à son
petit-fils.

Le comte de Soissons était son oncle paternel, fils du second mariage du
premier prince de Condé avec une Longueville, qui fut toujours du parti
catholique. L'émulation qui ne se trouve que trop souvent dans les
cadets d'une autre mère et dans les principaux des partis différents,
piqua ce prince de voir son aîné M. le Prince tout court, et le porta à
imaginer sur cet exemple à se donner aussi un nom singulier. Il se fit
donc appeler M. le Comte tout court par ses domestiques, puis par ses
créatures, par ses amis, enfin par la maison de Longueville et par ses
parents. Rien n'égale la promptitude et la facilité des Français à
suivre les modes, et à se soumettre aux prétentions. Sur l'exemple de
ceux qui prirent cet usage, et la connaissance que M. le comte de
Soissons y était attaché, il prévalut bientôt partout. Comme il ne
donnait ni rang ni avantage réel à ce prince, le roi laissa dire et
faire, en sorte que non seulement le comte de Soissons resta toute sa
vie M. le Comte tout court, mais que cette dénomination passa après lui
à M. son fils qui l'a conservée toute sa vie. Nul autre prince du sang
ne portait alors le titre de comte.

M. le Prince, quelque ennemis que le comte de Soissons et lui fussent,
n'eut garde de trouver mauvaise une distinction mise à la mode pour un
cadet de sa maison\,; mais elle lui donna l'idée de multiplier la
sienne, et de faire appeler le duc d'Enghien, son fils aîné, M. le Duc
tout court. Il y réussit avec la même facilité que son oncle avait
rencontrée à se fait appeler tout court M. le Comte, et ce nom tout
court de M. le Duc a passé depuis comme de droit acquis aux fils aînés
des deux derniers princes de Condé, en sorte qu'il y en eut quatre de
suite appelés M. le Prince, quatre M. le Duc, et deux M. le Comte, parce
que la branche de Soissons a fini au second, tué sans alliance à la
bataille de Sedan ou de la Marfée, 6 juillet 1641, à quarante-deux ans.

Ce prince n'avait point de frère et avait eu quatre soeurs. Deux étaient
mortes sans alliance, et l'aînée n'avait laissé qu'une fille du duc de
Longueville, qui épousa ensuite la fameuse soeur de M. le Prince le
héros. Cette fille du premier lit fut la dernière duchesse de Nemours,
dont il a été parlé plus d'une fois ici, et qui eut tant de procès avec
M. le prince de Conti. L'autre soeur, qui n'est morte qu'en 1692, à
quatre-vingt-six ans, porta, entre autres biens, le comté de Soissons au
prince Thomas, fils de Savoie, appelé le prince de Carignan, mort en
1656, dont elle eut entre autres deux fils, le fameux muet, père du
prince de Carignan, mort depuis peu à Paris, mari de la bâtarde du
premier roi de Sardaigne et de la comtesse de Verue\,; l'autre qui porta
le nom de comte de Soissons, qui de la nièce du cardinal Mazarin laissa,
entre autres enfants, un autre comte de Soissons, mort dans l'armée du
roi des Romains devant Landau, et le fameux prince Eugène. Le feu roi,
dans sa jeunesse et dans les premières années de son mariage, ne
bougeait de chez cette comtesse de Soissons, dont la faveur personnelle
jointe à la toute-puissance de son oncle, dominait la cour et en
distribuait les agréments et fort souvent les grâces. Ce nom de comtesse
de Soissons dans un éclat si grand lui fit imaginer d'abuser de la
servitude française, et de s'adopter, sur l'exemple des comtes de
Soissons, princes du sang, le nom de M\textsuperscript{me} la Comtesse
tout court, et à son mari celui de M. le Comte. Elle hasarda de se faire
nommer ainsi par ses domestiques et ses familiers. La fleur de la cour,
qui abondait chez elle, n'eut pas plutôt aperçu cette ambition qu'elle
s'y conforma. Le roi s'accoutuma à l'entendre sans le trouver mauvais,
et cet usage s'introduisit. Son mari, de qui rien ne dépendait, n'y
parvint pas si généralement, et ne vécut pas assez pour le bien établir.
Sa veuve étant tombée en disgrâce\,; l'usage s'interrompit\,: elle
redevint M\textsuperscript{me} la comtesse de Soissons, mais, par
habitude parmi beaucoup de gens, demeura M\textsuperscript{me} la
Comtesse jusqu'à sa fuite hors du royaume, qu'elle ne put s'en faire
suivre dans les pays étrangers. On voit ainsi jusqu'où et avec quelle
facilité les abus s'introduisent et s'établissent en France.

Le feu roi avait bien envie d'introduire l'usage d'appeler M. le comte
de Toulouse M. le Comte tout court. Parlant de lui, il ne disait jamais
que le Comte, et toute la maison de ce fils naturel ne disait jamais que
M. le Comte tout court. Il y avait néanmoins deux princes du sang qui
portaient le nom de comte de Charolais et de comte de Clermont, mais qui
ne pointèrent que sur la fin de son règne, et qui étaient fils de sa
fille naturelle M\textsuperscript{me} la Duchesse, lesquels alors ni
depuis n'ont pas songé à ce nom singulier. Je ne sais comment il est
arrivé que le comte de Toulouse, M. le Comte tout court dans le désir et
dans la bouche du roi, et dans celle de toute la marine, n'a jamais pu
l'être dans le public, excepté un très petit nombre de bas courtisans,
et qui encore n'osaient le hasarder hors de la présence du roi, ni
comment ce monarque, si flatté, si redouté, dont les moindres désirs
étaient adorés, et qui a conduit ses bâtards jusqu'à l'apothéose, n'a
jamais pu venir à bout de ce qui tout de plain-pied avait réussi à la
nièce du cardinal Mazarin, femme d'un prince de la maison de Savoie, par
le chausse-pied de la conformité du nom de comtesse de Soissons.

Les princes de Condé, pleinement possesseurs du nom héréditaire de M. le
Prince, et pour leur fils aîné de celui de M. le Duc, commencèrent à
prétendre cette distinction comme un droit de premier prince du sang. Le
roi et le monde le leur passa comme bien d'autres choses plus
importantes, mais cela même les leur a fait perdre.

M. le duc d'Orléans, vraiment premier prince du sang, négligea cette
qualité offusquée sous son rang si supérieur de petit-fils de France. On
a vu en son lieu comment elle passa à M. le Prince, à la mort de
Monsieur, qui dès auparavant, à la mort de M. son père, avait pris le
nom de M. le Prince tout court, par cette même raison que M. le duc
d'Orléans méprisait pour soi la qualité de premier prince du sang. M. le
Prince fit en même temps passer à M. son fils le nom tout court de M. le
Duc, qu'il portait auparavant. À la mort de M. le Prince dernier, le
roi, dans l'idée que ce nom singulier de M. le Prince avait été porté
par le premier prince du sang, et en dernier lieu par celui qu'il avait
fait tel sans l'être, ne voulut pas qu'il passât à M. son fils, à qui le
nom de M. le Duc tout court qu'il portait passa. M. le duc d'Orléans
avait, dès ce temps-là un fils portant le nom de duc de Chartres, qu'il
conserva.

M\textsuperscript{me} la duchesse d'Orléans avait alors des chimères
dans la tête, qu'elle n'a pu faire réussir comme on verra dans la suite.
Non contente du moderne rang de petit-fils de France dont elle jouissait
par M. son mari, elle ne pouvait souffrir que ses enfants ne fussent que
princes du sang, et voulait imaginer un entre-deux, avec un nom
d'arrière-petit-fils de France\,; c'est en effet ce qui empêcha M. le
duc de Chartres de s'appeler M. le Prince, et ce qui favorisa encore M.
le duc d'Enghien, celui que nous avons vu si courtement premier
ministre, à prendre à la mort de M. son père le nom qu'il avait porté de
M. le Duc tout court. Mais à la mort de celui-ci, en 1740, ce nom a péri
avec lui, quoique M. le duc de Chartres, premier prince du sang,
déterminé alors et rien plus, et portant le nom de duc d'Orléans depuis
la mort de M. son père, eût un fils qu'il fit appeler duc de Chartres.
Ainsi, soit que la maison de Condé n'ait osé hasarder le nom tout court
de M. le Duc au fils enfant que le dernier M. le Duc a laissé, soit
qu'elle se soit ménagé, durant son enfance, le temps d'essayer de lui
faire ressusciter le nom tout court de M. le Prince, par l'habitude de
la conformité de nom, sur l'exemple très sauvage de la comtesse de
Soissons dont je viens de parler, ils l'ont fait appeler le prince de
Condé, sans que jusqu'à présent, dans l'hôtel de Condé même, on l'ait
encore nommé M. le Prince tout court.

On ne peut disconvenir que les frères de Charles IX ne se trouvent
quelquefois l'un après l'autre appelés M. le Duc tout court, quelquefois
Monsieur tout court, dans les Mémoires de ces temps-là\,: Henri III
étant duc d'Anjou presque jamais, et depuis qu'il fut roi, le duc
d'Alençon un peu davantage. Jusqu'à eux on n'avait jamais ouï parler de
ces noms. Ils vinrent de leurs maisons, et ils y demeurèrent. Le gros du
monde n'y prit point. Toutes les histoires et la plupart des Mémoires
les nomment toujours ducs d'Anjou et d'Alençon\,; il ne paraît point
qu'ils aient affecté ces noms particuliers\,; ainsi ce que j'ai dit du
nom de M. le. Duc sur les fils aînés des princes de Condé demeure
certain, sans que ce peu qui s'est vu de ses fils de France y apporte de
variation.

De cela même on doit comprendre que Gaston, frère de Louis XIII, est le
premier fils de France qui ait été véritablement et continuellement
appelé tout court Monsieur, et qui l'ait affecté. Il est vrai que les
histoires et les Mémoires de son temps l'appellent aussi duc d'Orléans,
mais il n'est pas moins vrai qu'il y est très ordinairement nommé aussi
tout court Monsieur, et d'une fréquence suivie tout autrement que les
fils de France dont on vient de parler. Il est certain de plus que j'ai
ouï dire à mon père, qui l'a vu tant d'années sous Louis XIII et depuis,
qu'on ne lui donnait jamais d'autre nom en parlant de lui, et que je
l'ai su encore de tous ceux que j'ai vus qui ont vécu dans ces temps-là.
On doit donc regarder Gaston comme le premier qui ait véritablement
porté le nom de Monsieur, et qui, par l'idée qu'on y a attachée, l'a
consacré au premier frère du roi. Cela est si vrai qu'il l'a porté
jusqu'à sa mort, parce que les rangs, honneurs et distinctions une fois
acquis, ne se perdent point., à la différences des préséances. Gaston
cédait à M. le duc d'Anjou, frère de Louis XIV, qu'il a longtemps vu
puisqu'il n'est mort qu'en 1660, pendant le voyage du mariage du roi son
neveu, et néanmoins il demeurait Monsieur.

À sa mort M. le duc d'Anjou l'est devenu à sa place. Il est mort en
1701. Non seulement M. son fils, qui prit alors le nom de duc d'Orléans,
avec des honneurs et des avantages que le rang de petit-fils de France,
tout grand qu'il est, ne lui donnait pas, ne fut point appelé Monsieur
tout court\,; mais M. le duc de Berry, fils de France, de même rang que
Monsieur, et qui le précédait partout, ne le prit point parce qu'il
n'était pas frère du roi de France, quoiqu'il le fût du roi d'Espagne.
On voit donc que ces noms tout courts, qui paraissent si distingués,
n'ont dans le fond ni réalité ni avantages, et ne doivent leur être
qu'au hasard.

Il en est de même de celui de Madame, de M\textsuperscript{me} la
Princesse, de M\textsuperscript{me} la Duchesse, de
M\textsuperscript{me} la Comtesse. Les femmes prennent les noms de leurs
maris par une suite nécessaire. À l'égard des filles de France, la chose
est différente de tous temps elles ont été appelées Madame, par le
respect de leur naissance, et tout court, Madame, parce que n'ayant
point d'apanage comme les fils de France, elles n'ont point de nom que
celui de leur baptême et celui de France. Ainsi il peut {[}y avoir{]},
et il y a maintenant plusieurs Madame tout court, qui pour les cadettes
ne peuvent être distinguées que par leur nom de baptême, et il n'y peut
avoir qu'une Madame par son mari, parce qu'il n'y a qu'un seul prince
qui soit Monsieur tout court. On en a vu deux tant que la veuve Gaston a
vécu, mais comme douairière.

Le nom singulier de Mademoiselle est encore plus moderne. J'ai raconté
(t. Ier, p.~45) comment mon père engagea Louis XIII à former en sa
faveur le nouveau rang de petite-fille de France inconnu jusqu'alors.
Chez Monsieur, dont elle fut dix-huit ans fille unique, elle n'était
nommée que Mademoiselle tout court. Les Mémoires de ces temps-là
apprennent qu'elle figura de bonne heure, et les siens montrent bien
franchement le mépris qu'elle avait pour Madame, sa belle-mère, et
quelle différence, bien ou mal à propos, elle mettait entre elle et ses
soeurs parce qu'elles étaient du second lit. Elle voulut donc une
distinction au-dessus d'elles, bien que de rang égal, et à l'exemple du
nom singulier de Monsieur et de Madame tout court, elle voulut être
nommée tout court Mademoiselle. Cela n'ajoutait rien à son rang\,; elle
était bien l'aînée\,; point d'autres petites-filles de France
qu'elles\,; Gaston était chef des conseils et lieutenant général de
l'État pendant la minorité de Louis XIV, et alors craint, et ménagé de
tous les partis. Ce nom unique et nouveau passa donc avec la même
facilité que les autres dont on vient de parler\,; et comme elle ne se
maria point, à son très grand regret, elle fut tout court Mademoiselle
toute sa vie, quoique Monsieur, frère de Louis XIV, eût des filles, par
la même raison que lui-même n'était devenu Monsieur tout court que parla
mort de son oncle Gaston. Ce n'est pas qu'il ne le trouvât mauvais,
quoique très lié d'amitié avec Mademoiselle dont il ménagea toute sa vie
la succession, et qu'il ne fit appeler tant qu'il put lainée de ses
filles l'une après l'autre Mademoiselle tout court. Mais jamais cela ne
prévalut, et tout ce qu'il put obtenir de l'usage fut que peu à peu,
pour distinguer la fille de Gaston de la sienne, ou se mit à dire
Mademoiselle de la sienne, et la grande Mademoiselle de l'autre, dont la
taille était en effet fort haute\,; mais jamais Monsieur n'osa proposer
qu'elle ajoutât un nom à celui de Mademoiselle\,; et le roi, qui aimait
à la mortifier, et qui n'avait jamais perdu le souvenir du portereau
d'Orléans\footnote{M\textsuperscript{lle} de Montpensier se présenta
  devant Orléans, apanage de son père, le 27 mars 1652. Les magistrats
  lui en ayant refusé l'entrée, elle s'introduisit par un portereau, ou
  petite porte, qui donnait sur le quai et qu'elle fit rompre, par des
  bateliers. Elle n'était accompagnée que de quelques dames\,; mais
  bientôt elle obtint pour toute sa suite l'entrée dans la ville
  d'Orléans et en prit possession. Voy. les \emph{Mémoires de
  Mademoiselle} à l'année 1652.}**, ni du canon de la porte
Saint-Antoine\footnote{On sait que Mademoiselle sauva l'armée du prince
  de Condé en faisant tirer les canons de la Bastille contre l'armée du
  roi commandée par le maréchal de Turenne. Voy. les mêmes
  \emph{Mémoires}, à la date du 2 juillet 1652.}, ne songea jamais à
donner cet avantage à Monsieur. À sa mort, en 1693, il n'y eut plus de
difficultés\,; et la dernière fille de Monsieur, la seule alors non
mariée devint seule Mademoiselle tout court jusqu'à son mariage, en
1698, au duc de Lorraine.

Ce nom de Mademoiselle tout court passa ainsi dans l'esprit du monde
pour être affecté à la première petite-fille de France, comme on s'était
persuadé que Monsieur tout court était le nom distinctif du premier
frère du roi. Tant que Louis XIV vécut, personne ne crut qu'il pût
descendre plus bas, et M. le Prince et M. le Duc qui avaient l'un et
l'autre des filles non mariées depuis le mariage de
M\textsuperscript{me} la duchesse de Lorraine, tous deux si fertiles en
prétentions et si âpres à usurper, n'imaginèrent jamais qu'une princesse
du sang pût prétendre au nom tout court de Mademoiselle. M. le Duc, leur
fils et petit-fils, devenu premier ministre, osa tout. Il avait préféré
entre ses soeurs filles la cadette qu'il aimait, pour la faire
surintendante au mariage de la reine, à l'aînée qu'il n'aimait point,
qui en fut outrée. Plus entreprenante encore que lui, elle lui fournit
un moyen de la consoler, qu'il trouva tellement de son goût qu'il y
travailla à l'heure même.

Elle avait plus de trente-deux ans, et n'avait pas mené une vie à se
marier\,; demeurant fille, elle voulut être appelée tout court
Mademoiselle. Le monde, depuis qu'elle était née, était accoutumé à
l'appeler Mille de Charolais. M\textsuperscript{me} la duchesse de
Berry, fille de M. le duc d'Orléans, n'avait paru qu'une seule fois
avant son mariage\,; M\textsuperscript{lle}s ses soeurs, point du
tout\,; l'aînée était bien tout court Mademoiselle au Palais-Royal, mais
le monde n'avait pas eu à se ployer à cet usage, sinon comme en
avancement d'hoirie pour M\textsuperscript{me} le duchesse de Berry,
entre la déclaration et la conclusion de son mariage, et de même après
pour la reine d'Espagne (mais elles ne paraissaient point dans ces
courts intervalles, et on ne les nommait pas beaucoup).
M\textsuperscript{lle} de Charolais, au contraire, de branche si
reculée, qui n'avait point eu de tante Mademoiselle, et qui depuis si
longtemps passait sa vie à la cour et dans le plus grand monde, vit bien
qu'il aurait peine à se défaire du nom de Charolais\,; et M. le Duc,
pour rie pas se commettre avec le public, fit dans sa toute-puissance,
ce qui n'avait jamais été imaginé pour le nom singulier de Mademoiselle
ni pour tous les autres dont j'ai parlé. Il fit donner un brevet à
M\textsuperscript{lle} de Charolais pour être désormais appelée
Mademoiselle tout court. M\textsuperscript{lle} de Beaujolais, dernière
fille de M\textsuperscript{me} la duchesse d'Orléans\footnote{Le duc
  d'Orléans n'avait qu'une fille religieuse\,; elle se nommait
  Louise-Adélaïde et devint abbesse de Chelles le 14 septembre 1719.},
était morte\,; il n'en restait plus que mariées ou religieuse\,;
M\textsuperscript{lle} de Charolais se trouvait la première princesse du
sang fille et n'en craignait point d'autres, parce que M. le duc
d'Orléans était veuf et ne se voulait plus remarier. Ce prince n'imagina
pas que son fils pourvait avoir des filles, ou n'osa s'opposer à M. le
Duc qui l'accablait en tout. Ce fut l'époque que prirent M. le Duc et
M\textsuperscript{lle} de Charolais pour cette nouveauté et la faire
passer en titre. Le monde cria, murmura\,; il n'en fut autre chose, et
M\textsuperscript{lle} de Charolais est demeurée Mademoiselle tout court
par brevet.

Jamais Dauphin jusqu'au fils de Louis XIV n'avait été appelé
Monseigneur, en parlant de lui tout court, ni même en lui parlant. On
écrivait bien «\, Monseigneur le Dauphin,\,» mais on disait «\,Monsieur
le Dauphin,\,» et «\, Monsieur\,» aussi en lui parlant\,; pareillement
aux autres fils de France, à plus forte raison au-dessous. Le roi, par
badinage, se mit à l'appeler Monseigneur\,; je ne répondrais pas que le
badinage ne fût un essai pour ne pas faire sérieusement ce qui se
pouvait introduire sans y paraître, et pour une distinction sur le nom
singulier de Monsieur. Le nom de Dauphin le distinguait de reste, et son
rang si supérieur à Monsieur qui lui donnait la chemise et lui
présentait la serviette. Quoi qu'il en soit, le roi continua, peu à peu
la cour l'imita, et bientôt après non seulement on ne lui dit plus que
Monseigneur parlant à lui, mais même parlant de lui, et le nom de
Dauphin disparut pour faire place à celui de Monseigneur tout court. Le
roi, parlant de lui, ne dit plus que mon fils ou Monseigneur, à son
exemple, M\textsuperscript{me} la Dauphine, Monsieur, Madame, en un mot
tout le royaume. M. de Montausier, M. de Meaux qui l'avaient élevé\,;
Sainte-Maure, Florensac, ceux qui avaient été auprès de lui dans sa
première jeunesse, ne purent se ployer à cette nouveauté\,; ils cédèrent
à celle de lui dire Monseigneur, parlant à lui, mais en parlant de lui
ils continuèrent à l'appeler M. le Dauphin, et y ont persévéré toute
leur vie.

M. de Montausier, qui avait été son gouverneur, et qui, tant qu'il a
vécu, le servit assidûment de premier gentilhomme de sa chambre, ne lui
dit jamais que Monsieur, parlant à lui, et ne se contraignit pas de
déclamer contre l'usage qui s'était introduit de lui dire Monseigneur.
Il demandait plaisamment si ce prince était devenu évêque. C'est que peu
auparavant, dans une assemblée du clergé, les évêques, pour tacher à se
faire dire et écrire Monseigneur, prirent délibération de se le dire et
se l'écrire réciproquement les uns les autres. Ils ne réussirent à cela
qu'avec le clergé et le séculier subalterne. Tout le monde se moqua fort
d'eux, et on riait de ce qu'ils s'étaient monseigneurisés. Malgré cela
ils ont tenu bon, et il n'y a point eu de délibération parmi eux sur
aucune matière, sans exception, qui ait été plus invariablement
exécutée.

Monseigneur fut donc Monseigneur toute sa vie, et le nom de Dauphin
éclipsé. C'est le premier et jusqu'à présent l'unique Monseigneur tout
court qu'on ait connu. Longtemps après que l'usage de ne lui dire plus
que Monseigneur, parlant à lui, fut universellement établi, M. le Duc et
M. le prince de Conti, ou de hasard ou de familiarité avec eux, ou
d'adresse, commencèrent à être quelquefois appelés Monseigneur, à
l'armée, par leurs principaux domestiques. L'imitation et la fatuité ont
grand cours dans notre nation. De jeunes gens, et même grands seigneurs,
les plus dans leur privance, croyant se donner, avec eux un air de
liberté, commencèrent à faire comme leurs principaux domestiques de
retour à Paris, cela continua dans le particulier et les parties de
plaisir. D'une campagne à l'autre, le nombre augmenta. Quelques gens
moins familiers crurent devoir en user de même\,; on se moqua d'eux
d'abord, comme prenant une liberté dont ils n'étaient pas à portée. Cela
ne fut pas su assez à temps pour en instruire d'autres. Peu à peu les
domestiques de ces princes ne leur dirent plus que Monseigneur, parlant
à eux. Tout le subalterne de l'armée crut que ce serait manquer de
respect que de les traiter autrement. On s'aperçut qu'ils le trouvaient
fort bon. Nos Français ne connaissent ni bornes ni barrières\,; la
crainte de déplaire et l'exemple de l'un à l'autre gagna. À la fin
jusqu'aux officiers généraux, et les plus marqués, leur parlèrent de
même. Alors, les familiers les plus huppés, qui avaient commencé,
n'osèrent plus discontinuer\,; et comme cette façon de leur parler était
passée des intimes et des familiers à toute l'armée, au retour elle se
communiqua à Paris et à la cour, mais y demeura dans la jeunesse et dans
le subalterne. M. le duc d'Orléans, à qui toute sa vie personne n'avait
dit que Monsieur, devint à plus forte raison Monseigneur pour les mêmes.
M. du Maine et M. le comte de Toulouse, si égalés en tout aux princes du
sang, le furent en ce nouveau traitement d'usage, par la crainte et la
flatterie des mêmes, qui pourtant ne gagna pas jusqu'aux courtisans d'un
certain âge d'aucune espèce, pour aucun de ces princes. Cela dura de la
sorte jusqu'à la mort du roi. Alors le grand vol que prirent M. le Duc
et M. du Maine, l'un et l'autre ménagés par M. le duc d'Orléans, leur
rendit le Monseigneur plus commun. On crut sentir à leurs manières que
le Monsieur les blessait, et rapidement presque personne de tout âge et
de toutes conditions ne le leur dit plus, ducs, princes, étrangers,
chancelier, maréchaux de France, à l'exception d'un très petit nombre,
mais de qui que ce soit à l'égard du régent, qui, avec un air libre et
indifférent, laissait solider cet usage dont M. son fils devait
profiter.

Je tirai ce parti avec lui de mon ancienne et continuelle privante que
de ma vie, ni en public ni en particulier, je ne lui ai dit Monseigneur.
En opinant au conseil de régence, ou chez lui en des assemblées
particulières, on lui adressait toujours la parole. J'étais le seul qui
lui dit Monsieur. Plusieurs fois le maréchal de Villars, quelquefois le
maréchal de Villeroy, et souvent d'autres de cette distinction, m'en
reprenaient en particulier, et me disaient que cette singularité à la
fin lui déplairait. Je tins bon, et jamais il ne m'a fait apercevoir
qu'elle lui fût désagréable. À plus forte raison je n'ai jamais dit
Monseigneur au-dessous, qui me voyant toujours dire Monsieur à M. le duc
d'Orléans, n'osèrent le trouver mauvais, et jusqu'à présent encore je me
suis conservé ce pucelage. Je n'ai jamais dit Monseigneur qu'aux deux
fils de France, pour qui cet usage s'introduisit général fort peu après
le mariage de Mgr le duc de Bourgogne comme insensiblement, mais avec
rapidité, sans exception que des princes du sang et bâtards, encore
tortillaient-ils entre leurs dents. M. de Beauvilliers {[}ne dit{]}
jamais en sa vie que Monsieur, et presque toujours aussi M. de
Chevreuse. Les dames leur dirent aussi Monseigneur, et à la fin en sont
venues pendant la régence, mais surtout pendant que M. le Duc a été
premier ministre, à le dire presque toutes aux princes du sang, qui fut
le temps où presque de vive force le Monseigneur en leur parlant devint
général.

Comme tout va toujours croissant, M. de Vendôme dans son apogée
l'introduisit à l'armée d'Italie, où qui que ce soit peu à peu n'osa
plus lui dire Monsieur. Il soutint cet usage en Flandre\,; mais il
échoua tout à fait à Paris et à la cour dans les voyages qu'il y fit
dans sa plus grande splendeur. Il n'y eut pas jusqu'au maréchal de
Montrevel, dans son commandement de Guyenne, qui ne l'établit parmi tous
les officiers d'abord, et de là dans toute la noblesse pour le premier
commandant qui l'ait osé, et qui trouvait tout publiquement très mauvais
que qui que ce fût portant l'épée lui dit Monsieur. Il les y avait tous
ployés, et aucun ne s'y hasardait. D'abus en abus, quand on les souffre,
jusqu'où ne tombe-t-on pas\,?

La curiosité de cette digression me la fera allonger pour l'Altesse. Peu
à peu les rois ont pris la Majesté réservée à l'empereur, comme bien
plus anciennement les papes se sont réservé la Sainteté que prenaient
non seulement les patriarches mais les évêques. L'Altesse abandonnée, et
il n'y a pas encore si longtemps, par les petits rois, fut curieusement
ramassée par les autres souverains, et leur est demeurée privativement à
tous autres jusqu'aux commencement du dernier siècle, et avec eux les
fils et les frères des rois. Ceux-ci s'en contentèrent si bien, qu'on ne
voit point que les fils puînés d'Henri II aient jamais été traités
d'Altesse Royale. En Espagne, encore aujourd'hui, les infants, fils de
Philippe V, n'ont que la simple Altesse, mais on leur dit Monseigneur.
J'y fus averti de cela, et de me garder de leur donner de l'Altesse
Royale.

Gaston, frère de Louis XIII, prit le premier l'Altesse Royale. Cela
était encore si nouveau, que son régiment, qui n'eut point d'autre nom
que celui de l'Altesse, n'eut jamais celui d'Altesse Royale, non pas
même lorsque Gaston fut lieutenant-général de l'État pendant la minorité
de Louis XIV. C'est le seul fils de France qui l'ait pris. Monsieur,
frère de Louis XIV le dédaigna parce que les filles de Gaston l'avaient
pris avec le rang de petites-filles de France, quoique Monsieur leur
père et Madame sa seconde femme l'aient conservé toute leur vie. Ainsi
Monsieur, frère de Louis XIV, le fit prendre à ses enfants, et se serait
également offensé qu'on le lui eût donné, ou qu'on l'eût omis pour eux.
Tout le monde, même princes et princesses du sang, l'ont toujours donné
aux filles de Gaston et aux enfants de Monsieur en leur parlant, sans en
faire aucune façon.

M. de Savoie, depuis roi de Sardaigne, qui pièce à pièce obtint pour ses
ambassadeurs les honneurs partout de ceux des têtes couronnées, sur sa
prétention de roi de Chypre, et dont la mère, fille du duc de Nemours et
d'une fille du duc de Vendôme, bâtard d'Henri IV, avait la première pris
le nom bizarre et nouveau de Madame Royale, prit chez lui l'Altesse
Royale, après son mariage avec la fille de Monsieur, qui l'avait par
elle-même, et le donna aussi à Madame Royale. Peu à peu il l'obtint des
cours étrangères, et ce qu'il y a de rare dans cette usurpation, c'est
que son grand-père, avec la même prétention de Chypre, fils d'une fille
de Philippe II, roi d'Espagne, et mari d'une fille d'Henri IV, soeur de
Louis XIII, n'y avait jamais songé.

Le grand-duc à cet exemple, gendre de Gaston, le prit bien des années
après\,; et le duc de Lorraine s'en avisa aussi après son mariage avec
la fille de Monsieur, quoique son père, beau-frère de l'empereur
Léopold, ni son trisaïeul, gendre d'Henri II, et si follement favorisé
de Catherine de Médicis sa belle-mère, n'y eussent jamais pensé, et se
fussent contentés de l'Altesse simple. Le duc d'Holstein-Gottorp, père
de celui-ci, gendre du czar frère du fameux czar Pierre Ier, fils de la
soeur aînée du dernier fameux roi de Suède, et de même maison que le roi
de Danemark, se donna aussi et obtint de l'empereur l'Altesse Royale.
Ces trois derniers ne l'ont jamais pu obtenir du feu roi.

Ce nouveau titre d'Altesse royale de Gaston réveilla les souverains. Ils
ajoutèrent à leur Altesse simple le Sérénissime, qu'ils prirent
apparemment sur la sérénité des doges de Venise et de Gênes, lesquels ne
prennent point l'Altesse. Les princes du sang, qui ne s'étaient pas trop
attachés à l'Altesse, la voulurent, et la prirent Sérénissime, parce
qu'ils ne cèdent à aucuns souverains, et qu'ils ne voulurent pas les
laisser se hausser de titre sans s'approprier le même.

Alors les cadets de maisons souveraines ramassèrent l'Altesse simple
réservée aux seuls souverains qui venaient de l'abandonner. La preuve de
cette époque est claire. MM. de Guise, si maîtres en France durant la
Ligue, et par là même si considérés dans toute l'Europe, et qui ont,
pendant ce qui se peut appeler leur règne absolu, si fort augmenté le
rang de leur maison, n'ont jamais été traités d'Altesse. Cela se voit
dans tous les Mémoires et les histoires de tous ces temps-là, qui sont
pleines des lettres qu'ils ont écrites et qu'ils ont reçues de toutes
sortes de gens et de toutes sortes d'États, dont aucun ne les traite
d'Altesse\,; et ce qui en pousse l'évidence au dernier degré, c'est
qu'on y voit plusieurs lettres du secrétaire du duc de Mayenne à ce
prince, pendant qu'il était lieutenant général de l'État et qu'il
disputait à main armée la couronne à Henri IV, dans lesquelles il n'y a
point d'Altesse. Rien ne prouve donc plus clairement qu'ils ne la
prenaient point alors.

Lors donc que longtemps après ils la prirent à l'occasion que je viens
de dire, ils ne la prirent que simple, parce que, quelque grand rang
qu'ils aient conservé de leurs usurpations en ce genre pendant la Ligue,
il n'était plus temps pour eux, non pas de surpasser, mais même de
s'égaler aux princes du sang, qui l'avaient prise Sérénissime. Cela dura
ainsi jusqu'à ce que MM. de Rohan et de La Tour-Bouillon, étant devenus
princes de la manière que je l'ai rapporté (t. Il, p.~155 et t. V,
p.~313), et que longtemps après, c'est-à-dire quelques années, ils s'y
furent accoutumés et affermis, non contents d'être devenus égaux en
distinctions à la maison de Lorraine, ils hasardèrent pour dernier trait
de se faire comme eux donner par leurs gens de l'Altesse. Les princes
véritables, car en parlant de ceux de Lorraine j'entends aussi les
autres qui étaient pour lors en France et qui firent comme eux, indignés
déjà de voir ces deux maisons à leur niveau, ne purent souffrir la
communauté d'Altesse, et y ajoutèrent le Sérénissime. Cela leur était
aisé. Personne ne leur a jamais donné d'Altesse que ceux qui en
recevaient d'eux réciproquement, et les cardinaux pour en avoir
l'Éminence, et encore seulement en s'écrivant, et personne autre ni
écrivant ou en, parlant que leurs domestiques, et peut-être quelques
gens du plus bas étage\,; ainsi il ne leur fut pas difficile
d'accoutumer leurs gens à les traiter d'Altesse Sérénissime, qui déjà
leur donnaient l'Altesse. Ils n'en furent pas plus avancés. MM. de Rohan
et de Bouillon ne leur voulurent pas être inférieurs en cela non plus
qu'au reste, et se firent donner le Sérénissime chez eux, et on a vu ce
que les cardinaux de Bouillon et de Rohan ont arraché là-dessus de la
Sorbonne, qui est le seul lieu où ils l'aient obtenu en France.

\hypertarget{chapitre-xi.}{%
\chapter{CHAPITRE XI.}\label{chapitre-xi.}}

1709

~

{\textsc{Disgrâce de M. de Vendôme.}} {\textsc{- Éclat entre le duc de
Vendôme et Puységur, qui le perd radicalement auprès du roi.}}
{\textsc{- Affront reçu à Marly, de M\textsuperscript{me} la duchesse de
Bourgogne, par le duc de Vendôme.}} {\textsc{- {[}Il{]} est exclu de
Marly.}} {\textsc{- Vendôme exclu de Meudon.}} {\textsc{- Vendôme refusé
d'aller en Espagne.}} {\textsc{- Fortune, caractère et retraite du duc
de La Rochefoucauld.}}

~

La mort de M. le prince de Conti sembla au duc de Vendôme un avantage
d'autant plus considérable qu'il se voyait délivré d'un émule si
embarrassant par la supériorité de naissance, au moment qu'il l'allait
voir en sa place à la tête des armées, porté partout sur les pavais, et
qu'il le laissait encore auprès de Monseigneur sans aucun contrepoids.
J'ai déjà dit en son temps son exclusion des armées, parce que cet
événement ne se pouvait reculer hors de temps, par rapport aux
dispositions militaires qui ne se pouvaient transposer. La chute de ce
prince des superbes eut trois degrés, tant, de si haut, elle fut
profonde. Nous voici arrivés au deuxième qui laisse encore un espace
considérable jusqu'au dernier d'entre deux et trois mois\,; mais comme
ce dernier n'a de connexité avec aucun autre événement, je le
rapporterai tout de suite après avoir averti de l'intervalle pour
n'avoir plus à y revenir.

Quelques raisons de toute espèce qui dussent engager le roi à ôter à M.
de Vendôme le commandement de ses armées, je ne sais si tout l'art et le
crédit de M\textsuperscript{me} de Maintenon n'y eût pas succombé, et si
les menées de M. du Maine, qu'il lui cachait avec tant de soins, et
aidées du secours journalier des valets intérieurs, sans une aventure
qu'il faut expliquer ici pour mettre tout, à la fois ce grand tout, sous
les yeux, de la dernière issue de cette terrible lutte, et si poussée à
l'extrême entre Vendôme secondé de sa formidable cabale, et l'héritier
nécessaire de la couronne appuyé de son épouse qui faisait les délices
du roi et de M\textsuperscript{me} de Maintenon, qui pour trancher le
mot, dont le dedans et le dehors ont été trente ans durant témoins, le
gouvernait entièrement, et dont Vendôme avait si pleinement et si
insolemment triomphé.

On a vu qu'à son retour de Flandre, il avait eu une audience du roi,
unique et qui ne fut pas fort longue. Il n'y oublia pas Puységur, dont
il fit des plaintes amères, et en dit tout ce qui lui plut de pis, avec
son audace accoutumée à être cru sur parole.

Puységur, dont j'ai eu occasion de parler plus d'une fois, était fort
connu du roi, avec une sorte de privante que lui avait acquise le
rapport continuel au roi des détails si continuels de son régiment
d'infanterie, dont il se croyait le colonel particulier, dans lequel
Puységur avait passé jusqu'alors la plus grande partie de sa vie major
et lieutenant-colonel avec la confiance du roi. Elle s'était augmentée
par des rapports plus importants, lorsque, maréchal des logis de l'armée
de M. de Luxembourg, il en était l'âme et y faisait tout jusqu'aux
projets. La part qu'il eut après au secret et à l'exécution de
l'expulsion de toutes les garnisons hollandaises des places des Pays-Bas
espagnols, et de là en beaucoup d'autres choses importantes que le roi
lui confia, soit pour l'en consulter, soit pour l'en charger, dont il
s'était toujours acquitté avec toute la capacité et la droiture possible
en Flandre, en Espagne et partout où il fut employé, comme on l'a vu
quelquefois ici, avaient ajouté pour lui, dans le roi, le dernier degré
de confiance et d'estime. Lui et son ami Montriel, aussi du régiment du
roi et souvent son aide dans les détails des armées, avaient été mis
gentilshommes de la manche de Mgr le duc de Bourgogne, lorsque l'affaire
de M. de Cambrai en fit chasser Léchelle et Dupuis, comme je l'ai
rapporté alors. Il s'était extrêmement attaché à M. de Beauvilliers\,;
et, depuis que leur emploi fut fini, Puységur, dont il avait goûté la
vérité et la capacité, demeura dans son commerce et dans son amitié la
plus particulière, conséquemment très bien auprès de Mgr le duc de
Bourgogne, qui, s'il eût régné, ne lui eût pas fait attendre si
longtemps qu'on a fait le bâton de maréchal de France, si dignement
mérité, et qu'il n'a eu enfin que par la honte de ne le lui pas donner.
Dans cette situation à la cour et dans les armées il n'était pas
possible qu'il ne fût toujours tout au milieu de ce qu'il s'y passait et
le témoin de tous les démêlés de la campagne de Lille, dès lors
lieutenant général dans cette armée. Il y était le correspondant du duc
de Beauvilliers, fort exact, et plût à Dieu qu'on l'eût particulièrement
attaché à la personne de Mgr le duc de Bourgogne, au lieu de ceux qu'on
y mit. Sa capacité et sa vertu furent, dès le commencement de la
campagne, fort choquées de la conduite de M. de Vendôme, et le furent
dans la suite de plus en plus jusqu'au comble. Il voyait tout à revers,
et dans les sources il ne pouvait approuver rien de ce que faisait et
voulait le général. Il avait souvent occasion de le montrer et de le lui
témoigner à lui-même. À l'injonction du duc de Berwick, ami particulier
du duc de Beauvilliers, il s'était lié avec lui, et le fut toute la
campagne.

C'en était trop à la fois pour n'être pas exposé à la haine de Vendôme,
malgré tous les ménagements extrêmes qu'il avait constamment gardés avec
lui, qui ne purent adoucir un homme si superbe, et si ennemi né de tout
ce qui ne l'était pas du prince qu'il voulait perdre et qu'il ménageait
si peu, bien plus, de tout ce\,: qui lui était attaché. C'est ce qui
produisit les plaintes que Vendôme en fit au roi et à son retour, tout
ce qu'il lui en dit d'étrange, et non content de cette vengeance, de
tout ce qu'il en répandit publiquement en propos peu mesurés.

Puységur, si accoutumé aux fréquents particuliers avec le roi, comprit
qu'après une si épineuse campagne, il en aurait où il serait vivement
questionné s'il arrivait à la chaude et prudemment se mit six semaines
ou deux mois en panne, chez lui, en Soissonnais, avant que d'arriver à
Paris et à la cour. La curiosité refroidie, instruit d'ailleurs des
propos que le duc de Vendôme tenait sur lui, il ne voulut pas, par un
plus long séjour, donner à penser qu'il était embarrassé de se montrer.
Ainsi il arriva.

Peu de jours après, le roi qui l'avait toujours goûté, peiné de tout ce
que M. de Vendôme lui en avait dit, le fit entrer dans son cabinet, et
là tête à tête, lui demanda raison, avec bonté, de mille sottises
absurdes qui l'avaient embarrassé. Puységur l'en éclaircit si nettement,
que le roi, dans sa surprise, lui avoua que c'était M. de Vendôme qui
les lui avait dites. À ce nom, Puységur, qui se sentit piqué, saisit le
moment. Il dit au roi d'abord ce qui l'avait retenu si longtemps chez
lui sans paraître, puis détailla naïvement et courageusement les fautes,
les inepties, les obstinations, les insolences de M. de Vendôme, avec
une précision et une clarté qui rendit le roi très attentif et fécond en
questions, et en éclaircissements de plus en plus. Puységur qui les lui
donna tous, voyant tant d'ouverture, et le roi demeurer court et
persuadé à chaque fois, poussa sa pointe, et lui dit que, puisque
Vendôme l'épargnait si peu après toutes les mesures et les ménagements
qu'il avait toujours gardés avec lui, il se croyait permis, et même de
son devoir pour le bien de son service, de le lui faire connaître une
bonne fois. De là, il lui dépeignit le personnel du duc de Vendôme, sa
vie ordinaire à l'armée, l'incapacité de son corps, la fausseté de son
jugement, la prévention de son esprit, la fausseté et les dangers de ses
maximes, l'ignorance de toute sa conduite à la guerre\,; puis, reprenant
toutes ses campagnes d'Italie, et les deux dernières de Flandre, il le
démasqua totalement, mit au roi le doigt et l'oeil sur toutes ses
fautes, et lui démontra manifestement que c'était une profusion de
miracles si ce général n'avait pas perdu la France cent fois.

La conversation dura plus de deux heures. Le roi, convaincu de tout, et
de longue main persuadé par expériences, non seulement de la capacité de
Puységur, mais de sa droiture, de sa fidélité et de son exacte vérité,
ouvrit à ce coup tout à la fois les yeux sur cet homme que tant d'art
lui avait si bien caché jusqu'alors, et montré comme un héros et le
génie titulaire de la France. Il eut honte et dépit de sa crédulité, et
de cette conversation Vendôme demeura perdu dans son esprit, et bien
exclu du commandement des armées, exclusion qui tarda peu après à se
déclarer.

Puységur, naturellement humble, doux et modeste, mais vrai et piqué au
jeu, et qui n'avait plus de ménagement à garder avec M. de Vendôme après
l'éclat qu'il avait fait contre lui en public, et ce qu'il avait dit au
roi, content d'ailleurs du succès qu'il avait remarqué dans toute sa
conversation, la rendit sur-le-champ en gros dans la galerie, et brava
vertueusement Vendôme et toute sa cabale, qu'il n'ignorait point.

Elle en frémit de rage\,; Vendôme encore plus. Ils ne répondirent qu'en
répandant des raisonnements misérables qui ne firent impression sur
personne. Les plus avisés les jugèrent dès lors sur le côté. Le parti
opposé et jusqu'alors si, opprimé embrassa Puységur\,; et
M\textsuperscript{me} de Maintenon, M\textsuperscript{me} la duchesse de
Bourgogne, le duc de Beauvilliers même, surent faire valoir auprès du
roi ce qu'il avait enfin appris par lui.

La suite assez prompte, je l'ai racontée. Vendôme, exclu de servir,
vendit ses équipages, se retira à Anet où l'herbe commença à croître, et
supplia le roi de trouver bon qu'il ne lui fit guère sa cour qu'à Marly,
et Monseigneur qu'à Meudon, de tous les voyages desquels il continua
d'être. Cette légère continuation de distinction le soutenait un peu
dans la solitude qu'il s'était creusée\,; elle lui servit comme de
témoignage de la satisfaction demeurée au roi et à Monseigneur de ses
services et de sa conduite, que ses ennemis si puissants et si
nécessairement chers n'avaient pu lui enlever\,: c'est ainsi que sa
cabale s'en expliquait, et lui-même, avec un faux air de philosophie et
de mépris du monde dans lequel personne ne donna.

Tout abattu qu'il était, il soutenait à Marly et à Meudon le grand air
qu'il y avait usurpé dans les temps de sa prospérité. Après avoir
surmonté les premiers embarras, il y reprit sa hauteur, sa voix
élevée\,; il y tenait le dé. À l'y voir, quoique peu environné, on l'eût
pris pour le maître du salon\,; et à sa liberté avec Monseigneur, et
même, tant qu'il l'osait hasarder, avec le roi, on l'eût cru le
principal personnage. La piété de Mgr le duc de Bourgogne lui faisait
supporter sa présence et ses manières comme s'il ne se fût rien passé à
son égard\,; ses serviteurs particuliers en souffraient, et
M\textsuperscript{me} la duchesse de Bourgogne fort impatiemment\,; mais
sans oser rien dire, épiant les occasions.

Il s'en présenta une au premier voyage que le roi fit à Marly après
Pâques. Le brelan était à la mode\,; Monseigneur y jouait souvent dans
le salon d'assez bonne heure avec M\textsuperscript{me} la duchesse de
Bourgogne. Manquant d'un cinquième, il vit M. de Vendôme à un bout du
salon\,; il le fit appeler pour faire sa partie. À l'instant
M\textsuperscript{me} la duchesse de Bourgogne dit modestement, mais
fort intelligiblement, à Monseigneur, que la présence de M. de Vendôme à
Marly lui était bien assez pénible, sans l'avoir encore au jeu avec
elle, et qu'elle le suppliait de l'en dispenser. Monseigneur, qui n'y
avait pas fait la moindre réflexion, ne le put trouver mauvais\,; il
regarda par le salon et en fit appeler un autre. Vendôme, cependant,
arrivait à eux et en eut le dégoût en face et en plein devant tout le
monde. On peut juger à quel excès cet homme superbe fut piqué de
l'affront. Il ne servait plus, il ne commandait plus, il n'était plus
l'idole adorée, il se trouvait dans la maison paternelle du prince qu'il
avait si cruellement offensé, et c'était à son épouse chérie et outrée à
qui il avait affaire\,; il pirouetta, s'éloigna dès quille put, et
bientôt après gagna sa chambre, où il ragea à son loisir.

La jeune princesse fit cependant ses réflexions sur ce qu'il venait
d'arriver. Rassurée par la facilité qu'elle avait trouvée à ce qu'elle
venait de faire, en peine aussi comme le roi prendrait la chose, elle se
détermina, tout en jouant, à la pousser plus loin, ou pour y réussir, ou
au moins pour se tirer d'embarras, car, avec toute son intime
familiarité, elle s'embarrassait aisément parce qu'elle était douce et
timide. Sitôt donc que la partie de brelan fut finie, elle courut chez
M\textsuperscript{me} de Maintenon avant que le roi y fût encore entré,
et lui conta ce qu'il lui venait d'arriver. Elle lui dit que, après tout
ce qu'il s'était passé en Flandre, elle avait une peine extrême à voir
M. de Vendôme\,; que cette affectation continuelle de Marly, où elle ne
le pouvait éviter, sans jamais aller à Versailles, où elle ne le
rencontrait jamais, était une suite d'insultes à laquelle elle ne
pouvait s'accoutumer\,; que, de plus, ses fautes étant assez reconnues
pour lui avoir fait ôter le commandement des armées, il ne pouvait y
avoir d'autre raison de le souffrir à Marly que celle de l'amitié du roi
pour lui, et qu'elle ne pouvait supporter qu'avec la dernière douleur
qu'elle parût égale entre son petit-fils et elle d'une part, et M. de
Vendôme de l'autre. Cela fut vif, mais court, parce que le roi allait
arriver.

M\textsuperscript{me} de Maintenon, piquée contre Vendôme du fond des
choses, et plus dangereusement peut-être d'avoir si longuement lutté
contre lui en vain, parla ce soir là même au roi de cette affaire, lui
lit valoir les raisons de la princesse, sa douceur, sa modération
d'avoir été si longtemps sans en rien dire, et combien ces sentiments-là
étaient estimables, par rapport à son mari. Le propos réussit sur
l'heure. Le roi entièrement dégoûté du duc de Vendôme, et toujours peiné
d'avoir sous ses yeux ceux qu'il jugeait avec raison être mécontents,
comme il n'en pouvait douter\,; de celui-ci depuis qu'il ne servait
plus, ne fut pas fâché d'une occasion de se soulager de sa présence, et
avec le gré de sa petite-fille et de M\textsuperscript{me} de Maintenon.
Avant de se coucher, il chargea Bloin de dire de sa part, le lendemain
au matin, à M. de Vendôme de s'abstenir désormais de demander pour
Marly, ou se rencontrant sans cesse, et nécessairement, dans les mêmes
lieux que M\textsuperscript{me} la duchesse de Bourgogne qui avait peine
à le voir, il n'était pas juste de lui en laisser plus longtemps la
contrainte.

On ne peut imaginer en quel excès de désespoir il entra à ce message si
peu attendu, et qui sapait par le pied le fondement de toute espérance,
et de l'insolence de ses manières et de ses propos. Il se tut néanmoins
de peur de pis, n'osa parler au roi, et s'enfuit cacher sa rage et sa
honte à Clichy, chez Crosat. L'aventure du brelan avait fait grand
bruit, il avait retenti jusqu'à Paris. Les auteurs du compliment fait à
Vendôme en conséquence ne le cachèrent pas. Cette nouvelle fit un
nouveau fracas dans le monde, tellement que, lorsqu'on sut Vendôme si
brusquement à Clichy, le bruit courut partout qu'il avait été chassé de
Marly. Il le sut\,; et, pour montrer qu'il n'en était rien, il y
retourna deux jours avant la fin du voyage, qu'il passa dans la honte et
dans un continuel embarras. Il en partit pour Anet, en même temps que le
roi pour Versailles, et n'a jamais depuis remis les pieds à Marly.

Revenu des premiers transports, il se prit à ce qu'il put. Bloin ne lui
avait point parlé de Meudon\,; il s'assura d'être de tous les voyages,
et se mit à se vanter de l'amitié de Monseigneur à tout propos, comme
aurait fait un franc provincial. Réduit à ce retranchement, il arrivait
à Versailles la surveille de chaque voyage de Monseigneur pour faire sa
cour au roi, et logeait chez Bloin, parce qu'il avait prêté son logement
à M\textsuperscript{me} de Montbazon, soeur du comte d'Évreux, lorsqu'il
renonça à Versailles pour Marly et Meudon, quand il sut qu'il ne
servirait plus. Il passait à Meudon tout le temps que Monseigneur y
demeurait, lui qui dans sa splendeur lui donnait à peine un jour ou
deux, et de Meudon retournait droit à Anet. Il ne se faisait point de
voyages à Meudon que M\textsuperscript{me} la duchesse de Bourgogne n'y
allât voir Monseigneur et que Vendôme ne s'y présentât audacieusement
devant elle, comme pour lui faire sentir qu'au moins chez Monseigneur il
l'emportait sur elle. Conduite par l'expérience de l'expulsion de Marly,
la princesse souffrit doucement cette insolence\,; elle épia quelque
occasion.

Deux mois après, il arriva que, pendant un voyage de Monseigneur, le roi
et M\textsuperscript{me} de Maintenon y allèrent dîner avec
M\textsuperscript{me} la duchesse de Bourgogne, sans y coucher. C'était
une énigme que cette partie. Au roi cela lui était arrivé, quoique
rarement\,; quelquefois M\textsuperscript{me} de Maintenon, tout à fait
réunie avec M\textsuperscript{lle} Choin, la voulait entretenir à son
aise sans la faire venir à Versailles, et le roi, comme on peut croire,
était du secret. On verra bientôt quelle fut cette liaison. M. de
Vendôme, qui, à l'ordinaire, était à Meudon, eut le peu de sens de se
présenter des premiers à la descente du carrosse. M\textsuperscript{me}
la duchesse de Bourgogne, qui en fut très blessée, s'en contraignit
moins qu'à l'ordinaire, et détourna la tête avec affectation après une
apparence de révérence. Vendôme, qui le sentit, n'en poussa que mieux sa
pointe et il fit la folie de la poursuivre l'après-dînée à son jeu. Il
en essuya le môme traitement, et encore plus marqué. Piqué au vif, et à
la fin embarrassé de sa contenance, il monta dans sa chambre et n'en
descendit que fort tard. Pendant ce temps-là, M\textsuperscript{me} la
duchesse de Bourgogne fit sentir à Monseigneur le peu de ménagement que
Vendôme avait pour elle. Retournée le soir à Versailles, elle en parla à
M\textsuperscript{me} de Maintenon, et s'en plaignit ouvertement au roi.
Elle lui représenta combien il lui était dur d'être moins bien traitée
de Monseigneur que de lui-même, et que M. de Vendôme se fit ouvertement
contre elle un asile de Meudon, et une consolation de Marly.
M\textsuperscript{me} la princesse de Conti, avec quelques dames,
étaient de ce voyage avec Monseigneur, entre autres
M\textsuperscript{me} de Montbazon.

Le lendemain du jour que le roi y avait dîné, M. de Vendôme se plaignit
aigrement à Monseigneur de l'étrange persécution qu'il souffrait partout
de M\textsuperscript{me} la duchesse de Bourgogne\,; mais Monseigneur,
qu'elle avait prévenu la veille, répondit si froidement à Vendôme, qu'il
se retira les larmes aux yeux, résolu toutefois de ne point quitter
prise qu'il n'eût arraché de Monseigneur quelque sorte de satisfaction.
Il entretint longtemps dans un cabinet M\textsuperscript{me} de
Montbazon tête à tête, qui n'en sortit que pour aller prier
M\textsuperscript{me} la princesse de Conti d'y passer, avec qui elle
était fort bien, et qu'elle y suivit. Le colloque fut encore long entre
eux trois et la conclusion que M\textsuperscript{me} la princesse de
Conti parlât à Monseigneur le jour même en faveur de M. de Vendôme. Elle
ne réussit pas mieux. Tout ce qu'elle en tira fut qu'il fallait que M.
de Vendôme évitât M\textsuperscript{me} la duchesse de Bourgogne quand
elle viendrait à Meudon, et que c'était bien le moindre respect qu'il
lui devait, jusqu'à ce qu'il l'eût apaisée et se fût remis bien auprès
d'elle. Une réponse si sèche et si précise fut cruellement sentie\,;
mais il n'était pas au bout du châtiment qu'il avait si plus que
mérité\footnote{Cette vieille locution, \emph{il avait si plus que
  mérité}, peut se traduire par \emph{il n'avait que trop mérite}.}. Le
lendemain mit fin à tous ces mouvements et à ces pourparlers.

Vendôme jouait l'après-dînée à un papillon en un cabinet particulier,
lorsque d'Antin arriva de Versailles. Il s'approcha de ce jeu, demanda
où en était la reprise avec un empressement qui fit que M. de Vendôme
lui en demanda la raison. D'Antin lui dit qu'il avait à lui rendre
compte de ce dont il l'avait chargé. «\,Moi\,! dit Vendôme avec
surprise, je ne vous ai prié de ne rien. --- Pardonnez-moi, répliqua
d'Antin\,: vous ne vous souvenez donc pas que j'ai une réponse à vous
faire\,?» À cette recharge M. de Vendôme comprit qu'il y avait quelque
chose, quitta le jeu et entra dans une petite garde-robe obscure de
Monseigneur avec d'Antin, qui là, tête à tête, lui dit que le roi lui
avait ordonné de prier Monseigneur de sa part de ne le plus mener à
Meudon, comme lui-même avait cessé de le mener à Marly, que sa présence
choquait M\textsuperscript{me} la duchesse de Bourgogne, et que le roi
voulait aussi que le duc sût qu'il désirait qu'il ne s'y opiniâtrât pas
davantage. Là-dessus la fureur transporta Vendôme et lui fit vomir tout
ce qu'elle peut inspirer. Il reparla le soir à Monseigneur, qui ne s'en
émut pas davantage, et qui, avec le même sang-froid qu'il lui avait déjà
montré, l'éconduisit entièrement. Le peu qui restait du voyage s'écoula
dans l'embarras et dans la rage qu'il est aisé de penser, et le jour que
Monseigneur retourna à Versailles, il s'enfuit droit à Anet.

Mais, rie pouvant tenir nulle part, il s'en alla avec ses chiens, sous
prétexte de chasse, passer un mois à sa terre de la Ferté-Aleps, sans
logement et sans nulle compagnie, rager tout à son aise. Il revint de là
à Anet se fixer dans un abandon universel. Dans ce délaissement, dans
cette exclusion de tout si éclatante et si publique, incapable de
soutenir une chute si parfaite après une si longue habitude d'atteindre
à tout et de pouvoir tout, d'être l'idole du monde, de la cour, des
armées, d'y faire adorer jusqu'à ses vices et admirer ses plus grandes
fautes, canoniser tous ses défauts, d'oser concevoir le prodigieux
dessein de perdre et d'anéantir l'héritier nécessaire de la couronne,
sans avoir jamais reçu de lui que des marques de bonté et uniquement
pour s'établir sur ses ruines, et triomphé huit mois durant de lui avec
l'éclat et le succès le plus scandaleux, on vit cet énorme colosse
tomber par terre, par le souffle d'une jeune princesse sage et
courageuse, qui en reçut les applaudissements si bien mérités. Tout ce
qui tenait à elle fut charmé de voir ce dont elle était capable, et ce
qui lui était opposé et à son époux en frémit. Cette cabale si
formidable, si élevée, si accréditée, si étroitement unie pour les
perdre et régner après le roi sous Monseigneur en leur place, au hasard
de se manger alors les uns les autres à qui les rênes de la cour et du
royaume demeuraient\,; ces chefs mâles et femelles, si entreprenants, si
audacieux, et qui, par leur succès, s'étaient tant promis de grandes
choses, et dont les propos impérieux avaient tout subjugué, tombèrent
dans un abattement et dans des frayeurs mortelles. C'était un plaisir de
les voir rapprocher avec art et bassesse, et tourner autour de ceux du
parti opposé qu'ils jugeaient y tenir quelque place, et que leur
arrogance leur avait fait mépriser et haïr, surtout de voir avec quel
embarras, quelle crainte, quelle frayeur ils se mirent à ramper devant
la jeune princesse, tourner misérablement autour de Mgr le duc de
Bourgogne et de ce qui l'approchait de plus près, et faire à ceux-là
toutes sortes de souplesses.

M. de Vendôme, sans ressource que celle qu'il chercha dans ses vices et
parmi ses valets, ne laissa pas de se vanter souvent parmi eux de
l'amitié de Monseigneur, dont il était, disait-il, bien assura, et de la
violence qui avait été faite à ce prince à son égard. Il en était réduit
à cette misère d'espérer que cela se répandrait par eux dans le monde,
qu'on se le persuaderait, et que la considération du futur lui donnerait
de la considération. Mais le présent lui était insupportable. Pour s'en
tirer il songea au service d'Espagne\,; il écrivit à la princesse des
Ursins pour se faire demander. On y avait besoin de tout\,; il fut
demandé, mais sa disgrâce était encore trop fraîche pour devoir espérer
de l'adoucir. Le roi trouva mauvais que le duc de Vendôme voulût
s'accrocher à l'Espagne. Ses menées lui rompirent aux mains, le roi le
refusa tout plat, et rompit cette intrigue en Espagne, où nous verrons
pourtant qu'elle se renoua bientôt.

Personne ne gagna plus à cette chute si profonde que
M\textsuperscript{me} de Maintenon. Outre la joie de terrasser si
complètement un homme qui, par M. du Maine, lui devant presque tout ce
qu'il avait conquis, avait osé lutter contre elle, et avec un si long
avantage, elle en vit son crédit devenir de plus en plus l'effroi de la
cour, par un si grand exemple de puissance, dont personne ne douta que
le coup ne fût parti de sa main. Nous la verrons incessamment en lancer
un autre qui n'épouvanta pas moins.

Elle acheva en même temps d'être délivrée d'un favori, qui pour n'avoir
jamais ployé le genou devant elle, et qui l'avait constamment affecté
toute sa vie, lui était d'autant plus odieux que la connaissance qu'elle
avait du coeur du roi pour lui l'empêcha d'oser jamais travailler à
l'entamer. Je dis qu'elle acheva, parce que la faveur était usée, et que
l'âge et les yeux le jetèrent dans une retraite qui l'ôta de devant
elle. C'est du duc de La Rochefoucauld dont je parle, et dont j'ai fait
mention plus d'une fois, à propos du procès de préséance de M. de
Luxembourg et d'autres occasions, particulièrement sur le mariage du duc
de Noailles avec la nièce de M\textsuperscript{me} de Maintenon, dont le
roi mourait d'envie pour le prince de Marcillac, et sur lequel M. de La
Rochefoucauld fit opiniâtrement la sourde oreille. Quoi que ce soit en
lui ne faisait souvenir de son père, cet homme qui a tant fait de bruit
dans le monde par son esprit, sa délicatesse, sa galanterie, ses menées,
ses intrigues, et la part qu'il a eue dans les troubles de la minorité
de Louis XIV, dont il demeura ruiné, mais avec un grand bien qu'il remit
dans sa maison par le mariage de son fils, que j'ai expliqué à propos de
M\textsuperscript{me} de Vaudemont.

Tous les troubles finis, le cardinal Mazarin maître, le roi marié et ne
bougeant de chez la comtesse de Soissons avec l'élite de la cour, de
l'esprit, de la galanterie, du bon goût, des intrigues, parut le prince
de Marcillac avec une figure commune qui ne promettait rien et qui ne
trompait pas. Sans charge, sans emploi, portant encore sur le visage des
marques du combat du faubourg Saint-Antoine, fils d'un père à qui le roi
n'a voit jamais pardonné, et qui sans approcher de la cour faisait à
Paris les délices de l'esprit et de la compagnie la plus choisie, ce
fils ne fit peur à personne de ce qui environnait le roi. Je ne sais
comment cela arriva, et personne ne l'a pu comprendre, à ce que j'ai ouï
dire à M. de Lauzun, qui pointait fort dès lors, et aux vieillards de
son temps, mais en fort peu de jours il plut tellement au roi dont, au
milieu d'une cour en hommes et en femmes si brillante, si polie, si
spirituelle, le goût n'était pas fin ni délicat, qu'il lui donna des
préférences qui inquiétèrent Vardes, le comte de Guiche, et les plus
avant dans la privance du roi. Cette affection alla toujours croissant,
jusque-là que le père de concert avec son fils se roidit à ne se point
démettre de son duché pour en tirer par cette adresse, le rang de prince
étranger, qu'il ne se consolait point d'avoir vu arracher aux Bouillon
avec cet immense échange, et tirer ces grands établissements des mêmes
crimes qui lui étaient communs avec eux, parce qu'ils avaient plus
effrayé que lui. Cet artifice néanmoins échoua, et ne les mena qu'à
l'inutile distinction d'être traités de cousin. Mais le fils tira de sa
faveur la charge de grand maître de la garde-robe que le roi avait faite
pour Guitry, tué sans alliance au passage du Rhin, et celle de grand
veneur à la mort de Soyecourt, que le roi lui apprit lui-même par ce
billet dont on lui fit tant d'honneur, qu'il se réjouissait comme son
ami de la charge qu'il lui donnait comme son maître. On dit alors qu'il
l'avait fait son grand veneur pour avoir mis la bête dans les toiles. Il
était confident des aventures passagères du roi, et on l'accusa dans ce
temps-là de lui avoir fourni M\textsuperscript{lle} de Fontanges. Sa
mort prompte et soupçonnée de poison n'altéra point la faveur de son
ami. Il se lia alors étroitement avec M\textsuperscript{me} de
Montespan, M\textsuperscript{me} de Thianges et toute sa famille. Cette
liaison, qui fit son éloignement de M\textsuperscript{me} de Maintenon,
dura avec eux toute sa vie, et sa faveur aussi, qui lui fit donner avec
raison le nom de l'ami du roi, parce qu'elle fut solide au-dessus de
toute autre, et indépendante de tous appuis, comme inébranlable à toute
secousse. Il tira du roi des sommes immenses, qui lui paya trois fois
ses dettes, et lui faisait sans cesse et sourdement de gros présents.

C'était un homme haut, de beaucoup de valeur, et d'autant d'honneur
qu'en peut avoir un fort honnête homme, mais entièrement confit dans la
cour. Avec cela noble et magnifique en tout, au-dessus du faste,
officieux, serviable, et rompant auprès du roi les plus dangereuses
glaces pour ceux qu'il protégeait, et souvent pour des inconnus, du
mérite ou du malheur desquels il était touché, et les a très souvent
remis en selle.

Je ne sais qui l'avait mis en inimitié avec M. de Louvois, à moins que
ce ne fût une suite de ses liaisons avec M\textsuperscript{me} de
Montespan qui fut toujours aux couteaux avec ce ministre. Il était lors
au plus haut point de faveur et de puissance par les grands succès de la
guerre\,; mais elle était finie, c'était en 1679, et il craignait un
favori haut et fougueux qui lui-même n'appréhendait rien, parlait au roi
avec la dernière liberté, et s'expliquait au monde sans mesure. Il
songea donc à se le réconcilier par le mariage de sa fille avec son
fils, et de le faire avec tant de grâces et de richesses qu'il pût
désormais autant compter sur lui comme il avait eu lieu de le craindre.
Mais pour cette affaire-là il fallait être deux, et M. de La
Rochefoucauld n'en voulut pas ouï parler, jusqu'à ce que le roi,
entraîné par son ministre, et importuné des haines de gens qui à divers
titres l'approchaient de si près, se mit de la partie, et força plutôt
par autorité M. de La Rochefoucauld à consentir au mariage et à la
réconciliation qu'il ne le gagna, malgré tant de trésors dont ce mariage
fut la source, et la nouvelle érection de La Rocheguyon faite et
vérifiée en faveur de son fils qui en prit le nom. La réconciliation ne
dura guère entre deux hommes si impérieux et si gâtés. Jamais M. de La
Rochefoucauld n'aima sa belle-fille, ni ne la voulut souffrir à la cour
quoique son mérite et sa vertu l'ait fait généralement considérer, et
que son économie et son travail ait non seulement rétabli cette maison
ruinée (et par M. de La Rochefoucauld lui-même qui fui\,: toujours un
panier percé), mais qui la laissa une des plus puissantes du royaume.

M. de La Rochefoucauld était borné d'une part, ignorant de l'autre à
surprendre, glorieux, dur, rude, farouche et ayant passé toute sa vie à
la cour, embarrassé avec tout ce qui n'était pas subalterne ou de son
habitude de tous les jours. Il était rogue, en aîné des La Rochefoucauld
qui le sont tous par nature et par conséquent très repoussants. J'en ai
vu peu de ce nom qui aient échappé à un défaut si choquant, que M. de La
Rochefoucauld avait fort au-dessus d'eux tous\,; avec cela, bien plus
ami qu'ennemi, quoique ennemi dangereux, et même à incartades\,; mais
excepté un bien petit nombre, ami par fantaisie, sans goût et sans
choix. Il aimait moins que médiocrement ses enfants, et quoiqu'ils lui
rendissent de grands devoirs, il leur rendait la vie fort dure\,;
gouverné jusqu'au plus aveugle abandon par ses valets, à qui presque
tous il fit de grosses fortunes, partie par crédit., partie en se
ruinant pour eux, jusque-là qu'il fallut que sur la fin, son fils, le
bâton haut, y entrât pour tout ce qu'il voulut.

Les vieillards se souvenaient d'avoir vu Bachelier son laquais leur
donner à boire à sa table, en livrée, et s'étonnaient de le voir premier
valet de garde-robe du roi, dont le fils est aujourd'hui premier valet
de chambre, de la charge de Bloin qu'il a achetée. Il faut dire à
l'honneur du père qu'il n'y eut jamais homme si modeste, si
respectueux\,; qui se soit moins méconnu, ni qui ait toujours plus
exactement vécu à l'égard de M. de La Rochefoucauld et tout ce qui lui a
appartenu que s'il n'avait pas changé de condition\,; un fort honnête
homme, très sage, et qui se fit considérer. Il refusa beaucoup de M. de
La Rochefoucauld et a souvent obtenu de lui pour ses enfants ce
qu'eux-mêmes, ni d'autres pour, eux, n'avaient pu faire. On dit aussi du
bien de son fils.

Si M. de La Rochefoucauld passa sa vie dans la faveur la plus déclarée,
il faut dire aussi qu'elle lui coûta cher, s'il avait quelques
sentiments de liberté. Jamais valet ne le fut de personne avec tant
d'assiduité et de bassesse, il faut lâcher le mot, avec tant
d'esclavage, et il n'est pas aisé de comprendre qu'il s'en put trouver
un second à soutenir plus de quarante ans d'une semblable vie. Le lever
et le coucher, les deux autres changements d'habits tous les jours, les
chasses et les promenades du roi de tous les jours, il n'en manquait
jamais, quelquefois dix ans de suite sans découcher, d'où était le roi,
et sur le pied de demander congé, non pas pour découcher, car en plus de
quarante ans il n'a jamais couché vingt fois à Paris, mais pour aller
dîner hors de la cour et ne pas être à la promenade, jamais malade, et
sur la fin rarement et courtement {[}de{]} la goutte. Les douze ou
quinze dernières années il prenait du lait à Liancourt, et un congé de
cinq ou six semaines. Quatre ou cinq fois en sa vie il en a pris autant
pour aller chez lui à Verteuil en Poitou où il se plaisait fort, et où
la dernière il ne fut pas huit jours qu'il fallut revenir, sur un
courrier et un billet du roi qui lui mandait qu'il avait une anthrax, et
qui par amitié et confiance le voulut auprès de lui. Il allait dîner à
Paris trois ou quatre fois l'année, un peu plus souvent à une petite
maison près de Versailles où le roi fut quelquefois, mais il n'y coucha
jamais.

Son appartement à la cour était ouvert depuis le matin jusqu'au soir. Le
mélange des valets d'un trop bon maître, les égards qu'il fallait avoir
pour eux, les airs et le ton qu'y prenaient les principaux, en
bannissait la bonne compagnie, qui n'y allait que rarement et des
instants, embarrassée avec lui, et lui empêtré avec elle, qui y laissait
le champ libre aux désoeuvrés et aux ennuyeux de la cour, mêlés de
subalternes, tous gens qui n'auraient guère eu entrée ailleurs. Ils y
établissaient leur domicile et leurs repas, et y essuyaient les humeurs
du maître, qui dominait durement sur eux, et qui se trouvait toujours
déplacé avec mieux qu'eux.

Cette raison et son temps, que son assiduité rendait fort coupé,
l'avaient mis sur le pied qu'il ne faisait presque aucune visite, et
d'amitié il n'allait guère que chez le cardinal de Coislin, M. de
Bouillon et M. le maréchal de Lorges. Pour de femme, elles étaient
toutes ses bêtes\,; à peine pouvait-il souffrir ses parentes, encore
quand il les rencontrait, et ce hasard était fort rare.
M\textsuperscript{me} la maréchale de Lorges et M\textsuperscript{lle}
de Bouillon étaient les seules qui eussent trouvé grâce devant lui.
M\textsuperscript{me} Sforce, il allait quelquefois causer chez elle, et
elle par les derrières chez lui. C'était les restes de
M\textsuperscript{me} de Montespan et de M\textsuperscript{me} de
Thianges sa mère.

On aurait cru qu'il devait être heureux, et jamais homme ne le fut
moins. Tout le choquait\,; il se fâchait des choses les plus fortuites
et les plus indifférentes, et il était si accoutumé à réussir, que tout
ce qu'il obtenait pour soi ou pour autrui lui semblait toujours peu de
chose. En même temps jamais homme si envieux. Les grâces les moins à la
portée de gens en qui il s'intéressât, et les moins proportionnées à
lui, le chagrinaient essentiellement. Il était né piqué de tout, d'un
évêché, d'une abbaye\,; mais quand il en tombait sur des émules de
faveur, comme M. de Chevreuse, M. de Beauvilliers, M. le Grand, le
maréchal de Villeroy, il était au désespoir à ne pouvoir le cacher. Il
haïssait les trois premiers de jalousie, l'autre un peu moins, parce
qu'il était en respect avec lui. Il était toujours demeuré une sorte de
liaison de M. le Prince et de M. le prince de Conti à lui, de l'ancien
chrême des pères, mais sans rien d'apparent.

Sur les derniers temps, ses bas amis et ses valets abusèrent de lui pour
eux et pour les leurs, et lui firent faire au roi si souvent des
demandes âpres, importunes et si peu convenables, qu'il l'en fatigua et
l'accoutuma à le refuser, et lui à le gourmander de plaintes et de
reproches, {[}ce{]} qui mit un malaise entre eux, et lui donna des
pensées de retraite qui l'amusèrent et le trompèrent longtemps.

Sa voix était déjà fort affaiblie, elle ne lui permettait plus de monter
à cheval\,; il courait en calèche, et si on manquait, c'était à
l'ordinaire une furie jusqu'à la chasse suivante qu'on prenait. À la
mort du cerf, il se faisait descendre et mener au roi, pour lui
présenter le pied, qu'il lui fourrait souvent dans les yeux ou dans
l'oreille. Cela le peinait fort, et même le monde, et de le voir presque
couché dans sa calèche comme un corps mort. Quelquefois le roi hasardait
doucement de lui proposer de prendre du repos, et cela perçait le coeur
au favori, qui, ne pouvant plus suivre le roi ni le servir, faute de
vue, sentait qu'il lui devenait pesant de plus en plus.

Peu écouté, presque toujours éconduit, quelquefois, à force
d'importuner, refusé sèchement, le dépit vint au secours du courage. Il
se retira, mais pitoyablement. Il flottait entre sa maison de Paris et
Sainte-Geneviève, où la mémoire du cardinal de La Rochefoucauld l'eût
rendu maître de tout ce qu'il aurait voulu\footnote{Voy. notes à la fin
  du volume.}. En l'un et l'autre lieu il n'eût pas manqué de toute
espèce de compagnie et de secours\,; mais ses valets, qui étoient ses
maîtres, ne lui permirent ni l'un ni l'autre. Ils le voulurent à portée
de le faire marcher à leur gré chez le roi, pour en arracher des grâces
pour eux, et tirer ce qu'il pourrait d'un reste de crédit et de bonté du
roi pour lui. Ils le confinèrent au Chenil, à Versailles, lieu très
éloigné de tout, et où bientôt il demeura dans un entier abandon, à
l'ennui et à la douleur d'un aveugle déchu de toute occupation, de toute
faveur et de tout commerce. Il en fit encore quelques parties de main
pour importuner le roi, dans le cabinet duquel il allait par les
derrières, la plupart peu fructueuses, qui achevèrent de l'accabler. Il
finit ainsi fort amèrement sa vie, entièrement en proie à ses valets, et
avec peu de provisions pour se suffire.

\hypertarget{chapitre-xii.}{%
\chapter{CHAPITRE XII.}\label{chapitre-xii.}}

1709

~

{\textsc{Torcy en Hollande.}} {\textsc{- Cent cinquante mille livres de
brevet de retenue à La Vallière sur son gouvernement de Bourbonnais.}}
{\textsc{- Mariage du prince de Lambesc avec M\textsuperscript{lle} de
Duras.}} {\textsc{- Digne et rare procédé de M. le Grand.}} {\textsc{-
Mariage du marquis de Gesvres avec M\textsuperscript{lle} Mascrani.}}
{\textsc{- Mariage de Montendre avec M\textsuperscript{lle} de Jarnac.}}
{\textsc{- Mariage de Donzi avec M\textsuperscript{lle} Spinola.}}
{\textsc{- Mariage de Polignac avec M\textsuperscript{lle} de Mailly.}}
{\textsc{- Mort de Saumery\,; sa fortune\,; celle de son fils\,; leur
caractère.}} {\textsc{- Fortune d'Avaray.}} {\textsc{- Belle-Ile mestre
de camp général des dragons\,; sa fortune.}} {\textsc{- Mort, famille,
singularité étonnante et deuil du prince de Carignan.}} {\textsc{- Mort,
caractère et dépouille du duc de La Trémoille.}} {\textsc{- Mort,
fortune et caractère de La Reynie et de son fils.}} {\textsc{- Mort du
duc de Brissac.}} {\textsc{- Prince des Asturies juré par les cortès ou
états généraux d'Espagne.}} {\textsc{- Château d'Alicante rendu à
Philippe V.}} {\textsc{- Bataille gagnée par les Espagnols contre les
Portugais entièrement défaits.}} {\textsc{- Chamarande demandé et
accordé à Toulon.}}

~

Le roi alla le 1er mai, qui était un mercredi, à Marly. Ce fut l'époque
de la retraite de M. de La Rochefoucauld, qui n'y vint point, et qui
jusque-là, quoique aveugle, n'en avait point encore manqué de voyage. Ce
jour-là même, M. de Torcy alla à Paris, d'où il partit tout de suite
pour la Hollande dans le plus grand secret. Je ne sais comment M. de
Lauzun l'écuma\,; mais je le vis le lendemain matin dans le salon
accoster le duc de Villeroy et deux ou trois autres, à qui il demanda
s'ils n'avaient point vu M. de Torcy qui lui dirent que non. «\,Il est
pourtant revenu hier au soir fort tard de Paris, leur répondit-il, et je
sais qu'il aura des choses singulières aujourd'hui à son dîner, que je
ne veux pas vous dire. Je compte bien d'en aller manger ma part\,; vous
devriez bien y venir.\,» Ils donnèrent dans le panneau. Torcy faisait
une chère fort délicate, et il était sur le pied qu'il n'allait chez lui
que la meilleure compagnie, et sans prier. Les dupes y furent tard,
parce qu'il dînait tard à Marly, et travaillait jusqu'à ce qu'il fût
servi. Ils trouvèrent la porte fermée\,; ils frappèrent\,; point de
réponse. Enfin ils s'aperçurent qu'il n'y avait personne, et tous les
uns après les autres les voilà à pester contre M. de Lauzun, et leur
sottise d'avoir donné dans cette bourde, et à chercher où dîner\,; et le
soir M. de Lauzun à leur demander s'ils avaient fait bonne chère chez
Torcy, et à se moquer d'eux. Cette plaisanterie, qui se répandit dans
Marly, fit qu'on y sut plus tôt le voyage de Torcy que le roi n'aurait
voulu.

La Vallière eut en ce même temps cent cinquante mille livres de brevet
de retenue sur son gouvernement de Bourbonnais, que son père avait eu
pendant la faveur de M\textsuperscript{me} de La Vallière la carmélite.

Il se fit aussi trois mariages\,: le prince de Lambesc, fils unique du
comte de Brionne, qui était fils aîné de M. le Grand, épousa la fille
aînée du feu duc de Duras, frère aîné du maréchal-duc de Duras
d'aujourd'hui, tous deux fils du feu maréchal-duc de Duras, qui était
belle comme le jour, très bien faite et fort riche. Elle n'avait qu'une
soeur, qui épousa depuis le comte d'Egmont. Le procédé qu'eut M. le
Grand, quelque temps après ce mariage, mérite de n'être pas omis. La
duchesse de Duras, leur mère, était en procès avec son beau-frère pour
les biens de ses filles\,; elle prétendait beaucoup, et poussait
l'affaire avec grand soin. M. le Grand refusa tout net de la solliciter,
défendit à tous ses enfants de le faire, à sa petite-belle-fille
elle-même, dit que s'il le pouvait honnêtement, il solliciterait pour le
duc de Duras\,; qu'il n'avait pas pris sa nièce pour le ruiner et sa
maison\,; que sa belle-petite-fille était assez riche pour que trois ou
quatre cent mille livres de plus ou de moins ne lui fussent pas moins
considérables que d'avoir un oncle paternel et chef de sa maison ruiné.
L'autre procédé fut pour les par-, tapes entre les deux soeurs. Il
voulut que l'abbé de Lorraine, son fils, mort évêque de Bayeux, fût
présent à tout, et le chargea de céder et de faire régler en faveur de
la cadette tout ce qui pouvait être litigieux, parce qu'il trouvait sa
petite-fille assez riche\,; mais qu'il ne lui était pas indifférent à
lui, après l'avoir fait épouser à son petit-fils, que sa soeur la
demeurât assez pour faire une alliance qui leur fût à tous convenable.
La vérité {[}est{]} que c'est là penser et agir avec grandeur, car tout
fut exécuté de la sorte\,; mais il est vrai aussi que
M\textsuperscript{me} d'Armagnac était morte, qui n'aurait pas laissé
faire M. le Grand.

Le duc de Tresmes maria son fils aîné, le marquis de Gesvres, avec
M\textsuperscript{lle} Mascrani, prodigieusement riche. Elle n'avait ni
père, ni mère, ni frère, ni soeur. Son père avait été maître des
requêtes, sa mère était soeur de Caumartin, ami intime du duc de
Gesvres, qui fit ce mariage, lequel bientôt après se tourna fort
étrangement, et donna au public des farces fort singulières.

M\textsuperscript{lle} de Jarnac, aussi sans père ni mère, aussi fort
riche, et du nom de Chabot, épousa un cadet de Montendre, de la maison
de La Rochefoucauld, qui n'avait ni biens, ni figure, mais beaucoup
d'esprit et fort orné, {[}beaucoup{]} d'amis et d'envie de faire. Ce fut
elle qui, ayant l'âge de disposer d'elle, le choisit, et qui voulut
demeurer chez elle, dans ce beau château de Jarnac, sur la Charente, et
n'être point obligée d'en sortir, comme jusqu'alors elle y était
toujours demeurée. C'était une personne pourtant plutôt bien que mal,
avec de l'esprit, et qui voulait être maîtresse.

Quelque temps assez court après, il s'en fit deux autres M. de Donzi,
fils du feu duc de Nevers, qui n'avait pu obtenir le brevet de son père,
et à qui, avec ses grands biens, il fâchait fort de n'en pouvoir
espérer. Il passa ici un marquis Spinola, gouverneur d'Ath, lieutenant
général des armées d'Espagne, qui avait acheté la grandesse de Charles
II, et le titre de prince de l'empire de l'empereur Léopold, et qui
n'avait que deux filles, dont l'aînée héritait de la grandesse. Il {[}M.
de Donzi{]} l'épousa, et prit en se mariant le nom de prince de
Vergagne, que le public, qui aime à se jouer sur les mots, et qui
n'approuvait pas sa vie, appela le prince de Vergogne. Son beau-père lui
fit peu attendre sa dignité, et M. le duc d'Orléans, devenu régent,
moins encore celle de duc et pair, sans avoir jamais rien fait, ni été à
la guerre, ni même à la cour\footnote{Passage supprimé dans les
  précédentes éditions depuis \emph{Quelque temps assez court après}.}.

La comtesse de Mailly maria sa dernière fille à Polignac, dont il aurait
été le grand-père. Elle était fort belle, et ne tarda pas à montrer que
Polignac n'était pas heureux en mariage, ni sa mère en éducations.

Le vieux Saumery mourut chez lui, près de Chambord, à quatre-vingt-six
ans. C'était un beau et grand vieillard, très bien fait et de la vieille
roche, plein d'honneur et de valeur, pour qui le roi avait de la bonté,
et qui était estimé. Henri IV, entre autres bagages, avait amené deux
valets de Béarn\,: l'un avait nom Joanne, c'était peut-être son nom de
baptême, car force Basques s'appellent Joannès chez leurs maîtres\,;
l'autre Béziade\,: ils furent longtemps bas valets.

Lorsque Henri IV parvint à la couronne et à en jouir, Joanne devint
jardinier de Chambord, et par succession concierge, mais concierge
nettoyeur et balayeur, comme sont ceux des particuliers, et non pas
comme le sont devenus ceux des maisons royales. Son fils peu à peu se
mit sur ce dernier pied\,; mais, toutefois sentant encore le valet, et
s'y enrichit pour son état. Cela lui fit épouser une soeur de
M\textsuperscript{me} Colbert, dont le père était un bourgeois de Blois
qui s'appelait Charon, dont le petit-fils, par la fortune de M. Colbert,
devint intendant de Paris, eut la terre de Ménars, et est mort président
à mortier\,; peu éclairé, mais fort bon homme et fort honnête homme et
fort droit. Lors du mariage de Saumery, c'était encore la petite
bourgeoisie de Blois, et M. Colbert un très petit garçon. Arrivé dans la
confiance et les affaires du cardinal Mazarin dont il fut intendant, il
y donna accès à Saumery son beau-frère, et lui procura de petits emplois
dans les troupes, où il montra de la valeur. Devenu personnage, il le
protégea tant qu'il put, suivant sa portée si nouvelle, et le fit enfin
gouverneur et capitaine des chasses de Chambord et de Blois. Il laissa
deux fils entre autres et deux filles. Monglat, chevalier de l'ordre en
1661, et maître de la garde-robe, dont nous avons de si bons Mémoires,
se trouvant ruiné, espéra tout de M. Colbert en mettant son fils dans
son alliance. Il avait eu Cheverny, de sa femme, petite-fille du
chancelier de Cheverny, dont ce fils portait le nom. Il le maria à la
fille de Saumery. Chambord et Cheverny ne sont qu'à cieux lieues. C'est
le même Cheverny qui eut des emplois au dehors, qui fut menin de
Monseigneur et attaché à Mgr le duc de Bourgogne, dont j'ai parlé
quelquefois. Des deux fils, l'aîné était un grand homme, très bien fait,
et d'une représentation imposante, qui avait été estropié d'un genou en
un de ces combats de M. de Turenne. Il n'avait été que subalterne
quelques campagnes, et se retira chez lui, où il se recrépit d'une
charge de grand maître des eaux et forêts. Il épousa une fille de
Besmaux, gouverneur de la Bastille, dont le crédit, joint à la bonté du
roi pour son père, lui obtint la survivance du gouvernement de Chambord
et de la capitainerie de Blois. Avec ces établissements, il comptait
avoir fait une grande fortune et en jouissait chez lui, lorsque M. de
Beauvilliers fut gouverneur des enfants de France, et que le roi lui
laissa le choix de tout ce qui devait composer leur éducation et leur
maison, excepté du premier valet de chambre seul, comme je l'ai dit
ailleurs. Il dénicha Saumery des bords de la Loire, et le fit
sous-gouverneur. D'abord souple, respectueux, obséquieux, attaché à son
emploi, il tâcha de reconnaître un terrain si nouveau pour lui, après de
s'y ancrer\,: il courtisa les ministres et les personnages. Ce qu'il
avait d'esprit était tout tourné à l'intrigue, que la probité ne
contraignit pas, ni la reconnaissance. Il se mit à voir des femmes
importantes, et à mettre, comme il le fit dire de lui, son pied dans
tous les souliers. Jamais homme ne fit tant de chemin tous les jours par
tout le château de Versailles, et ne montait tant d'escaliers\,; jamais
homme aussi ne tira si grand parti d'une vieille blessure. À la fin il
se crut un personnage\,; il fit le gros dos et l'important, et ne
s'aperçut jamais qu'il n'était qu'un impertinent. Il ne parlait plus
qu'à l'oreille, ou sa main devant sa bouche, souvent riochant\footnote{Riocher
  (ou \emph{rioter})\,: rire dédaigneusement. Voy. \emph{Littré}.} et
s'enfuyant, toujours des riens qu'il ramassait toujours mystérieusement.
J'ai parlé de sa femme à propos de M. de Duras, qui lui donna de fâcheux
ridicules, et devant qui il n'osait souffler, quelque impudent qu'il fût
devenu.

À force d'adresse et de manéges et de duperies de M. de Beauvilliers, il
trouva moyen de tirer du roi près de quatre-vingt mille livres de rente
pour lui ou pour ses enfants qui eurent pour rien les plus gros
régiments, avec cela toujours plaintif en dehors, et frondeur en
dessous. Il avait pris l'habitude de ne dire \emph{monsieur} de personne
ni \emph{madame} non plus, de ceux-là mêmes dont l'habitude et le
respect en avait rendu le nom plus inséparable. \emph{Monsieur} était
son plus grand effort, et il citait de la sorte les plus considérables
personnages, dont il se donnait pour avoir eu la confiance, et qui lui
avaient dit ceci et appris cela.

Je me souviens qu'étant venu à Dampierre où j'étais chez M. de
Chevreuse, il vit à table un portrait de M\textsuperscript{me} la
princesse de Conti. «\,Ah\,! dit-il, voilà un assez joli portrait de la
princesse de Conti\,!» De là se mit à raconter {[}ce{]} que «\,ce pauvre
prince de Conti lui disait,\,» et puis «\,un marin nommé Preuilly,\,» et
c'était le frère du maréchal d'Humières. Il vint après à M. de Turenne,
qu'il n'appela jamais que M. Turenne, et dont il rapportait des propos
avec lui, très jeune subalterne, et dont sûrement il n'avait jamais su
le nom, qu'il aurait eus à peine avec un officier général de sa
confiance. Et par-ci par-là riochant d'autorité\,: «\,Le vieux vicomte,
disait-il, ou ce pauvre vieux vicomte,\,» et on était tout étonné que
c'était de M. de Turenne. C'était trop de sa fatuité favorite pour
qu'elle fût ignorée, et pour qu'elle nous fût nouvelle\,; mais il en
entassa tant ce jour-là, que nous nous mimes à lui en présenter des
occasions pour nous en divertir davantage, et nous y réussîmes
pleinement. Nous mourions de rire, et il ne doutait pas que ce ne fût
des gentillesses qu'il racontait avec une autorité et une dignité
merveilleuse.

Le lendemain Sassenage, Louville, le petit Renault et moi étions le
matin chez M\textsuperscript{me} de Chevreuse à parler de l'excès de ces
impertinences. Il vint quelqu'un. Nous nous mîmes dans une fenêtre sous
le rideau à continuera Mais nous en disions là de bonnes, et tout haut
se mit à dire lé petit Renault\,: «\,Mais nous serions bien étonnés si
M. de Saumery nous entendait et venait à lever le rideau.\,» Il n'eut
pas achevé que la chose arriva. Nous, au lieu d'être embarrassés, à
pâmer de rire\,; et lui qui peut-être ne nous avait pas écoutés, à
demander à qui nous en avions. Les rires furent si démesurés, et si bien
répondus par presque tout le reste de la chambre, qui savait de quoi il
s'agissait, que, tout effronté qu'il était, il en demeura confondu.

Ce galant homme était du naturel des rats, qui se hâtent de sortir d'un
logis lorsqu'il est près de crouler\,; mais il n'eut pas le nez bon. Il
furetait tout et en tant de sortes de lieux qu'il ne lui fut pas
difficile de voir le vol que le duc d'Harcourt prenait, et la décadence
de M. de Beauvilliers, à qui il devait en totalité être et fortune. Le
drôle ne balança point de se donner à Harcourt, qui le recueillit comme
un transfuge par lequel il espérait de savoir beaucoup de choses sur des
gens qu'il voulait culbuter pour s'élever sur leurs ruines, et avec
lesquels Saumery demeurait en commerce, sans qu'ils voulussent
s'apercevoir d'une conduite que chacun voyait. Il était particulièrement
attaché à M. le duc de Bourgogne, quoique Denonville fût l'ancien des
trois gouverneurs, et y était demeuré ensuite, lorsque Cheverny, d'O, et
Gamaches y furent mis. Cheverny avait la santé ruinée depuis son
ambassade en Danemark, et n'était pas sur le pied de suivre à la chasse
ni à la guerre. Saumery, sous prétexte de son genou, s'exempta de la
chasse, et lorsqu'il fut question de la guerre, il fut malade une
fois\,; les deux autres, il eut besoin des eaux. Il en revint pendant la
campagne de Lille à Versailles, où, trouvant les rieurs pour M. de
Vendôme, il se mit de leur côté\,; et pour être à la mode et s'initier
parmi la cabale triomphante, en dit pis que pas un. M. de Chevreuse et
M. de Beauvilliers, dont l'aveugle charité n'avait voulu rien voir ni
écouter sur la désertion de Saumery, et qui le traitaient bien,
lorsqu'il leur faisait l'honneur d'aller chez eux, eurent bien de la
peine à entendre ce qu'on leur dit de ses propos sur Mgr le duc de
Bourgogne. À la fin pourtant, la publicité les convainquit. Ils furent
un peu plus froids, mais ce fut tout. Saumery y gagna M. du Maine, qui
le fit dans la suite nommer par le roi mourant un des sous-gouverneurs
du roi d'aujourd'hui. Sur la fin, c'était un seigneur qui se trouvait
fort maltraité de n'être pas chevalier de l'ordre\,; on va voir que,
quelque fou que cela fût, il n'avait pas tout le tort\footnote{Voy. t.
  II, note, p.~452 et suiv. Voy. aussi les notes à la fin de ce volume.}.

Béziade, camarade de Joanne (qui est devenu le nom de famille de
Saumery), eut un emploi à la porte de je ne sais quelle ville, pour les
entrées, que Henri IV lui fit donner et continuer. Le fils de celui-ci
le continua dans ce métier, mais il monta en emploi, et s'enrichit si
bien que son fils n'en voulut point tâter, et préféra un mousquet. Il
montra de la valeur et de l'aptitude, il eut des emplois à la guerre, il
épousa une soeur de Foucault, longtemps après intendant de Caen, enfin
conseiller d'État, qui était une femme pleine d'esprit d'intrigue et qui
eut des amis considérables. En se mariant il prit le nom de d'Avaray\,;
il est devenu lieutenant général. Il a bien clabaudé de n'être pas
maréchal de France et de voir ses cadets y être arrivés, et à la fin on
l'a fait chevalier de l'ordre, qu'il n'a fait la grâce d'accepter
qu'avec beaucoup de répugnance et de délais. Il avait été quelque temps
ambassadeur en Suisse, et n'y avait point mal réussi.

Une autre fortune commença cette année en ce temps-ci à poindre grande,
et peu espérable alors, traversée depuis d'une manière terrible, montée
ensuite au comble avec la rapidité des plus incroyables hasards, mais
conduite et soutenue par l'esprit, le travail, la persévérance
infatigable, l'art et la capacité de deux frères également unis et
amalgamés ensemble, qui peuvent passer pour les prodiges de ce siècle.
Belle-Ile, petit-fils de M. Fouquet, si célèbre par sa fortune et sa
plus que profonde disgrâce, était fils d'un homme qui s'était présenté à
tout, et dont le roi n'avait voulu pour rien à cause de son père, et
l'avait tenu plus de vingt ans en exil. Son mariage avec une soeur du
duc de Lévi (je dis duc pour faire connaître l'alliance, car il ne le
fut que trente ou trente-cinq ans depuis)\,; ce mariage, dis-je,
étrange, et encore plus étrangement fait, acheva de le mettre à
l'aumône. Sa femme n'avait rien, et sa famille, bien loin de lui donner,
fut plus de vingt ans sans vouloir ouïr parler ni d'elle ni de son mari.
Ils furent réduits à vivre chez l'évêque d'Agde, frère de M. Fouquet,
longues années exilé hors de son diocèse. Revenus enfin à Paris au pot
de M\textsuperscript{me} Fouquet, mère de Belle-Ile, jusqu'à la mort de
cette espèce de sainte, ils se trouvèrent bien à l'étroit. Belle-Ile
était un cadet du surintendant\,; ses aînés emportaient les débris
qu'ils avaient pu sauver, mais qui à la fin se sont réunis par la mort
de M. de Vaux, sans enfants, et du P. Fouquet, de l'Oratoire. Le fils
aîné de Belle-Ile et de la soeur de M. de Lévi prit le nom de comte de
Belle-Ile, et son frère celui de chevalier de Belle-Ile. Je m'étends sur
eux parce qu'il sera souvent mention d'eux, dans la suite, et beaucoup
plus dans les histoires et dans les Mémoires de ce temps-ci qui
dépasseront les miens.

Tous deux entrèrent dans le service. L'aîné fut refusé avec aigreur d'un
régiment de cavalerie. Le roi dit que ce serait beaucoup encore s'il lui
accordait, avec le temps, l'agrément d'un régiment de dragons. Il
l'obtint enfin. Il se signala dans Lille. Il fut fait, comme on l'a dit,
brigadier en sortant\,; il y fut dangereusement blessé. Le maréchal de
Boufflers le servit si bien\,; que Hautefeuille ayant demandé à se
défaire de sa charge de mestre de camp général des dragons, Belle-Ile en
eut la préférence, et pour deux cent quatre-vingt mille livres, qui
était la même somme que Hautefeuille en avait donnée au duc de Guiche,
et que celui-ci l'avait achetée de Tessé\,; et Belle-Ile eut aussi cent
vingt mille livres de brevet de retenue dessus, comme Hautefeuille
l'avait obtenu lorsqu'il eut la charge. C'était un furieux pas, et sous
le feu roi, pour, d'où il était parti. Quel prodige et comment le voir
aujourd'hui gouverneur absolu d'une grande place et d'une province
frontière, chevalier de l'ordre, les entrées chez le roi, et tout à coup
maréchal de France, duc vérifié, ambassadeur extraordinaire pour
l'élection de l'empereur, général d'armée, et le dictateur de
l'Allemagne\,!

Le prince de Carignan mourut le 23 avril en sa soixante-dix-neuvième
année. Il était fils du prince Thomas ou de Carignan, et de la fille et
soeur des deux comtes de Soissons, dernière princesse du sang de cette
branche cadette de Bourbon. Le prince Thomas était fils de l'infante
Catherine, fille de Philippe II, roi d'Espagne, soeur du roi Philippe
III, grand-père de la reine épouse de Louis XIV, et du célèbre
Charles-Emmanuel, duc de Savoie, vaincu par l'industrie, le courage et
l'épée de Louis XIII au fameux pas de Suse. Ce prince de Carignan, de la
mort duquel je parle, était né sourd et muet. Il était l'aîné du comte
de Soissons, mari de la nièce du cardinal Mazarin, de laquelle j'ai
souvent parlé, et oncle, par conséquent, du comte de Soissons, si
étrangement marié en France, tué parmi les ennemis devant Landau, et du
célèbre prince Eugène\,; et de cette branche de Soissons-Savoie, il n'en
reste plus.

Cette cruelle infirmité affligea d'autant plus la maison de Savoie que
ce prince montrait tout l'esprit, le sens et l'intelligence dont son
état pouvait être capable. Après avoir tout tenté, on prit enfin un
parti extrême\,: ce fut de l'abandonner à un homme qui promit de le
faire parler et entendre, pourvu qu'il en fût tellement le maître, et
plusieurs années, qu'on ignorerait même tout ce qu'il ferait de lui. La
vérité est qu'il en usa comme les dresseurs de chiens, et ces gens qui
de temps en temps font voir pour de l'argent toutes sortes d'animaux
dont les tours et l'obéissance étonnent, et qui paraissent entendre et
expliquer par signes tout ce que leur maître leur dit la faim, la
bastonnade, la privation de lumière, les récompenses à proportion. Le
succès en fut tel, qu'il le rendit entendant tout aidé du mouvement des
lèvres et de quelques gestes, comprenant tout, lisant, écrivant, et même
parlant, quoique avec assez de difficulté. Lui-même, profitant après des
cruelles leçons qu'il avait reçues, s'appliqua avec tant d'esprit, de
volonté et de pénétration, qu'il posséda plusieurs langues, quelques
sciences, et parfaitement l'histoire. Il devint bon politique jusqu'à
être fort consulté sur les affaires d'État, et faire à Turin plus de
personnage par sa capacité que par sa naissance. Il y tenait sa petite
cour, et faisait la sienne avec dignité toute sa longue vie, qui put
passer pour un prodige.

Il épousa en 1684 une Este-Modène, fille du marquis de Scandiano qui
envoya un gentilhomme au roi pour lui donner part de cette mort, et lui
présenter une lettre de son fils, à laquelle le roi répondit, et prit le
noir pour quinze jours.

Ce fils prit le nom de prince de Carignan, épousa par amour et pour
plaire au duc de Savoie, depuis premier roi de Sardaigne, la bâtarde
qu'il avait de la comtesse de Verue, lesquels brouillés à Turin et venus
ici sous un rare incognito, comme en lieu de conquête assurée pour tout
étranger, on les a vus courtiser bassement les gens en place de les
servir pendant la jeunesse du roi, prendre partout, faire toutes sortes
d'indignes affaires\,; la femme la complaisante de celle du garde des
sceaux Chauvelin, et le mari se faire le fermier de l'Opéra et le
surintendant de ce spectacle, et avec des millions de rapines\,; le mari
dans l'obscurité et dans la basse débauche, la femme, dans l'intrigue de
toute espèce, et l'écorce de la plus haute dévotion, caressant tout le
monde, ménageant tout, se fourrant partout, se moquer de leurs
créanciers, et vivre en bohémiens\,; le mari mort dans cette crapule à
Paris, en 1740, la femme se raccrocher aux Rohan par le mariage de sa
fille avec M. de Soubise\,; et son fils devenu prince de Carignan, ôté
d'avec eux longtemps avant la mort du père par le roi de Sardaigne et
élevé à Turin, et marié par lui à la soeur de sa seconde femme
Hesse-Rhinfels, et de la seconde femme de M. le Duc, les deux soeurs
mortes et M. le Duc aussi.

En même temps mourut le duc de La Trémoille, dont j'ai parlé plus d'une
fois, à cinquante-quatre ans, que je regrettai extrêmement, et qui,
malgré la disproportion de nos âges, était demeuré extrêmement de mes
amis, depuis que notre commun procès de préséance contre M. de
Luxembourg avait formé notre liaison. C'était un fort grand homme, le
plus noblement et le mieux fait de la cour, et qui, avec un fort vilain
visage, sentait le mieux son grand seigneur\,: sans esprit que l'usage
du monde, sans dépense avec des affaires fort mal rangées, et une femme
fort avare et fort maîtresse qu'il avait perdue depuis assez peu\,; sans
crédit de faveur, et sans grand commerce. Il avait tant d'honneur, de
droiture, de politesse et de dignité, que cela lui tint lieu d'esprit,
lui fit garder une conduite toujours honnête et digne, et lui acquit
partout de la considération, même du roi et des ministres, à qui il ne
se prodigua jamais. Il ne laissa qu'un fils, et une fille mariée au duc
d'Albret. Il mourut dans la douleur, dont il m'avait entretenu souvent,
de n'avoir pu obtenir la survivance de sa charge de premier gentilhomme
de la chambre pour son fils, et de trouver le roi inflexible sur la
règle qu'il s'était faite de n'en plus donner. C'était celle de mon
père. Il m'en souhaitait souvent une d'un camarade avec qui il vivait
fort bien, mais qu'il supportait avec impatience dans sa même dignité et
dans sa même charge. M. de Beauvilliers et lui étaient fort amis, et je
ne sais comment il était arrivé que lui et moi avions assez les mêmes
goûts et les mêmes éloignements.

Son fils, à sa mort, était considérablement malade. La duchesse de
Créqui, sa grand'mère, qui avait été dame d'honneur de la reine jusqu'à
sa mort, vint le lendemain matin parler au roi avant son grand lever, et
emporta la charge avec quelque difficulté. Hors la jeunesse que le roi
n'aimait pas pour les grandes charges, il n'y avait aucunes raisons d'en
faire. Enfin le nouveau duc de La Trémoille l'eut\,; il ne la garda
guère. Son fils, enfant, l'eut après lui de M. le duc d'Orléans au
commencement de sa régence. Il vient de laisser un seul fils dans la
première enfance, et sa charge en proie à la toute-puissance du cardinal
Fleury, qui pourtant, à toute peine et bien évidente, l'arracha pour le
duc Fleury, petit-fils de sa soeur.

Peu de jours après mourut La Reynie, un des plus anciens conseillers
d'État, des plus capables, des plus intègres, grand magistrat, et de
l'ancienne roche, modeste et désintéressé, qui a formé la place de
lieutenant de police dans l'importance où elle est montée, et qui ne
l'avait pas mise sur le dangereux pied et honteux où peu à peu, pour
plaire et se faire valoir, ses successeurs l'ont conduite. Il y avait
bien des années que La Reynie ne l'était plus. Son nom était Nicolas, et
homme de fort peu, que son mérite et sa vertu élevèrent, et par les
mains duquel il a passé bien des choses importantes et secrètes. Son
fils unique lui échappa jeune, s'en alla à Rome, d'où jamais il ne put
le faire revenir quoique exprès il l'y laissât manquer de tout. Après la
mort de son père, il y voulut demeurer, et y est mort longues années
après, ne voyant presque personne que des curieux obscurs, et ne se
pouvant lasser, sans débauche, de la vie paresseuse et des beautés de
Rome, et du \emph{far niente} des Italiens, sans s'être jamais marié. Je
le rapporte `comme une chose fort singulière.

Le duc de Brissac le suivit de près. Quelques mois auparavant étant à
Meudon, il s'avisa, au sortir de table de Monseigneur, de me prendre sur
la terrasse, et de me demander pardon de son procès, et de ce qu'il
avait fait contre moi, après tout ce qu'il me devait, et l'avoir fait
duc et pair. Il mourut à Paris, subitement chez lui, d'apoplexie, à
quarante et un ans, comme il allait moleter en carrosse pour s'en aller
à Meudon.

L'extrémité où les affaires se trouvaient réduites par les malheurs de
la guerre en tous lieux, et par la disette et la misère où la France fut
cette année, firent craindre au roi et à la reine d'Espagne, un abandon
à leurs propres forces, dont il se parlait depuis quelque temps à
l'oreille\footnote{Voy. notes à la fin du volume.}. Le prince des
Asturies avait près de vingt mois et se portait fort bien. Ces soupçons
leur firent prendre la résolution de s'assurer et de se lier de plus en
plus les Espagnols, en renouvelant une ancienne cérémonie qui est ce
qu'ils appellent faire jurer le prince, c'est-à-dire de le faire
reconnaître pour le successeur de la couronne, et de lui faire rendre
hommage et prêter serment de fidélité, comme tel, et comme roi futur et
nécessaire pour tous les membres de la monarchie.

Les cortès, c'est-à-dire les états généraux, furent convoquées pour
cela, et s'assemblèrent le 7 avril dans l'église des Jéronimites du
palais de Buen-Retiro, tout à l'extrémité de Madrid. Le palais et le
couvent de ces religieux sont très grands et très magnifiques\,; ils se
tiennent, à aller à couvert de l'un dans l'autre par plusieurs endroits,
et l'église, grande et belle, sert de chapelle au palais. Le roi et l'a
reine sous leur dais du côté de l'évangile, les grands tout de suite sur
leurs bancs, les grands officiers, les conseils, les ordres, les députés
des villes, vis-à-vis et au bas en face de l'autel, et les évêques des
deux côtés de l'autel\,; le cardinal Portocarrero, archevêque de Tolède
et diocésain officiant\,; le prince porté par la princesse des Ursins,
auprès de la reine. La fonction dura trois heures et fut fort
pompeuse\,; tous les ordres du royaume y témoignèrent une grande
affection. Après la messe le petit prince fut confirmé par le patriarche
des Indes, confirmation étrangement prématurée.

En ces occasions il y a toujours dispute à qui des députés de Tolède et
de Valladolid prêtera son serment et sa foi et hommage la première.
Valladolid est la première ville de la vieille Castille\,; Tolède de la
nouvelle, mais décorée de la première métropole qui se prétend primatie.
Toutes deux sont appelées ensemble les premières de toutes les villes,
et toutes deux arrivent de leur place à toute course au pied de l'autel,
à qui s'y trouvera la première\,; mais, quoi qu'il en réussisse,
Valladolid est admise la première, et toujours sans conséquence. Les
villes comme représentant le peuple ne sont appelées que les dernières.

Tôt après, le château d'Alicante se rendit, la ville l'était de
l'automne précédent. Le château était demeuré bloqué tout l'hiver\,; une
mine qui joua à propos y fit un grand désordre, et à la fin opéra la
reddition, qui fut très importante. Ce succès fut suivi d'un autre fort
considérable au commencement de mai\,: l'armée portugaise, plus forte de
quatre ou cinq mille hommes que celle d'Espagne, commandée par le
marquis de Bay, la vint attaquer, et fut si bien reçue qu'elle fut
entièrement défaite et son infanterie tout à fait perdue. Le marquis
d'Ayetone, de la maison de Moncade, et grand d'Espagne, y commandait
l'infanterie d'Espagne, et s'y distingua extrêmement de tête et de
valeur, ainsi que Fiennes, aussi lieutenant général des troupes de
France, qui commandait la gauche, et Caylus, maréchal de camp dans
celles d'Espagne. Toute la cavalerie ennemie prit la fuite et abandonna
trois régiments anglais qui furent pris entiers, outre huit ou neuf
cents Portugais et quatre ou cinq mille tués. Milord Galloway, qui
commandait les Anglais, rejeta toute la faute sur le comte de
Saint-Jean, général de leur armée. Les Espagnols perdirent fort peu.

Chamarande, qui avait commandé à Toulon, la campagne précédente, s'y
était si dignement conduit, que tous les habitants écrivirent au duc de
Berwick dès qu'ils le surent destiné à commander l'armée de Dauphiné, et
à Chamillart pour obtenir qu'il leur fût donné encore celle-ci. La
demande fut accordée, et Chamarande destiné pour Toulon en cas
d'entreprise de Ai. de Savoie en Provence.

\hypertarget{chapitre-xiii.}{%
\chapter{CHAPITRE XIII.}\label{chapitre-xiii.}}

1709

~

{\textsc{Villars et ses fanfaronnades.}} {\textsc{- Modeste habileté
d'Harcourt.}} {\textsc{- Chamillart ébranlé, puis apparemment
raffermi.}} {\textsc{- Chamillart rudement attaqué.}} {\textsc{-
Sarcasme d'Harcourt sur Chamillart.}} {\textsc{- Conseil de guerre
devant le roi fort orageux, et l'unique de sa vie à la cour.}}
{\textsc{- Petits désordres à Paris.}} {\textsc{- Billets fous.}}
{\textsc{- Placards insolents.}} {\textsc{- Procession de
Sainte-Geneviève.}} {\textsc{- Harcourt bien pourvu à Strasbourg.}}
{\textsc{- Dangereuses audiences pour Chamillart.}} {\textsc{- Surville
dans Tournai avec dix-huit bataillons.}} {\textsc{- Manquement de tout
en Flandre.}} {\textsc{- Retour de Hollande de Torcy.}} {\textsc{-
Princes ne vont point aux armées qu'ils devaient commander.}} {\textsc{-
Besons maréchal de France.}} {\textsc{- Duchesse de Grammont.}}
{\textsc{- Vaisselles portées à l'orfèvre du roi et à la monnaie.}}
{\textsc{- Le roi et la famille royale en vermeil et en argent\,; les
princes et les princesses du sang en faïence.}} {\textsc{- Inondations
de la Loire.}} {\textsc{- Rouillé de retour de Hollande.}} {\textsc{-
Les armées assemblées.}} {\textsc{- Cardinal de Bouillon rapproché à
trente lieues.}} {\textsc{- Superbe du roi.}}

~

Peu de jours après la déclaration des généraux d'armée, le maréchal de
Villars, qui devait commander en Flandre sous Monseigneur, travailla
avec lui à Meudon, puis avec lui chez le roi, et de là s'en alla en
Flandre, à la mi-mars, y disposer toutes choses. Il en revint dans les
premiers jours de mai rendre compte de son voyage pour repartir peu
après. Les troupes n'étaient {[}pas{]} payées, et de magasins on n'en
avait pu faire nulle part. Villars, toutefois, se mit à pouffer à la
matamore, et à tenir à son ordinaire des propos insensés. Il ne
respirait que batailles, publiait qu'il n'y avait qu'une bataille qui
pût sauver l'État, qu'il en livrerait une dans les plaines de Lens à
l'ouverture de la campagne, se mit en défi, et, par un tissu de
fanfaronnades folles, faisait transir tout ce qu'il y avait de gens
sages de voir la dernière ressource de l'État commise en de telles
mains. Ce n'était pas pourtant qu'il ne sentit le poids du fardeau\,;
mais il pensait étourdir le monde, les ennemis même à qui ces propos
reviendraient, rassurer le roi et M\textsuperscript{me} de Maintenon, et
donner de grandes idées de lui. Il travailla avec le roi et plusieurs
fois avec Monseigneur, se donna pour lui rendre un compte exact de
toutes choses\,; et ce prince ne fut pas insensible à l'air de se mêler
de quelque chose d'important. Sur cette piste, Chamillart et Desmarets
lui parlèrent aussi d'affaires, l'un sur les projets et la disposition
des troupes, l'autre sur les fonds.

Harcourt, plus sage et plus mesuré, avait refusé l'armée de Flandre\,;
il avait modestement allégué qu'il n'était plus depuis longtemps dans
l'habitude de la guerre, qu'il n'avait jamais commandé que de petits
corps, qu'il ne se sentait pas assez fort pour une armée si nombreuse et
pour des événements si importants. Il aima mieux se conserver la faculté
de pouvoir de loin blâmer ce qui s'y ferait, commander une armée aussi à
l'abri des événements qu'une armée le pouvait être, et, déjà bien avec
Monseigneur, saisir l'occasion de débaucher au duc de Beauvilliers son
pupille, ou de faire au moins autel contre autel. Il suivit à l'égard du
fils la trace que Villars marquait à celui du père. Il travailla avec
Mgr le duc de Bourgogne\,; mais en rusé compagnon, il alla plus loin. Il
proposa au jeune prince que M\textsuperscript{me} la duchesse de
Bourgogne fût présente à leur travail, et les charma tous deux de la
sorte. Il avait réservé les choses principales pour les déployer devant
elle\,; finement il la consulta, admira tout ce qu'elle dit, le fit
valoir à Mgr le duc de Bourgogne, allongea la séance, et y mit tout son
esprit à étaler dextrement sa capacité pour leur en donner grande idée,
et à persuader la princesse de son plus entier attachement. Elle en fut
flattée\,; d'Harcourt la ménageait de longtemps\,; il était trop à
M\textsuperscript{me} de Maintenon, et elle à lui, pour que la princesse
ne fût pas déjà bien disposée pour lui\,; elle était fort sensible à se
voir ménagée et recherchée par les personnages.

La destination des généraux fut fort approuvée. Je fus en cela du
sentiment de tous\,; mais je ne pouvais goûter que Chamillart eût laissé
remettre Harcourt en voie, et lui donner de plus les moyens de s'emparer
de Mgr le duc de Bourgogne. J'en parlai fortement aux ducs de Chevreuse
et de Beauvilliers qui, à leur ordinaire, tout en Dieu et froids sur les
cabales et les événements, n'en firent pas grand cas, séduits peut-être
par la raison que Chamillart m'en avait lui-même donnée, qu'il aimait
mieux éloigner ce censeur de la cour. Mais le pauvre homme ne voyait pas
qu'en l'éloignant en apparence, il le rapprochait en effet, en lui
donnant lieu, par cette armée, d'entrer dans tout de l'un à l'autre avec
Mgr le duc de Bourgogne, avec M\textsuperscript{me} la duchesse de
Bourgogne, et de plus belle avec M\textsuperscript{me} de Maintenon et
avec le roi, dont les trois premiers ne lui avaient pas pardonné sa
conduite de Flandre et son opiniâtre partialité pour le duc de Vendôme
contre Mgr le duc de Bourgogne.

Plus de six semaines avant la déclaration des généraux des armées, il
avait couru de fort mauvais bruits de ce ministre, à la place duquel on
avait publiquement parlé de mettre d'Antin. J'en avais averti sa fille
Dreux, la seule de la famille à qui on pût parler avec fruit. La mère,
avec très peu d'esprit et de conduite de cour, pleine d'apparente
confiance et de fausse finesse en effet, prenait mal tous les avis. Les
frères étaient des imbéciles, le fils un enfant et un innocent, les deux
autres filles trop folles\,; et Chamillart se piquait de mépriser tout
et de compter sur le roi comme sur un appui qui ne pouvait lui manquer.
J'avais aussi souvent averti M\textsuperscript{me} Dreux du ressentiment
de M\textsuperscript{me} la duchesse de Bourgogne\,; elle lui en avait
reparlé. La princesse lui avait fort froidement dit qu'il n'en était
rien, et, faute de pouvoir mieux, l'autre s'en était contentée. Je
l'avais pressée de forcer son père à parler au roi sur ces bruits de
d'Antin. Il le fit à la fin, malgré sa sécurité\,; mais il ne le fit
qu'à demi, il lui dit bien les bruits, mais il fit la faute capitale de
ne lui nommer personne. Ce qu'il fit de mieux fut qu'il ajouta que s'il
avait le malheur que ceux qui arrivaient en ses affaires le dégoûtassent
de lui, il le lui dît sans s'en contraindre. Le roi parut touché, lui
donna toutes sortes de marques et d'assurances d'estime et d'amitié,
jusqu'à lui faire son éloge, et le renvoya comblé et en apparence mieux
que jamais avec lui. Je ne sais si déjà Chamillart touchait à sa perte,
et si cette conversation le remit\,; mais du jour qu'il l'eut eue, les
bruits qui s'étaient toujours soutenus sur lui tombèrent tout court, et
on le crut tout à fait rétabli.

Ces apparences ne purent me rassurer\,; je ne pouvais douter de
l'extrême mauvaise volonté pour lui de filme de Maintenon et de
M\textsuperscript{me} la duchesse de Bourgogne, et il était sans cesse
coiffé par deux rudes lévriers. Le maréchal de Boufflers ne l'avait
jamais aimé\,; il se plaignait nouvellement et avec amertume de tout ce
dont il avait manqué à Lille. Il lui était revenu qu'il avait tu
quelques-unes des blessures qu'il y avait reçues, que le roi avait
apprises d'ailleurs avec surprise. Impuissance peut-être pour l'un, et
pour l'autre ne vouloir pas alarmer, ce n'était pas là des crimes, mais
le maréchal, sensible, court, littéral, les trouvait tels. Il m'en avait
fait souvent des plaintes, sans que j'eusse pu lui remettre l'esprit
là-dessus. Il était persuadé de plus que le poids était trop fort pour
Chamillart. Encouragé par M\textsuperscript{me} de Maintenon qui était
pour lui, et entraîné par Harcourt, il se contraignait peu sur ce
ministre, et il s'en faisait comme un point d'honneur et de bon citoyen.

Le maréchal d'Harcourt le mettait savamment en pièces dans tous les
particuliers qu'il avait. Un jour, entre autres, qu'il déclamait
rudement contre lui chez M\textsuperscript{me} de Maintenon, à qui il ne
pouvait douter que cela ne déplaisait pas, elle lui demanda qui donc il
mettrait en sa place. «\,M. Fagon, madame,\,» lui répondit-il
froidement. Elle se mit à rire, et à lui remontrer qu'il n'était point
question de plaisanter. «\,Je ne plaisante point aussi, madame,
répliqua-t-il. M. Fagon est bon médecin, et point homme de guerre\,; M.
Chamillart est magistrat et point homme de guerre non plus. M. Fagon de
plus est homme de beaucoup d'esprit et de sens\,; M. Chamillart n'a ni
l'un ni l'autre. M. Fagon, d'entrée et faute d'expérience, pourra faire
des fautes, il les corrigera bientôt à force d'esprit et de réflexion\,;
M. Chamillart en fait aussi, et ne cesse d'en faire et qui perdront
l'État, et avec cela il n'y a en lui aucune ressource ainsi, je vous
répète très sérieusement que M. Fagon y vaudrait beaucoup mieux.\,»

Il n'est pas concevable le mal que ce sarcasme fit à Chamillart, et le
ridicule qu'il lui donna. Le fin Normand comptait bien sur les plaies
profondes que ferait à Chamillart ce bizarre parallèle, et si
cruellement soutenu. Il fut au roi, et de là à bien des gens qui en
jugèrent de même.

Mais il se passa en même temps une scène entre d'Antin et le fils de
Chamillart, devant beaucoup de monde, chez M\textsuperscript{me} la
Duchesse, dont je passe l'inutile détail, qui, plus que, tout dut faire
trembler le ministre. D'Antin, si mesuré, si valet de la faveur et des
places, d'ailleurs si maître de soi, s'aigrit de commande dans la
dispute, et y traita si mal le père et le fils, que la duchesse de La
Feuillade sortit en colère. L'éclat de cette aventure embarrassa
pourtant d'Antin, qui, de propos délibéré, avait voulu faire le chien de
meute et plaire à ce qui prenait le dessus. Il en vint à de fort sottes
excuses, après avoir tâché d'en sortir en badinant. Il n'y eut personne
à la cour qui eût quelque lumière qui ne sentit que Chamillart était
fort ébranlé, puisque d'Antin s'échappait de la sorte et sans cause
d'inimitié. Lui seul se tenait fort assuré, et dédaignait de rien
craindre\,; et sa famille l'imitait en cette sécurité. Ses vrais amis,
et ceux-là en bien petit nombre, gémissaient de cet aveuglement. MM. et
M\textsuperscript{me}s de Chevreuse, de Beauvilliers et de Mortemart
m'en témoignaient souvent leur inquiétude\,: c'était inutilement que
nous cherchions des remèdes dont il s'éloignait toujours.

Quelque peu après, le roi fit une chose fort extraordinaire pour lui, et
qui fit fort parler le monde. Il entretint dans son cabinet les
maréchaux de Boufflers et de Villars ensemble, en présence de
Chamillart. Ce fut l'après-dînée du vendredi 7 mai, à Marly. Au sortir
de là, Villars s'en alla à Paris avec ordre d'être de retour à Marly
pour le dimanche suivant au matin. Il revint dès le lendemain, samedi au
soir.

Si on avait été surpris de cette manière de petit conseil de guerre de
la veille, on le fut bien plus le lendemain après midi\,: le roi tint
pour la première fois de sa vie dans sa cour un vrai conseil de guerre.
Il en avertit Mgr le duc de Bourgogne en lui disant un peu aigrement\,:
«\,A moins que vous n'aimiez mieux aller à vêpres.\,» En ce conseil
furent Monseigneur et Mgr le duc de Bourgogne, les maréchaux de
Boufflers, de Villars et d'Harcourt, MM. Chamillart et Desmarets, l'un
pour les troupes, l'autre pour les fonds. Il dura près de trois heures
et fut fort orageux. On y traita des opérations de la campagne et de
l'état des frontières et des troupes. Les maréchaux, un peu émancipés de
la tutelle des ministres, les vexèrent, l'un affaibli, l'autre nouveau
et non encore bien ancré. Tous trois tombèrent sur Chamillart, Villars
avec plus de réserve que les deux autres. Le roi ne prit point son parti
et le laissa malmener par Boufflers et Harcourt qui se renvoyaient la
balle, jusque-là que Chamillart, doux et modéré, mais qui n'était pas
accoutumé au poinçon, s'aigrit et s'emporta de sorte qu'on l'entendit du
petit salon voisin de la chambre du roi où était la scène. Il s'agissait
du dégarnissement des places, et du mauvais état des troupes, sur quoi
Desmarets voulut aussi dire son mot, mais le roi le réprima aussitôt.

Les gardes du corps n'étaient pas payés depuis longtemps. Boufflers,
capitaine des gardes en quartier, en avait parlé au roi. Il en avait été
mal reçu. Il avait insisté, le roi lui dit qu'il était mal informé, et
qu'ils étaient payés. Boufflers piqué s'était muni d'un rôle exact et
détaillé de ce qui était dû à chacun et l'avait mis dans sa poche. Le
conseil levé il arrêta la compagnie, tira ce rôle, supplia le roi d'être
persuadé qu'il était bien informé quand il lui parlait de quelque chose,
et ouvrant le rôle, fit voir en un coup d'œil, avec la plus grande
netteté, la misère des gardes, du corps, et qu'il n'avait rien avancé
que d'exact. Le roi, qui ne s'attendait à rien moins, se redressa, et
jetant à Desmarets un regard sévère, lui demanda ce que cela voulait
dire, et s'il ne lui avait pas bien assuré que ses gardes étaient payés.
Desmarets demeura court, et tout confus, prit le rôle et barbouilla
quelque chose entre ses dents, sur quoi Boufflers piqué au jeu lui parla
fort vivement. Desmarets en silence laissa passer l'ondée, puis avoua au
roi qu'il avait cru les gardes payés et qu'il s'était trompé, sur quoi
Boufflers, de nouveau à la charge, lui fit entendre qu'il fallait être
sûr de son fait avant d'en répondre si bien, et répéta au roi qu'il le
suppliait de croire qu'il ne lui parlait jamais que bien informé. Les
deux autres maréchaux gardaient cependant un profond silence, et
Chamillart, qui jusque-là s'était contenté de rire dans sa barbe, ne put
s'empêcher de rendre à son tour un lardon au contrôleur général.
Boufflers étant sur la fin de sa romancine, Chamillart ajouta qu'il
suppliait le roi de croire qu'il en allait ainsi de beaucoup de choses,
qu'il n'y avait pas un seul régiment de payé, et que les preuves en
seraient bientôt apportées. Cela fut dit avec grande émotion. Le roi,
fatigué d'une fin de conseil si aigre et si peu attendue, interrompit
Chamillart par un mot assez ferme à Desmarets de mieux s'assurer de ce
qu'il avançait, et de mieux pourvoir aux choses, et tout de suite les
congédia tous.

Boufflers et Villars n'avaient pas toujours été d'accord dans leurs
avis, sur les opérations de la campagne qui s'allait ouvrir, mais le
premier avec retenue, et le second avec un air de respect, en sorte
qu'Harcourt s'y comporta le plus paisiblement. Au sortir de ce conseil
Villars prit congé et s'en retourna en Flandre.

Il y avait eu divers désordres dans les marchés de Paris, ce qui fit
retenir plus de compagnies des régiments des gardes françaises et
suisses qu'à l'ordinaire. Argenson, lieutenant de police, courut même
fortune à Saint-Roch, où il était accouru sur une grande émeute de la
populace, fort grossie et fort insolente, à l'occasion d'un pauvre qui
était tombé et avait été foulé aux pieds. M. de La Rochefoucauld, retiré
au Chenil, y reçut un billet anonyme atroce contre le roi, qui marquait
en termes exprès qu'il se trouvait encore des Ravaillacs, et qui, à
cette folie, ajoutait un éloge de Brutus. Là-dessus le duc accourt à
Marly, et, tout engoué, fait dire au roi pendant le conseil qu'il a
quelque chose de pressé à lui dire. Cette apparition si prompte d'un
aveugle retiré, et son empressement de parler au roi, fit raisonner le
courtisan. Le conseil fini, le roi fit entrer M. de La Rochefoucauld qui
avec emphase lui donna le billet et lui en rendit compte. Il fut fort
mal reçu. Comme à la fin tout se sait dans les cours, on sut ce que M.
de La Rochefoucauld était venu faire, et que les ducs de Bouillon et de
Beauvilliers, qui avaient reçu les mêmes billets, et les avaient portés
au roi, en avaient été mieux reçus, parce qu'ils l'avaient fait plus
simplement. Le roi en fut pourtant fort peiné pendant quelques jours,
mais, réflexion faite, il comprit que des gens qui menacent et qui
avertissent ont moins dessein de se commettre à un crime que d'en donner
l'inquiétude.

Ce qui piqua le roi davantage, fut l'inondation des placards les plus
hardis et les plus sans mesure contre sa personne, sa conduite et son
gouvernement, qui longtemps durant furent trouvés affichés aux portes de
Paris, aux églises, aux places publiques, surtout à ses statues, qui
furent insultées de nuit en diverses façons, dont les marques se
trouvaient les matins et les inscriptions arrachées\,: il y eut aussi
une multitude de vers et de chansons où rien ne fut épargné.

On en était là, lorsqu'on fit, le 16 mai, la procession de
Sainte-Geneviève, qui ne se fait que dans les plus pressantes
nécessités, en vertu des ordres du roi, des arrêts du parlement et des
mandements de l'archevêque, de Paris et de l'abbé de
Sainte-Geneviève\textsuperscript{{[}Voy. notes à la fin du
volume.{]}}{[}Voy. notes à la fin du volume.{]}. Les uns en espérèrent
du secours, les autres amuser un peuple mourant de faim.

Harcourt, habile en tout, et dont les sorties sur Chamillart avaient
intimidé Desmarets avec lui, ne voulut point partir que très bien assuré
de pain, de viande et d'argent pour son armée du Rhin. Il entretint fort
Monseigneur à Meudon tête à tête, y prit congé de lui, fut le lendemain
fort longtemps seul avec le roi, et partit les derniers jours de mai. Ce
même jour de la dernière audience du maréchal d'Harcourt, le roi en
donna une fort longue aussi dans son cabinet au maréchal de Tessé. Le
prétexte des unes fut le prochain départ pour l'armée (car Harcourt en
avait eu plusieurs\,; et Boufflers sans cesse, sans qu'elles parussent à
l'abri de ses grandes entrées)\,; celle de Tessé pour rendre le compte
de ses négociations d'Italie, elles étoient alors plus que prescrites et
en fumée. La vérité fut que toutes ces audiences regardèrent Chamillart,
comme on le verra bientôt, et toutes ameutées et procurées par
M\textsuperscript{me} de Maintenon.

Surville eut permission de saluer le roi, et fut envoyé aussitôt après
commander dans Tournai, avec dix-huit bataillons.

L'armée de Flandre ne fut pas si heureuse que celle d'Allemagne\,; aussi
n'avait-elle pas un général si madré, et si craint des ministres. Elle
manquait de tout. On fit les derniers efforts pour lui envoyer de
l'argent les premiers jours de juin, et y procurer des blés de Bretagne,
et en voiturer de Picardie. De l'argent et du pain, il n'y en vint que
chiquet à chiquet\,; et cette armée se trouva abandonnée souvent à sa
propre industrie là-dessus, et souvent pendant de longs intervalles,
avec une frontière fort resserrée. Les armées de Dauphiné et de
Catalogne étaient beaucoup mieux pour les subsistances, et les troupes
en bon état. Il y avait déjà du temps que le duc de Berwick était à la
sienne, et qu'il faisait un camp retranché sous Briançon.

J'ai déjà averti que je ne dirais rien ici des négociations ni des
voyages de Rouillé, de Torcy, du maréchal d'Huxelles et de l'abbé de
Polignac ensuite, et j'en ai dit la raison. Tout cela se trouvera bien
au long et fort en détail et d'original dans les Pièces. Je me
contenterai donc de marquer ici que Torcy arriva de la Haye à
Versailles, le samedi 1er juin, après un mois juste d'absence. Il ne
rapporta rien d'agréable, et fut médiocrement reçu du roi et de
M\textsuperscript{me} de Maintenon chez laquelle il alla d'abord rendre
compte au roi. Chamillart et M\textsuperscript{me} de Maintenon avaient
fort blâmé son voyage, parce qu'elle ne l'aimait pas et que la chose
avait été faite sans elle, Chamillart, par jalousie de métier et dépit
du traité, dont j'ai parlé, qu'il fut obligé de signer à Torcy.

Ce retour fit presser dès le lendemain le départ de tous les officiers
généraux. L'électeur de Bavière que Torcy avait vu à ilions, et le
maréchal de Villars qu'il avait entretenu à Arras, étaient informés de
l'état des affaires. En même temps on déclara qu'aucun des princes
destinés aux armées ne sortirait de la cour\,; et le roi envoya le bâton
de maréchal de France à Besons qui commandait l'armée de Catalogne. Il
fut fait seul, et n'était pas des plus anciens lieutenants généraux. M.
Je duc d'Orléans pressait fort le roi pour lui depuis assez longtemps\,;
mais nous verrons bientôt que son crédit n'était pas grand alors. Le roi
lui fit entendre que Monseigneur et Mgr le duc de Bourgogne demeurant à
la cour, il convenait qu'il y demeurât aussi, d'autant plus qu'il
pouvait se trouver peut-être dans peu dans la triste nécessité de
retirer ses troupes d'Espagne.

Si M\textsuperscript{me} de Maintenon fut bien fatale dans le plus
grand, cette vilaine que le duc de Grammont avait épousée la fut en
petit\,: c'est le sort de toutes ces créatures. Celle-ci, revenue de
Bayonne par ordre du roi, où ses pillages et d'adresse et de forée
avaient trop éclaté, où elle avait impunément volé les perles de la
reine d'Espagne, et manqué de respect en toutes façons, était au
désespoir de se retrouver à Paris exclue du rang et des honneurs de son
mariage.

En attendant Rouillé, qui, à l'arrivée de Torcy, eut ordre de revenir,
on avait jugé à propos de ranimer le zèle de tous les ordres du royaume
en leur faisant part des énormes volontés, plutôt que propositions, des
ennemis, par une lettre imprimée du roi aux gouverneurs des provinces
pour l'y répandre et y faire voir jusqu'à quel excès le roi s'était
porté pour obtenir la paix, et combien il était impossible de la faire.
Le succès en l'ut tel qu'on avait espéré. Ce ne fut qu'un cri
d'indignation et de vengeance, ce ne furent que propos de donner tout
son bien pour soutenir la guerre, et d'extrémités semblables pour
signaler son zèle.

Cette Grammont crut trouver dans cette espèce de déchaînement un moyen
d'obtenir ce qui lui était interdit et qu'elle désirait avec tant de
passion. Elle proposa à son mari d'aller offrir au roi sa vaisselle
d'argent, dans l'espérance que cet exemple serait suivi, et qu'elle
aurait le gré de l'invention, et la récompense d'avoir procuré un
secours si prompt, si net et si considérable. Malheureusement pour elle
le duc, de Grammont en parla au maréchal de Boufflers son gendre, comme
il allait exécuter ce conseil. Le maréchal trouva cela admirable, s'en
engoua, alla sur les pas de son beau-père offrir la sienne dont il avait
en grande quantité, et admirable, et en fit tant de bruit pour y
exhorter tout le monde, qu'il passa pour l'inventeur, et qu'il ne fut
pas seulement mention de la vieille Grammont, ni même du duc de
Grammont, qui en furent les dupes, et elle enragée. Il en avait parlé à
Chamillart, son ancien ami du billard, pour en parler au roi. Cette
offre entra dans la tête du ministre, et par lui dans celle du roi à qui
Boufflers alla tout droit. Lui et son beau-père furent fort remerciés.

Aussitôt la nouvelle en vola au Chenil. M. de La Rochefoucauld à
l'instant se fit mener chez le roi qu'il trouva allant passer chez
M\textsuperscript{me} de Maintenon, et l'embarrassa par une vive sortie
de plaintes et de reproches qui n'étonnèrent pas moins le courtisan, car
cette fois il l'attendit à son passage. La fin de ce torrent et de ces
convulsions énergiques, la cause de son mauvais traitement, de son
profond malheur, fut que le roi, voulant bien accepter la vaisselle de
tout le monde, ne lui eût pas fait la grâce de lui demander d'abord la
sienne. À ces mots le roi s'en tint quitte à bon marché, et pour la
première fois le courtisan au lieu d'applaudir s'écoula en silence en
levant les épaules. Le roi répondit qu'il n'avait encore rien résolu sur
cela, que s'il acceptait les vaisselles il serait averti, et qu'il lui
savait gré de son zèle. Le duc redoubla d'empressement et de cris en
aveugle qu'il était, avec lesquels il suivit le roi tant qu'il put, au
lieu des termes qui ne se présentaient pas souvent à lui, et bien
content de soi, il s'en retourna dans son Chenil.

Ce bruit de la vaisselle fit un grand tintamarre à la cour. Chacun
n'osait ne pas offrir la sienne\,; chacun y avait grand regret. Les uns
la gardaient pour une dernière ressource dont il les fâchait fort de se
priver\,; d'autres craignaient la malpropreté de l'étain et de la
terre\,; les plus esclaves s'affligeaient d'une imitation ingrate dont
tout le gré serait pour l'inventeur. Le lendemain, le roi en parla au
conseil des finances, et témoigna pencher fort à recevoir la vaisselle
de tout le monde.

Cet expédient avait déjà été proposé et rejeté par Pontchartrain,
lorsqu'il était contrôleur général, qui, devenu chancelier, n'y fut pas
plus favorable. On objectait que l'épuisement était depuis ces temps-là
infiniment augmenté et les moyens également diminués. Ce spécieux ne le
toucha point. Il opina fortement contre, représenta le peu de profit par
rapport à l'objet, si considérable pour chaque particulier, et un profit
court et peu utile, qui tôt perçu n'apporterait pas un soulagement qui
tînt lieu de quelque chose\,; l'embarras et la douleur de chacun, et la
peine dans l'exécution de ceux-là mêmes qui le feraient de meilleur
coeur\,; la honte de la chose en elle-même\,; la bigarrure de la cour et
de la première volée d'ailleurs en vaisselle de terre, et des
particuliers de Paris et des provinces en vaisselle d'argent, si on en
laissait la liberté\,; et si on ne la laissait pas, le désespoir
général, et la ressource des cachettes\,; le décri des affaires qui,
après cette ressource épuisée, et qui la serait en un moment, et
paraîtrait extrême et dernière, sembleraient n'en avoir plus aucune\,;
enfin le bruit que cela ferait chez les étrangers, l'audace, le mépris,
les espérances que les ennemis en concevraient\,; le souvenir de leurs
railleries lorsqu'en la guerre de 1688 tant de précieux meubles d'argent
massif qui faisaient l'ornement de la galerie et des grands et petits
appartements de Versailles et l'étonnement des étrangers, furent envoyés
à la Monnaie, jusqu'au trône d'argent\,; du peu qui en revint, et de la
perte inestimable de ces admirables façons plus chères que la
matière\footnote{Voy. ce qu'en dit Mine de Sévigné. Elle écrivait le I1
  décembre 1689\,: «\,M. le Dauphin et Monsieur ont envoyé leurs meubles
  à la Monnaie.\,» Et le 21 décembre\,: «\,Que dites-vous de tous ces
  beaux meubles de la duchesse du Lude et de tant d'autres qui vont
  après ceux de Sa Majesté à l'hôtel des Monnoies\,?.. Les appartements
  du roi ont jeté six millions dans le commerce.\,» D'après une note
  publiée dans les \emph{Oeuvres de Louis XIV} (t. VI, p.~507), cette
  somme ne s'éleva qu'à deux millions cinq cent cinq mille six cent
  trente-sept livres. Ce qui confirme ce que dit Saint-Simon \emph{du
  peu, qui en revint}.}, et que le luxe avait introduites depuis sur les
vaisselles, ce qui tournerait nécessairement en pure perte pour chacun.
Desmarets, quoique celui qui portait le poids des finances et que cela
devait soulager de quelques millions, opina en méfie sens et avec la
même force.

Nonobstant de si bonnes raisons et si évidentes, le roi persista à
vouloir non pas forcer personne, mais recevoir la bonne volonté de ceux
qui présenteraient leurs vaisselles, et cela fut déclaré ainsi et
verbalement, et on indiqua deux voies à faire le bon citoyen\,: Launay,
orfèvre du roi, et la Monnaie. Ceux qui donnèrent leur vaisselle à pur
et à plein l'envoyèrent à Launay, qui tenait un registre des noms et du
nombre de marcs qu'il recevait. Le roi voyait exactement cette liste, au
moins les premiers jours, et promettait à ceux-là, verbalement et en
général, de heur rendre le poids qu'il recevait d'eux quand ses affaires
le lui permettraient, ce que pas un d'eux ne crut ni n'espéra, et de les
affranchir du contrôle, monopole assez nouveau, pour la vaisselle qu'ils
feraient refaire. Ceux qui voulurent le prix de la leur l'envoyèrent à
la Monnaie. On l'y pesait en y arrivant\,; on écrivait les noms, les
marcs et la date, suivant laquelle on y payait chacun à mesure qu'il y
avait de l'argent. Plusieurs n'en furent point fâchés pour vendre leur
vaisselle sans honte, et s'en aider dans l'extrême rareté de l'argent.
Mais la perte et le dommage furent inestimables de toutes ces admirables
moulures, gravures, ciselures, de ces reliefs et de tant de divers
ornements achevés dont le luxe avait chargé la vaisselle de tous les
gens riches et de tous ceux du bel air.

De compte fait, il ne se trouva pas cent personnes sur la liste de
Launay, et le total du produit en don ou en conversion ne monta pas à
trois millions. La cour et Paris, encore les grosses têtes de la ville
qui n'osèrent s'en dispenser, et quelque peu d'autres qui crurent se
donner du relief, suivirent le torrent\,; nuls autres dans Paris, ni
presque dans les provinces. Parmi ceux même qui cessèrent de se servir
de leur vaisselle, qui ne furent pas en grand nombre, la plupart la
mirent dans le coffre pour en faire de l'argent, suivant leurs besoins,
ou pour la faire reparaître dans un meilleur temps.

J'avoue que je fis l'arrière-garde, et que, fort las des monopoles, je
ne me soumis point à un volontaire. Quand je me vis presque le seul de
ma sorte mangeant dans de l'argent, j'en envoyai pour un millier de
pistoles à la Monnaie, et je fis serrer le reste. J'en avais peu de
vieille de mon père, et sans façons, de sorte que je la regrettai moins
que l'incommodité et la malpropreté.

Pour M. de Lauzun, qui, en avait quantité et toute, admirable, son dépit
fut extrême, et l'emporta sur le courtisan. Le duc de Villeroy lui
demanda s'il l'avait envoyée\,; j'étais avec lui, le duc de La
Rocheguyon et quelques autres. «\,Non, encore, répondit-il d'un ton tout
bas et tout doux. Je ne sais à qui m'adresser pour me faire la grâce de
la prendre, et puis, que sais-je s'il ne faut pas que tout cela passe
sous le cotillon de la duchesse de Grammont\,?» Nous en pensâmes tous
mourir de rire\,; et, lui, de faire la pirouette et nous quitter.

Tout ce qu'il y eut de grand ou de considérable se mit en huit jours en
faïence, en épuisèrent les boutiques, et mirent le feu à cette
marchandise, tandis que tout le médiocre continua à se servir de son
argenterie.

Le roi agita de se mettre à la faïence\,; il envoya sa vaisselle d'or à
la Monnaie, et M. le duc d'Orléans le peu qu'il en avait. Le roi et la
famille royale se servirent de vaisselle de vermeil et d'argent\,; les
princes et les princesses du sang de faïence. Le roi sut peu après que
plusieurs avaient fait des démonstrations frauduleuses, et s'en expliqua
avec une aigreur qui lui était peu ordinaire, mais qui ne produisit
rien. Elle serait mieux tombée sur le duc de Grammont et sa vilaine
épousée, causes misérables d'un éclat si honteux et si peu utile. Ils
n'en furent pas les dupes, ils, encoffrèrent leur belle et magnifique
vaisselle\,; et la femme elle-même porta leur vieille à la Monnaie, oie,
elle se la fit très bien payer.

Pour d'Antin, qui en avait de la plus achevée et en grande quantité, on
peut juger qu'il fut des premiers sur la liste de Launay\,; mais, dès
qu'il eut le premier vent de la chose, il courut à Paris choisir force
porcelaine admirable, qu'il eut à grand marché, et enlever deux
boutiques de faïence qu'il fit porter pompeusement à Versailles.

Cependant les donneurs de vaisselle n'espérèrent pas longtemps d'avoir
plu. Au bout de trois mois, le roi sentit la honte et la faiblesse de
cette belle ressource, et avoua qu'il se repentait d'y avoir consenti.
Ainsi allaient alors les choses et pour la cour et pour l'État.

Les inondations de la Loire qui survinrent en même temps, qui
renversèrent les levées, et qui firent les plus grands désordres, ne
remirent pas de bonne humeur la cour ni les particuliers, par les
dommages qu'ils causèrent, et les pertes qui furent très grandes, qui
ruinèrent bien du monde et qui désolèrent le commerce intérieur.

Rouillé, à qui Torcy, le lendemain de son arrivée, avait envoyé ordre de
revenir, arriva incontinent après, sur quoi les armées de part et
d'autre s'assemblèrent en Flandre les ennemis commandés à l'ordinaire
par le duc de Marlborough et le prince Eugène\,; et le maréchal de
Villars dans les plaines de Lens.

Torcy eut aussi ordre d'envoyer au cardinal de Bouillon de pouvoir
s'approcher de la cour et de Paris, à la distance de trente lieues. On
fut surpris que cet adoucissement fût venu du mouvement du roi, sans que
personne lui en eût parlé. Avant la disgrâce de M. de Vendôme, il lui
avait parlé en faveur du grand prieur, en même temps que le P. Tellier
l'avait pressé pour le cardinal de Bouillon. Il les avait refusés tous
deux. Il demanda ensuite à Torcy si M. de Bouillon ne lui avait pas
parlé souvent pour son frère. Torcy lui dit qu'il ne lui en avait point
parlé. du tout. «\,Cela est fort extraordinaire, répliqua le roi d'un
air piqué, qu'un frère ne parle pas pour son frère\,; M. de Vendôme m'a
bien pressé pour le sien.\,» C'est que le roi aimait que toute une
famille se sentit affligée d'une disgrâce, et que, lors même qu'il la
voulait le moins adoucir, il était blessé du peu d'empressement, et
qu'on ne lui fournit pas l'occasion de refuser et d'humilier.

\hypertarget{chapitre-xiv.}{%
\chapter{CHAPITRE XIV.}\label{chapitre-xiv.}}

1709

~

{\textsc{Fautes de Chamillart à l'égard de Monseigneur.}} {\textsc{-
Énormes procédés de M\textsuperscript{lle} de Lislebonne à l'égard de
Chamillart.}} {\textsc{- Vues et menées de d'Antin contre Chamillart.}}
{\textsc{- Réunion contre Chamillart de M\textsuperscript{me} de
Maintenon avec Monseigneur et M\textsuperscript{lle} Choin, qui refuse
pension, Versailles et Marly.}} {\textsc{- Bruits fâcheux sur
Chamillart.}} {\textsc{- Bon mot de Cavoye.}} {\textsc{- Grands
sentiments et admirable réponse de Chamillart.}} {\textsc{- Durs propos
de Monseigneur à Chamillart, qui achève de le perdre.}} {\textsc{-
Cusani, nonce du pape, comble la mesure contre Chamillart.}}

~

Les armées étaient assemblées et les frontières en fort mauvais état\,;
elles étaient toutefois plus tranquilles que l'intérieur de la cour, où
la fermentation était extrême. Depuis qu'à la mort du cardinal Mazarin
le roi s'était mis à gouverner lui-même, C'est-à-dire en quarante-huit
ans, on n'avait vu tomber que deux ministres\,: Fouquet, surintendant
des finances, qu'il ne tint pas à Colbert et à Le Tellier qu'il ne
perdît la vie, et qui fut confiné dans le château de Pignerol, où, après
trois ans de Bastille, il passa le reste de ses jours, qui durèrent plus
de seize ans, jusqu'en mars 1681, qu'il mourut à soixante-cinq ans. M.
de Pomponne est l'autre que MM. de Louvois et Colbert, d'ailleurs si
ennemis, mais réunis pour le perdre, firent chasser, par leurs
artifices, de sa charge de secrétaire d'État des affaires étrangères en
1679\footnote{Louis XIV, parlant dans ses Mémoires (t. II, p.~458) de la
  disgrâce d'Arnauld de Pomponne, s'exprime ainsi\,: «\,Il a fallu que
  je lui ordonnasse de se retirer, parce que tout ce qui passait par lui
  perdait de la grandeur et de la force qu'on doit avoir en exécutant
  les ordres d'un roi de France qui n'est pas malheureux.\,»}, assez
contre le goût du roi, qui le rappela douze ans après dans le ministère
à la mort de Louvois. Celui-ci mort subitement, la veille du jour qu'il
devait être arrêté, ne peut passer pour le troisième exemple. Chamillart
le fut, et le dernier de ce règne, et peut-être le plus difficile de
tous à chasser, sans toutefois d'autre appui que la, seule affection du
roi, qui ne céda qu'à regret à toutes les forces qui furent employées à
le lui arracher.

Sans répéter ce que j'ai déjà dit des causes qui le perdirent et qui lui
déchaînèrent M\textsuperscript{me} de Maintenon et M\textsuperscript{me}
la duchesse de Bourgogne, il faut parler d'une faute précédente qu'il
aggrava sur la fin, mais d'une nature qui n'a été funeste qu'à lui seul.
Jamais il n'avait ménagé Monseigneur. Ce prince, qui était timide et
mesuré sous le poids d'un père qui, jaloux à l'excès, ne lui lais soit
pas prendre le moindre crédit, ne hasardait que bien rarement de
recommandations aux ministres, encore était-ce pour peu de chose, et
poussé par quelque bas domestiques de sa confiance. Du Mont était celui
qu'il en chargeait, et qui, accoutumé à trouver Pontchartrain, lorsqu'il
était contrôleur général, prompt à plaire à Monseigneur, et à en
rechercher les occasions, se trouva bien étonné lorsqu'il eut affaire à
Chamillart, successeur de l'autre aux finances. Celui-ci, faussement
préoccupé, que, avec le roi et M\textsuperscript{me} de Mainte non pour
lui, tout autre appui lui était inutile, et que, sur le pied où était
Monseigneur avec eux, il se nuirait en faisant la moindre chose qui, en
leur revenant, leur donnerait soupçon qu'il voulait s'attacher à lui,
n'eut aucun égard aux bagatelles que Monseigneur désirait, en garde même
qu'on ne {[}se{]} servit de son nom, reçut du Mont si mal, que celui-ci,
glorieux de la faveur et de la confiance de son maître, et de la
considération qu'elle lui, attirait des ministres et de tout ce qui
était le plus relevé à la cour, se plaignit souvent à Monseigneur, le
pria de charger tout autre que lui des commissions pour le contrôleur
général, et l'aigrit extrêmement contre lui.

Je m'étais bien aperçu, à un voyage de Meudon, que Monseigneur n'était
pas content de Chamillart. Quelques propos de du Mont et quelques
bagatelles ramassées m'en avaient mis sur les voies. J'en avertis ses
filles à Meudon même, où elles vinrent deux fois ce voyage-là. Elles
s'informèrent et trouvèrent qu'il était vrai. Elles en firent parler à
Monseigneur, qui en usa comme j'ai dit qu'avait fait
M\textsuperscript{me} la duchesse de Bourgogne en pareil cas, et cela
demeura ainsi jusqu'à la catastrophe de Turin.

La Feuillade, noyé à sors retour, et dès auparavant courtisan assidu de
M\textsuperscript{lle} Choin, comprit que de la lier à son beau-père,
leur pouvait être à tous deux fort utile un jour, et à lui, en
attendant, d'un grand usage auprès de Monseigneur. Il la tourna si bien,
qu'elle y mordit\,; elle ne pouvait rien par Monseigneur, qui était en
brassière fort étroite. Elle était donc réduite à ce que sa confiance
lui donnait de considération pour l'avenir, et elle comprit que, en
attendant, l'amitié et le commerce de Chamillart lui pourrait servir à
beaucoup de choses.

La Feuillade, ravi d'avoir pu apprivoiser une créature si importante que
la politique rendait si farouche, parla à son beau-père, et fut fort
surpris de le trouver très froid. Il le pressa, il déploya son
éloquence, et le tout pour néant. Il espéra en venir à bout, et
cependant amusa bille Choin de compliments, de voyages et de temps mal
arrangés. Elle ne laissa pas d'être surprise de voir ses avances
languir, elle qui n'était occupée que de parades et de refus de commerce
avec ce qu'il y avait de plus important qui faisait tout pour y être
admis.

L'entrevue se différant toujours, parce que Chamillart n'y voulait point
entendre, et que son gendre palliait toujours de prétextes,
M\textsuperscript{lle} Choin en parla à M\textsuperscript{lle} de
Lislebonne, si intimement avec Chamillart. Celle-ci craignit que cette
liaison se fît sans elle, et d'être privée du mérite des deux côtés d'y
avoir travaillé, se hâta d'en parler à Chamillart, qui, d'un ton de
confiance, et d'un air de complaisance, pour ne pas dire de mépris, lui
apprit que cette connaissance se serait faite depuis fort longtemps,
s'il l'avait voulu\,; qu'on l'en pressait toujours\,; que La Feuillade
le voulait\,; mais que, pour lui, il ne savait pas à quoi cela serait
bon à M\textsuperscript{lle} Choin et à lui\,; qu'il était trop vieux
pour des connaissances nouvelles\,; qu'il ne lui en fallait point au
delà de son cabinet\,; que le roi et M\textsuperscript{me} de Maintenon
lui suffi soient, et que les intrigues et les cabales de cour ne lui
allaient point.

Qui fut étonné\,? ce fut M\textsuperscript{lle} de Lislebonne. Elle
n'avait pas le même intérêt que La Feuillade\,; elle sentit le fait
qu'il n'avait osé avouer à M\textsuperscript{lle} Choin qu'il amusait
cependant\,; elle connaissait assez Chamillart pour comprendre que, avec
ces belles maximes dont il s'applaudissait, elle ne lui en ferait pas
changer\,; ainsi elle ne lui en dit pas davantage, pour ne pas lui
déplaire inutilement. Mais ce que fit d'honnête cette bonne et sûre
amie, sur laquelle Chamillart comptait si fort, fut de rendre à
M\textsuperscript{lle} Choin cette conversation tout entière sans y
manquer d'un mot, pour se faire un mérite auprès d'elle d'avoir
découvert en un moment à quoi il tenait qu'elle ne vit Chamillart, et
l'empêcher d'être plus longtemps la dupe du beau-père et du gendre.

Il est aisé de comprendre quel fut l'effet de ce rapport si fidèle dans
une créature devant qui tout rampait, à commencer par Mgr et
M\textsuperscript{me} la duchesse de Bourgogne, que, comme
M\textsuperscript{me} de Maintenon, elle voyait de son fauteuil sur un
tabouret, et n'appelait, et devant Monseigneur, que «\,la duchesse de
Bourgogne,\,» à continuer par M\textsuperscript{me} la Duchesse et par
tout ce que la cour avait de plus grand, de plus distingué, de plus
accrédité. La Feuillade sentit bientôt quelque altération dans cet
esprit contre son beau-père\,; et M\textsuperscript{lle} de Lislebonne,
qui connaissait parfaitement le terrain, compta d'un air de simplicité
ce qui s'était passé aux filles de Chamillart, comme un office de
prudence, pour faire passer plus doucement ce qu'une continuation de
suspens eût bientôt révélé et avec plus d'aigreur\,; et le rare est,
qu'elle les persuada, tant il est vrai qu'il est des personnes à qui
nulle énormité ne nuit, et d'autres destinées à un aveuglement
perpétuel. La bonne Lorraine, sachant bien à qui elle avait affaire, mit
ce gabion devant elle, de peur de demeurer brouillée avec Chamillart, si
sa délation lui revenait {[}autrement{]} que palliée de cet air de
franchise qui n'y entendait point finesse. Chamillart n'y fit pas plus
de réflexion qu'en avaient fait ses filles, et on a vu jusqu'où
M\textsuperscript{lle} de Lislebonne et son cher oncle le conduisirent
sur les affaires de Flandre. Longtemps après ce trait, il en arriva
encore un autre presque tout pareil.

M\textsuperscript{lle} Choin avait un frère major dans le régiment de
Mortemart, qu'elle désirait passionnément avancer. Il était bon sujet,
et passait pour tel dans ce régiment et dans les troupes. Il était
question l'obtenir un de ces petits régiments d'infanterie de nouvelle
création, qui vaquait, dont on avait donné plusieurs à des gens qui ne
le valaient pas. Quelque rebutée et dépitée qu'elle fût sur
Chamillart\,; l'extrême désir d'avancer ce frère, et l'impossibilité d'y
réussir sans le secrétaire d'État de la guerre, la forcèrent d'en parler
à La Feuillade. Celui-ci, ravi d'une occasion si naturelle de l'apaiser
par son beau-père, se chargea avec joie de l'affaire. Il en parla à
Chamillart, ne doutant pas d'emporter d'emblée une chose si raisonnable
en soi, dans un temps encore où les avancements avaient si peu de règle,
et où celui-ci devait sembler si précieux à Chamillart pour réparer le
passé s'il était possible\,; mais quelques raisons qu'il pût lui
alléguer, quelque crédit qu'il eût auprès de lui, jamais il ne put rien
gagner. Il se figura gauchement un mérite auprès du roi de laisser ce
major dans la poussière des emplois subalternes il s'irrita des plus
essentielles raisons de l'en tirer\,; en deux mots, sa soeur lui devint
un obstacle invincible auprès du ministre.

La Feuillade, outré, espéra de sa persévérance, et amusa encore une
fois. M\textsuperscript{lle} Choin qui, surprise dès le premier délai et
instruite par l'autre aventure, lâcha encore en celle-ci
M\textsuperscript{lle} de Lislebonne à Chamillart, ou pour réussir par
ce surcroît auprès de lui, ou pour en avoir le coeur net.
M\textsuperscript{lle} de Lislebonne en parla à La Feuillade, et tous
deux ensemble à Chamillart pour essayer de le réduire\,; mais ce fut en
vain jusque-là qu'il s'irrita de nouveau, et qu'il s'échappa un peu sur
le crédit que M\textsuperscript{lle} Choin se figurait qu'elle pouvait
prétendre. Le régiment fut incontinent donné à un autre, et
M\textsuperscript{lle} Choin instruite de point en point de ce qui
s'était passé par M\textsuperscript{lle} de Lislebonne. Ce dernier
procédé mit le comble dans le coeur de M\textsuperscript{lle} Choin, et
la rendit la plus ardente ennemie de Chamillart et la plus acharnée.

Je sus ces deux anecdotes dans les premiers moments, trop tard pour y
pouvoir rien faire\,; je n'aurais pas même espéré de réussir où La
Feuillade et M\textsuperscript{lle} de Lislebonne avaient échoué, mais
j'en augurai mal. D'Antin était trop initié dans les mystères de Meudon
pour ignorer ces diverses lourdises, le dépit de M\textsuperscript{lle}
Choin, tous les mauvais offices qu'elle rendait à Chamillart auprès de
Monseigneur, d'ailleurs irrité contre lui de plus ancienne date, que du
Mont n'adoucissait pas. D'Antin n'ignorait pas, comme je l'ai dit plus
haut, la haine que M\textsuperscript{me} la duchesse de Bourgogne et
M\textsuperscript{me} de Maintenon avaient conçue contre ce ministre à
qui il se flattait de succéder, et dans cette vue il mit
M\textsuperscript{me} la duchesse de Bourgogne au fait de tout ce qui
vient d'être expliqué. Il eut bientôt après le contentement de le voir
germer.

M\textsuperscript{me} de Maintenon n'était pas à s'apercevoir de toutes
les forces dont elle avait besoin pour arracher au roi un ministre en
qui il avait mis toute sa complaisance. Vendôme subsistait encore, et
tout cela ne faisait qu'un, et lui était également odieux. Pour la
première fois de sa vie elle crut avoir besoin de Monseigneur. C'est ce
qui l'engagea à déterminer le roi à lui destiner l'armée de Flandre,
afin de les mettre dans la nécessité, Monseigneur de se mêler de ce qui
regardait cette armée, et le roi de le trouver bon, pour se servir après
contre Chamillart du fils auprès du père qui, sans ce chausse-pied,
n'aurait osé parler. De là profitant de quelque chose que le roi marqua
sur les voyages de Meudon, si continuels pendant l'été, qui emmenaient
du monde, et laissait Versailles fort seul, elle le ramassa en ce
temps-ci\,; et pour le faire court, persuada au roi que pour les rendre
rares et combler Monseigneur à bon marché, il fallait donner à
M\textsuperscript{lle} Choin une grosse pension, un logement à
Versailles, la mener tous les voyages à Marly, et mettre ainsi
Monseigneur en liberté de la voir publiquement, ce qui le rendrait plus
sédentaire à Versailles, et les Meudons moins fréquents.

Jusqu'alors ces deux si singulières personnes s'étaient comme ignorées.
Un si grand changement flatta Monseigneur\,; il combla
M\textsuperscript{lle} Choin, mais il ne séduisit ni l'un ni l'autre.
Monseigneur, en acceptant, y aurait perdu la liberté qu'il croyait
trouver à Meudon\,; et M\textsuperscript{lle} Choin, qui y primait,
n'aurait été que fort en second vis-à-vis M\textsuperscript{me} de
Maintenon. Elle craignit de plus qu'un tel changement, qui ne serait
plus soutenu de l'imagination du mystère, car il n'en restait encore que
cela, n'apportât avec le temps du changement à sa fortune, qui n'était
pas comme celle de M\textsuperscript{me} de Maintenon appuyée de la base
du sacrement. Elle se jeta donc dans les respects, la confusion,
l'humilité, le néant\,; Monseigneur, sur ce qu'il ne l'avait pu
résoudre\,; et refusa jusqu'à la pension, sur ce que, dans la situation
malheureuse des affaires et à la vie cachée qu'elle menait et voulait
continuer, elle en avait assez.

Tout cela se conduisit avec une satisfaction tellement réciproque, que
d'Antin par qui une partie de ces choses avaient passé, fut chargé des
confidences contre Chamillart, et que le dîner qu'on a vu que le roi et
M\textsuperscript{me} de Maintenon firent à Meudon, sans y coucher, et
qui causa la dernière catastrophe de M. de Vendôme, ne fut à l'égard du
roi que pour presser M\textsuperscript{lle} Choin par
M\textsuperscript{me} de Maintenon elle-même, qui n'avait jamais
occasion de la voir, d'accepter ce qu'on vient de voir qui lui était
offert et qui était dès lors refusé, mais en effet pour s'entretenir de
toutes les mesures à prendre pour la chute de Chamillart, et y faire
agir Monseigneur pour la première fois de sa vie qu'il fût entré avec le
roi en chose importante, si on en excepte le conseil d'État.

Ces mesures réciproques firent encore que non seulement Villars, chargé
du commandement de l'armée de Flandre sous Monseigneur, travailla
plusieurs fois avec lui, mais qu'Harcourt y travailla aussi, quoiqu'il
allât sur le Rhin, et que, après mémé qu'il fut déclaré qu'aucun des
princes ne sortirait de la cour, ces généraux, contre tout usage,
continuèrent de travailler avec Monseigneur, parce que
M\textsuperscript{me} de Maintenon voulut qu'Harcourt le pût conduire
sur ce qu'il avait à faire et à dire contre Chamillart, et qu'il lui fit
même sa leçon pour jusqu'après son départ. La même raison de pousser
Chamillart fit tenir au roi et l'assemblée et le conseil de guerre
desquels j'ai parlé, et qui excita tout ce qu'on put à attaquer ce
ministre.

Toutes ces choses qui touchèrent Monseigneur par une considération qu'à
sort âge il n'avait pas encore éprouvée, le rapprochèrent de
M\textsuperscript{me} de Maintenon. Jusqu'alors ils étaient
réciproquement éloignés. Il lui fit deux ou trois visites tête à tête.
Là se prirent les dernières résolutions contre Chamillart, et ce prince
{[}y prit{]} le courage et l'appui qui lui étaient nécessaires pour
venger son ancien mécontentement, et servir la haine de
M\textsuperscript{lle} Choin, en l'attaquant à découvert auprès du roi,
comme un sacrifice indispensable au soutien des affaires.

Harcourt, lâché par M\textsuperscript{me} de Maintenon, avait jusqu'à
son départ eu de longues et de fréquentes audiences du roi, et y avait
frappé de grands coups. Villars, qui avait été mal avec lui, mais qui
était raccommodé, y fut plus sobre\,; mais il ne put refuser ni se
hasarder pour autrui de tromper M\textsuperscript{me} de Maintenon.
Boufflers était l'enfant perdu par les raisons qu'on a vues et par son
dévouement à M\textsuperscript{me} de Maintenon. Il avait les grandes
entrées\,; il était en quartier de capitaine des gardes\,; il jouissait
encore auprès du roi de toute la verdeur de ses lauriers. Il avait cent
occasions par jour de particuliers avec le roi\,; il en était toujours
bien reçu. Il marchait en puissante troupe. Il rompit glaces et lances,
et ne donna aucun repos au roi. Monseigneur fit son personnage avec
force\,; et jusqu'à M. du Maine, que le pauvre Chamillart croyait son
protecteur, n'osa refuser à M\textsuperscript{me} de Maintenon des
lardons secrets et acérés. Tout marchait en ordre et en cadence, et
toujours avec connaissance et sagesse, pour ne pas rebuter en poussant
toujours et toujours avec la même ardeur.

Le roi, déjà accoutumé par M\textsuperscript{me} de Maintenon, par lés
généraux de ses armées, par d'autres cardinaux plus obscurs mais qui
n'en réussissaient pas moins, par M\textsuperscript{me} la duchesse de
Bourgogne, par quelques mots de Mgr le duc de Bourgogne que son épouse
obtenait de lui, par d'Antin excité par l'espérance, à entendre dire
beaucoup de mal de son ministre, et c'était déjà beaucoup, était ébranlé
par raison, mais le coeur tenait ferme. Il le regardait comme son choix,
comme son ouvrage dans tous ses emplois, jusqu'au comble où il l'avait
porté, et dans ce comble même comme son disciple. Pas un de tous ses
ministres ne lui avait tenu les rênes si lâches\,; et, depuis que toute
puissance lui eut été confiée, le roi n'en avait jamais senti le joug.
Tout l'hommage lui en était reporté. Une habitude longue avant qu'il fût
en place, une dernière confiance depuis plus de dix ans, sans aucune
amertume la plus passagère, le réciproque attentif de cette confiance
par une obéissance douce, et un compte exact de tout, avaient joint le
favori au ministre. Une admiration vraie et continuelle, une
complaisance naturelle avaient poussé le goût jusqu'où il pouvait aller.
C'était donc beaucoup que tant de coups concertés et redoublés eussent
pu ébranler la raison. Elle l'était\,; mais quel obstacle ne restait-il
point à vaincre par ce qui vient d'être expliqué\,! Plus il était grand
et plus il irritait, et plus il donnait d'inquiétude à ceux qui
formaient l'attaque, et qui commandaient les travailleurs.

M\textsuperscript{me} de Maintenon, qui savait que Monseigneur avait
fortement parlé, et qu'il avait été écouté, redoubla d'instances auprès
de M\textsuperscript{lle} Choin et de lui pour le faire recharger. Ce
prince s'était laissé persuader par d'Antin de travailler à lui faire
tomber la guerre. L'estime et l'amitié sont rarement d'accord chez les
princes\,; celui-ci désira de tout son coeur de mettre là d'Antin, et
s'en flatta beaucoup. M\textsuperscript{me} de Maintenon, sans
s'engager, se montra favorable pour mieux les exciter.

Tant de machines ne pouvaient être en si grand mouvement sans quelque
sorte de transpiration. Il s'éleva au milieu de la cour je ne sais
quelle voix confuse, sans qu'on en pût désigner les organes immédiats,
qui publiait qu'il fallait que l'État ou Chamillart périssent\,; que
déjà son ignorance avait mis le royaume à deux doigts de sa perte\,; que
c'était miracle que ce n'en fût déjà fait, et folie achevée de le
commettre un jour de plus à un péril qui était inévitable tant que ce
ministre demeurerait en place. Les uns ne rougissaient pas des injures,
les autres louaient ses intentions, et parlaient avec modération des
défauts que beaucoup de gens lui reprochaient aigrement. Tous
convenaient de sa droiture, mais un successeur tel qu'il fût ne leur
paraissait pas moins nécessaire. Il y en avait qui, croyant ou voulant
persuader qu'ils porteraient l'amitié jusqu'où elle pouvait aller,
protestaient de la conserver toujours et de n'oublier jamais les
plaisirs et les services qu'ils avaient reçus de lui, mais qui avouaient
avec délicatesse qu'ils préféraient l'État à leur avantage particulier
et à l'appui qu'ils s'affligeaient de perdre, mais que si Chamillart
était leur frère, ils concluraient également à l'ôter, par l'évidence de
la nécessité de le faire. Sur la fin on ne comprenait pas ni comment il
avait pu être choisi, ni comment il était demeuré en place

Cavoye, à qui un si long usage de la cour et du grand monde tenait lieu
d'esprit et de lumière, et fournissait quelquefois d'assez bons mots,
disait que le roi était bien puissant et bien absolu et plus qu'aucun de
ses prédécesseurs, mais qu'il ne l'était pas assez pour soutenir
Chamillart en place contre la multitude. Les choses les plus
indifférentes lui étaient tournées à crime ou à ridicule. On eût dit
que, indépendamment de toute autre raison, c'était une victime que le
roi ne pouvait plus refuser à l'aversion publique. Force gens s'en
expliquaient tout nettement ainsi, et pas un qui pût énoncer une seule
accusation particulière. On s'en tenait à un vague qui se pouvait
appliquer à qui on voulait, sans que de tant de personnes qu'il avait si
fort obligées aucune prît sa défense, parmi tant d'autres qui, naguère
adorateurs de la fortune, se piquaient de louanges, d'admiration et
d'une adulation servile pour un homme qu'ils voyaient si rudement
attaqué\,; et si l'excès de ce qui se donnait en reproches poussait
quelqu'un à répondre, on insistait à demander des comptes, ou absurdes,
ou de choses sur lesquelles un respect supérieur fermait la bouche. Les
troupes dénuées de tout, les places dégarnies, les magasins, vides
sautaient aux yeux\,; mais on ne voulait plus se souvenir des deux
incroyables réparations des armées, l'une après Hochstedt en trois
semaines, l'autre en quinze jours seulement après Ramillies, qui
tenaient du Prodige, et qui néanmoins avaient deux fois sauvé l'État,
pour ne parler que de deux faits si importants et si publics. Il n'en
restait plus la moindre trace, une fatale éponge avait passé dessus\,;
et si quelqu'un encore osait les alléguer, faute de, réponse on tournait
le dos. Tels furent les derniers présages de la chute de Chamillart.

Je ne lui laissai pas ignorer tant de menaces, ni tous les ressorts qui
se remuaient contre lui, et je le pressai de parler au roi, comme il a
voit déjà fait une autre fois à ma prière, et dont il s'était si bien
trouvé que l'orage prêt, à fondre sur lui en avait été dissipé\,; mais
il pensa trop grandement pour un ministre de robe. Il me répondit qu'il
ne croyait pas que sa place valût la peine de soutenir un siège, ni
devoir, ajouter au travail qu'elle demandait celui de s'y défendre\,;
que tant que l'amitié du roi serait d'elle-même assez forte, il y
demeurerait avec agrément, mais que si cet appui avait besoin d'art,
l'art le dégoûterait de l'appui et lui rendrait son état
insupportable\,; qu'en un mot, des temps aussi fâcheux demandaient un
homme tout entier au timon de la guerre\,; que se partager entre les
affaires de l'État et les siennes particulières ne pouvait aller qu'à
une lutte honteuse à lui et dommageable au gouvernement par la
dissipation où il se laisserait aller, d'où il résultait qu'il fallait
laisser aller les choses au gré du sort, ou, pour mieux dire, de la
Providence, content de céder à un homme plus heureux, ou de continuer
son ministère avec honneur et tranquillité. Des sentiments pratiques si
relevés me touchèrent d'une admiration qui me fit redoubler d'efforts
pour l'engager de parler au roi. Jamais il ne voulut y entendre, ni
s'écarter d'une ligne de son raisonnement\,; et dès lors je compris sa
chute très prochaine et sans remède..

Lés choses en étaient là lorsque Chamillart fut à Meudon rendre compte à
Monseigneur de l'état de la frontière et de l'armée de Flandre, et lui
dire, ce qu'il savait déjà par le roi, qu'il ne la commandait plus.
Monseigneur, qui avait déjà parlé contre lui au roi avec une force qui
lui avait été jusque-là inconnue, et qu'il ne tenait que des
encouragements de M\textsuperscript{lle} Choin et de
M\textsuperscript{me} de Maintenon, prit ce temps pour reprocher à
Chamillart que tous ces manquements n'arrivaient que par ses fautes, et
alla jusqu'à lui dire que son La Cour aurait mieux fait de bien fournir
les vivres des armées, dont il avait été chargé, que de lui bâtir de si
belles maisons, puis sortit avec lui de son bâtiment neuf où cette
conversation s'était faite tête à tête, et revenus au gros du monde, le
lui montra tout entier comme s'il ne s'était rien passé, et se hâta
après d'aller se vanter à M\textsuperscript{lle} Choin de ce qu'il
venait de dire. Elle applaudit fort à de si rudes propos, et s'en
avantagea pour exciter Monseigneur à ne pas différer auprès du roi
d'achever un ouvrage si nécessaire et si bien commencé, ce qu'il exécuta
aussi, et il donna le dernier coup de mort à ce ministre.

Un hasard lui en prépara la voie et combla la mesure de tout ce qui
s'était brassé contre lui. J'ai parlé, il y a peu, d'une longue audience
que le maréchal de Tessé eut du roi pour lui rendre compte de son voyage
d'Italie. Cusani, Milanais, mort cardinal il n'y a pas fort longtemps,
avait été accepté ici pour succéder au cardinal Gualterio. Il était
frère d'un des généraux de l'empereur, et se montra si autrichien,
pendant tout le cours de sa nonciature, qu'on eut lieu de se repentir de
s'y, être si lourdement mépris. Ce fut avec lui que se négocia à paris
la ligue d'Italie, dont on a parié, et lui qui sollicita la permission
des levées et de l'achat des armes pour le pape en Avignon, qui ne fut
accordée qu'avec des difficultés et une lenteur inexcusables. Ce nonce
en avait fait des plaintes amères en ce temps-là.

Étant le mardi 4 juin dans la galerie de Versailles, attendant que le
roi allât à la messe, il avisa le maréchal de Tessé qui causait avec le
maréchal de Boufflers, tous deux seuls et séparés de tout le monde. Le
nonce, qui n'avait point vu Tessé depuis son retour, alla à lui\,; et
après les premières civilités, se mirent bientôt sur les affaires qui
avaient mené Tessé en Italie. Les plaintes dont je viens de parler
trouvèrent promptement leur place dans la conversation, auxquelles
Cusani ajouta qu'il ne serait jamais venu à bout d'obtenir la permission
qu'il demandait, sans un millier de pistoles qu'il s'était enfin avisé
de faire offrir à la femme de Chamillart, dont le payement avait opéré
avec promptitude.

Il parlait à deux ennemis de Chamillart, et il ne fut guère douteux
qu'il ne s'y méprenait pas. On a vu les causes de l'acharnement du
maréchal de Boufflers contre le ministre. Tessé, plus en douceur, ne le
haïssait pas moins\,: il ne pouvait lui pardonner ce qu'il avait exigé
de lui en Dauphiné, en Savoie et en Italie, en faveur de La Feuillade,
{[}ce{]} qu'on a vu en son lieu, pour le porter rapidement au
commandement des armées, ce qui ne put se faire qu'à ses dépens. En
flexible Manceau il s'y était prêté de bonne grâce dans cette
toute-puissance de Chamillart, mais il n'en avait pas moins senti
l'injure d'être obligé de s'anéantir, et de se faire lui-même le pont de
La Feuillade pour lui monter sur les épaules et le chasser pour lui
succéder, sans oser n'en être pas lui-même le complice. En arrivant il
trouva le temps de la vengeance venu, et de l'exercer encore en plaisant
à M\textsuperscript{me} de Maintenon, à Monseigneur, à Mgr {[}le duc{]}
et à M\textsuperscript{me} la duchesse de Bourgogne, à tous les gens
encore avec qui il tâchait d'être uni, et qui étaient tous des
personnages. Il se jeta donc à eux tout en arrivant.

Ce fut le lendemain de cette aventure qu'il devait avoir audience de
M\textsuperscript{me} de Maintenon, et du roi ensuite, pour la première
fois depuis son retour. Soit de hasard, soit de concert, Boufflers alla
le même lendemain matin chez M\textsuperscript{me} de Maintenon, où les
portes lui étaient toujours ouvertes, et y trouva le maréchal de Tessé.
Boufflers lui demanda s'il avait bien rendu compte de toutes choses,
M\textsuperscript{me} de Maintenon en tiers. «\,De toutes celles que
madame m'a demandées, répondit Tessé. --- Mais cela ne suffit pas,
répliqua le maréchal de Boufflers, il ne lui faut laisser rien
ignorer.\,» Et par ce petit débat la curiosité de M\textsuperscript{me}
de Maintenon étant excitée, elle voulut en savoir la raison. Il y eut
encore quelques circuits adroits. Boufflers demanda à Tessé s'il avait
rendu compte à madame du discours que le nonce leur avait tenu la
veille, et publiquement. Tessé ayant répondu que non d'un air à
augmenter la curiosité, M\textsuperscript{me} de Maintenon voulut en
être informée. Tessé lui en fit le récit, mais en se récriant que cela
ne pouvait pas être, et se fondant sur la modicité de la somme, et prise
d'un étranger. Boufflers, au contraire, exagéra le crime, et tout ce
dont était capable une femme en cette place, qui n'avait pas honte de
recevoir si peu, et d'un étranger\,; combien de malversations elle avait
faites puisqu'elle avait pu se porter à celle-là, comment le roi pouvait
être servi, puisqu'une affaire de cette importance s'achetait, et ne
réussissait que par un présent\,; qu'enfin une femme tentée, et
succombant à si peu, l'était de tout, depuis un écu jusqu'à un million.
Tessé peu à peu se mit doucement de la partie, et sans mettre en aucun
doute la vérité de ce que le nonce leur avait dit, ils paraphrasèrent le
danger de laisser les affaires entre les mains du mari d'une femme si
avide, et laissèrent M\textsuperscript{me} de Maintenon presque
persuadée du fait, et ravie de la découverte.

Deux heures après, Tessé entra dans le cabinet du roi pour son audience.
Boufflers, qui vit le roi de loin à l'ouverture de la porte, fit
quelques pas en dedans après Tessé, et le prenant par le bras, lui dit
d'un ton élevé pour que le roi l'entendit\,: «\,Au moins, monsieur, vous
devez la vérité au roi. Dites-lui bien tout et ne lui laissez rien
ignorer.\,» Il répéta encore une autre fois plus haut et se retira,
laissant {[}au{]} roi un grand sujet de curiosité, et au maréchal de
Tessé la nécessité de lui dire ce qu'il avait déjà appris à
M\textsuperscript{me} de Maintenon.

Les deux maréchaux avaient déjà répandu le discours du nonce\,; qui fit
un étrange bruit, et ce bruit fut le dernier éclair qui précéda le coup
de foudre, qu'une dernière conversation que Monseigneur, venu exprès un
matin de Meudon, eut ensuite avec le roi, acheva de déterminer.
Cependant le roi ne fit aucun semblant d'avoir su cette histoire, ni
M\textsuperscript{me} de Maintenon\,; et ce silence de leur part fut une
des choses que les ducs de Chevreuse et de Beauvilliers regardèrent
comme un signal le plus sinistre. Ils ne s'y trompèrent pas.

Je ne sais s'il eût encore été temps pour Chamillart. Cette audience de
Tessé fut le mercredi, et Chamillart m'a conté depuis sa disgrâce, que,
près de succomber, il avait toujours éprouvé le même accueil et le même
visage du roi, jusqu'au vendredi, surveille de sa chute et surlendemain
de l'audience de Tessé\,; que ce jour-là, il le remarqua embarrassé avec
lui\,; et que, frappé qu'il fut d'un changement si soudain, il fut sur
le point de lui demander s'il n'avait plus le bonheur de lui plaire, et
en cas de ce malheur, la permission de se retirer plutôt que de le
contraindre. S'il l'eût fait, il y a lieu de croire par tout ce qui
parut depuis que le roi n'aurait pu y tenir et qu'il serait demeuré en
place. Mais il hésita, et le roi, qui craignit peut-être qu'il n'en vint
là, et qui, par la faiblesse qu'il se sentait peut-être, ne lui donna
pas le temps, à ce qu'il m'ajouta, de se déterminer lui-même, et ce fut
la dernière faute qu'il fit contre soi-même, et peut-être la plus lourde
de toutes\,; et si, avant ce dernier coup de poignard de l'audience de
Tessé et de la conversation de Monseigneur avec le roi ensuite,
Chamillart m'eût voulu encore croire à son retour de Meudon à l'Étang,
où il me conta ces propos si durs que Monseigneur lui avait tenus dans
son bâtiment neuf, et que, comme je l'en pressai pour la seconde fois
vainement de parler au roi il l'eût fait, il ne paraît pas douteux qu'il
ne se fût raffermi.

Dans ces derniers jours, M\textsuperscript{me} de Maintenon, se comptant
sûre enfin de la perte de Chamillart, et de n'avoir plus besoin de
Monseigneur ni de d'Antin pour jeter par terre un homme qu'elle tenait
pour sûrement abattu, ne crut plus avoir de mesures à garder, et se
donna tout entière à profiter de tous les instants pour élever sa
créature. Le détail de ce fait si pressé et si court, et qui n'eut point
de témoins entre le roi et elle, m'a échappé\,; elle ne l'a raconté
depuis à personne, ou, si elle l'a fait, l'anecdote n'en est pas venue
jusqu'à moi. Tout ce qu'on en a pu conjecturer, c'est qu'elle n'y
réussit pas sans peine, par deux faits qui suivirent incontinent et qui
seront remarqués en leur temps. Je n'ai pu découvrir non plus si le roi
en voulait un autre, ou s'il n'était point fixé. Monseigneur l'avait osé
presser pour d'Antin, profitant de la nouvelle liberté, qu'à l'appui de
M\textsuperscript{me} de Maintenon il avait usurpée sans danger, de
parler au roi de la situation des affaires et de la nécessité d'en ôter
Chamillart et de se voir écouté. D'Antin était reçu aussi à parler au
roi de ses troupes, de ses frontières, à lui en montrer des états qu'il
s'était fait envoyer, aller même jusqu'à se faire écouter sur des
projets d'opérations de campagne, appuyé de Monseigneur, ayant M. du
Maine favorable et M\textsuperscript{me} de Maintenon, et à ce qu'il se
figurait de leurs discours obligeants, il espérait tout dans ces
derniers jours de la crise, et fut bientôt après outré de douleur, et
Monseigneur fort fâché de s'y trouver trompé. Le samedi coula à
l'ordinaire et sans rien de marqué.

\hypertarget{chapitre-xv.}{%
\chapter{CHAPITRE XV.}\label{chapitre-xv.}}

1709

~

{\textsc{Disgrâce de Chamillart.}} {\textsc{- Magnanimité de
Chamillart.}} {\textsc{- Caractère de Chamillart et de sa famille.}}
{\textsc{- Voysin secrétaire d'État\,; sa femme\,; leur fortune\,; leur
caractère.}} {\textsc{- Spectacle de l'Étang.}} {\textsc{- Procédé
infâme de La Feuillade.}} {\textsc{- Accueil du roi à Cani.}} {\textsc{-
Beau procédé de Le Guerchois.}}

~

Le dimanche 9 juin, sur la fin de la matinée, la maréchale de Villars,
qui logeait porte à porte de nous, entra chez M\textsuperscript{me} de
Saint-Simon, comme elle faisait souvent, et d'avance nous demanda à
souper pour causer, parce qu'elle croyait qu'il y aurait matière. Elle
nous dit qu'elle s'en allait dîner en particulier avec Chamillart\,;
qu'un temps était que c'eût été grande grâce, mais que, pour le présent,
elle croyait la grâce de son côté. Ce n'était pourtant pas qu'elle sût
rien, à ce qu'elle nous assura depuis, mais elle parlait ainsi sur les
bruits du monde, qui, surtout depuis le mardi et le mercredi que le
discours du nonce s'était su, étaient devenus plus forts que jamais.

Ce même matin, le roi, en entrant au conseil d'État, appela le duc de
Beauvilliers, le prit en particulier, et le chargea d'aller
l'après-dînée dire à Chamillart qu'il était obligé, pour le bien de ses
affaires, de lui demander la démission de sa charge et celle de la
survivance qu'en avait son fils\,; que néanmoins il voulait qu'il
demeurât assuré de son amitié, de son estime, de la satisfaction qu'il
avait de ses services\,; que, pour lui en donner des marques, il lui
continuait sa pension de ministre, qui est de vingt mille livres, lui en
donnait une autre particulière, encore à lui, d'autres vingt mille
livres, et une à son fils aussi de vingt mille livres\,; qu'il désirait
que son fils achetât la charge de grand maréchal des louis de sa maison,
à quoi il avait disposé Cavoye, lequel, sa vie durant, en conserverait
le titre, les fonctions et les appointements, que le futur secrétaire
d'État lui payerait les huit cent quatre-vingt mille livres de son
brevet de retenue, y compris la charge de secrétaire du roi\,; qu'il
aurait soin de son fils\,; que, pour lui, il serait bien aise de le
voir, mais que, dans ces premiers temps, cela lui ferait peine\,; qu'il
attendit qu'il le fît avertir\,; qu'il ferait bien de se retirer ce
jour-là même\,; qu'il pouvait demeurer à Paris, aller et venir partout
où il voudrait\,; et réitéra tant et plus les assurances de son amitié.
M. de Beauvilliers, touché au dernier point de la chose et d'une
commission si dure, voulut vainement s'en décharger. Le roi lui dit
qu'étant ami de Chamillart, il l'avait choisi exprès pour le ménager en
toutes choses.

Un moment après, il rentra dans le cabinet du conseil, suivi du duc, où
le chancelier, Torcy, Chamillart et Desmarets se trouvèrent. C'était
conseil d'État, dans lequel il ne se passa rien, même dans l'air et dans
le visage du roi, qui pût faire soupçonner quoi que ce fût. Il s'y parla
même d'une affaire sur laquelle le roi avait demandé un mémoire à
Chamillart, qui, à la fin du conseil, en prit encore son ordre. Le roi
lui dit de le lui apporter le soir en venant travailler avec lui chez
M\textsuperscript{me} de Maintenon.

Beauvilliers, dans une grande angoisse, demeura le dernier des ministres
dans le cabinet, où, seul avec le roi, il lui exposa franchement sa
peine, et finit par le prier de trouver bon, au moins, qu'il s'associât
dans sa triste commission le duc de Chevreuse, ami comme lui de
Chamillart, pour en partager le poids, à quoi le roi consentit, et dont
M. de Chevreuse fut fort affligé.

Sur les quatre heures après midi, les deux beaux-frères s'acheminèrent
et furent annoncés à Chamillart, qui travaillait seul dans son cabinet.
Ils entrèrent avec un air de consternation qu'il est aisé d'imaginer. À
cet abord, le malheureux ministre sentit incontinent qu'il y avait
quelque chose d'extraordinaire, et sans leur donner le temps de
parler\,: «\,Qu'y a-t-il donc, messieurs\,? leur dit-il d'un visage
tranquille et serein. Si ce que vous avez à me dire ne regarde que moi,
vous pouvez parler, il y a longtemps que je suis préparé à tout.\,»
Cette fermeté si douce les toucha encore davantage. À peine purent-ils
lui dire ce qui les amenait. Chamillart l'entendit sans changer de
visage, et du même air et du même ton dont il les avait interrogés
d'abord\,: «\, Le roi est le maître, répondit-il. J'ai tâché de le
servir de mon mieux, je souhaite qu'un autre le fasse plus à son gré et
plus heureusement. C'est beaucoup de pouvoir compter sur ses bontés, et
d'en recevoir en ce moment tant de marques.\,» Puis leur demanda s'il ne
lui était pas permis de lui écrire, et s'ils ne voulaient pas bien lui
faire l'amitié de se charger de sa lettre, et sur ce qu'ils l'assurèrent
qu'oui, et que cela ne leur était pas défendu, du même sang-froid il se
mit incontinent à écrire une page et demie de respects et de
remercîments qu'il leur lut tout de suite, comme tout de suite il
l'avait écrite en leur présence. Il venait d'achever le mémoire que le
roi lui avait demandé le matin\,; il le dit aux deux ducs, comme en s'en
réjouissant, le leur donna pour le remettre au roi, puis cacheta sa
lettre, y mit le dessus et la leur donna. Après quelques propos
d'amitié, il leur parla admirablement sur son fils, et sur l'honneur
qu'il avait d'être leur neveu par sa femme. Après quoi les deux ducs se
retirèrent, et il se prépara à partir.

Il écrivit à M\textsuperscript{me} de Maintenon, la fit souvenir de ses
anciennes bontés, sans y rien mêler d'autre chose, et prit congé d'elle.
Il écrivit un mot à La Feuillade, à Meudon où il était, pour lui
apprendre sa disgrâce, manda verbalement à sa femme, qu'il attendait de
Paris ce jour-là, de le venir trouver à l'Étang où il allait, sans lui
dire pourquoi tria ses papiers, puis fit venir l'abbé de La Proustière,
les lui indiqua, et lui donna ses clefs pour les remettre à son
successeur. Tout cela fait sans la moindre émotion, sans qu'il lui fût
échappé ni soupirs, ni regret, ni reproches, pas une plainte, il
descendit son degré, monta en carrosse et s'en alla à l'Étang tête à
tête avec son fils, comme s'il ne lui fût rien arrivé, sans que
longtemps après on en sût rien à Versailles.

Son fils aussi porta ce malheur fort constamment. En arrivant à l'Étang,
où sa femme l'avait devancé de quelques moments, il entra dans sa
chambre, où il la manda avec sa belle-fille, où étant tous quatre seuls,
il leur confirma ce qu'elles commençaient déjà fort à soupçonner. Il
parla principalement à sa belle-fille sur l'honneur de son alliance, la
combla de respects et d'amitiés qu'elle méritait par sa conduite, et par
la manière dont elle vivait avec eux. Après avoir été quelque temps
témoin de leurs larmes, il vit son frère l'évêque de Senlis, et passa
chez la duchesse de Lorges, au lit, incommodée, qui avait sa soeur de La
Feuillade auprès d'elle, et M\textsuperscript{me} de Souvré, qui de
hasard s'y rencontra. On peut juger de l'amertume de cette première
entrevue. M\textsuperscript{me} Dreux, qui était à Versailles, et qui
avait appris la disgrâce par l'abbé de La Proustière que son père en
avait chargé en partant, eut une force qui mérite de n'être pas oubliée.
Elle sentit le néant où elle retombait, mariée si différemment de ses
soeurs, et le besoin qu'elle avait de tout. Elle s'en alla chez
M\textsuperscript{me} la Duchesse qu'elle trouva jouant au papillon, qui
commençait, et la pria qu'elle lui pût parler en particulier après sa
reprise. M\textsuperscript{me} la Duchesse lui offrit plusieurs fois de
l'interrompre, M\textsuperscript{me} Dreux ne voulut pas\,; et ce qui
est d'étonnant, c'est qu'on ne s'aperçut d'abord de rien à son air\,;
dans la suite on remarqua que les larmes lui roulaient dans les yeux. Ce
jeu dura une heure entière, après lequel elle suivit
M\textsuperscript{me} la Duchesse dans son cabinet. Elle lui apprit son
infortune, et lui parla comme une personne qui avait passé avec elle la
plupart du temps que son père avait été en place, et qui s'en voulait
faire une protection. La réponse fut pleine d'amitié, après quoi
M\textsuperscript{me} Dreux se sauva chez elle, qui était tout proche,
et de là à l'Étang.

M\textsuperscript{me} de Maintenon, en rentrant de Saint-Cyr chez elle,
avait reçu la lettre de Chamillart. En même temps, M\textsuperscript{me}
la duchesse de Bourgogne y entra. M\textsuperscript{me} de Maintenon lui
demanda si elle ne savait rien, et lui montra la lettre de Chamillart.
Quoique, après tout ce qui avait précédé, l'adieu qu'il lui disait fût
assez clair, toutes deux n'y comprirent rien, ce qui toutefois est
inconcevable jusque-là que M\textsuperscript{me} de Maintenon pria
M\textsuperscript{me} la duchesse de Bourgogne de passer dans le cabinet
de Mgr le duc de Bourgogne, qui par les derrières était tout contre,
savoir s'il n'était pas plus instruit.

Dans ce moment-là même, le roi entra, et ce qui n'arrivait jamais, le
duc de Beauvilliers à sa suite. Le roi fit à l'ordinaire sa révérence à
M\textsuperscript{me} de Maintenon, congédia le capitaine des gardes, et
prit Beauvilliers dans une fenêtre, qui tira des papiers de sa poche,
c'était la lettre et le mémoire de Chamillart, et tous deux se mirent à
parler bas. M\textsuperscript{me} la duchesse de Bourgogne, voyant cela,
dit à M\textsuperscript{me} de Maintenon qu'apparemment c'était pour
elle, et qu'elle s'allait retirer pour les laisser en liberté. En effet,
comme elle allait sortir par le grand cabinet, elle vit le roi s'avancer
vers M\textsuperscript{me} de Maintenon, et le duc de Beauvilliers s'en
aller. Ce mouvement ne mit encore rien au jour\,; et
M\textsuperscript{me} la Duchesse n'avait rien voulu dire chez elle
depuis que M\textsuperscript{me} Dreux en fut sortie.

J'allai chez le chancelier, comme je faisais fort souvent les soirs, que
je trouvai avec La Vrillière. Un peu après, son fils y entra, qui lui
parla bas, et s'en alla aussitôt. C'était la nouvelle qu'il venait lui
apprendre, et que par considération pour moi ils ne me voulurent pas
dire. Revenu chez moi, je me mis à écrire en haut quelque chose sur les
milices de Blaye, ce que je cite parce qu'on en verra de grandes suites.
Comme j'y travaillais, la maréchale de Villars entra en bas qui me
demanda. J'envoyai mon mémoire à Pontchartrain, et je descendis. Je
trouvai la maréchale debout et seule, parce que M\textsuperscript{me} de
Saint-Simon était sortie, qui me demanda si je ne savais rien, et qui me
dit\,: «\,Le Chamillart n'est plus.\,» À ce mot, il m'échappa un cri
comme à la mort d'un malade quoique dès longtemps condamné et dont
pourtant on attend la fin à tous moments. Après quelques lamentations,
elle s'en alla au souper du roi, et moi par les cours, pour n'être point
vu, et sans flambeaux, chez M. de Beauvilliers, que je venais
d'apprendre par la maréchale de Villars avoir été chez lui le congédier.
M. de Beauvilliers, qui était d'année, était allé chez le roi, quoique
le duc de Tresmes servît toujours pour lui les soirs. Je trouvai
M\textsuperscript{me} de Beauvilliers avec M\textsuperscript{me} de
Chevreuse, Desmarets et Louville. Je jetai d'abord un regard sur le
contrôleur général dans la curiosité de le pénétrer, et je n'eus pas de
peine à sentir un homme au large et qui cachait sa joie avec effort.
J'abordai M\textsuperscript{me} de Beauvilliers, qui avait les larmes
aux yeux, et de qui je ne sus pas grand'chose dans cette émotion. J'y
fus peu et me retirai chez moi, où la maréchale de Villars vint souper.

M\textsuperscript{me} de Saint-Simon était allée faire sa cour à
M\textsuperscript{me} la duchesse de Bourgogne dans ce grand cabinet de
M\textsuperscript{me} de Maintenon, où elle entendit quelque bruit
confus et tout bas de la nouvelle. Elle demanda à M\textsuperscript{me}
la duchesse de Bourgogne si cela avait quelque fondement. Elle ne savait
rien, parce qu'elle n'avait pas été rappelée dans la chambre depuis
qu'elle en toit sortie, et n'avait osé y rentrer ce soir-là d'elle-même.
Apparemment que les grands coups s'y ruaient pour le successeur, dont
personne ne parlait encore, et que c'était pour cela qu'on la laissait
dehors. Elle dit à M\textsuperscript{me} de Saint-Simon d'aller au
souper du roi, où elle lui apprendrait ce qu'elle aurait découvert en
passant dans la chambre. M\textsuperscript{me} de Saint-Simon y fut et
s'y trouva, assise derrière M\textsuperscript{me} la duchesse de
Bourgogne, qui lui dit la disgrâce, les pensions, et la charge de
Cavoye. Au sortir du souper, que M\textsuperscript{me} de Saint-Simon
trouva bien long, M\textsuperscript{me} la duchesse de Bourgogne, prête
à entrer dans le cabinet du roi, vint à elle, et la chargea de faire
mille amitiés pour elle aux filles de Chamillart, mais plus
particulièrement à l'aînée, et à la duchesse de Lorges qu'elle aimait,
de leur dire combien elle les plaignait, et de les assurer de sa
protection et de tous les adoucissements à leur malheur qui pourraient
dépendre d'elle.

Le duc de Lorges n'était content d'aucun de la famille. Il passa jusque
fort tard avec nous et s'en alla à l'Étang, en résolution de faire
merveilles pour eux, et les fit en effet constamment. Je le chargeai
d'un mot de tendre amitié pour Chamillart\,; et par mon billet je le
priai de me mander verbalement s'il voulait absolument être seul ce
premier jour, ou s'il voulait bien nous voir.

Par tout ce qui a été dit de lui en différentes occasions, on a vu quel
était son caractère, doux, simple, obligeant, vrai, droit, grand
travailleur, aimant l'État et le roi comme sa maîtresse, attaché à ses
amis, mais s'y méprenant beaucoup, nullement soupçonneux ni haineux,
allant son grand chemin à ce qu'il croyait meilleur, avec peu de
lumière\,; opiniâtre à l'excès, et ne croyant jamais se tromper,
confiant sur tous chapitres, et surtout infatué que, marchant droit et
ayant le roi pour lui, comme il n'en douta jamais, tout autre
ménagement, excepté M\textsuperscript{me} de Maintenon, était inutile\,;
et avec cette opinion, trop ignorant de la cour au milieu de la cour, il
se l'aliéna par le mariage de son fils, il augmenta son aversion par son
entraînement en faveur de M. de Vendôme contre Mgr le duc de Bourgogne,
comme un aveugle qui ne voit que par autrui, enfin il se la déchaîna
sciemment par amour de l'État, et par sa passion pour la personne du
roi, et pour sa gloire, et par le projet de le mener reprendre Lille
sans elle.

Cette cabale si puissante, qui lui fit voir, croire et faire tout ce
qu'elle voulut, sans aucun ménagement, sur les choses d'Italie, mais
surtout sur celles de Flandre, ne lui fut après d'aucun usage. M. de
Vendôme était perdu\,; M. de Vaudemont sur le côté pour avoir trop
prétendu\,; M\textsuperscript{lle} de Lislebonne, on a vu comme elle en
usa entre M\textsuperscript{lle} Choin et lui, conséquemment sa soeur
qui n'était qu'un avec elle\,; et M. du Maine avait trop besoin de
M\textsuperscript{me} de Maintenon pour ne lui pas sacrifier Chamillart,
après lui avoir sacrifié sa propre mère.

Chamillart eut un autre malheur, qui est extrême pour un ministre. Il
n'était environné que de gens qui n'avaient pas le sens commun, et qui
n'avaient pu acquérir à la cour et dans le monde les connaissances les
plus communes\,; et, ce qui n'est pas moins fâcheux que le défaut du
solide, qui tous avaient un maintien, des façons et des propos
ridicules.

Tels étaient ses deux frères\,; tels, et très impertinents de plus,
étaient Le Rebours, son cousin germain, et Guyet, beau-père de son
frère, qu'il avait faits intendants des finances. Ses deux cadettes
étaient les meilleures créatures du monde, et la duchesse de Lorges,
avec de l'esprit, mais des folles dont l'ivresse de la fortune et des
plaisirs a même cessé à peine à sa disgrâce. L'aînée était la seule qui,
avec de l'esprit, eût du sens et de la conduite, et qui se fit aimer,
estimer, plaindre et recueillir de tout le monde. Mais outre qu'elle ne
voyait et ne savait pas tout, elle n'était pas bastante pour arrêter et
gouverner les autres, ni être le conseil de son père, qui n'aimait ni ne
croyait aucun avis. M\textsuperscript{me} Chamillart passait ses
matinées entre son tapissier et sa couturière, son après-dînée au jeu,
ne savait pas dire deux mots, ignorait tout, et comme son mari ne
doutait de rien, et voulant être polie se faisait moquer d'elle, quoique
la meilleure femme du monde\,; sans avoir en elle de quoi ni tenir ses
filles ni leur donner la moindre éducation, incapable de tous soins de
ménage, de dépense, de bien et d'économie, qui fut abandonné en total à
l'abbé de La Proustière, leur parent, qui y entendait aussi peu qu'elle,
et qui mit leurs affaires en désarroi.

Le lundi matin, on sut que le triomphe de M\textsuperscript{me} de
Maintenon était entier\,; et qu'à la place de Chamillart, chassé la
veille, Voysin, sa créature, tenait cette fortune de sa main. Il
figurera maintenant jusqu'à la mort du roi si grandement et si
principalement qu'il faut faire connaître ce personnage et sa femme, qui
lui fit sa fortune.

Voysin avait parfaitement la plus essentielle qualité, sans laquelle nul
ne pouvait entrer et n'est jamais entré dans le conseil de Louis XIV en
tout son règne, qui est la pleine et parfaite roture, si on en excepte
le seul duc de Beauvilliers\,; car M. de Chevreuse, quoiqu'il en fût,
n'y entra et n'y parut jamais, le premier maréchal de Villeroy ne fut
point ministre, et l'autre ne l'a pas été un an.

Voysin était petit-fils du premier commis au greffe criminel du
parlement, qui le devint après en chef, et qui mourut dans cette charge.
On juge bien qu'il ne faut pas monter plus haut. Le frère aîné du père
de Voysin, dont je parle, passa avec grande réputation d'intégrité et de
capacité par les intendances, fut prévôt des marchands, et devint
conseiller d'État très distingué. C'était de ces modestes et sages
magistrats de l'ancienne roche, qui était fort des amis de mon père, et
que j'ai vu souvent chez lui. Il maria sa fille unique, très riche
héritière, à Lamoignon, mort président à mortier, fils du premier
président, et frère aîné du trop célèbre Bâville\,; et le père de notre
Voysin fut maître des requêtes et eut diverses intendances, dans
lesquelles il mourut. Son heureux fils fut le seul des trois frères qui
parût dans le monde, et une seule fille, mariée à Vaubourg mort
conseiller d'État après beaucoup d'intendances, frère aîné de Desmarets,
contrôleur général.

Voysin épousa, en 1683, la fille de Trudaine, maître des comptes, et
cinq ans après, étant maître des requêtes, fut, je ne sais par quel
crédit, envoyé intendant en Hainaut, d'où il ne sortit que conseiller
d'État en 1694. Sa femme avait un visage fort agréable, sans rien
d'emprunté ni de paré. L'air en était doux, simple, modeste, retenu et
mesuré, et d'être tout occupée de son domestique et de bonnes oeuvres\,;
au fond, de l'esprit, du sens, du manège, de l'adresse, de la conduite,
surtout une insinuation naturelle, et l'art d'amener les choses sans,
qu'il y parût. Personne ne s'entendait mieux qu'elle à tenir une maison,
et à la magnificence quand cela convenait sans offenser par la
profusion\,; à être libérale avec choix et avec grâce, et à porter
l'attention à tout ce qui lui pouvait concilier le monde.

L'opulence de sa maison, et plus encore ses manières polies et
attrayantes, mais avec justesse à l'égard des différences des personnes,
l'avaient extrêmement fait aimer, surtout des officiers, pour le
soulagement desquels elle fit merveille, pendant les sièges et après les
actions qui se passèrent en Flandre, et de soins et d'argent et de
toutes façons. Elle avait fait beaucoup de liaison avec M. de Luxembourg
qui y commandait tous les ans les armées, et avec la fleur la plus
distinguée des généraux qui y servirent, surtout avec M. d'Harcourt qui
y eut toujours des corps séparés.

M. de Luxembourg l'avertit de bonne Heure de ce qu'il fallait faire pour
plaire à M\textsuperscript{me} de Maintenon venant sur la frontière, et
elle en sut profiter parfaitement. Elle la reçut chez elle à Dinan, où
elle fut pendant que le roi assiégeait Namur, la salua à son arrivée,
pourvut avec le dernier soin à la commodité et à l'arrangement de son
logement, courtisa jusqu'à ses moindres domestiques, se renferma après
dans sa chambre sans se montrer à elle, ni aux autres dames de la cour,
que précisément pour le devoir, donnant ordre à tout de cette retraite,
de manière à contenter tout le monde, mais comme si elle n'eût pas
habité sa maison. Une réception si fort dans le goût de
M\textsuperscript{me} de Maintenon la prévint favorablement pour son
hôtesse. Ses gens, charmés d'elle, s'empressèrent à lui raconter tout ce
qu'elle avait fait après Neerwinden pour les officiers et les soldats
blessés, la libéralité, le bon ordre de sa maison, et à lui vanter sa
piété et ses bonnes oeuvres.

Une bagatelle heureuse, et heureusement prévue, toucha tout à fait
M\textsuperscript{me} de Maintenon. En un instant le temps passa d'une
chaleur excessive à un froid humide et qui dura longtemps\,; aussitôt
une belle robe de chambre, mais modeste et bien ouatée, parut dans un
coin de sa chambre. Ce présent, d'autant plus agréable que
M\textsuperscript{me} de Maintenon n'en avait point apporté de chaude,
ne lui en parut que plus galant par la surprise, et par la simplicité de
s'offrir tout seul.

La retenue de M\textsuperscript{me} Voysin acheva de la charmer. Souvent
deux jours de suite sans la voir, non pas même à son passage. Elle
n'allait chez elle que lorsqu'elle l'envoyait chercher, à peine s'y
voulait-elle asseoir\,; toujours occupée de la crainte d'importuner, et
de l'attention à saisir le moment de s'en aller. Une telle
circonspection, à quoi M\textsuperscript{me} de Maintenon n'était pas
accoutumée, tint lieu du plus grand mérite. La rareté devint la source
du désir, qui attira à l'habile hôtesse les agréables reproches qu'elle
était la seule personne qu'elle n'eût pu apprivoiser. Elle prit un
véritable goût à sa conversation et à ses manières.
M\textsuperscript{me} Voysin ne s'ingéra jamais de rien, même après
qu'elle fut initiée, et finalement plut si fort à M\textsuperscript{me}
de Maintenon, dans ce long séjour qu'elle lit chez elle, qu'elle
s'offrit véritablement à elle, et lui ordonna de la voir toutes les fois
qu'elle irait à Paris. Il parut toujours plus d'obéissance dans
l'exécution que d'empressement, et {[}elle{]} réussit de plus en plus
par ses manières si respectueuses et si réservées. Le voyage de Flandre
de 1693 donna un nouveau degré à cette amitié, qui valut, l'année
suivante, une place de conseiller d'État à Voysin. Fixés de la sorte à
Paris, sa femme se tint dans sa même réserve, ne voyait
M\textsuperscript{me} de Maintenon que rarement, presque toujours
mandée\,; et, devenue plus familière venait quelquefois d'elle-même par
reconnaissance, par attachement, toujours de loin en loin, toujours
obscurément, en sorte que ce commerce demeura fort longtemps inconnu, à
l'abri de l'envie, des réflexions et des mauvais offices.

Avec le même art, mais diversifié suivant les convenances, elle sut
cultiver tous les gens principaux qu'elle avait le plus vus en Flandre,
et jusqu'à Monseigneur qui y avait commandé en 1694, et à qui M. de
Luxembourg, général de l'armée sous lui, en avait dit mille biens, et
d'autres gens encore depuis.

Le mari, de son côté, assidu à ses fonctions, ne parut songer à rien,
jusqu'à ce que Chamillart, trop chargé d'affaires, remit celles de
Saint-Cyr, que M\textsuperscript{me} de Maintenon donna à Voysin. La
relation par ce moyen devint entre eux continuelle, et la femme de plus
en plus rapprochée, et tous deux d'autant plus goûtés qu'ils se tinrent
toujours sagement dans leurs mêmes bornes de retenue qui les avait si
bien servis. Alors néanmoins les yeux s'ouvrirent sur eux, et Voysin
devint comme le candidat banal de toutes les grandes places. Lassé de
n'en espérer aucune par la stabilité où il voyait toutes celles du
ministère, il désira ardemment, et M\textsuperscript{me} de Maintenon
pour lui, celle de premier président. Il fut heureux que Chamillart tint
ferme pour Pelletier, pour plaire au duc de Beauvilliers, et pour
soi-même, {[}ce{]} qui par la cascade fit avocat général un fils de son
ancien ami Lamoignon, qui tôt après le paya d'une étrange ingratitude.
Comme on juge par les événements, on regarda comme une faute grossière
en Chamillart de ne s'être pas défait de ce rival à toutes places, en
lui faisant tomber celle de premier président. Mais, comme je l'ai
remarqué en son temps, rien n'eut tant de part à la promotion de
Pelletier que le crédit que son père, qui ne mourut de plus de quatre
ans après, conserva toute sa vie auprès du roi, qui se piqua toujours de
l'aimer, et qui lui fit plus de grâces pour sa famille, depuis sa
retraite, qu'il n'en avait obtenu pendant son ministère.

Voysin eut grand besoin de la femme dont la Providence le pourvut.
Devenu maître des requêtes sans avoir eu le temps d'apprendre dans les
tribunaux, et de là passé promptement à l'intendance, il demeura
parfaitement ignorant. D'ailleurs sec, dur, sans politesse ni
savoir-vivre, et pleinement gâté comme le sont presque tous les
intendants, surtout de ces grandes intendances, il n'en eut pas même le
savoir-vivre, mais tout l'orgueil, la hauteur et l'insolence. Jamais
homme ne fut si intendant que celui-là, et ne le demeura si parfaitement
toute sa vie, depuis les pieds jusqu'à la tête, avec l'autorité toute
crue pour tout faire et pour répondre à tout. C'était sa loi et ses
prophètes\,; c'était son code, sa coutume, son droit\,; en un mot,
c'était son principe et tout pour lui. Aussi excella-t-il dans toutes
les parties d'un intendant, et grand, facile et appliqué travailleur,
d'un grand détail et voyant et faisant tout par lui-même\,; d'ailleurs
farouche et sans aucune société, non pas même devenu conseiller d'État
et après ministre\,; incapable jusque de faire les honneurs de chez lui.
Le courtisan, le seigneur, l'officier général et particulier, accoutumés
à l'accès facile et à l'affabilité de Chamillart, à sa patience à
écouter, à ses manières douces, mesurées, honnêtes, proportionnées de
répondre, même à des importuns et à des demandes et à des plaintes sans
fondement, et au style semblable de ses lettres, se trouvèrent bien
étonnés de trouver en Voysin tout le contre-pied\,: un homme à peine
visible et fâché d'être vu, refrogné, éconduiseur, qui coupait la
parole, qui répondait sec et ferme en deux mots, qui tournait le dos à
la réplique, ou fermait la bouche aux gens par quelque chose de sec, de
décisif et d'impérieux, et dont les lettres dépourvues de toute
politesse n'étaient que la réponse laconique, pleine d'autorité, ou
l'énoncé court de ce qu'il ordonnait en maître\,; et toujours à tout\,:
«\,le roi le veut ainsi.\,» Malheur à qui eut avec lui des affaires de
discussion dépendantes d'autres règles que de celles des intendants\,!
elles le sortaient de sa sphère, il sentait son faible, il coupait court
et brusquait pour finir. D'ailleurs il n'était ni injuste pour l'être,
ni mauvais par nature, mais il ne connut jamais que l'autorité, le roi
et M\textsuperscript{me} de Maintenon, dont la volonté fut sans réplique
sa souveraine loi et raison.

Quelque apparent qu'il fût, vers les derniers temps de Chamillart, que
Voysin lui succéderait, l'incertitude en dura jusqu'à sa déclaration. Le
choix ne fut déterminé que le soir même de la retraite de Chamillart
entre le roi et M\textsuperscript{me} de Maintenon. Au sortir du souper,
Bloin eut ordre de mander à Voysin, à Paris, de se trouver le lendemain
de bon matin chez ce premier valet de chambre, et sans paraître, qui le
mena par les derrières dans les cabinets du roi, qui là lui parla seul
un moment après son lever, et qui lui fit un accueil médiocre\,; il le
déclara ensuite. Voysin avait auparavant été remercier et recevoir les,
ordres et les instructions de sa bienfaitrice.

De chez le roi, il alla dans le cabinet de son prédécesseur, prit
possession des papiers et des clefs que lui donna et montra l'abbé de La
Proustière, manda les commis, et de ce jour habita l'appartement avec,
les meubles de Chamillart, en sorte qu'il n'y parut de changement qu'un
autre visage jusqu'au mercredi suivant qu'on alla à Marly, pendant
lequel les meubles se changèrent.

Le soir, M\textsuperscript{me} Voysin arriva à petit bruit droit chez
M\textsuperscript{me} de Caylus, son amie d'ancien temps, et avant
qu'elle fût rappelée à la cour. Celle-ci aussitôt la conduisit chez sa
tante, où les transports de la protectrice et le néant où se jeta la
protégée furent égaux. Peu après, le roi entra, qui l'embrassa jusqu'à
deux fois différentes pour plaire à sa dame, l'entretint de l'ancienne
connaissance de Flandre, et la pensa faire rentrer sous terre. De là, se
dérobant à toute la cour, elle regagna son carrosse et Paris pour y
donner ordre à tout, et se mettre en état de ne plus quitter son mari à
qui plus que jamais elle était nécessaire auprès de
M\textsuperscript{me} de Maintenon, et à porter l'abord du monde et le
poids délicat de la cour qui s'empressa autour d'eux avec sa bassesse
ordinaire, et jusqu'à Monseigneur se piqua de dire qu'il était des amis
de M\textsuperscript{me} Voysin depuis leur connaissance de Flandre. Il
oublia ainsi de s'être mépris pour d'Antin, et d'Antin lui-même se fit
un de leurs plus grands courtisans. Vaudemont et ses nièces, si intimes
de Chamillart, s'oublièrent auprès d'eux moins que personne, et avec les
plus grands empressements.

La Feuillade, ce gendre si chéri, avait gardé le secret, à Meudon, de
l'avis qu'il avait reçu par le billet de son beau-père. Dès le lundi
matin, l'air libre et dégagé, il vint prier le roi, qui allait à la
messe, de se souvenir qu'il avait donné sa vaisselle, et de lui
conserver le logement que Chamillart lui avait donné. Le roi ne répondit
que par un froid et méprisant signe de tête. Son maintien ne réussit pas
mieux dans le public, et tout à la fin de la matinée, il se résolut
enfin d'aller à l'Étang.

J'y allai au sortir de table avec M\textsuperscript{me} de Saint-Simon
et la duchesse de Lauzun. Quel spectacle\,! une foule de gens oisifs et
curieux, et prompts aux compliments, un domestique éperdu, une famille
désolée, des femmes en pleurs dont les sanglots étaient les paroles,
nulle contrainte en une si amère douleur. À cet aspect, qui n'eût
cherché la chambre de parade et le goupillon pour rendre ce devoir au
mort\,? On avait besoin d'effort pour se souvenir qu'il n'y en avait
point, et pour ne trouver pas à redire qu'il n'y eût point de tenture et
d'appareil funèbre\,; et on était effrayé de voir ce mort, sur qui on
venait pleurer, marcher et parler d'un air doux, tranquille, le front
serein, sans rien de contraint ni d'affecté, attentif à chacun, point ou
très peu différent de ce qu'il avait coutume d'être.

Nous nous embrassâmes tendrement. Il me remercia, pénétré des termes de
mon billet de la veille. Je l'assurai que je n'oublierais point les
services et les plaisirs que j'en avais reçus, et je puis dire que je
lui ai tenu plus que parole, et à sa famille après lui.

Son fils parut tout consolé, moins sensible à une chute qui le mettait
en poudre qu'à l'a délivrance d'un travail dont il n'avait ni le goût ni
l'aptitude\,; des frères stupides qui parfois s'émerveillaient comment
le roi s'était pu séparer de leur frère. La Feuillade voltigeait et
philosophait sur l'instabilité des fortunes, avec une liberté d'esprit
qui ne scandalisa pas moins qu'il avait indigné le matin à Versailles.

Tout est mode et curiosité à la cour. Des uns aux autres il n'y eut
personne qui n'allât à l'Étang\,; et à y voir Chamillart y répondre à
tout le monde, on eût dit qu'encore en place, il y donnait audience à
toute la cour, tant il y paraissait tranquille et naturel. Une ignorance
de magistrat de beaucoup de choses de la cour et du monde qu'aucun des
siens ne suppléait, et un air excessif de naïveté, avec une démarche
dandinante, lui avait fait grand tort et nier trop entièrement l'esprit.
Le mardi se passa dans le même abord, ou plutôt dans la même foule. Nous
y passâmes encore ce jour-là et le lendemain\,; mais il leur vint le
mardi tant d'avis de l'aigreur avec laquelle M\textsuperscript{me} de
Maintenon s'en expliquait, de son dépit de ce qu'elle prit pour une
marque de considération, du blâme amer de ce que Chamillart avait laissé
forcer, puis ouvert sa porte, que de peur de pis, quoique le roi ne
l'eût pas trouvé mauvais, Chamillart accepta l'offre de sa maison des
Bruyères près de Ménilmontant, où il s'en alla le mercredi, où nous
fûmes toujours avec lui, et où M. de Lorges n'épargna rien pour qu'il
s'y trouvât au mieux qu'il fût possible.

Le mercredi matin que le roi devait aller coucher à Marly, Cani alla
pour lui faire la révérence\,; il attendit à la porte du cabinet, avec
tout le monde, qu'il rentrât de la messe. Le roi s'arrêta à lui, le
regarda d'un air d'affection et de complaisance, l'assura qu'il aurait
soin de lui, et qu'il lui voulait faire du bien\,; et, se sentant
attendrir, il se hâta d'entrer. On fut bien surpris que quelques moments
après le roi rouvrit la porte du cabinet, les yeux rouges qu'il venait
d'essuyer, rappela Cani, lui répéta encore les mêmes choses, et plus
fortement.

On vit par là quel fut l'effort que le roi se fit pour se laisser
arracher son ministre, combien il fallut de puissants et d'habiles
ressorts, et qu'il ne put encore leur céder que lorsque, par le retour
de Torcy, il vit la paix tout à fait, désespérée. Le froid accueil fait
contre sa coutume à un ministre au moment de son choix, qu'on a vu que
Voysin avait essuyé, ce que nous verrons bientôt qui lui arriva encore
dans une nouveauté toujours si brillante, et cette réception faite à
Cani, montra bien que, si son père m'eût voulu croire une seconde fois
et parler au roi, ce monarque ne se serait jamais pu défendre de lui, et
qu'il serait demeuré en place.

La famille de la femme de son fils, bien empêchée de lui à son âge, le
détermina, et la sienne, à entrer dans le service, quelque dégoût qu'il
y eût pour lui, qui en avait été comme le petit roi, de dépendre du
successeur de son père et de lui-même, d'avoir affaire à ses propres
commis, et de devenir camarade, et beaucoup moins, de cette foule de
jeunes gens qui lui faisaient leur cour.

Le Guerchois, qui avait la Vieille-marine\footnote{Le régiment de la
  Vieille-marine.} et qui venait d'être fait maréchal de camp, et que
Chamillart, à ma prière, avait fort servi, n'eut pas plutôt appris ce
dessein par le public, qu'il lui envoya d'où il était sa démission sans
stipulation quelconque, et tous les autres régiments vendus. Chamillart
en fut fort touché et lui en donna le prix, sans que Le Guerchois s'en
voulût mêler en façon quelconque. Le jeune homme, qui, par un prodige
unique, ne s'était point gâté dans la place qu'il avait occupée, s'y fit
aimer, et de tous les militaires, s'y fit estimer, et y servit le peu
qu'il vécut avec une valeur, une distinction et une application qui dans
un autre genre lui aurait réconcilié la fortune\,; et le roi, qui prit
toujours plaisir à en ouïr dire du bien, ne cessa point de le traiter
avec une amitié tout à fait marquée.

\hypertarget{chapitre-xvi.}{%
\chapter{CHAPITRE XVI.}\label{chapitre-xvi.}}

1709

~

{\textsc{Voysin ministre.}} {\textsc{- Voysin rudement réprimandé par le
roi.}} {\textsc{- Boufflers évangéliste de Voysin.}} {\textsc{-
Chamillart poursuivi par Boufflers.}} {\textsc{- Louable mais grande
faute de Chamillart.}} {\textsc{- Chamillart chassé de Paris par
M\textsuperscript{me} de Maintenon.}} {\textsc{- Raisons qui me
persuadent la retraite.}} {\textsc{- Trois espèces de cabales à la
cour\,: des seigneurs, des ministres, de Meudon.}} {\textsc{- Crayon de
la cour.}}

~

Voysin alla à Meudon le mardi matin, le lendemain de sa déclaration, et
y fût longtemps seul avec Monseigneur, qui n'avait pas dédaigné de
recevoir les compliments qu'on osa lui faire de la part qu'il avait eue
à la disgrâce de Chamillart. Le lendemain mercredi, le roi le manda au
conseil d'État et le fit ainsi ministre. Cette promptitude n'avait point
eu d'exemple, et son prédécesseur eut plus d'un an les finances avant de
l'être, et le fut beaucoup plus tôt qu'aucun. Le roi lui dit que ce
n'était pas la peine de lui faire attendre cette grâce, que
M\textsuperscript{me} de Maintenon lui valut encore, à quoi personne ne
se méprit, et à laquelle elle ne fut pas insensible, quelque accoutumée
qu'elle fût à régner.

Un si rapide éclat ne laissa pas incontinent après d'être mêlé
d'amertume. Le maréchal de Villars envoya cinq différents projets pour
recevoir les ordres du roi. La face des affaires, sur laquelle on
s'était réglé, avait un peu changé en Flandre, et c'était sur quoi il
s'agissait de prendre un nouveau plan. Voysin reçut ces projets à Marly.
Il avait toujours ouï dire et su depuis, par les officiers principaux
depuis qu'il fut en Flandre, peut-être même par M. de Luxembourg, qui
avec grande raison s'en plaignait souvent, que Louvois, Barbezieux, et
depuis Chamillart les décidaient et faisaient les réponses toutes prêtes
qu'ils montraient seulement au roi. Sur ces exemples il en voulut user
de même, mais le coup d'essai se trouva trop fort pour lui, et il ne
put. Il sentit que déterminer un plan de campagne et les partis à
prendre sur ses diverses opérations était besogne qui passait un
intendant de frontière et un conseiller d'État, qu'il n'y connaissait
rien, et que la chose dépassait tout à fait ses lumières. Il porta donc
au roi tous les projets, et lui dit qu'il était si nouveau dans sa place
qu'il croyait pouvoir lui avouer sans honte que le choix de ces projets
le passait, et qu'en attendant qu'il en sait davantage il le suppliait
de vouloir bien le décider lui-même.

Ce n'était pas là le langage du pauvre Chamillart, ni celui de Louvois
même. C'était lui qui avait réduit les généraux à ce point, après qu'il
fut délivré de M. le Prince et de M. de Turenne. Mais il savait combien
le roi était jaloux, et à quel point il se piquait d'entendre la guerre.
Il fit donc là-dessus, comme depuis Mansart sur les projets de son
métier, il fit tout, mais avec l'art de faire accroire au roi que
c'était lui-même qui faisait, dont il exécutait et expédiait seulement
les ordres. Son fils en usa de même\,; mais Chamillart, tout de bon,
laissait tout au roi.

Il fut donc également surpris et irrité d'un langage si nouveau. Il se
fâcha de voir un homme de robe vouloir à l'avenir décider sur la guerre,
et le prétendre comme un apanage de sa place, tandis qu'il la donnait
principalement à la robe pour en savoir plus qu'eux et pouvoir compter
tout faire. Il se redressa d'un pied, et prenant un ton de maître, lui
dit qu'il voyait bien qu'il était neuf, de prétendre décider de quelque
chose\,; qu'il voulait donc qu'il apprît, et de plus qu'il retint bien
pour ne l'oublier jamais, que sa fonction était de prendre ses ordres et
de les expédier, et la sienne à lui d'ordonner de toutes choses, et de
décider des plus grandes et des plus petites. Il prit ensuite les
projets, les examina, prescrivit la réponse que bon lui sembla, et
renvoya sèchement Voysin, qui ne savait plus où il en était, et qui eut
grand besoin de sa femme pour lui remettre la tête, et de
M\textsuperscript{me} de Maintenon pour le raccommoder, et pour
l'endoctriner mieux qu'elle n'avait encore eu loisir de faire.

Cette romancine fut suivie d'un autre chagrin, aussi nouveau dans cette
place que contraire au goût, à l'esprit, aux maximes et à l'usage du
roi. Il défendit à Voysin de rien expédier sans le maréchal de
Boufflers, et ordonna à celui-ci de tout examiner, tellement qu'on vit
aller continuellement le maréchal et le nouveau ministre l'un chez
l'autre, et plus souvent le dernier portant le portefeuille chez le
maréchal, et les deux commis des lettres les porter tous les jours, une
et souvent plusieurs fois chez lui, avec le projet des réponses
auxquelles le maréchal effaçait, ajoutait et corrigeait ce qu'il jugeait
à propos. L'humiliation était grande pour un ministre d'avoir sans cesse
à présenter son thème à la correction d'un seigneur qui n'entrait point
dans le conseil, et qui n'allait point commander d'armée. Une fonction
si haute et si singulière mit le maréchal dans une grande privance
d'affaires avec le roi, et dans une considération éclatante, ajoutée
encore à celle où Lille l'avait mis, et à la part publique qu'il avait
eue à la disgrâce de Chamillart. Voysin fut souple, et sûr de
M\textsuperscript{me} de Maintenon, et par elle du maréchal même,
attendit du bénéfice du temps le moment de sortir de tutelle, sans
témoigner de s'en lasser, et moins qu'à personne au tuteur qui lui avait
été donné.

Chamillart ayant passé quelque temps aux Bruyères, vint à Paris, dont il
avait toute liberté, et où un si grand changement de fortune demandait
sa présence pour le nouvel arrangement de ses affaires. Pendant qu'il y
était, Bergheyck vint faire un tour à la cour, et y travailla deux
heures avec le roi et Torcy. Il trouva le ministère changé et son ami
hors de place, qu'il voulut embrasser avant de s'en retourner. C'était
les premiers jours de juillet\,; j'étais aussi à Paris, où je fus
surpris de voir entrer chez moi le maréchal de Boufflers tout en colère,
et qui, à peine assis, me dit que tout à l'heure il avait pensé arriver
une belle affaire\,; qu'étant chez le duc d'Albe, Chamillart y était
venu avec Bergheyck\,; qu'heureusement Chamillart avait été sage,
qu'ayant vu son carrosse dans la cour, il n'avait pas voulu entrer et
avait descendu Bergheyck à la porte\,; qu'il avait bien fait, parce que,
s'il eût monté et se fût avisé de dire quelque chose, il lui aurait fait
la sortie qu'il méritait, et qu'il continuait de mériter, puisque, hors
du ministère et non content de demeurer à Paris, il conservait commerce
avec les ministres étrangers, visitait les ambassadeurs et se voulait
encore mêler d'affaires. Le maréchal s'échauffa de plus en plus, se
lâcha contre ce mort, comme il faisait de son vivant, et finit par me
dire que je ferais bien de l'avertir de prendre garde à sa conduite,
pour ne s'attirer pas pis, et de lui conseiller encore de sortir de
Paris, où il était hardi de demeurer. Je tâchai de l'adoucir, de peur de
pis en effet pour le malheureux ex-ministre, et j'y réussis assez bien
en ne le contredisant pas sur des choses inutiles.

Je fus ensuite chez Chamillart, que je voyais fort assidûment, qui me
conta que Bergheyck l'étant allé voir, et lui ayant affaire dans le
quartier du duc d'Albe, chez qui Bergheyck voulait aller au sortir de
chez lui, il l'y avait mené sans aucun dessein d'y descendre, et
seulement pour être plus longtemps avec Bergheyck. Ce qu'il y eut de
rare, c'est que le roi demanda à ce dernier s'il n'avait pas été surpris
de ne plus trouver son ami Chamillart en place\,; et comme Bergheyck
répondit mollement en tâtant le pavé, le roi le rassura en lui en disant
du bien, mais comme en passant et comme quelque chose qui lui échappait
avec plaisir. J'avais fait en sorte de faire parler Chamillart sur cette
prétendue visite au duc d'Albe sans lui dire pourquoi\,; mais le vacarme
qu'en fit Boufflers ailleurs encore que chez moi fit du bruit qui revint
à Chamillart, et qui fit qu'il me demanda si le maréchal ne m'en avait
point parlé. Je le lui avouai, mais sans entrer dans un fâcheux détail.

Là-dessus Chamillart, le coeur gros de l'aventure, m'apprit que, sans
lui, Boufflers n'eût pas eu la survivance de ses gouvernements de
Flandre et de Lille pour son fils\,; qu'il fut même obligé d'en presser
le roi à plus d'une reprise, et qu'il lui arracha cette grâce pour le
défenseur de Lille, plutôt qu'il ne l'obtint. C'est ainsi que les
bienfaits qui semblent le plus naturellement couler de source ne sont
souvent que le fruit d'offices redoublés\,; et une des choses en quoi
Chamillart se manqua le plus principalement à soi-même fut de ne se
faire valoir d'aucun, pour en laisser au roi tout le gré et l'honneur,
dont sa disgrâce fut le salaire.

J'ai touché déjà les raisons pour lesquelles le maréchal ne l'aimait
pas, entre lesquelles son revêtement de M\textsuperscript{me} de
Maintenon, pour ainsi parler de son dévouement pour elle, et la
partialité du ministre pour Vendôme, et son abandon à cette étrange
cabale l'avaient tellement aigri qu'il se déchaîna à découvert, et que
le braillant de son retour de Lille, joint à l'opinion de sa droiture,
de sa vérité, de sa probité, qui en effet étaient parfaites, firent
peut-être plus de mal à Chamillart que M\textsuperscript{me} de
Maintenon même, et que tout ce qu'elle avait su ameuter et organiser
contre lui. Mais si le maréchal eût su qu'il lui devait la survivance de
Flandre pour son fils, jamais il ne se fût porté à le perdre, et il
était homme si généreux et si reconnaissant que, tout politique qu'il
était, je l'ai connu assez intimement pour avoir lieu de douter que
M\textsuperscript{me} de Maintenon, toute telle qu'elle fût pour lui,
l'eût pu empêcher de le servir.

De tous ses ennemis, il n'y eut presque que le maréchal qui ne le visita
point et qui ne lui fit rien dire, et il eut raison après s'être si
ouvertement déclaré. Le chancelier même et Pontchartrain son fils, l'un
lui écrivit, l'autre le visita\,; et tous ceux qui lui avaient été le
plus opposés se piquèrent de procédés honnêtes.

Mais la poursuite menaçante de M\textsuperscript{me} de Maintenon, qui
craignait même son ombre, le contraignit de retourner aux Bruyères, et
bientôt après à Mont-l'Évêque, maison de campagne de l'évêché de Senlis,
parce qu'elle le trouvait trop près de Paris. J'y fus des Bruyères avec
lui et j'y demeurai plusieurs jours. Le grand écuyer y vint dîner avec
lui de Royaumont. La proximité des Bruyères de Paris lui avait procuré
quantité de visites\,; l'éloignement de Mont-l'Évêque ne l'en priva pas.
M\textsuperscript{me} de Maintenon fut piquée à l'excès que sa disgrâce
ne fit pas son abandon général\,; elle s'en expliqua avec tant de dépit
et lui fit revenir tant de menaces sourdes, s'il ne s'éloignait
entièrement, qu'il jugea devoir céder à une si dangereuse persécution.
Il n'avait point de terres, il en cherchait pour placer une partie du
prix de sa charge, il ne savait où se retirer au loin. Il prit le parti
forcé d'aller visiter lui-même les terres qu'on lui proposait, pour
s'éloigner sous ce prétexte, en attendant qu'il pût être fixé quelque
part au loin.

La Feuillade avait fait l'effort de coucher une nuit aux Bruyères et
deux à Mont-l'Évêque. Le surprenant est qu'il avait tellement ensorcelé
son beau-père qu'il lui fut obligé de ce procédé, tandis qu'il n'y eut
personne, jusqu'à ses ennemis même, qui n'en fût indigné.

Il y avait longtemps que je m'apercevais que l'évêque de Chartres ne
m'avait que trop véritablement averti des mauvais offices qu'on m'avait
rendus auprès du roi, et de l'impression qu'ils y avaient faite. Son
changement à mon égard ne pouvait être plus marqué\,; et, quoique je
fusse encore des voyages de Marly, je ne pouvais pas douter que ce
n'était pas sur mon compte\,; piqué de tant de cheminées qui, pour ainsi
dire, m'étaient tombées sur la tête en allant mon chemin, de ne pouvoir
démêler le véritable apostume ni son remède par conséquent, d'avoir
affaire à des ennemis puissants et violents que je ne m'étais point
attirés, tels que M. le Duc et M\textsuperscript{me} la Duchesse, et que
les personnage de la cabale de Vendôme et les envieux et les ennemis
dont les cours sont remplies, et, d'autre part, à des amis faibles ou
affaiblis, comme Chamillart et le chancelier, le maréchal de Boufflers
et les ducs de Beauvilliers et de Chevreuse, qui ne pouvaient m'être
d'aucun secours avec toute leur volonté\,; vaincu par le dépit, je
voulus quitter la cour et en abandonner toutes les idées.

M\textsuperscript{me} de Saint-Simon, plus sage que moi, me représentait
les changements continuels et inattendus des cours, ceux que l'âge y pou
voit apporter, la dépendance où on en était non seulement pour la
fortune, mais pour le patrimoine même, et beaucoup d'autres raisons. À
la fin, nous convînmes d'aller passer deux ans en Guyenne, sous prétexte
d'y aller voir un bien considérable que nous ne connaissions point par
nous-mêmes, faire ainsi une longue absence sans choquer le roi, laisser
couler le temps et voir après le parti que les conjonctures nous
conseilleraient de prendre.

M. de Beauvilliers, qui se voulut adjoindre M. de Chevreuse dans la
consultation que nous lui en fîmes, le chancelier à qui nous en parlâmes
après, furent de cet avis, dans l'impuissance où ils se virent de me
persuader de demeurer à la cour\,; mais ils nous conseillèrent de parler
d'avance de ce voyage, pour éviter l'air de dépit, et qu'il ne se
répandît aussi que j'eusse été doucement averti de m'éloigner.

Il fallut la permission du roi pour s'écarter si loin et si longtemps\,;
je ne voulus pas lui en parler dans la situation où je me trouvais. La
Vrillière, fort de mes amis, et qui avait la Guyenne dans son
département, le fit pour moi, et le roi le trouva bon.

Le maréchal de Montrevel commandait en Guyenne\,; j'ai déjà remarqué,
lors de sa promotion au bâton, quelle espèce d'homme c'était. La tête
avoir achevé de lui tourner en Guyenne\,; il s'y croyait le roi, et avec
des compliments et des langages les plus polis, usurpait peu à peu toute
l'autorité dans mon gouvernement. Ce n'est pas ici le lieu d'expliquer
ce dont il s'agissait entre nous, qui se trouvera nécessairement
ailleurs. Il suffit de dire ici en gros qu'il ne m'était pas possible
d'aller à Blaye, que cela ne fût fini avec une manière de fou pour qui
le roi avait eu toute sa vie du goût, et avec qui les raisons mêmes qui
me menaient en Guyenne ne me laissaient pas espérer que raison, droit et
justice de mon côté, fussent des armes dont je me pusse défendre. Il y
avait deux ans que lui et moi étions convenus de nous en rapporter à
Chamillart, sans que ce ministre eût pu prendre le temps de finir cette
affaire. Je me mis donc à l'en presser par la nécessité où je me
trouvais là-dessus. Le même défaut de loisir, affaires, voyages, temps
rompus, la différèrent toujours, tant qu'enfin arriva sa chute qui lui
ôta tout caractère de décider entre nous, et à Montrevel toute envie de
s'y soumettre.

Si, depuis cinq ou six mois, je m'étais déterminé à la retraite, cet
événement ne fit que m'y confirmer et m'en presser. Un ami éprouvé dans
une telle place et dans une telle faveur est d'un grand et continuel
secours pour les choses et pour les apparences, et laisse un grand vide
par sa disgrâce. Elle m'ôtait de plus le logement de feu M. le maréchal
de Larges au château, qu'il me fallut rendre au duc de Lorges, logé
jusqu'alors dans celui de son beau-père, dont le roi disposa\,; et la
cour, non seulement à demeure, comme j'y avais toujours été, mais même à
fréquenter, est intolérable et impossible sans un logement que je
n'étais pas alors à portée d'obtenir. Depuis le Marly où éclata le
départ de Torcy pour la Hollande, j'en avais été éconduit\,: ainsi la
main du roi s'appesantissait peu à peu en bagatelles, peut-être en
attendant occasion de pis\,; d'aller en Guyenne sans que rien fût
terminé entre Montrevel et moi, il n'y avait pas moyen d'y penser\,; je
pris donc le parti d'aller à la Ferté, résolu d'y passer une et
plusieurs années, et de ne revoir la cour que par moments et pas même
tous les ans, s'il m'était possible sans manquer au tribut sec et pur du
devoir le plus littéral.

Mon assiduité auprès de Chamillart à l'Étang, aux Bruyères, à
Mont-l'Évêque, à Paris, avait déjà déplu. Je partis un mois après qu'il
fut allé chercher des terres pour s'éloigner de Paris. Ses filles
vinrent s'établir et l'attendre à la Ferté, où il revint de ses
tournées, et où je le reçus avec des fêtes et des amusements que je ne
lui aurais pas donnés dans sa faveur et dans sa place, mais dont je
n'eus pas de scrupule, parce qu'il n'y avait plus de cour à lui faire,
ni rien à attendre de lui\,: aussi y fut-il vivement sensible. Il fut
assez longtemps chez moi\,; il y laissa ses filles, et s'en alla à Paris
pour finir plusieurs affaires et le marché de la terre de Courcelles,
dans le pays du Maine, qu'il acheta à la fin. Je demeurai chez moi dans
ma résolution première, où toutefois je ne laissai pas d'être informé de
ce qu'il se passait. Reprenons maintenant le affaires devant et depuis
mon départ de la cour, et qui le retardèrent de beaucoup, et après
lequel je soupirais avec un dépit ardent.

L'expression me manque pour ce que je veux faire entendre. La cour, par
ces grands changements d'état et de fortune de Vendôme et de Chamillart,
était plus que jamais divisée. Parler de cabales, ce serait peut-être
trop dire, et le mot propre à ce qui se passait ne se présente pas.
Quoique trop fort, je dirai donc cabale, en avertissant qu'il dépasse ce
qu'il s'agit de faire entendre, mais qui, sans des périphrases
continuelles, ne se peut autrement rendre par un seul mot.

Trois partis partageaient la cour qui en embrassaient les principaux
personnages, desquels fort peu paraissaient à découvert, et dont
quelques-uns avaient encore leurs recoins et leurs réserves
particulières. Le très petit nombre n'avait en vue que le bien de
l'État, dont la situation chancelante était donnée par tous comme leur
seul objet, tandis que la plupart n'en avaient point d'autre que
soi-même, chacun suivant ce qu'il se proposait de vague ou de
considération, d'autorité, et en éloignement de puissance\,; d'autres de
places et de fortunes à embler\,; d'autres, plus cachés ou moins
considérables, tenaient à quelqu'une des trois, et formaient un sous
ordre qui donnait quelquefois le branle aux affaires, et qui entretenait
cependant la guerre civile des langues.

Sous les ailes de M\textsuperscript{me} de Maintenon se réunissait la
première, dont les principaux, en curée de la chute de Chamillart, et
relevés par celle de Vendôme qu'ils avaient aussi poussoté tant qu'ils
avaient pu, étoient ménagés et ménageaient réciproquement
M\textsuperscript{me} la duchesse de Bourgogne, et étaient bien avec
Monseigneur. Ils jouissaient aussi de l'opinion publique et du lustre
que Boufflers leur communiquait. À lui se ralliaient les autres, pour
s'en parer et pour s'en servir\,; Harcourt, même des bords du Rhin, en
était le pilote, Voysin et sa femme, leurs instruments, qui
réciproquement s'appuyaient d'eux. En deuxième ligne était le
chancelier, qui {[}était{]} dégoûté à l'excès par l'aversion que
M\textsuperscript{me} de Maintenon avait prise pour lui, conséquemment
par l'éloignement du roi\,; Pontchartrain, de loin, à l'appui de la
boule\,; le premier écuyer, vieilli dans les intrigues, qui avait formé
l'union d'Harcourt avec le chancelier, et qui les rameutait tous\,; son
cousin Huxelles, philosophe apparent, cynique, épicurien, faux en tout,
et dont on peut voir le caractère ci-devant (t. IV, p.~92), rongé de
l'ambition la plus noire, dont Monseigneur avait pris la plus grande
opinion par la Choin que Beringhen, sa femme et Bignon, en avaient
coiffée\,; le maréchal de Villeroy qui, du fond de sa disgrâce, n'avait
jamais perdu les étriers chez M\textsuperscript{me} de Maintenon, et que
les autres ménageaient par là et par cet ancien goût du roi qui, par
elle, pouvait renaître\,; le duc de Villeroy, remué par lui, mais avec
d'autres allures, et La Rocheguyon qui, ricanant sans rien dire, tendait
des panneaux, et par Bloin et d'autres souterrains savaient tout et
avaient toute créance de jeunesse auprès de Monseigneur, et qui, quoique
de loin, ne laissaient pas que d'avoir influé à la perte de Vendôme et
de Chamillart, ayant en tiers la duchesse de Villeroy, dont le peu
d'esprit était compensé par du sens, beaucoup de prudence, un secret
impénétrable et la confiance de M\textsuperscript{me} la duchesse de
Bourgogne en beaucoup de choses, qu'elle savait tenir de court et haut à
la main.

D'autre part, sous l'espérance que nourrissait la naissance, la vertu et
les talents de Mgr le duc de Bourgogne, tout de ce côté, par affection
décidée, était le duc de Beauvilliers, le plus apparent de tous\,; le
duc de Chevreuse en était l'âme et le combinateur\,; l'archevêque de
Cambrai, du fond de sa disgrâce Est de son exil, le pilote\,; en
sous-ordre, Torcy et Desmarets\,; le P. Tellier, les jésuites et
Saint-Sulpice, d'ailleurs si éloignés des jésuites, et réciproquement\,;
Desmarets, ami du maréchal de Villeroy et du maréchal d'Huxelles et
Torcy bien avec le chancelier, uni avec lui sur les matières de Rome,
conséquemment contre les jésuites et Saint-Sulpice, et en brassière sur
ce recoin d'affaires avec ses cousins de Chevreuse et surtout de
Beauvilliers, ce qui met toit entre eux du gauche et souvent des
embarras.

Ceux-ci, plus amis entre eux, au besoin, toujours plus concertés, en
occasion continuelle de se voir sans air de se chercher, affranchis des
sarbacanes par leurs places, et voyant tout immédiatement, en état
d'amuser les autres par des fantômes, et d'un coup de nain de rendre
fantômes les réalités les mieux amenées, et par voir et savoir de
source, de rompre la mesure à leur gré, tant était-il vrai, de tout ce
règne, que le ministère donnait tout en affaires, quelque tondante que
M\textsuperscript{me} de Maintenon y eût usurpée, qui n'osait
questionner ni montrer rien suivre, à qui les choses ne venaient par le
roi qu'à bâtons rompus, et qui par là avait si grand besoin d'avoir un
ministre tout à elle. Ceux-ci n'admirent personne avec eux sans une
vraie nécessité, et pour le moment seulement de la nécessité. Ils
n'avaient qu'à parer, et comme ils étaient en place, ils n'avaient qu'à
se défendre et rien à conquérir\,; mais les rieurs n'étaient pas pour
eux. Leur dévotion les tenait en brassière, était tournée aisément en
ridicule\,; le bel air, la mode, l'envie étaient de l'autre côté, avec
la Choin et M\textsuperscript{me} de Maintenon.

Ces deux cabales se tenaient réciproquement en respect. Celle-ci
marchait en silence\,; l'autre, au contraire, avec bruit, et saisissait
tous les moyens de nuire à l'autre. Tout le bel air de la cour et des
armées était de son côté, que le dégoût et l'impatience du gouvernement
grossissait encore, et quantité de gens sages, entraînés par la probité
de Boufflers et les talents d'Harcourt.

D'Antin, M\textsuperscript{me} la Duchesse, M\textsuperscript{lle} de
Lislebonne et sa soeur, leur oncle, inséparable d'elles, et
l'intrinsèque cour de Meudon formaient le troisième parti. Aucun des
deux autres ne voulait d'eux\,; l'un et l'autre les craignaient et s'en
défiaient\,; mais tous les ménageaient, à cause de Monseigneur, et
M\textsuperscript{me} la duchesse de Bourgogne elle-même.

D'Antin et M\textsuperscript{me} la Duchesse n'étaient qu'un\,; ils
étaient également décriés\,; ils étaient pourtant à la tête de ce parti,
d'Antin, par ses privantes avec le roi, qui augmentaient chaque jour, et
dont mieux qu'homme du monde il savait se parer et même s'avantager
solidement\,; lui et M\textsuperscript{me} la Duchesse pour les leurs,
avec Monseigneur. Ce n'était pas que les deux Lorraines n'eussent\,:
encore plus sa confiance et celle de M\textsuperscript{lle} Choin au
moins plus que les deux autres\,; elles avaient de plus un autre
avantage, mais alors et longtemps depuis inconnu, dont j'ai parlé
d'avance (t. V, p.~427), qui était cette liaison avec
M\textsuperscript{me} de Maintenon si honteusement mais si solidement
fondée, et pour cela même si cachée. Mais elles étaient encore étourdies
des deux coups de foudre qui venaient de tomber sur Vendôme et
Chamillart. Boufflers, Harcourt et leurs principaux tenants détestaient
l'orgueil du premier et la suprématie de rang et de commandement où il
s'était élevé. Chevreuse, Beauvilliers et les leurs, par ces raisons, et
plus encore par rapport à Mgr le duc de Bourgogne, n'étaient pas moins
éloignés de lui\,: pas un de ces deux partis n'était donc pas pour se
rapprocher de ce troisième, qui était proprement la cabale de Vendôme,
encore troublée du coup, ni les derniers, de plus, de d'Antin, qui, dans
la folle espérance d'avoir la part principale à la dépouille de
Chamillart, avait travaillé si fortement à sa ruine.

Pour être mieux entendu, donnons un nom aux choses, et nommons ces trois
partis\,: la cabale des seigneurs, qui est le nom qui lui fut donné
alors, celle des ministres, et celle de Meudon.

Cette dernière avait été plus touchée de la fâcheuse épreuve de ses
forces que de la chute de Vendôme\,; elle ne le portait que pour perdre
Mgr le duc de Bourgogne par les raisons qui en ont été expliquées\,; ce
grand coup à la fin manqué à demi, Vendôme de moins les mettait plus au
large auprès de Monseigneur et ramassait tout plus à eux. Je dis manqué
à demi, car il avait pleinement porté par leurs artifices auprès de
Monseigneur qui n'en est jamais revenu pour Mgr le duc de Bourgogne et
qui le lui fit sentir le reste de sa vie, même grossièrement. À l'égard
de Chamillart, ce coup manqué auprès du roi, on a vu par le trait que
lui fit par deux fois M\textsuperscript{lle} de Lislebonne auprès de
M\textsuperscript{lle} Choin, combien peu ils s'en soucièrent dès qu'ils
le virent sur le penchant\,; elle et sa soeur comptèrent bien sur le
successeur par elles-mêmes à cause de Monseigneur, encore plus quand
elles virent Voysin l'être par leurs secrets rapports avec
M\textsuperscript{me} de Maintenon.

Pour Vaudemont, outre qu'il n'était qu'un avec ses nièces, éconduit
qu'il était sans retour des usurpations de rang qu'il avait essayées,
établi d'ailleurs comme il était, tout cela lui importait assez peu, et
sa considération déjà tombée demeurait sans souffrir une plus grande
diminution.

M. du Maine, régnant dans le coeur du roi et de M\textsuperscript{me} de
Maintenon, ménageait tout, n'était à aucun qu'à soi-même, se moquait de
beaucoup, nuisait à tous tant qu'il pouvait, et tous aussi le
craignaient et le connaissaient. Voysin, tout à M\textsuperscript{me} de
Maintenon, lui valait mieux que Chamillart qui s'était livré à lui\,; et
Vendôme ayant péri dans son entreprise des Titans, l'entreprise échouée,
du Maine se trouvait soulagé d'un audacieux qui n'aurait pas voulu être
inférieur à ses enfants, et dont la parité réelle était un titre
embarrassant.

M. le Duc laissait faire, embourbé qu'il était dans son humeur qui
éloignait tout le monde de lui comme d'une mine toujours prête à sauter,
dans ses affaires de la mort de M. le Prince, dans ses plaisirs obscurs,
et dans sa santé qui commençait à devenir mauvaise.

Le comte de Toulouse non plus que M. le duc de Berry ne prenaient part à
rien\,; M. le duc d'Orléans n'était pas en volonté, ni, comme on le
verra bientôt, en état d'entrer en quoi que ce soit, et Mgr le duc de
Bourgogne, enfoncé dans la prière et dans le travail de son cabinet,
ignorait ce qui se passait sur la terre, suivait les impressions douces
et mesurées des ducs de Beauvilliers et de Chevreuse, n'avait figuré en
rien dans les disgrâces de Vendôme et de Chamillart, et s'était contenté
de les offrir à Dieu comme il avait fait les tribulations qu'ils lui
avaient causées.

À l'égard de M\textsuperscript{me} la duchesse de Bourgogne, on a vu
qu'elle procura l'une et qu'elle ne s'épargna pas pour l'autre\,; cela
joint à ce qu'elle était à M\textsuperscript{me} de Maintenon, et
M\textsuperscript{me} de Maintenon a elle, la jetait naturellement du
côté de la cabale des seigneurs avec le goût qu'Harcourt lui avait donné
pour lui, l'estime qu'elle ne pouvait refuser à Boufflers, et son amitié
pour la duchesse de Villeroy. Mais éloignée à l'excès des ducs de
Beauvilliers et de Chevreuse qu'elle craignait en cent façons auprès de
Mgr le duc de Bourgogne, elle s'en était fort rapprochée à l'occasion
des choses de Flandre, et comme elles avaient duré longtemps, ses
préventions s'étaient fort amorties par le commerce qu'elle avait eu
avec eux par elle-même, et par M\textsuperscript{me} de Lévi fort bien
avec elle, une de ses dames du palais, qui avait tout l'esprit possible,
et qui avait saisi ces temps favorables à son père et à son oncle, de
manière qu'elle ne leur était pas opposée, et qu'elle nageait entre les
deux cabales. Pour celle de Meudon, la même de Vendôme, elle ne gardait
que les mesures dont elle ne se pouvait dispenser sagement à cause de
Monseigneur et de la qualité de bâtarde du roi de M\textsuperscript{me}
la Duchesse, avec laquelle on a vu qu'indépendamment du reste elle était
personnellement mal. Le seul d'Antin en fut excepté par l'usage qu'elle
en avait tiré sur la Flandre, et qu'elle s'en promettait encore au
besoin par ses privances avec le roi.

Tallard, enragé de n'être, de rien, parce qu'on ne se fiait à lui
d'aucun côté, ne tenait qu'à Torcy qu'il avait toujours ménagé, et au
maréchal de Villeroy de toute sa vie son parent et son protecteur, sous
la disgrâce duquel il gémissait. Quoique livré aux Rouan, si uns avec
M\textsuperscript{lle} de Lislebonne et sa soeur, cela n'avait point
pris avec lui, et il petillait de se fourrer de quelque chose sans y
pouvoir réussir. Les ministres avaient moins d'éloignement pour lui que
les deux autres partis, mais cela n'allait pas jusqu'à l'admettre. Il
mourait de jalousie contre ceux qui lui étaient préférés dans le
commandement des armées, il pâmait d'envie du brillant du maréchal de
Boufflers, souple toutefois avec eux, mais hors de toute portée.

Villars ne doutait, ni de soi, ni du roi, ni de M\textsuperscript{me} de
Maintenon. Le bonheur infatigable pour lui et l'expérience lui en
répondaient\,; il était content, incapable de suite et de vues hors les
purement personnelles\,; il n'était de rien, il ne se souciait pas d'en
être, et aucun des partis ne le désirait.

Berwick ménageait et était ménagé des deux premiers. Les affaires
d'Angleterre l'avaient lié avec Torcy\,; la piété et la dernière
campagne de Flandre, avec les ducs de Chevreuse et de Beauvilliers\,; il
était fort bien d'ancienneté avec d'Antin, et c'était le seul de la
cabale de Meudon avec qui il fût de la sorte\,; le maréchal de Villeroy
était son ami et son protecteur, et il était ami d'Harcourt qu'il avait
toujours cultivé.

Tessé, ami de Pontchartrain, était suspect aux seigneurs et aux
ministres. Les personnages qu'il avait faits ne lui avaient acquis
l'estime ni la confiance de personne. Sa conduite à l'égard de Catinat
l'avait perdu dans l'esprit de tous les honnêtes gens et empêcha même
les autres de se lier avec lui\,; et sa bassesse à l'égard de Vaudemont,
de Vendôme, de La Feuillade, avait achevé de l'anéantir. Son ambassade à
Rome ne le releva pas, ni ses lettres ridicules au pape, qu'il n'eut pas
honte de publier partout. Il était donc souffert dans la cabale de
Meudon, mais rien au delà, et rejeté des deux autres. Noailles, riche en
calebasses de toutes les sortes, nageait partout, tâtant tout, reçu
honnêtement partout à cause de sa tante et de son langage\,; mais admis
à rien encore en jeune homme qu'on ne connaissait pas assez, et dont le
grand vol et les nombreux crampons tenaient en égale attention et
défiance.

Ces cabales, au reste, avaient leurs subdivisions. Dans celle des
seigneurs, Harcourt avait ses réserves avec tous les autres, quoique
cheminant avec eux et souvent par eux, et ne faisait comparaison avec
aucun, pour me servir de ce terme vulgaire, excepté le chancelier, mais
qui n'était bon que pour le conseil dans la situation où il se trouvait
avec le roi et M\textsuperscript{me} de Maintenon, qui l'excluait de
pouvoir être acteur en rien, sinon quelquefois au conseil, où il était
sans milieu, nul ou emportant la pièce avec feu, adresse et subtilité,
qui était son talent naturel\,; ce qu'il ne faisait qu'aux grandes
occasions pour tomber sur le duc de Beauvilliers sans l'attaquer
directement, mais embarrasser un avis et tacher de lui donner un air
ridicule.

Le maréchal de Villeroy, le moins ardent de tous, par la futilité de son
esprit, son incapacité naturelle et la chute de Vendôme et de
Chamillart, ses deux objets de haine, était de longue main ami
particulier de Desmarets par ses anciennes liaisons avec Bechameil, son
beau-père, fort attaché et protégé du chevalier de Lorraine et d'Effiat.
Malgré sa disgrâce, on a vu qu'il avait conservé l'amitié et souvent la
confiance de M\textsuperscript{me} de Maintenon, une relation assez
fréquente avec elle, la privante de longues conversations avec elle\,;
toutes les fois qu'il allait à Versailles, ce qui n'était pas fréquent.
Beaucoup plus souvent des lettres de l'un et de l'autre, et des mémoires
sur les choses de Flandre qu'elle lui demandait, et qui étaient toujours
biens reçus. Leurs paquets passaient le plus ordinairement par
Desmarets, rarement par la duchesse de Villeroy. Il était assez bien
avec Torcy, et en quelque mesure avec Beauvilliers, qui tous deux n'en
faisaient nul compte, et tous deux fort haïs de La Rocheguyon et du duc
de Villeroy autant qu'il en était capable\,; en cela, comme en bien
d'autres points, divisé d'avec son père, quoique très uni sur le
principal, et mieux ensemble depuis que leur différent genre de vie,
depuis que la disgrâce du père et la charge du fils les avait séparés de
lieux. Chevreuse et Beauvilliers, sans secret l'un pour l'autre, étaient
réservés avec les leurs, et, bien que cousins germains de Torcy, un
fumet de janséniste les écartait de lui fort au delà du but.

D'Antin et M\textsuperscript{me} la Duchesse, entièrement unis de vue,
de besoins réciproques de vices et de lieux, se déficient fort des deux
Lorraines, avec des confidences néanmoins et l'extérieur le plus intime,
que le dessein commun soutenait pendant la vie du roi, en attendant
qu'ils s'entr'égorgeassent tous après, pour la possession unique de
Monseigneur, devenu roi. Cette cabale frayait avec celle des
seigneurs\,; mais elle en était découverte et intérieurement haïe et
crainte comme ayant été celle de Vendôme.

Pour celle des ministres, rien de plus opposé, quoique Torcy et
M\textsuperscript{me} la Duchesse, et par conséquent d'Antin, eussent
des ménagements réciproques par la Bouzols, soeur de Torcy, amie intime
de tous les temps, et de toutes les façons, de M\textsuperscript{me} la
Duchesse, et qui, avec une figure hideuse, était charmante dans le
commerce, avec de l'esprit comme dix démons.

Telle était la face intérieure de la cour dans ce temps orageux, signalé
par deux chutes si profondes, qui semblaient en préparer d'autres.

\hypertarget{chapitre-xvii.}{%
\chapter{CHAPITRE XVII.}\label{chapitre-xvii.}}

1709

~

{\textsc{Blécourt relève Amelot en Espagne, mais avec caractère
d'envoyé.}} {\textsc{- Tournai investi, bien muni\,; Surville et
Mesgrigny dedans.}} {\textsc{- Affaire du rappel des troupes
d'Espagne.}} {\textsc{- Éclat à Marly sur le rappel des troupes
d'Espagne.}} {\textsc{- Boufflers aigri contre Chevreuse.}} {\textsc{-
Conversation sur les deux cabales, et en particulier sur le maréchal de
Boufflers, avec le duc de Beauvilliers, puis avec le duc de Chevreuse,
et ma situation entre les cabales.}}

~

Amelot était rappelé depuis quelque temps, et Blécourt, qui avait déjà
été deux fois en Espagne, l'allait relever, mais avec simple caractère
d'envoyé\footnote{Voy., à la fin de ce volume, la note sur la situation
  de l'Espagne à l'époque où Blécourt vint remplacer Amelot.}. Les
affaires avaient retenu Amelot, qui était là à la tête de toutes sous la
princesse des Ursins, mais si bien avec elle et si capable que, pour ce
qui était affaires, il faisait tout. On verra bientôt que son retour fut
une époque effrayante pour tous les ministres.

Tournai était investi. Surville, lieutenant général, y commandait\,;
Mesgrigny, lieutenant général et principal ingénieur après Vauban, était
gouverneur de la citadelle. Il y avait treize bataillons, quatre
escadrons de dragons, et sept compagnies franches en tout de quatre
cents hommes, Ravignan, maréchal de camp, et profusion de toutes sortes
de munitions de guerre et de bouche\,; avec cela notre armée de Flandre
manquait de tout, et on en était à la cour, à Paris et partout aux
prières de quarante heures.

Il y avait longtemps que l'Espagne commençait à être regardée de mauvais
oeil, et que les oreilles s'ouvraient au spécieux prétexte que les
alliés ne se lassaient point de semer, que cette monarchie était la
pierre d'achoppement. Personne n'avait été d'avis de passer carrière sur
les énormes propositions qui avaient été faites à Torcy à la Haye, mais
il semblait qui, trop crédules, on eût désiré que l'Espagne se trouvât
ruinée d'elle-même, et que par là il se rouvrît une porte à la paix.

De tous temps j'avais pris la liberté d'avoir un sentiment bien
opposé\,; jamais je n'avais cru que l'Espagne fût un obstacle sérieux à
terminer la guerre. Je ne me figurais point les alliés de l'empereur
assez épris de la grandeur de sa maison, pour ne s'épuiser que pour
elle. J'étais d'ailleurs persuadé que pas un ne voulant la paix, de rage
contre la personne du roi, et de jalousie contre la France, tous avaient
saisi un prétexte plausible de l'écarter, durable tant qu'ils voudraient
par sa nature\,; et j'en concluais que le seul moyen de le leur ôter
était de secourir si puissamment le roi d'Espagne et de seconder si
fermement ses succès et le bon ordre déjà rétabli dans ses troupes et
dans ses finances, et la grande volonté des peuples, que de préférence à
tout on rendît ses frontières libres, pour ôter aux alliés tout espoir
d'y revenir, et faire tomber cet éternel prétexte d'Espagne dont ils
faisaient bouclier contre toutes propositions, puisque le roi d'Espagne,
délivré de la sorte, ce qui avait été aisé quatre ans durant, il n'eût
plus été soutenable aux ennemis de rien mettre en avant là-dessus, et se
seraient vus réduits, lorsqu'en effet ils auraient voulu la paix, à la
traiter à des conditions qui, à la vérité, eussent fort diminué la
puissance des deux couronnes, leur seul intérêt essentiel. On était
encore à temps d'y revenir\,; mais on n'aimait pas à approfondir, et on
aimait à se flatter dans l'extrême besoin où les désastres avaient
réduit le royaume, dont on a vu ici les causes expliquées en plus d'une
occasion.

On voulut donc se fermer les veux à tout autre raisonnement qu'à celui
d'avancer nous-mêmes le renversement d'un trône qui nous avait coûté
tant de sang et d'argent à maintenir, et par ce moyen nous dérober à la
honte et à la nécessité de nous mettre du côté de nos ennemis communs
pour y travailler conjointement avec eux à force ouverte, et cependant
les adoucir en produisant le même effet qu'ils voulaient exiger de notre
concours d'une manière plus dure ou plutôt barbare. La base de ce
raisonnement était la présupposition qu'ils voulaient bien la paix,
pourvu que la monarchie d'Espagne revînt à la maison d'Autriche, sans
faire réflexion que tout montrait qu'ils ne voulaient point de paix, et
qu'ils ne songeaient qu'à leurrer leurs peuples qui soutenaient le poids
de la guerre, et à leur cacher leur dessein qui ne tendait qu'à une
destruction générale de la France, qu'ils ne leur osaient pas montrer,
et qui, une fois découvert par la continuation opiniâtre de la guerre,
après leur avoir ôté manifestement toute espérance sur l'Espagne par les
armes, produirait nécessairement la paix malgré le triumvirat qui les
gouvernait tous par ses artifices, et qui seul voulait éterniser la
guerre, comme on le verra dans les Pièces des négociations de Torcy à la
Haye, et depuis du maréchal d'Huxelles à Gertruydemberg. Mais on était
si loin de raisonner ainsi, qu'on trouvait que les alliés n'avaient pas
tort, et qu'il n'y avait d'issue qu'en les satisfaisant sur un point
essentiel pour eux, ce qui ne se pouvait opérer sans une honte déclarée,
que par les moyens obliques de laisser périr l'Espagne d'elle-même. Il
fut donc agité de congédier le duc d'Albe, de faire revenir d'Espagne
toutes les troupes françaises, de cesser d'y faire ou même d'y laisser
passer aucune Sorte de secours, et d'en rappeler Amelot et
M\textsuperscript{me} des Ursins même. On ne voulait pas douter que les
alliés, peu crédules à nos paroles, ne le devinssent à nos actions\,;
que le roi d'Espagne sans ressource ne fût bientôt réduit à revenir en
France, ou à. se contenter du très peu que ses ennemis lui voudraient
bien laisser par grâce, pour ne pas dire par aumône, et que la paix ne
suivît incontinent. Ce fut dans celte pensée qu'Amelot fut rappelé, que
M\textsuperscript{me} des Ursins eut ordre de se disposer aussi à
quitter l'Espagne, et Besons, celui de passer de Catalogne en Espagne
pour en ramener toutes nos troupes. Le roi et la reine d'Espagne, dans
la dernière alarme d'un parti si violent, se mirent aux hauts cris et à
demander au moins qu'on laissât tout en l'état jusqu'à ce qu'Amelot eût
achevé de mettre ordre à des affaires importantes prêtes à
terminer\footnote{Voy. aux notes, placées à la fin du volume, les
  extraits des dépêches d'Amelot.}.

Dans cet intervalle, les alliés qui ne voulaient point de paix, ou
plutôt le triumvirat qui s'était rendu maître des affaires, ajoutèrent
les conditions énormes du passage de leur armée par la France, et autres
qui se trouvent parmi les Pièces de la négociation de Torcy à la Haye,
qui rompirent tout. Malgré la rupture, on voulut toujours rappeler nos
troupes, non plus dans la vile de la paix, qui ne se pouvait plus
espérer, mais dans celle de la défense de nos frontières, sans
considérer qu'elles consommeraient le meilleur temps de la campagne à se
rendre où on les destinerait. Parmi ces incertitudes, Besons reçut ordre
de suspendre, suivant la demande du roi d'Espagne, jusqu'à ce qu'Amelot
eût achevé ce qu'il avait commencé, tellement qu'étant déjà en Espagne
et dans cette espèce de suspension de ramener ses troupes, il n'osait
les mettre en corps d'armée et les opposer au comte de Staremberg, qui
mettait les siennes en mouvement.

Un voyage de Marly arrivé dans ces entrefaites devint fort
remarquable\,; et pour en faire entendre le principal, il faut en
expliquer l'accessoire. On a vu (t. VI, p.~183 et suiv.) que le duc de
Chevreuse était très réellement ministre d'État sans entrer dans le
conseil, et la considération de sa femme et ses privances avec le roi et
chez M\textsuperscript{me} de Maintenon même à cause de lui, que
l'affaire de M. de Cambrai n'avait pu affaiblir que pendant quelques
mois\,; sa santé ne lui permettait pas, depuis quelque temps, de mettre
un corps, et quoique le grand ami des dames fût banni de Marly, elles
n'y pouvaient pourtant paraître qu'habillées avec un corps et une robe
de chambre. Cette raison avait éloigné M\textsuperscript{me} de
Chevreuse de Marly, qui y allait tous les voyages\,; mais toujours en se
présentant, dont personne n'était dispensé. Le roi s'en était plaint,
et, à la fin, voulut qu'elle y vint sans corps. Alors elle ne paraissait
ni dans le salon ni à la table du roi, mais le voyait tous les jours
chez M\textsuperscript{me} de Maintenon, et à des promenades
particulières. M. de Chevreuse, qui aimait sa maison de Dampierre, à
quatre lieues de Versailles, le particulier, la solitude même et la
retraite par piété, profitait tant qu'il pouvait du prétexte de la santé
de M\textsuperscript{me} de Chevreuse, pour se dispenser des Marlys, ce
que le roi trouvait souvent mauvais, et avait peine à le lui accorder, à
cause du fil des affaires. Malgré cette facilité d'y aller sans corps,
M\textsuperscript{me} de Chevreuse évitait encore, et le roi se fâchait,
mais ils ne laissaient pas d'esquiver.

À celui-ci ils y furent, et la rareté donna de l'attention, parce
qu'avec toute cette rareté, M. de Chevreuse avait été du dernier voyage,
et depuis longtemps on ne l'y voyait plus deux fois de suite. Les grands
coups s'y devaient ruer tout de bon sur le rappel des troupes d'Espagne.
Le duc de Beauvilliers était le grand promoteur de l'affirmative, Mgr le
duc de Bourgogne l'y secondait, les ministres suivaient la plupart, le
chancelier même ne s'en éloignait pas, et par une singularité qu'on
n'aurait pas attendue, Desmarets était de l'avis opposé, Voysin aussi,
mais avec faiblesse, soit par sa nouveauté et son peu d'expérience, soit
pour voir démêler la fusée, et se tenir cependant un peu à quartier.
Monseigneur, toujours ferme en faveur de son fils, et ferme à l'excès,
mais uniquement sur ce chapitre, contestait formellement pour la
négative, malgré lequel l'autre avis `l'emporta, et le rappel des
troupes fut résolu.

Ce débat ne s'était point passé sans émotion. Il fut su dès le jour
même, et ce qui avait été résolu, et le maréchal de Boufflers en parla
au roi, qui lui avoua le fait, et sans se laisser ébranler. Le maréchal
alla au duc de Beauvilliers, qui, averti de l'aveu du roi au maréchal,
ne disconvint point du fait. Boufflers lui demanda ses raisons pour y
opposer les siennes. Beauvilliers, avec ses précisions, refusa de
s'expliquer parce qu'il était ministre, et renvoya le maréchal au duc de
Chevreuse, en l'assurant qu'il était aussi instruit que liai, quoiqu'il
n'entrât pas au conseil, et que, n'étant tenu à rien, il le trouverait
en état de le satisfaire. Chevreuse prêta donc le collet au maréchal, et
se promettait bien de sa dialectique de mettre bientôt à bout le peu
d'esprit du maréchal. Au lieu d'y réussir, il échauffa son homme, qui,
plein de l'importance de la chose, en entretint chacun.

Tout ce qui était à Marly ne s'entretint d'autre chose, et le courtisan,
ravi d'oser parler tout haut d'une affaire de cette sorte, se partialisa
selon son goût, mais avec tant de chaleur, qu'elle sembla être devenue
celle d'un chacun. Le nombre et l'espèce de ceux qui tenaient pour la
négative l'emporta fort sur ceux qui soutenaient l'affirmative, dont le
courage accrut tellement au maréchal de Boufflers, qu'il fut trouver
M\textsuperscript{me} de Maintenon et lui en parla de toute sa force. M.
le duc d'Orléans, du même avis, criait de son côté qu'il connaissait
l'Espagne et les Espagnols, et mille raisons particulières tirées de
cette connaissance. Il plut tellement par là au maréchal qu'il proposa à
M\textsuperscript{me} de Maintenon que, puisqu'il était question d'une
si importante affaire, qui regardait l'Espagne où ce prince avait si
bien servi, le roi l'en devrait consulter. Mais Boufflers ignorait le
fatal trop bon mot qui avait rendu M\textsuperscript{me} de Maintenon et
M\textsuperscript{me} des Ursins ses plus mortelles ennemies, et ne put
gagner ce point. Le duc de Villeroy et La Rocheguyon, son beau-frère,
recueillaient les voix, échauffèrent Monseigneur avec qui ils étaient à
portée de tout, et poussèrent Boufflers à lui aller parler.

Ce prince, bien embouché et qui ne fut jamais ardent de soi que pour le
roi d'Espagne, parla au roi avec force contre le rappel de ses troupes
et l'abandon. Le duc d'Albe, averti de tout ce vacarme, hasarda une
chose du tout inusitée jusqu'alors. Il alla à Marly sans demander si on
le trouvait bon, et, tout en arrivant, {[}sollicita{]} une audience que
le roi lui donna aussitôt, dont il usa avec tout l'esprit et la force
possible, tandis qu'en même temps le duc de Chevreuse livrait chance à
tout le monde en plein salon, et y disputait contre tout venant. Tant de
bruit étonna le roi enfin, et le porta, par M\textsuperscript{me} de
Maintenon, à ce qu'il n'avait jamais fait sur une affaire discutée et
résolue. Il suspendit les ordres, et rassembla le conseil d'État pour
délibérer de nouveau sur cette affaire. Le débit de part et d'autre y
fut très vif, Monseigneur parla fort hautement, dont la conclusion fut
un \emph{mezzo-termine}, tous ordinairement fort mauvais.

Il fut résolu de laisser soixante-six bataillons au roi d'Espagne, pour
ne le pas tout à fait abandonner à l'entrée d'une campagne, et sans l'en
avoir averti à temps\,; et de faire revenir le maréchal de Besons avec
tout le reste des troupes françaises, en laissant Asfeld général de
celles qui demeureraient avec quelques officiers généraux.

Ce parti pris et déclaré ne satisfit personne. Ceux qui voulaient
soutenir l'Espagne s'en prévalurent pour crier qu'ils avaient donc eu
raison, et pour blâmer d'autant plus de n'y laisser qu'une partie des
troupes, et en rendre le tout inutile\,: en Espagne par ce grand
retranchement, à nos frontières par la longue marche que celles qu'on
rappelait auraient à faire pour se rendre à nos armées du Dauphiné et de
Roussillon dont nous avions à garder les frontières peu couvertes des
Catalans assistés des ennemis, peu occupés qu'ils seraient par le roi
d'Espagne si affaibli et partagé à faire tête à eux, au Portugal, et
même en d'autres lieux plus intérieurs. Ceux qui voulurent le rappel
entier demeurèrent dans le silence, honteux d'avoir perdu leur cause
devant le tribunal du public, et de ne l'avoir pas gagnée dans la
révision qui s'en était faite au conseil. Mais ils n'en furent pas plus
persuadés. Les ordres furent expédiés aussitôt conformément à cette
dernière résolution.

Le lendemain qu'elle eut été, prise, Chevreuse, prenant Boufflers par le
bras, suivant tous deux le roi qui sortait de la messe, lui dit en
riant, comme pour se raccommoder avec lui\,: «\,Vous avez vaincu.\,»
Mais le maréchal, bouillant encore, et dépité du parti mitoyen, lui fit
une si vive repartie, qu'elle déconcerta le duc, bien qu'elle n'eût rien
d'offensant. Cet incident acheva de les éloigner l'un de l'autre, et
Beauvilliers conséquemment.

Une bagatelle de discussion entre un garde du corps et un chevau-léger
de la garde avait commencé cet éloignement il y avait deux ou trois
mois. Le maréchal de Boufflers, impatienté des longs raisonnements du
duc de Chevreuse, était venu chez moi m'exposer l'affaire et me prier de
lui en dire mon sentiment\,; et comme dans le vrai il n'y avait pas
ombre de difficulté pour le garde, et que je le dis franchement au
maréchal, il voulut que j'en parlasse au duc de Chevreuse. Je le fis et
je ne pus le persuader. Dans ce mécontentement que Boufflers prit aussi
avec trop d'amertume, vint tout ce qui a été raconté de la disgrâce de
Chamillart et du rappel des troupes d'Espagne, où tous deux se
trouvèrent d'avis et de partis si opposés.

Le reste de ce voyage de Marly se sentit de la vivacité de cette
dernière affaire, et les courtisans remarquèrent en M. de Chevreuse un
air d'empressement qui lui était entièrement nouveau. Ils s'aperçurent
qu'il cherchait à s'approcher de M\textsuperscript{me} la duchesse de
Bourgogne, et qu'il en était bien reçu. Cela n'était pas étrange\,; elle
savait combien il s'était intéressé pour Mgr le duc de Bourgogne pendant
la dernière campagne de Flandre par le duc de Beauvilliers et par
M\textsuperscript{me} de Lévi si bien et si libre avec elle\,; ce qui
l'avait très favorablement changée pour les deux beaux-frères.

Un soir entre autres qu'elle s'amusait dans le salon à s'instruire du
hoca\footnote{Jeu de hasard qui avait été introduit en France par le
  cardinal Mazarin. C'était une espèce de loterie.},
M\textsuperscript{me} de Beauvilliers lui dit que M. de Chevreuse le
savait très bien pour y avoir beaucoup joué autrefois. Là-dessus la
princesse l'appela, et il demeura jusqu'à une heure après minuit dans le
salon à le lui apprendre. Cette singularité fit une nouvelle, car il
n'en faut pas davantage à la cour. Les gens des autres cabales en
riaient et disaient tout haut qu'ils allaient envoyer charitablement
avertir chez la duchesse de Chevreuse et chez le duc de Beauvilliers, où
à heure si indue on les croyait sûrement perdus.

Cette cabale des seigneurs tâcha de prendre l'ascendant et soutint
longtemps l'autre, à force de hardiesse. Peu après le retour de cet
orageux Marly à Versailles, M. de Chevreuse, raisonnant dans la chambre
du roi avec quelques personnes, en attendant qu'il allât à la messe, le
maréchal de Boufflers les joignit et brusqua le ducs d'humeur, et pour
le coup sans raison, et s'engoua de dire, et de dire si mal, que
quelques-uns des siens, qui par hasard s'y trouvèrent, ne purent
s'empêcher de l'avouer, toutefois sans rien d'offensant.

Toutes ces choses me firent beaucoup de peine par les suites d'aversion
que j'en craignais. Tous deux étaient intimement mes amis, et les ducs
de Chevreuse et de Beauvilliers n'étaient qu'un\,; autre raison du plus
grand poids pour moi. Je connaissais leur naturelle faiblesse, et
combien le maréchal était poussé, qui jusqu'alors avait bien vécu avec
eux, au moins avec mesure. Je redoutais un orage conduit par
M\textsuperscript{me} de Maintenon, pressé par sa cabale, tous gens
fermes et actifs. J'essayai donc d'abord d'adoucir Boufflers, et je
reconnus que la chose n'était pas en état d'être précipitée\,; en même
temps je fis des pas vers les deux ducs, tant pour les ramener au
maréchal que pour les exciter à se cramponner bien, mais sans leur rien
dire de tout ce que je voyais, pour ne pas intimider des gens déjà trop
timides.

M. de Beauvilliers m'étant venu voir dans ces entrefaites, et m'ayant
trouvé seul, je voulus en profiter. Je le mis sur ce qui s'était passé à
Marly, il me le conta sobrement et avec indifférence, mais
franchement\,; je lui contestai son avis sur le rappel des troupes dont
le sort était jeté uniquement pour entrer mieux en matière, et de cette
façon je vins au point que je voulais traiter avec lui, qui était la
cabale opposée, et qui en voulait à tous les ministres, qui commençait à
prendre force et à parler haut. Il me dit que tout cela ne lui importait
guère, qu'il disait son avis comme il le pensait, parce qu'il avait
droit de le dire au conseil\,; que, du reste, il lui importait peu en
son particulier qu'il fût goûté, ou non, pourvu qu'il fit l'acquit de sa
conscience, moins encore de la cabale qu'il voyait bien toute formée et
toute menaçante\,; que je l'avais vu, dans la crise des affaires de M.
de Cambrai, dans un état bien plus hasardeux, puisqu'il était près alors
d'être congédié à tous les instants\,; que je lui pouvoir être témoin
que je ne l'en avais vu ni plus ému ni plus embarrassé, aussi content de
se retirer en sa maison que de vivre parmi les affaires, et même
davantage\,; qu'il regardait les choses du même œil présentement\,; qu'à
son âge, dans l'état où se trouvait sa famille, et pensant comme il
faisait depuis longtemps sur ce monde et sur l'autre, il ne regardait
pas comme un malheur d'achever sa vie chez lui, en solitude, à la
campagne, et de s'y préparer avec plus de tranquillité à la mort\,;
qu'il ne se pouvait retirer avec bienséance dans la confusion présente
des affaires\,; mais qu'il était bien éloigné de regarder comme un mal
la nécessité de le faire qui lui donnerait du repos.

Je lui répondis que personne n'était plus persuadé que je l'étais de la
sincérité et de la solidité de ses sentiments, et ne les admirait
davantage, et en cela je disais ce que je pensais, et je ne me trompais
pas, mais que j'avais un dilemme à lui opposer que je le suppliais
d'écouter avec attention, auquel je ne croyais pas de réplique\,: que
si, charmé des biens et de la douceur de la retraite, et de n'avoir plus
à songer qu'aux années éternelles, il se persuadait que son âge (il
avait alors soixante et un ans), l'état de sa famille et ses propres
réflexions sur les affaires présentes, le dussent affranchir de tout
autre soin que de celui de vaquer uniquement à son salut, je n'avais
nulle volonté de lui rien opposer, encore que je me persuadasse que je
ne manquerais pas de bonnes raisons de conscience pour le faire\,; qu'en
ce cas-là il devait dès aujourd'hui remettre ses emplois, se retirer
dans le lieu qu'il jugerait le plus propre à son dessein, et abdiquer
tout soin de ce monde\,: mais que, s'il pensait que chacun devait
travailler en sa manière dans sa vocation particulière, et selon la voie
à Dieu avait conduit et établi les divers particuliers de ce monde,
chacun dans son état, pour rendre compte à Dieu de ses talents et de ses
rouvres, et qu'il ne crût pas sa carrière remplie, il n'était pas
douteux qu'il ne dût demeurer dans le monde, et dans les fonctions où il
avait plu à la Providence de l'appeler, non pour en jouir à sa manière,
niais pour y servir Dieu et l'État, et que de cela il compterait devant
Dieu comme ferait un moine de sa règle\,; que cela étant ainsi, il ne
lui devait pas suffire d'aller par routine aux différents conseils où il
avait sa voix, et d'y dire son avis par forme et avec nonchalance,
content d'avoir parlé selon ce qu'il croyait meilleur, et peu en peine
de l'effet de son avis comme ferait un moine qui, assidu au choeur,
psalmodierait avec les autres, content d'avoir prononcé les psaumes dans
la cadence accoutumée, peu en peine d'y appliquer son esprit et son
cœur, ni de réfléchir que sa présence corporelle et l'articulation de
ses lèvres était insuffisante sans cette double application\,; que
l'état de ministre, surtout dans des conjonctures aussi critiques que
celles où on se trouvait actuellement, demandait en ses avis non
seulement la probité et la sincérité, mais la force pour les soutenir et
les faire valoir leur juste poids, et de s'opposer généreusement, non
pour son intérêt particulier, mais pour le bien de l'État trop
chancelant, à des cabales dont le but était d'arriver à des fins
particulières, et qui par sa destruction priveraient l'État de ses avis,
qui néanmoins lui paraissaient tels à lui-même que sa conscience
l'empêchait de l'en priver en se retirant maintenant du monde et des
affaires\,; qu'il n'était donc pas seulement de son devoir de dire son
avis, mais de le faire valoir, mais de demeurer en place pour avoir
droit de le dire, mais de demeurer tellement qu'il n'opinât pas sans
fruit, mais de faire toutes les choses nécessaires et convenables pour y
demeurer, et y demeurer en autorité, sans quoi il vaudrait autant pour
l'État qu'il n'y fût plus, et mieux pour lui et pour son repos et son
loisir\,; qu'une situation mitoyenne était, quant au bien de ce monde et
aux devoirs concernant l'autre, la pire de toutes\,; que vivre ainsi
content de tout était une tranquillité et un repos anticipés hors de
place, de temps, et de saison, une usurpation de retraite, un synonyme
de prévarication.

M. de Beauvilliers sourit de la chaleur que je mêlais à ce discours, et
ne laissa pas de l'écouter avec grande attention\,; il m'interrompit
peu, et je repris les détails où je descendis, qu'il était en état de
procurer et lui seul sans qu'ils pussent être suppléés par personne par
rapport à Mgr le duc de Bourgogne, et même à la façon dont il était
auprès du roi. Il en convint, ensuite je passai aux autres ministres
dont la ruine amenait la sienne, et je lui dis avec hardiesse ces
propres termes dont je m'étais déjà servi une autre fois lorsque je le
forçai de parler au roi sur l'entrée résolue du duc d'Harcourt au
conseil, qu'il fit avorter. Qu'il n'y avait point à se mécompter, que
ç'avait été un miracle qu'il n'eût pas succombé sous la main puissante
de M\textsuperscript{me} de Maintenon lors des affaires du quiétisme\,;
que l'estime solide du roi, la confiance de sa place de gouverneur des
enfants de France, ni celle du ministre dont il était revêtu ne
l'auraient pas tiré d'affaire\,; que son salut, il ne le devait qu'à ses
entrées de gouverneur, qui, entées sur celles de premier gentilhomme de
la chambre, avaient si bien accoutumé le roi à le voir dans ses heures
les plus privées, et à l'y voir en toutes depuis si longtemps, qu'elles
avaient fait de lui, à son égard, une espèce de garçon bleu renforcé qui
seul avait soutenu le seigneur, le ministre, l'homme de confiance,
lequel, sans cela eût péri\,; que c'était donc à ce titre qu'il devait
oser se cramponner et s'affermir en toutes manières, attaquer la cabale
contraire sans crainte ni mollesse, en mettre en garde le roi, par des
vérités fortes et bien assenées, non pas se laisser frapper sans montrer
le sentier, et par cette sorte de dévotion si mal entendue, enhardir les
frappeurs, y accoutumer le roi, devenir inutile, et se laisser enfin
porter par terre lui et les siens.

De toutes les différentes fois que j'aie parlé à M. de Beauvilliers,
excepté celle de l'entrée du maréchal d'Harcourt au conseil, je ne le
fis jamais tant de suite, je ne dis pas de raisonnements, mais, si cela
se peut dire, exhortations, ni avec une si grande impression sur lui.

Il se mit d'abord sur la défensive, non plus pour quitter et se retirer,
car il était convenu d'abord que ce n'en était pas le temps, non plus
même sur sa faiblesse par dévotion, car, à mon raisonnement, il sentit
bien qu'il n'y avait rien de solide à répondre\,; mais d'abord sur la
cabale\,; il s'effaroucha de ce mot, je ne le lui contestai pas. Il se
persuadait qu'il n'y en avait point, ses précisions le lui faisaient
croire ainsi, mais l'effet du terme je l'empêchai d'en disputer. Il se
mit sur les difficultés de pratiquer ce que je lui voulais persuader de
faire, et l'embarras des moyens en ne voulant dire mal de personne.

Je répondis que cela n'empêchait pas la force dans ses avis, les
répliques étendues, ni les insinuations et les raisonnements
particuliers\,; qu'après cela, la cabale opposée était composée de
diverses sortes de personnes parmi lesquelles il y en avait de bons et
de mauvais\,; que les mauvais étaient ceux qui, couverts du manteau du
bien des affaires, ne travaillaient que pour eux-mêmes\,; que ceux-là
étaient les maréchaux d'Harcourt et d'Huxelles, que par cela même il
était permis de faire connaître pour tels, de les démasquer à propos et
d'énerver auprès du roi, de sorte que tout leur esprit et leur sens si
vanté par les leurs ne servît qu'à leur nuire en donnant ombrage de
leurs sentiments et de leurs avis, ce qui les écarterait aisément dans
la suite\,; que la piété bien entendue le de mandait, loin de s'y
opposer, et que c'était là ce qu'il falloir faire.

Nous disputâmes assez là-dessus, et je crus n'avoir pas peu gagné de
l'avoir fait convenir que tout ce que j'avançais à leur égard n'était
pas à rejeter, pourvu que cela se fit par nécessité et avec modération.
Je battis encore le duc là-dessus, enclin à n'y trouver jamais la
nécessité assez décisive, ni la modération assez compassée, sur quoi je
lui ôtai la plupart de ses réponses. De cette discussion nous passâmes à
celle des bons, parmi lesquels je citai le maréchal de Boufflers pour
exemple\,; le duc en convint avec empressement, et saisissant le
triomphe me demanda d'un air content ce que je vouloir qu'il fit à
celui-là qui certainement ne prenait feu que de bonne foi. «\,Ce que je
veux, répliquai-je, que vous le regagniez absolument, et que deux hommes
aussi purs et aussi bien intentionnés que vous l'êtes tous deux ne
demeuriez pas plus longtemps opposés, ni la cabale où il est plus
longtemps décorée d'un homme si estimable, et qui la fortifie avec tant
d'avantages contre vous.\,»

De là je lui dis, comme il était vrai, que j'avais toujours reconnu du
goût pour lui fondé sur l'estime dans le maréchal\,; que j'étais même
surpris que les autres l'eussent entraîné assez avant pour l'aigrir au
point qu'ils avaient fait\,; que c'était un bon homme, doux, aisé à
ramener par des avances de considération, d'estime et d'amitié, et
pareillement aisé à éloigner par l'indifférence, et un air d'autorité et
de supériorité\,; que les premières manières étaient tellement les
siennes à lui, M. de Beauvilliers, qu'il n'y aurait nulle peine\,; que
pour les secondes qui lui ressemblaient si peu, il y fallait néanmoins
prendre garde dans le raisonnement, qui, étant court dans le maréchal,
devait être ménagé en rie lui contestant pas les bagatelles, et
réservant l'effort de la persuasion pour les choses importantes, mais
avec art et douceur, tâchant de l'amener comme de lui-même\,; surtout de
ne lui laisser sentir nul poids de ministre ni de supériorité d'esprit
ou d'expérience dans les affaires, et s'aider adroitement de flatteries
sur sa capacité à la guerre, sur les choses qu'il y a effectivement
faites, et sur ses bonnes intentions qu'on ne pouvait douter être les
seuls qui le menassent et sans aucun intérêt\,; qu'en s'y prenant de la
sorte avec application et suite, j'étais persuadé que Boufflers serait
d'abord touché du cas qu'il sentirait être fait de lui, et par là
deviendrait bientôt capable d'entrer en raison\,; qu'il ne serait pas
difficile de lui ôter les impressions que les autres étaient venus à
bout de lui donner, et sinon de le détacher tout à fait d'eux, de le
rendre du moins un instrument dont ils ne feraient pas dans la suite
tout l'usage qu'ils projetaient et qu'ils avaient déjà commencé d'en
faire.

Beauvilliers goûta au dernier point mon discours, et s'ouvrant de plus
en plus\,: «\,Eh qui, me dit-il, n'a pas envie de le raccrocher, et de
faire tout ce qu'il faut pour cela\,?» Puis convint que ce que je lui
proposais était le meilleur, et qu'il fallait incessamment travailler
sur ce plan-là.

Je me gardai bien de lui en nommer aucuns autres. Je connaissais trop
l'antipathie naturelle de l'esprit et de l'humeur du chancelier pour lui
proposer rien à son égard pour les rapprocher l'un de l'autre, bien
moins encore pour nuire au chancelier, mon ami au point qu'il l'était,
ni sur l'aversion des ducs de La Rocheguyon et de Villeroy, glissant
ainsi pour ne pas commettre nies amis d'une part, et ne les pas laisser
dupes de l'autre. Avant finir, je repris encore un peu le propos de
nuire à ceux qui ne valaient rien, et je le fis souvenir de la pacifique
et silencieuse conduite de Mgr le duc de Bourgogne qui l'avait abattu
sous le duc de Vendôme à tel point, qu'il en demeurait meurtri après
même la chute de ce colosse. Je lui remis que lui-même n'avait pas
approuvé cette douceur cruelle, et comme il s'éleva contre la
comparaison, par sa disproportion d'avec ce jeune prince, je m'élevai à
mon tour, et le mis hors de défense par la compensation de l'importance
de ses places, et le devoir dont il était comptable au roi et à l'État.

Nous nous séparâmes enfin, lui très satisfait de toutes mes réponses, et
persuadé qu'il devait faire plus d'usage de son crédit et de son esprit,
et moi en large et content au possible de m'être si utilement déchargé
le coeur avec lui, et de lui avoir de plus vivement reproché d'être si
peu instruit de mille choses qui se passaient à la cour, qui, petites en
apparence auprès des affaires d'État, ne laissaient pas de découvrir
mille intrigues nécessaires à savoir et dont l'ignorance conduisait
pourtant assez souvent à celles de choses qui influaient tellement à la
justesse du raisonnement en choses considérables, qu'on se trouvait au
besoin court par ce défaut, et hors d'état de prendre de justes mesures
et à temps.

C'était aussi mon grief contre le duc de Chevreuse auquel je l'avais
très souvent reproché, et qui prétendait s'en disculper en m'opposant
qu'il n'était chargé de rien avec ses précisions désespérantes, parce
qu'il n'entrait pas au conseil, quoiqu'il fût en effet ministre et
entrant dans tout avec le roi, et avec les autres ministres, comme je
l'avais découvert il y avait longtemps, et que M. de Beauvilliers et
lui-même ensuite me l'eussent avoué des lors, ainsi que je l'ai remarqué
(t. VI, p.~183), il était de plus l'âme de la cabale des ministres, et
considéré comme tel par toutes les autres.

Je lui contai dès le lendemain la conversation que j'avais eue avec M.
de Beauvilliers quoiqu'il fût accoutumé à ma franchise et à ma liberté
avec son beau-frère et avec lui, il ne laissa pas d'être extrêmement
surpris de la hardiesse dont j'avais usé dans les choses et dans les
termes, et il m'en remercia, d'où je pris occasion de lui reprocher
fortement pourquoi il ne parlait pas de même, puisqu'il trouvait cette
force nécessaire avec son beau-frère, avec lequel il était à toute
portée, en toute confiance et intimité, et si entièrement au fait de
tout, au lieu d'entretenir ses mesures étroites et sa faiblesse par la
sienne propre.

Il s'excusa avec plus de gentillesse que de solidité, et convint
pourtant de l'excès des mesures du duc de Beauvilliers, et du tort que
cela faisait aux affaires, par ne vouloir pas user de son esprit et de
son crédit, demeurer dans des entraves continuelles de réserve, de
retenue et d'inaction qui arrêtaient tout de leur part, et donnaient jeu
aux autres dont ils savaient bien profiter, jusque-là, qu'il m'avoua que
M\textsuperscript{me}s de Chevreuse et de Beauvilliers n'en étaient pas
plus contentes que lui, et que tous trois y échouaient continuellement.

Nous approfondîmes fort la matière, et même avec un grand détail. Je
n'en crus pas le temps perdu, parce qu'en lui inculquant les choses que
je croyais nécessaires, c'était parler avec le même succès à eux tous et
jusqu'à Mgr le duc de Bourgogne\,; la suite me le persuada encore
davantage\,; ils devinrent plus éveillés sur tout ce qu'il se passait,
plus attentifs à m'en demander des nouvelles, à en raisonner avec moi,
plus occupés à parer les coups et même à en porter, et M. de
Beauvilliers encore plus au large avec moi et sur tous chapitres. Je
m'aperçus bien par le maréchal de Boufflers même qu'ils n'étaient pas
demeurés oisifs pour le rapprocher, en quoi ils auraient mieux et plus
tôt réussi, s'ils l'eussent fait plus ouvertement, à quoi je suppléais
autant qu'il m'était possible.

Ce que le monde nomme hasard, et qui, comme toutes choses n'est qu'une
disposition de la Providence, qui toute ma vie m'avait lié avec une
singularité marquée à presque toutes les personnes opposées, en usait de
même à mon égard sur ces deux cabales des seigneurs et des ministres.

Entièrement uni aux ducs de Beauvilliers et de Chevreuse, et à presque
toute leur famille, lié intimement à Chamillart jusque dans sa plus
profonde disgrâce, fort bien avec les jésuites, et avec Mgr le duc de
Bourgogne, comme on l'a vu à propos des choses de Flandre, bien aussi,
quoique de loin et par les deux ducs, avec M. de Cambrai sans
connaissance immédiate, mon coeur était à cette cabale qui pouvait
compter Mgr le duc de Bourgogne à elle envers et contre tous.

D'autre part, dépositaire de la plus entière confiance domestique et
publique du chancelier et de toute sa famille, comme on le verra encore
bientôt en continuelle liaison avec le duc et la duchesse de Villeroy,
et par eux avec le duc de La Rocheguyon, qui n'était qu'un avec eux, en
confiance aussi avec le premier écuyer, avec du Mont, avec Bignon, lui
et sa femme dans toute celle de M\textsuperscript{lle} Choin, et ces
derniers de la cabale de Meudon, qui ne seraient pas même péris avec
elle, et qui y surnageaient, je ne pouvais désirer qu'aucune des deux
autres succombât, d'autant plus que les ménagements constants d'Harcourt
pour moi étaient tels qu'ils m'ôtaient tout lieu de le craindre, et me
donnaient tout celui d'entrer plus avant avec lui toutes les fois que je
l'aurais voulu.

Je n'oserais dire que l'estime de tous ces principaux personnages,
jointe à l'amitié que plusieurs d'eux avaient pour moi, leur donnait,
Harcourt excepté, une liberté, une aisance, une confiance entière à me
parler de tout ce qui se passait de plus secret et de plus important,
non quelquefois sans qu'il leur échappât quelque chose sur ceux de mes
amis qui leur étaient opposés et sans que les tireurs en fussent en
peine. J'en savais beaucoup plus par le chancelier et par le maréchal de
Boufflers que par les ducs de Chevreuse et de Beauvilliers, peu
vigilants, souvent ignorants.

À ces connaissances sérieuses, j'ajoutais celles d'un intérieur intime
de cour par les femmes les plus instruites, et les plus admises en tout
avec M\textsuperscript{me} la duchesse de Bourgogne, qui, vieilles et
jeunes en divers genres, voyaient beaucoup de choses par elles-mêmes, et
savaient tout de la princesse, de sorte que jour à jour j'étais informé
du fond de cette curieuse sphère\,; et fort souvent par les mêmes voies,
de beaucoup de choses secrètes du sanctuaire de M\textsuperscript{me} de
Maintenon. La bourre même en était amusante, et parmi cette bourre
rarement n'y avait pas quelque chose d'important, et toujours
d'instructif pour quelqu'un fort au fait de toutes choses.

J'y étais mis encore quelquefois d'un autre intérieur, non moins
sanctuaire, par des valets très principaux, et qui, à toute heure dans
les cabinets du roi, n'y avaient pas les yeux ni les oreilles fermés.

Je me suis donc trouvé toujours instruit journellement de toutes choses
par des canaux purs, directs et certains, et de toutes choses grandes et
petites. Ma curiosité, indépendamment d'autres raisons, y trouvait fort
son compte\,; et il faut avouer que, personnage ou nul, ce n'est que de
cette sorte de nourriture que l'on vit dans les cours, sans laquelle on
n'y fait que languir.

Mon attention continuelle était à un secret extrême des uns aux autres
sur tout ce qui pouvait les intéresser\,; à un discernement scrupuleux
des choses qui pouvaient avoir des suites, et pour cela même à les
taire, quoique apparemment indifférentes\,; et sur celles qui l'étaient
en effet, à les conter pour payer et nourrir la confiance, ce qui
faisait l'entière sûreté de mon commerce avec tous et l'agrément de ce
commerce, où je rendais souvent autant et plus que j'en recueillais,
sans qu'il me soit arrivé d'avoir trouvé jamais refroidissement,
défiance, moins d'ouverture même dans pas un\,; encore qu'ils sussent
très bien tous que j'étais dans le même intrinsèque avec plusieurs de la
cabale opposée à la leur, et que les uns et les autres me parlassent de
cette intimité très librement, quand l'occasion s'en présentait, et
toujours avec mesure sur ces personnes, par égard pour moi, hors
quelques occasions rares de vivacités échappées auxquelles je fermais
les yeux.

\hypertarget{chapitre-xviii.}{%
\chapter{CHAPITRE XVIII.}\label{chapitre-xviii.}}

1709

~

{\textsc{Affaire d'Espagne de M. le duc d'Orléans.}} {\textsc{- Flotte
arrêté en Espagne et Renaut aussi.}} {\textsc{- Déchaînement contre M.
le duc d'Orléans.}} {\textsc{- Villaroël et Manriquez, lieutenants
généraux, arrêtés en Espagne.}} {\textsc{- Terrible orage contre M. le
duc d'Orléans, à qui on veut faire juridiquement le procès.}} {\textsc{-
Le chancelier m'oblige à lui dire mon avis juridique sur le crime imputé
à M. le duc d'Orléans, en est frappé, et tout tombe là-dessus, desseins
et bruits incontinent après.}} {\textsc{- Triste état du duc d'Orléans
après l'avortement de l'orage.}}

~

Il faut maintenant retourner un peu en arrière, pour voir tout de suite
cette affaire de M. le duc d'Orléans sur l'Espagne, qui éclata en ce
temps-ci, et qui a été la source de tout ce qui a depuis accompagné sa
vie d'amertumes et de détresses, qui se sont de là répandues même sur
les temps les plus affranchis et les plus libres de sa vie, et dans
lesquels il a été revêtu seul de tout le pouvoir souverain.

Sans s'entendre ici sur le caractère avant le temps, il suffira de
remarquer que son oisiveté, continuellement trompée par des voyages de
Paris, amusée par des curiosités de chimie fort déplacées, et des
recherches de l'avenir qui l'étaient bien davantage, livrée à
M\textsuperscript{me} d'Argenton, sa maîtresse, à la débauche et à la
mauvaise compagnie avec un air de licence, de peu de compte de la cour,
et de beaucoup moins de M\textsuperscript{me} sa femme, lui avait fait
grand tort dans l'esprit du monde et surtout dans celui du roi, lorsque
la nécessité des affaires le força de l'envoyer relever le duc de
Vendôme en Italie, et après le malheur de Turin, arrivé de tous points
malgré sa prévoyance, et tout ce qu'il fit pour faire entendre raison à
Marsin, et, depuis sa blessure, pour rentrer avec l'armée en Italie,
porta le roi à l'en consoler par le commandement des armées en Espagne.

Le roi lui avait témoigné qu'il désirait qu'il vécût bien avec
M\textsuperscript{me} des Ursins, qu'il ne se mêlât que des choses qui
concernaient la guerre, et qu'il n'entrât en rien de toutes les autres
affaires. M. le duc d'Orléans avait exactement suivi cet ordre.
M\textsuperscript{me} des Ursins n'avait cherché qu'à lui plaire. Elle
avait affecté de m'en écrire de ces sortes de louanges que l'on compte
bien qui reviendront. Je savais les ordres du roi sur elle, j'étais ami
des deux au dernier point, je désirais leur union qui faisait leur bien
réciproque et plus encore celui de M. le duc d'Orléans, qui y était plus
attaché, et j'avais eu soin de lui faire passer tout ce qui pouvait y
contribuer. J'avais cimenté ces dispositions pendant le court séjour de
M. le duc d'Orléans ici entre ses deux voyages d'Espagne, et je n'avais
rien oublié pour lui en faire sentir toute l'importance pour lui, par
l'unité de M\textsuperscript{me} des Ursins et de la reine d'Espagne, et
de la même ici avec M\textsuperscript{me} de Maintenon.

Tout alla bien entre eux jusqu'à son retour en Espagne, que, mal content
du peu de dispositions faites pour la campagne, malgré les soins qu'il
en avait pris avant son retour et les promesses qui lui avaient été
faites, et outré de ce que les mêmes manquements lui avaient fait perdre
d'occasions glorieuses l'autre campagne, qu'il prévoyait lui devoir être
aussi nuisibles pour celle qu'il allait commencer, ce mot, d'autant plus
cruel qu'il était incomparable, lui échappa en plein souper, comme je
l'ai remarqué (t. VI, p.~301), qui lui rendit M\textsuperscript{me} des
Ursins et M\textsuperscript{me} de Maintenon sourdement
irréconciliables. L'intelligence sembla continuer entre lui et
M\textsuperscript{me} des Ursins, nonobstant les altercations fréquentes
auxquelles les vivres et les autres fournitures pour l'armée donnèrent
lieu. Il ne laissa pas de sentir, à plusieurs petites choses, qu'on lui
cherchait noise, et qu'il était bon d'y prendre garde de près\,; et je
l'en avertis fortement sur ce bruit répandu ici de son amour prétendu
pour la reine d'Espagne avec des circonstances ajustées, sur lequel il
me rassura, comme je l'ai dit ailleurs, dont il ne fut pas la moindre
mention en Espagne, et qui, en effet, n'avait pas eu le moindre
fondement.

Dès la fin de sa première campagne en ce pays-là, et plus encore dans
son séjour après à Madrid, il sentit les fautes que l'ambition et
l'avarice faisaient commettre à la princesse des Ursins. Il n'eut pas
peine à démêler qu'elle était extrêmement crainte et haïe. Peut-être la
simple curiosité le porta-t-elle à écouter quelques mécontents
principaux\,; les princes sur tous les hommes veulent être aimés. Tout
retentit en Espagne, et d'Espagne ici, de ses louanges en toutes façons,
travail, détails, capacité, valeur, courage d'esprit, industrie,
ressources, affabilité, douceur\,; et je ne sais s'il ne prit point les
hommages des désirs rendus au rang et au pouvoir pour les hommages des
coeurs, ni jusqu'à quel point il en fut flatté et séduit. Après s'être
aperçu par des effets, quoique assez peu perceptibles, mais qu'il ne put
méconnaître, de l'imprudence de ce bon mot fatal, il n'en fut que plus
curieux, pendant sa seconde campagne et son séjour après Madrid, sur les
déportements de la princesse des Ursins\,; il n'en fut aussi que d'un
accès plus ouvert aux plaintes des mécontents, sans toutefois en faire
d'usage.

Stanhope, cousin de celui qui, de mon temps, fut ambassadeur en Espagne,
et depuis secrétaire d'État en Angleterre, commandait les Anglais, et
était la seconde personne de l'armée du comte de Staremberg, opposée à
celle que M. le duc d'Orléans commandait. Ce général anglais avait été
fort débauché. Il avait passé du temps à Paris. Alors assez jeune, il y
avait connu l'abbé Dubois, comme on dit, entre la poire et le fromage,
et de là M. le duc d'Orléans, qui avait fait avec lui tout un hiver et
un été force parties, toutes des plus libres. Le prince et le général,
devenus personnages en Espagne, vis-à-vis l'un de l'autre, se souvinrent
du bon temps, se le témoignèrent autant qu'ils le purent réciproquement,
et saisirent également, pour s'écrire par des trompettes, des occasions
de passeports, d'échanges de prisonniers et, autres semblables.

Les mécontents du gouvernement et de M\textsuperscript{me} des Ursins se
rassemblèrent autour de lui. Il en fit si peu de mystère que, de retour
de l'armée à Madrid, il parla pour plusieurs, en remit quelques-uns en
grâce, obtint pour d'autres ce qu'ils désiraient, et répondit aux
plaintes que lui en fit M\textsuperscript{me} des Ursins, en présence du
roi et de la reine, qu'il avait cru les servir en se conduisant de la
sorte, pour jeter à ces gens-là un milieu entre Madrid et Barcelone où
ils se seraient précipités, s'il n'avaient eu recours à lui, et s'il ne
les eût retenus par ses paroles et son secours. Pas un des trois n'eut
le mot à répondre, et sur ce qu'il offrit de n'en plus écouter, ils le
prièrent de continuer à le faire. Ils le pressèrent de hâter son retour
en Espagne, et se séparèrent à ce qu'il parut, fort contents.

Il laissa, dans ce dessein d'une fort courte absence, tous ses équipages
avec un nommé Renaut, que le duc de Noailles lui avait donné, et qui lui
servait souvent de secrétaire, pour presser de sa part, en son absence,
les préparatifs convenus pour la campagne suivante, lui en rendre compte
et des choses dont il désirerait d'être instruit. Le comte de Châtillon,
premier gentilhomme de sa chambre, seigneur fort pauvreteux, et père du
duc de Châtillon, qui, sans y penser, a rapidement fait la plus grande
fortune, demeura aussi en Espagne, sous prétexte de s'épargner six cents
lieues en si peu de temps, en effet pour courtiser M\textsuperscript{me}
des Ursins et tâcher d'attraper une grandesse. {[}Renaut{]} demeura
aussi. Ce Renaut, que je n'ai jamais vu, était, par ouï dire, un drôle
d'esprit et d'entreprise, actif, hardi, intelligent. On verra bientôt
que le jugement n'était pas de la partie.

Vers la fin de l'hiver, le roi demanda à son neveu ce que c'était que
Renaut, pourquoi il ne l'avait pas ramené\,; et ajouta qu'il ferait bien
de le rappeler, parce que c'était un intrigant, qui se fourrait
indiscrètement parmi les ennemis de M\textsuperscript{me} des Ursins, à
qui cela faisait de la peine. M. le duc d'Orléans répondit aux
questions, et dit qu'il allait mander à Renaut de revenir, et il le lui
ordonna en effet. Renaut répondit qu'il s'allait préparer au retour, et
M. le duc d'Orléans n'y songea pas davantage.

Quelque temps après, le roi lui demanda s'il avait bien envie de
retourner en Espagne. Il répondit d'une manière qui, témoignant son
désir de servir, ne marquait aucun empressement\,; et ne fit nulle
attention qu'il pût y avoir rien d'important caché sous cette question.

Il me le conta. Je blâmai la mollesse de sa réponse. Je lui représentai
combien il lui importait que la paix seule mît fin à ses campagnes\,;
que, cessant de servir pendant la guerre, il se trouverait au niveau des
autres généraux d'armée remerciés, et tout ce qu'il avait fait oublié,
sans qu'il lui restât d'autre considération que celle de sa naissance,
au lieu qu'achevant cette guerre et continuant d'y bien faire, il était
difficile qu'il ne demeurât pas de quelque chose à la paix. D'ailleurs
(car on comptait encore alors que Monseigneur et Mgr le duc de Bourgogne
serviraient) nul autre pays ne lui convenait comme l'Espagne, où,
éloigné de concurrence d'envie et de courriers du cabinet, il était en
liberté. De servir en Flandre sous Monseigneur, ou en Allemagne sous Mgr
le duc de Bourgogne, ce n'était plus commander une armée\,; en Flandre,
c'était figurer péniblement dans une cour qui aurait ses épines, risquer
sa réputation si la politique l'emportait, sinon s'exposer à des
contradictions fâcheuses dont le poids de l'envie et des mauvais offices
retomberait sur lui, selon que les événements seraient bons ou mauvais,
lorsqu'ils auraient paru les suites de son opinion\,; en Allemagne,
c'était un voyage et non une campagne où le duc d'Harcourt et le duc
d'Hanovre ne chercheraient qu'à subsister. Ne servir plus, outre ce qui
a été d'abord remarqué, ce serait, en cas de malheurs et de discussions,
s'exposer à être saisi comme une ressource pour aller réparer des fautes
peut-être peu réparables, et peut-être également dangereuses à réparer
pour la politique, et à ne pas réparer pour I'État et sa propre
réputation, se perdre aisément en acceptant, et plus sûrement encore en
refusant. Ces raisons parurent déterminer M. le duc d'Orléans à un désir
plus effectif de retourner en Espagne.

À peu de jours de là, le roi lui demanda comment il se croyait être avec
la princesse des Ursins\,; et parce qu'il lui répondit qu'il avait lieu
de se persuader d'être bien avec elle, parce qu'il n'avait rien fait
pour y être mal, le roi lui dit qu'elle craignait pourtant fort son
retour en Espagne, qu'elle demandait instamment qu'on ne l'y renvoyât
pas\,; qu'elle se plaignait qu'encore qu'elle eût tout fait pour lui
plaire, il s'était lié à tous ses ennemis\,; que ce secrétaire Renaut
entretenait avec eux un commerce étroit et secret qui l'avait obligée à
demander son rappel, dans la crainte qu'il ne lui fit de la peine par le
nom de son maître.

M. d'Orléans répondit qu'il était infiniment surpris de ces plaintes de
M\textsuperscript{me} des Ursins\,; qu'il avait toujours eu grand soin,
comme Sa Majesté le lui avait recommandé, de ne se mêler d'aucune
affaire que de celles de la guerre\,; qu'il n'avait rien oublié pour
ôter à M\textsuperscript{me} des Ursins tout ombrage qu'il voulût entrer
en rien, et pour lui témoigner qu'il voulait vivre, en union et en
amitié avec elle, comme il y avait en effet vécu. Il conta au roi
l'éclaircissement qu'il avait eu avec elle, et que j'ai rapporté
ci-dessus, dont elle était demeurée très satisfaite, ainsi que Leurs
Majestés Catholiques qui y étaient présentes, et qui tous trois
l'avaient prié de continuer à écouter et ramener les mécontents, et â
presser son retour en Espagne dont il était lors près de partir.

Il ajouta qu'il était vrai qu'il savait beaucoup de malversations et de
dangereux manéges de la princesse des Ursins, qui ne pouvaient tourner
qu'à la ruine de Leurs Majestés Catholiques et de leur couronne\,; que
M\textsuperscript{me} des Ursins, qui s'en doutait peut-être, craignait
en lui ces connaissances, et pour cela ne voulait pas qu'il retournât\,;
mais qu'il avait si bien retenu ce que Sa Majesté lui avait prescrit,
qu'il osait la prendre elle-même à témoin que c'était là la première
fois qu'il prenait la liberté de lui en parler\,; que, quelque nécessité
qu'il vit à lui en rendre compte, il l'eût toujours laissé dans le
silence, s'il ne l'eût lui-même obligé à le rompre là-dessus en lui
parlant de l'éloignement de M\textsuperscript{me} des Ursins pour lui,
également ignoré et non mérité par lui.

Le roi pensa un moment, puis lui dit que, les choses en cet état, il
croyait plus à propos qu'il s'abstînt de le renvoyer en Espagne\,; que
les affaires se trouvaient en une crise où on doutait à qui elle
demeurerait\,; que, si son petit-fils en sortait, ce n'était pas la
peine d'entrer en rien sur l'administration de M\textsuperscript{me} des
Ursins\,; que, s'il conservait cette couronne, il serait à propos alors
de parler à fond de cette administration, et qu'il serait en ce temps-là
bien aise d'en consulter son neveu.

M. le duc d'Orléans s'en tint là, et me le conta, médiocrement fâché à
ce qu'il me parut, et moi plus que lui par les raisons qui ont été
rapportées. Il me dit que cette intrigue s'était toute conduite de Le
des Ursins à M\textsuperscript{me} de Maintenon immédiatement, et
c'était du roi qu'il l'avait appris, c'est-à-dire que
M\textsuperscript{me} des Ursins s'était adressée à
M\textsuperscript{me} de Maintenon là-dessus sans aucun autre canal
intermédiaire, aussi n'en avait-elle pas besoin, surtout sur une
vengeance commune.

Peu après il devint public que M. le duc d'Orléans ne retournerait point
en Espagne, parce que, ne s'y agissant guère que d'en ramener les
troupes françaises, cet emploi ne lui convenait pas. Alors le roi dit à
M. d'Orléans d'en faire revenir ses équipages, et lui ajouta à l'oreille
d'y envoyer les chercher par quelqu'un de sens, qui, dans la conjoncture
présente, pût être le porteur de ses protestations à tout événement, si
par un traité Philippe V quittait le trône d'Espagne, et son neveu
conserver ses droits en faisant doucement recevoir ses protestations. Au
moins fut-ce ce que m'en dit alors M. le duc d'Orléans, et ce que peu de
gens voulurent croire dans la suite, car il faut parler avec exactitude.

Ce prince choisit pour cet emploi un nommé Flotte, que je n'ai jamais
vu, non plus que Renaut, parce que je n'ai jamais eu d'habitude dans sa
maison, et n'y ai connu personne. C'était un homme de beaucoup d'esprit,
d'adresse, de hardiesse, à ce que j'ai ouï dire à M. de Lauzun qui en
faisait cas, qui avait été à lui au temps de ses plus importantes
affaires avec Mademoiselle, qui s'en était beaucoup mêlé, à laquelle il
était passé ensuite, mais comme l'instrument principal de tout entre
eux, dans les temps les plus fâcheux, et dans ceux de la prison de M. de
Lauzun, jusqu'à son retour et ses brouilleries depuis avec Mademoiselle,
à la mort de laquelle il était entré chez Monsieur, et à la mort de
Monsieur il était demeuré à M. le duc d'Orléans, qui s'en était servi à
la guerre d'aide de camp de confiance en Italie et en Espagne.

Cet homme, nourri comme on voit dans l'intrigue, s'en alla droit à
Madrid. En chemin il reçut des nouvelles de Renaut, qui y était toujours
demeuré, qui lui donnait avis du jour de son départ et du lieu où il le
rencontrerait. Flotte ne le trouva point au rendez-vous. Il crut qu'il
avait différé son départ et qu'il le rencontrerait plus loin. Avançant
toujours sans le voir, il ne douta pas qu'il ne le trouvât encore à
Madrid, et qu'il l'y attendait. Il y arriva, y séjourna quelque temps, y
chercha Renaut inutilement. Il y vit quelques personnes, et même
quelques grands en commerce avec Renaut qui ne purent lui en donner de
nouvelles. Je n'ai point su ce que Flotte en pensa\,; mais il séjourna
assez à Madrid, puis s'en alla à l'armée, qui était encore répandue dans
ses quartiers d'hiver.

Il y salua le maréchal de Besons, pour lequel il n'avait point de
lettres, et demeura trois semaines à rôder de quartier en quartier, sans
rien répondre de précis ni de juste à Besons qui ne voyait point de
fondement à ce long séjour dont il était surpris, et qui le pressait de
retourner en France. Enfin Flotte fut prendre congé du maréchal, et lui
demander une escorte pour s'en aller de compagnie avec un commissaire
des vivres qui voulait aussi repasser les Pyrénées. Lui et ce
commissaire partirent un matin de chez Besons, tous deux dans une chaise
à deux, avec vingt dragons d'escorte.

Comme ils s'éloignaient du quartier du maréchal, le commissaire vit de
loin deux gros escadrons qui s'approchaient d'eux peu à peu, qu'il
reconnut être de la cavalerie du roi d'Espagne. Le soupçon qu'il en prit
lui fit bientôt passer la tête par la portière, d'où il vit que ces
escadrons les suivaient\,; il le dit à Flotte qui d'abord n'en prit
point d'ombrage, mais qui à demi lieue de là commença aussi à s'en
inquiéter. Ils raisonnèrent ensemble dans la chaise et firent encore
deux lieues, au bout desquelles ils remercièrent leur escorte comme n'en
ayant plus besoin, pour voir alors ce que deviendraient ces deux
escadrons. Les dragons, qui étaient Français, insistèrent un peu à les
suivre par civilité, puis voulurent les quitter\,; mais aussitôt que les
escadrons s'en aperçurent, ils vinrent au trot et empêchèrent les
dragons de se retirer. Ce bruit si proche obligea le commissaire à
regarder ce que ce pouvait être, et voyant alors qu'il y a voit dessein
sur eux, il le dit à Flotte, et lui demanda s'il n'avait point de papier
sur lui. Flotte fit bonne contenance\,; mais un moment après, remarquant
quelques cavaliers détachés qui les côtoyaient, il pria le commissaire
de se charger d'un porte-lettres qu'il lui fit doucement couler. Il n'en
était plus temps, un des cavaliers le remarqua. Il arrêta la chaise que
les escadrons enveloppèrent en même temps. Les dragons là-dessus firent
mine de la vouloir défendre\,; mais celui qui commandait les escadrons
s'approcha du lieutenant de dragons, lui dit civilement qu'il avait ses
ordres, que l'inégalité du nombre le devait retenir puisqu'il
s'opposerait vainement à ce qu'il devait faire, et qu'enfin il serait
fâché d'être obligé de les faire désarmer. À cela il n'y avait et n'y
eut point de réplique. Les dragons se retirèrent. Un exempt des gardes
du corps du roi d'Espagne, jusque-là mêlé parmi les cavaliers s'avança à
la chaise, se fit connaître par un ordre par écrit qu'il montra, fit
mettre pied à terre à Flotte et au commissaire, fouilla entièrement la
chaise puis Flotte partout, et, averti qu'il fut, il commanda au
commissaire de lui remettre ce que Flotte lui avait fait couler, et
l'avertit de ne s'exposer pas au mauvais traitement qui l'attendait s'il
lui donnait la peine de le fouiller. Le commissaire ne se le fit pas
dire deux fois et donna le porte-lettres, après quoi l'exempt lui dit
qu'il était libre, et lui permit de remonter en chaise et de continuer
son voyage. En même temps Flotte fut mis sur un cheval, environné
d'officiers, qui s'assurèrent bien de sa personne, et conduit chez le
marquis d'Aguilar au même quartier d'où il venait de partir\footnote{Voy.,
  sur cette affaire, les notes placées à la fin du volume.}.

Le marquis d'Aguilar, grand d'Espagne, fils du vieux marquis de
Frigilliane, est le même qui vint à Paris persuader le malheureux siège
de Barcelone sur lequel je me suis étendu (t. V, p.~73). Il commandait
alors en chef les troupes d'Espagne sous le maréchal de Besons. Il était
lors vendu à M\textsuperscript{me} des Ursins, et il se retrouvera
encore dans la suite. Dès qu'il fut averti de la capture, il alla
trouver Besons, à qui il dit tout ce qu'il put de plus soumis pour
excuser ce qu'il venait de faire exécuter sans sa permission ni sa
participation, dans son armée, fondé sur un ordre par écrit, de la main
propre du roi d'Espagne, qu'il lui fit lire. Besons, tout irrité qu'il
était, l'écouta sans l'interrompre, et lut l'ordre du roi d'Espagne
positif pour cette exécution, et pour ne lui en rien communiquer. En le
rendant au marquis d'Aguilar, il lui dit qu'il fallait que Flotte, qu'il
avait connu et cru un garçon fort sage, fût bien coupable, puisque,
appartenant à M. le duc d'Orléans, le roi d'Espagne se portait à cette
extrémité.

Il congédia Aguilar étonné au dernier point\,; mais sans perdre le
jugement, il manda l'aventure à M. le duc d'Orléans, l'avertit qu'il
n'en rendrait compte au roi que par l'ordinaire, qui ne pourrait arriver
que six jours après un courrier qu'il venait {[}de{]} dépêcher, le fit
rattraper avec ce billet, avec ordre de le rendre à M. le duc d'Orléans
à l'insu de qui que ce fût, de manière que ce prince en fut averti six
jours entiers avant le roi et avant personne. Il tint le cas si secret
qu'il m'en fit un à moi-même, et cependant je ne sais quel usage il fit
de l'avis reçu si fort à temps. Il vint au roi par l'ordinaire qui
arriva le 12 juillet de l'armée et de Madrid. Le roi le dit à son neveu,
qui fit le surpris et qui avait eu le loisir de se préparer. Il répondit
au roi qu'il était étrange qu'on arrêtât ainsi un de ses gens\,;
qu'ayant l'honneur de lui appartenir de si près, c'était à Sa Majesté à
en demander raison, et à lui à l'attendre de sa justice et de sa
protection. Le roi repartit que l'injure le regardait plus en effet
lui-même, que son neveu, et qu'il allait donner ordre à Torcy d'écrire
là-dessus comme il fallait en Espagne.

Il n'est pas difficile de comprendre qu'un tel éclat fit grand bruit en
Espagne et en France\,; mais quel qu'il fût d'abord, ce ne fut rien en
comparaison des suites. J'en parlai alors à M. le duc d'Orléans qui me
dit ce qui a été raconté de ses protestations, et qui me parut tout
attendre de l'effet des lettres du roi. Je lui demandai, à cette
occasion, des nouvelles de Renaut\,; et j'ai appris qu'il n'en avait eu
aucune depuis la réponse qu'il lui avait faite à l'ordre de revenir\,;
que Flotte ne l'avoir trouvé ni sur la route ni à Madrid, et qu'on ne
savait ce qu'il était devenu. Tout cela me fit entrer en soupçon qu'il y
avait du plus en cette affaire, que Renaut avait été arrêté, et que ces
choses ne s'étaient point exécutées sans la participation du roi. Je dis
à M. {[}le duc{]} d'Orléans que cela seul de n'avoir point eu de
nouvelles de Renaut depuis le départ de Flotte lui aurait dû donner de
l'inquiétude de l'un et des précautions pour l'autre. Il en convint,
puis me dit que, Flotte n'étant allé que sur ce que le roi lui avait dit
de ses protestations, il n'avait pu prendre de défiance\,; qu'à la façon
dont le roi lui avoir parlé il ne pouvait croire qu'il y fût entré, mais
un coup de hardiesse et de curiosité de M\textsuperscript{me} des
Ursins, qui donnait en cela un second tome des dépêches de l'abbé
d'Estrées, pour découvrir à quels ennemis elle avait affaire, et cacher
la sienne sous le prétexte l'une affaire d'État, dont les moindres
soupçons excusent tous les éclats. Ce raisonnement, que la connaissance
des artifices et de la hardiesse de la princesse des Ursins m'avait déjà
fourni en moi-même, me persuada encore plus de la bouche de M. le duc
d'Orléans, et je crus qu'il fallait suspendre tout raisonnement jusqu'à
l'arrivée de la réponse d'Espagne.

Cependant, on ne l'attendit pas pour exciter le déchaînement contre M.
le duc d'Orléans. La cabale de Meudon avait manqué à demi son coup sur
Mgr le duc de Bourgogne, mais elle l'avait détruit auprès de
Monseigneur. L'occasion était trop belle contre le seul du sang royal
qui pût figurer pour n'en pas profiter dans toute son étendue, et se
faire place nette. Cette politique se trouvait aiguisée de la haine
personnelle de M\textsuperscript{me} la Duchesse, fondée sur les
distinctions de rang auquel les princes du sang ne pouvaient
s'accoutumer, plus vivement encore sur de ces choses de galanterie qui
pour avoir vieilli ne se pardonnent point, enfin sur la jalousie du
commandement des armées, quoiqu'elle fût fort éloignée d'aimer M. le
Duc, lequel ne se contraignit point de dire et de faire du pis qu'il
put. Il se publia que M. le duc d'Orléans avait essayé de se faire un
parti qui le portât sur le trône d'Espagne en chassant Philippe V, sous
prétexte de son incapacité, de la domination de M\textsuperscript{me}
des Ursins, de l'abandon de la France retirant ses troupes\,; qu'il
avait traité avec Stanhope pour être protégé par l'archiduc, dans l'idée
qu'il importait peu à l'Angleterre et à la Hollande qui régnât en
Espagne, pourvu que l'archiduc demeurât maître de tout ce qui était hors
de son continent, et que celui qui aurait la seule Espagne fût à eux,
placé de leur main, dans leur dépendance, et de quelque naissance qu'il
fût, ennemi, ou du moins séparé de la France. Voilà ce qui eut le plus
de cours.

Il y en avait qui allaient plus loin. Ceux-là ne parlaient de rien moins
que de la condition de faire casser à Rome le mariage de
M\textsuperscript{me} la duchesse d'Orléans comme indigne et fait par
force, et conséquemment déclarer ses enfants bâtards, à la sollicitation
de l'empereur\,; d'épouser la reine, soeur de l'impératrice et veuve de
Charles II, qui avait encore alors des trésors, monter avec elle sur le
trône, et, sûr qu'elle n'aurait point d'enfants, épouser après elle la
d'Argentan\,; enfin, pour abréger les formes longues et difficiles,
empoisonner M\textsuperscript{me} la duchesse d'Orléans. Grâce aux
alambics, au laboratoire, aux amusements de physique et de chimie, et à
la gueule ferrée et soutenue des imposteurs, M. le duc d'Orléans ne
laissa pas d'être heureux que M\textsuperscript{me} sa femme, qui était
grosse et qui eut en ce même temps une très violente colique qui
redoubla ces horreurs, s'en tirât heureusement, et bientôt après
accouchât de même, dont le rétablissement ne servit pas peu à les faire
tomber.

Cependant la réponse d'Espagne n'arrivait point. La plus saine partie de
la cour commençait à se hérisser, M. le duc d'Orléans l'attendait
toujours. Le roi, et plus encore Monseigneur, le traitaient avec un
froid qui le mettait fort mal à l'aise\,; à cet exemple, la plupart de
la cour se retirait ouvertement de lui.

J'étais alors, comme je l'ai remarqué, en espèce de disgrâce\,: je
n'allais plus à Marly, et cette situation désagréable était visible. Ma
liaison si étroite avec M. le duc d'Orléans inquiéta mes amis qui me
pressèrent de m'en écarter un peu. L'expérience que j'avais de ce que
savaient faire ceux qui me haïssaient ou me craignaient, surtout la
cabale de Meudon qui était celle de Vendôme, en particulier M. le Duc et
M\textsuperscript{me} la Duchesse, me fit bien faire réflexion à
moi-même que, dans l'état où je me trouvais avec le roi, cette liaison
si grande leur donnait beau jeu. Mais, tout considéré, je crus qu'à la
cour comme à la guerre il fallait de l'honneur et du courage, et savoir
avec discernement affronter les périls\,; je ne {[}crus{]} donc pas en
devoir témoigner la moindre crainte, ni marquer la moindre différence
dans ma liaison ancienne et si intime avec M. le duc d'Orléans au temps
de son besoin, par l'étrange abandon qu'il éprouvait.

Enfin les réponses d'Espagne venues depuis assez longtemps sans qu'on en
eût parlé, ce prince m'avoua que plusieurs gens considérables, grands
d'Espagne et autres, lui avaient persuadé qu'il n'était pas possible que
le roi d'Espagne s'y pût soutenir, et de là lui avaient proposé de hâter
sa chute et de se mettre en sa place\,; qu'il avait rejeté cette
proposition avec l'indignation qu'elle méritait, mais qu'il était vrai
qu'il s'était laissé aller à celle de s'y laisser porter si Philippe V
tombait de lui-même sans aucune espérance de retour, parce qu'en ce cas
il ne lui causerait aucun tort, et ferait un bien au roi et à la France
de conserver l'Espagne dans sa maison, qui ne lui serait pas moins
avantageux qu'à lui-même\,; que cela se faisant sans la participation du
roi, il ne se trouverait point embarrassé de renoncer par la paix, ni
les ennemis en peine d'un prince porté sur le trône par le pays même,
séparément de la France, avec qui l'apparence d'union et de liaison ne
pourrait pas être telle qu'avec Philippe V.

Cet aveu ne me donna pas opinion du projet, ni désir de presser pour en
savoir davantage, supposé qu'il y eût dit plus. Je me rabattis dans
cette crainte à remontrer à ce prince l'absurdité d'un projet si vide de
sens que ce serait perdre le temps que de s'amuser à raconter ici tout
ce que je lui en dis, et démontrai bien aisément. Je lui conseillai
ensuite de faire l'impossible pour pénétrer jusqu'où le roi en savait,
pour éviter de lui donner soupçon de plus en matières si jalouses, et de
suites, au mieux qu'elles se tournassent, si fâcheuses en éloignement et
en défiances irrémédiables, lui avouer ce qu'il lui apprendrait, ou, si
le roi était informé, lui raconter ce qu'il venait de me dire, surtout
lui en faisant bien remarquer les bornes et l'intention\,; lui demander
pardon de ne lui en avoir pas fait la confidence et reçu ses ordres\,;
s'en excuser sur ce qu'il n'y avait rien de mauvais dans le projet
contre son service, ni contre le roi d'Espagne, et sur ce que, l'ayant
su, la conscience de Sa Majesté aurait pu être embarrassée sur les
renonciations à faire à la paix, si alors elles lui étoient demandées.
J'ajoutai qu'avec tout cela je ne voyais point une plus mauvaise
affaire, plus triste, ni en même temps plus folle, ni plus impossible,
ni un plus grand malheur pour lui que de s'y être laissé entraîner, dont
toutefois, â force d'esprit, de conduite, de naissance, il fallait qu'il
tâchât de sortir au moins mal qu'il se pourrait, et qu'il ne
s'abandonnât pas soi-même dans le triste état d'abandon général et de
clameurs les plus cruelles où déjà il se trouvait réduit. Il goûta fort
mon conseil, convint à demi de la faute et de la folie, m'avoua qu'il
avait laissé Renaut en Espagne pour la suivre, que Flotte devait aussi
s'y concerter avec lui.

M\textsuperscript{me} des Ursins avait trop d'espions de tous les
genres, elle était trop occupée de sa haine contre M. le duc d'Orléans,
elle avait conçu trop de défiance de la protection qu'il avait donnée
aux mécontents\,; elle avait trop de soupçons de la conduite de Renaut,
laissé en Espagne depuis qu'elle avait procuré qu'il en fût rappelé\,;
enfin elle y fut trop confirmée par l'arrivée de Flotte, sous un
prétexte aussi frivole que celui de venir chercher des équipages qui ne
manquaient pas de gens pour les ramener\,; {[}elle avait{]} un trop vif
intérêt à pénétrer et à faire des affaires à M. le duc d'Orléans pour
n'être pas instruite.

Renaut se conduisit, à ce que j'ai ouï dire depuis, avec la dernière
imprudence. Il ne ménagea ni ses allées et venues, ni ses commerces très
justement suspects à M\textsuperscript{me} des Ursins, parce qu'il
n'était lié qu'avec ses ennemis. La tête de cet homme se tourna\,; il ne
put porter le poids d'une confiance si importante, de l'entremise de
choses si hautes\,; il se crut l'arbitre des récompenses de tout ce qui
entrerait dans le parti, et jusqu'à ses discours le trahirent, et le
firent arrêter secrètement un peu avant l'arrivée de Flotte, qui moins
indiscret, niais marchant à tâtons sans Renaut, donna dans des piéges
qui le perdirent. L'intervalle de ce rappel en tout, puis en partie des
troupes françaises, leur parut une conjoncture d'ébranlement à en
profiter. Ceux qui, en Espagne, avaient séduit M. le duc d'Orléans de
l'extravagance de ce projet impossible saisirent la même conjoncture
pour grossir le parti, et tous avec si peu de précautions, que leur
conduite, aussi insensée que leur projet même, le fit aisément
découvrir, et causa tout cet affreux scandale.

Tandis que j'arraisonnais M. le duc d'Orléans comme je viens de
l'expliquer, et qu'il se préparait à en faire usage (et que parmi ces
conversations je n'ai jamais bien démêlé jusqu'où l'affaire en était,
moins encore jusqu'où le roi en savait ni depuis), le roi consultait
là-dessus et sa famille et son conseil. Il savait le projet dès lors
qu'il ordonna à son neveu de faire revenir Renaut d'Espagne. Par les
papiers qui lui furent trouvés en l'arrêtant, et depuis par ceux de
Flotte, il en apprit beaucoup plus, et peut-être davantage encore,
lorsque quinze jours après, le marquis de Villaroël, lieutenant général
dans les troupes d'Espagne, fut arrêté à Saragosse, et en même temps don
Boniface Manriquez, aussi lieutenant général, le fut à Madrid, et dans
une église, qui est un asile en Espagne qu'on ne viole qu'avec de
grandes mesures pour en tirer les plus grands criminels.

Ce fut un éclat si grand pour le pays, qu'il ne s'y pouvait rien
ajouter. C'était aussi ce que voulait la princesse des Ursins, pour
exciter les clameurs de toute l'Espagne nécessaire à révolter toute la
France, sous les secrets auspices de M\textsuperscript{me} de Maintenon.
L'une et l'autre sentaient bien le vide du fond du complot, et qu'il
avait besoin d'autant plus de vacarmes qu'il s'agissait de brusquer et
d'entraîner aux plus forts partis contre un petit-fils de France, neveu
du roi, oncle de la reine d'Espagne et de M\textsuperscript{me} la
duchesse de Bourgogne, qu'il était trop dangereux d'attaquer vainement.
Le succès passa leur espérance.

Jamais clameurs si universelles, jamais d'un si grand fracas, jamais
abandon semblable à celui où M. le duc d'Orléans se trouva, et pour une
folie\,; car s'il y eût eu du crime, à la fin on l'aurait su\,; il ne
fut pas ménagé à le tenir caché, et dès là, qui que ce soit n'en sut que
ce que j'ai raconté. J'en infère que le roi, que M\textsuperscript{me}
de Maintenon, que M\textsuperscript{me} des Ursins elle-même, n'en
surent pas davantage, elles qui poussèrent sans cesse au plus violent,
et qui par conséquent se trouvaient si intéressées aux preuves qu'il
était mérité, sans que d'aucune part il en ait été allégué ni {[}en
ait{]} transpiré plus que ce que je viens de raconter, ni lors ni en
aucun temps depuis.

Monseigneur se signala entre tous pour sévir au plus fort\,; on a vu
qu'il a toujours aimé le roi d'Espagne\,; tout ce qui l'environnait, à
deux ou trois près, était contraire à M. le duc d'Orléans, duquel ils
avaient éloigné Monseigneur de longue main. La cabale de Meudon, dont
j'ai montré les raisons, menait, ou se faisait redouter de tout ce qui
approchait d'un prince qu'elle gouvernait, dont l'intelligence était
nulle, à qui on persuadait les choses les plus éloignées de toute
apparence, et dont l'année suivante fournira un exemple qui peut être
dit prodigieux. M\textsuperscript{lle} Choin n'avait garde de ne pas
suivre M\textsuperscript{me} la Duchesse et ses deux amies si intimes,
M\textsuperscript{lle} de Lislebonne et M\textsuperscript{me} d'Espinoy,
en chose qui leur importait si fort, à la première de haine, aux deux
autres et à elle-même de politique, et de ne seconder pas encore une
fois M\textsuperscript{me} de Maintenon, avec laquelle elle était restée
unie depuis l'affaire de la disgrâce de Chamillart, qui, sans oser
rallier comme à l'égard de ce ministre, eut soin de se montrer assez aux
gens dont elle compta faire usage pour faire presque le même effet sur
eux, que plus à découvert elle avoir obtenu contre le ministre. Elle
n'oublia pas les ressorts intérieurs des cabinets du roi qu'elle avoir
si utilement su remuer contre Chamillart. M. du Maine y avait le même
intérêt qui l'avait si vivement, mais si cauteleusement mis en mouvement
en faveur du duc de Vendôme contre Mgr le duc de Bourgogne, et, en cette
occasion-ci, au lieu d'avoir à se cacher de M\textsuperscript{me} de
Maintenon, il en avait l'aveu et le désir. Toute leur peine fut de ne
pouvoir associer ce prince à leurs cris. Il demeura ferme à vouloir des
preuves et de l'évidence, à soutenir que, quand bien même il s'en
trouverait de telles, il fallait cacher, non pas manifester à leur honte
commune le crime du sang royal. Il est pourtant très certain que la
partie était faite pour le répandre, à tout le moins de le déshonorer
par une condamnation et par la prétendue clémence d'une commutation de
peine qui anéantit le duc d'Orléans pour jamais. Force gens y trouvaient
leur compte pour les futurs contingents, quelques-uns pour leur haine,
les deux dominatrices surtout deçà et delà des Pyrénées pour leur
vengeance.

L'affaire fut donc donnée en Espagne et en France comme le complot d'un
prince si prochain des deux couronnes, et propre oncle maternel de la
reine d'Espagne, qui, abusant du diplôme qui le rappelait à son rang de
succession à la monarchie d'Espagne, nonobstant le silence du testament
de Charles II à son égard, abusant du pouvoir du commandement des
armées, de la confiance dans les affaires, du traitement d'infant, se
servait de toutes ces choses pour imiter l'usurpation du prince d'Orange
sur son beau-père, chasser d'Espagne la famille régnante et en occuper
la place sur le trône.

Monseigneur, toujours si enseveli dans l'apathie la plus profonde, et
qui, à force d'art et de machines, en avait été tiré pour la première
fois de sa vie contre Chamillart, poussé par les mêmes, montra jusqu'à
de la furie, et n'insista à rien moins qu'à une instruction juridique et
criminelle. Voysin et Desmarets, trop attachés à M\textsuperscript{me}
de Maintenon, l'un de reconnaissance, l'autre de crainte, n'osaient pas
être d'un autre avis, que le premier appuyait avec chaleur. Torcy était
flottant et dans l'embarras. Pour le duc de Beauvilliers, il s'y
trouvait bien davantage. Le cri public l'étourdissait\,; les mœurs et la
conduite de M. {[}le duc{]} d'Orléans lui rendaient tout croyable, il ne
pouvait oublier sa tendresse de gouverneur pour le roi d'Espagne.
Toutefois, il ne voyait rien de clair\,; la considération de M. de
Chevreuse, qui aimait M. le duc d'Orléans par des rapports de science et
des conversations par lesquelles il espérait le convertir à Dieu,
l'arrêtaient. Ce prince n'avait jamais biaisé sur l'archevêque de
Cambrai\,; il avait toujours conservé des liaisons avec lui, et ce
prélat était le coeur et l'âme des deux ducs. Beauvilliers enfin
déférait à la délicatesse de Mgr le duc de Bourgogne. Le chancelier,
effrayé d'un scandale si monstrueux dans la famille royale, n'était pas
moins éloigné de M. le duc d'Orléans par sa, conduite et par ses moeurs.
Il était extrêmement bien avec Monseigneur, sans qu'il y parût par les
raisons que j'ai marquées\,; il ne voulait pas perdre un si précieux
avantage, lié d'ailleurs avec Harcourt, qui l'avait, comme on a vu,
réuni avec filme des Ursins\,; mais l'acharnement de son fils, qu'il
connaissait à fond, et dont il détestait tout, hors le soutien et la
fortune, le ramenait vers l'avis de Mgr le duc de Bourgogne. Tout cela
se préparait et se cuisait sous la cendre, dès le temps que le roi parla
à son neveu de ne plus retourner en Espagne, et d'en faire revenir
Renaut, qui tôt après fut arrêté. La capture si éclatante de Villaroël,
et surtout de Manriquez, donna un tel coup de fouet à cette terrible
affaire, qu'elle mit toute autre en silence, et agita violemment
jusqu'aux visages de tout le monde.

Dans ce tourbillon, M. le duc d'Orléans parla au roi longtemps, qui ne
l'écouta qu'en juge, quoiqu'il lui avouât alors le fait tel qu'il me
l'avait dit et que je l'ai raconté ici. Ce fait, tel qu'il le lui
exposa, était bien une idée extravagante, mais qui ne pouvait jamais
passer pour criminelle, et toutefois ce n'était pas ce qui revenait
d'Espagne, ni ce qui était soufflé d'ici. On y employa tout le manége et
toute l'application possible, pour soutenir le roi dans la persuasion
que l'aveu que lui avait fait M. le duc d'Orléans n'était qu'un tour
d'esprit d'un criminel qui se voit près d'être convaincu, et qui pour
échapper donne le change, mais un change dont la grossière ineptie
faisait seule la preuve de ce qui se trouverait, si, en l'arrêtant et le
livrant aux formes, on faisait disparaître tout ce qui le rendait trop
respectable et trop à craindre pour que, sans une démarche si
nécessaire, on pût espérer de faire dire la vérité, retenue par la
frayeur de sa naissance et de sa personne, mais dont toute considération
tomberait quand on le verrait abandonné et livré à l'état des criminels,
puisque, à travers l'éclat et la terreur qui le protégeaient encore,
cette humble vérité se rendait déjà si palpable et se faisait si bien
sentir telle, par M. d'Orléans même, qu'avec tout son esprit, il n'avait
dû imaginer qu'une folie pour l'obscurcir, et une folie destituée de
toute sorte d'apparence.

Contre tant de machines, d'artifices, de hardiesse, de haine et
d'ambition, M. le duc d'Orléans se trouvait seul à se défendre, sans
autre appui que les larmes méprisées d'une mère et les languissantes
bienséances d'une femme, la volonté impuissante du comte de Toulouse
qui, avec son froid naturel, aurait voulu le servir, et les discours
dangereux de l'autre beau-frère, qui protestait de ses désirs et y
mêlait de légers et d'inutiles conseils, qu'il fallait écouter sans
montrer de défiance.

Le roi, à tous moments en proie à tous les accès de ses cabinets, sans
repos chez M\textsuperscript{me} de Maintenon, persécuté sans cesse
d'Espagne, accablé de Monseigneur qui lui demandait continuellement
justice pour son fils, peu retenu par le sage avis de Mgr le duc de
Bourgogne, dont le poids était resté en Flandre, ni par
M\textsuperscript{me} la duchesse de Bourgogne, qui désirait de tout son
coeur délivrer son oncle, mais qui, timide de son naturel, tremblante
sous Monseigneur, et plus encore sous M\textsuperscript{me} de Maintenon
dont elle apercevait la volonté, n'osait lâcher que des demi paroles, le
roi, dis-je, ne sachant à quoi se résoudre, parlait au conseil d'État
qu'il trouvait encore partagé. À la fin, il se rendit à tant de clameurs
si intimes et si bien organisées, et ordonna au chancelier d'examiner
les formes requises pour procéder à un pareil jugement. Le chancelier
travailla trois ou quatre fois seul avec le roi, après que les autres
ministres étaient sortis du conseil. Comme il n'avait aucun département,
il ne travaillait jamais avec le roi, avec tout ce qui était répandu sur
cette affaire, qui seule faisait alors tout l'entretien, cette nouveauté
mit bientôt le doigt sur la lettre à la cour et à la ville.

J'allais presque tous les soirs causer avec le chancelier, dans son
cabinet, et cette affaire y avait été quelquefois traitée
superficiellement à cause de quelques tiers. Un soir que j'y allai de
meilleure heure, je le trouvai seul, qui, la tête baissée et ses deux
bras dans les fentes de sa robe, s'y promenait, et c'était sa façon
lorsqu'il était fort occupé de quelque chose. Il me parla des bruits qui
se renforçaient, puis, voulant venir doucement au fait, ajouta qu'on
allait jusqu'à parler d'un procès criminel, et me questionna, comme de
pure curiosité et comme par le hasard de la conversation, sur les formes
dont il me savait assez instruit, parce que c'est celle de pairie. Je
lui répondis ce que j'en savais, et lui en citai des exemples. Il se
concentra encore davantage, fit quelques tours de cabinet, et moi avec
lui, sans proférer tous deux une seule parole, lui regardant toujours à
terre, et moi l'examinant de tous mes yeux\,; puis tout à coup le
chancelier s'arrêta, et se tournant à moi comme se réveillant en
sursaut\,: «\,Mais vous, me dit-il, si cela arrivait, vous êtes pair de
France, ils seraient tous nécessairement ajournés et juges puisqu'il les
faudrait convoquer tous, vous le seriez aussi, vous êtes ami de M. le
duc d'Orléans, je le suppose coupable, comment feriez-vous pour vous
tirer de là\,? --- Monsieur, lui dis-je avec un air d'assurance, ne vous
y jouez pas, vous vous y casseriez le nez. --- Mais, reprit-il encore
une fois, je vous dis que je le suppose coupable et en jugement\,;
encore un coup, comment feriez-vous\,? --- Comment je ferais\,? lui
dis-je, je n'en serais pas embarrassé. J'y irais, car le serment des
pairs y est exprès, et la convocation y nécessite. J'écouterais
tranquillement en place tout ce qui serait rapporté et opiné avant
moi\,; mon tour venu de parler, je dirais qu'avant d'entrer dans aucun
examen des preuves, il est nécessaire d'établir et de traiter l'état de
la question\,; qu'il s'agit ici d'une conspiration véritable ou supposée
de détrôner le roi d'Espagne, et d'usurper sa couronne\,; que ce fait
est un cas le plus grief de crime de lèse-majesté, mais qu'il regarde
uniquement le roi et la couronne d'Espagne, en rien celle de France\,;
par conséquent, avant d'aller plus loin, je ne crois pas la cour
suffisamment garnie de pairs, dans laquelle je parle, compétente de
connaître d'un crime de lèse-majesté totalement étrangère, ni de la
dignité de la couronne de livrer un prince que sa naissance en rend
capable, et si proche, à aucun tribunal d'Espagne, qui seul pourrait
être compétent de connaître d'un crime de lèse-majesté qui regardé
uniquement le roi et la couronne d'Espagne. Cela dit, je crois que la
compagnie se trouverait surprise et embarrassée, et, s'il y avait débat,
je ne serais pas en peine de soutenir mon avis.\,» Le chancelier fut
étonné au dernier point, et après quelques moments de silence en me
regardant\,: «\,Vous êtes un compère\,; me dit-il en frappant du pied et
souriant en homme soulagé, je n'avais pas pensé à celui-là, et en effet
cela a du solide.\,» Il raisonna encore très peu de moments avec moi, et
me renvoya, ce qu'il n'avait jamais accoutumé à ces heures-là, parce que
sa journée était faite et n'était plus alors que pour ses amis
familiers. Comme je sortais, le premier écuyer y entra.

Je trouvai l'impression que j'avais faite au chancelier si grande, que
je l'allai sur-le-champ conter à M. le duc d'Orléans, qui m'embrassa de
bon coeur. Je n'ai jamais su ce que le chancelier en fit, mais le
lendemain il travailla encore seul avec le foi à l'issue du conseil. Ce
fut la dernière fois, et moins de vingt-quatre heures après, les bruits
changèrent tout d'un coup\,: il se dit tout bas, puis tout haut, qu'il
n'y aurait point de procès, et aussitôt {[}ces bruits{]} tombèrent.

Le roi se laissa entendre en des demi particuliers pour être répandu
qu'il avait vu clair en cette affaire, qu'il était surpris qu'on en eût
fait tant de bruit, et qu'il trouvait fort étrange qu'on en tint de si
mauvais propos\footnote{On peut comparer ce que dit à ce sujet le
  marquis d'Argenson (\emph{Mémoires}, édit. 1825, p.~190 et 191), et
  les détails qu'il donne sur les services que son père, alors
  lieutenant de police, rendit au duc d'Orléans.}.

Cela fit taire en public, non en particulier, où on s'en entretint
encore longtemps. Chacun en crut ce qu'il voulut, suivant ses affections
et ses idées. Le roi en demeura éloigné de son neveu\,; et Monseigneur,
qui n'en revint jamais, le lui fit sentir non seulement en toute
occasion, mais jusque dans la vie ordinaire, d'une façon très
mortifiante. La cour en était témoin à tous moments et voyait le roi sec
avec son neveu, et l'air contraint avec lui. Cela ne rapprocha pas le
monde de ce prince, dont le malaise et la contrainte, après quelque
temps d'une conduite un peu plus mesurée, l'entraîna plus que jamais à
Paris par la liberté qu'il ne trouvait point ailleurs, et {[}pour{]}
s'étourdir par la débauche.

Si M\textsuperscript{me} des Ursins fut mortifiée de n'avoir fait que
toucher au but qu'elle s'était proposé, M\textsuperscript{me} de
Maintenon et ses consorts, M\textsuperscript{lle} Choin et les siens
n'en furent pas plus contents, et prirent grand soin de nourrir et de
tourner en haine et aux plus fâcheux soupçons l'éloignement du roi et de
Monseigneur, et de tenir le monde dans l'opinion que c'était mal faire
sa cour que de voir M. le duc d'Orléans aussi son abandon demeura-t-il
le même. Il le sentait, mais, abattu de sa situation avec le roi et
Monseigneur, il ne fit pas grand'chose pour se rapprocher le monde, qui
néanmoins ne le fuyait plus comme dans le fort de cette affaire et
l'incertitude de ce qu'elle deviendrait.

\hypertarget{chapitre-xix.}{%
\chapter{CHAPITRE XIX.}\label{chapitre-xix.}}

1709

~

{\textsc{Mérite et capacité d'Amelot.}} {\textsc{- Tous les ministres
menacés.}} {\textsc{- Singulière consultation du chancelier et de la
chancelière avec moi.}} {\textsc{- Mesures de retraite à la Ferté.}}
{\textsc{- Conversation particulière et curieuse sur ma situation de
M\textsuperscript{me} de Saint-Simon avec M\textsuperscript{me} la
duchesse de Bourgogne.}} {\textsc{- Causes de l'éloignement du roi pour
moi.}} {\textsc{- Folle ambition d'O et de sa femme qui me tourne à
danger.}} {\textsc{- Changements en Espagne.}} {\textsc{- Amelot, refusé
d'une grandesse pour sa fille, arrive à Paris, perdu.}}

~

C'était, ce semble, le temps des orages à la cour\,; il en grondait un
qui menaçait tous les ministres. Celui qui fut si près d'accabler M. le
duc d'Orléans ne fut pas plutôt passé que l'autre sembla se renouveler.

Le retour d'Amelot, toujours à la veille de partir d'Espagne, parut une
bombe en l'air qui les menaçait tous. Il y avait été à la tête de toutes
les affaires qu'il avait trouvées dans le plus grand chaos et dans un
épuisement étrange\,; il gouverna les finances, le commerce, la marine,
avec tant d'application et de succès, que malgré le malheur de la guerre
il les rétablit dans le plus grand ordre, les augmenta considérablement,
acquitta une infinité de choses, régla les troupes, les rendit plus
belles, plus choisies, plus nombreuses, les paya exactement, et peu à
peu remplit toutes les sortes de magasins. Cela parut une création, et
ce qui ne fut pas moins merveilleux, c'est qu'avec une fermeté que rien
n'affaiblissait et qui se faisait ponctuellement obéir, il ne laissa pas
de s'acquérir les cours de tous les ordres de l'Espagne par ses manières
douces, prévenantes, polies, respectueuses, au milieu de ce grand
pouvoir, comme sa capacité lui en acquit l'estime, et sa probité la
confiance, et cela tout d'une voix, et cependant toujours très bien, et
même en amitié avec la princesse des Ursins.

Cette grande réputation, qui depuis tant d'années dure encore en
bénédiction en Espagne, et où, douze ans après son retour, tout ce que
j'y vis me demanda de ses nouvelles avec empressement, se répandit sur
ses louanges, et en étonnement de ce qu'il n'était pas en première place
en France, était pleinement connue en notre cour, où on sentait le
besoin de ministres d'un mérite aussi éprouvé que le sien. On parla de
lui pour les affaires étrangères où il avait si bien réussi dans ses
ambassades, et Torcy avait tout à craindre de M\textsuperscript{me} de
Maintenon et des jésuites. On en parla pour les finances qu'il avait
rétablies et augmentées. On en parla pour la guerre, parce qu'il n'avait
pas moins bien réussi pour les troupes, et en ce cas de donner les
finances à Voysin où sur tous les autres départements,
M\textsuperscript{me} de Maintenon voulait avoir celui-là à elle\,;
ainsi Desmarets se crut en l'air fort longtemps, parce que le retour
d'Amelot se différait toujours.

Enfin on en parla pour la marine avec plus d'apparence encore par les
créations, s'il faut ainsi parler, qu'il avait faites dans celle
d'Espagne, qui fut toute formée\,; rétablie et mise en ordre et en
nombre par ses soins, par les connaissances qu'il avait plus
particulièrement acquises du commerce par l'administration immédiate de
celui des Indes, enfin par la haine générale de Pontchartrain, qui
n'avait plus le bouclier de sa femme, et dont le père était
personnellement si mal avec M\textsuperscript{me} de Maintenon, l'évêque
de Chartres et les jésuites.

Le comte de Toulouse s'était repenti plus d'une fois de ne l'avoir pas
perdu lorsqu'il l'avait pu\,; M\textsuperscript{me} la duchesse de
Bourgogne ne le pouvoir souffrir\,; il était abhorré de la marine et de
ses propres confrères dans les affaires. Il ne tenait au roi que par
l'inquisition de Paris, qu'il avait mise sur ce pied-là\,; encore le
secret et les affaires qui tenaient de l'important lui avaient-ils été
soufflés par d'Argenson en qui le roi avait toute confiance, et qui
s'était acquis l'affection de beaucoup de gens considérables, en
soustrayant au roi et à Pontchartrain les aventures de leurs enfants et
de leurs proches, qui les auraient perdus si elles avaient été sues. Les
meilleurs amis même du chancelier n'étaient rien moins que les siens, et
avec toute sa bassesse pour les jésuites et pour Saint-Sulpice il
n'avait pu gagner leur amitié. Dans cet état, son père, qui le
connaissait mieux que personne, mais qui ne pouvait faire qu'il ne fût
son fils, tremblait pour lui, se voyait aisément entraîné dans sa
disgrâce, conséquemment la ruine d'une famille qu'il n'aurait élevée que
pour la douleur de la voir périr.

Dans cette anxiété, il me pressa d'un voyage à Pontchartrain, où
j'allais assez souvent avec eux\,; et là, sans peur et sans aveuglement,
il me fit l'honneur de me consulter dans son cabinet, où il appela la
chancelière en tiers. Là, il m'exposa ses craintes et la matière de la
consultation sans s'ouvrir, pour me donner lieu de dire plus
naturellement ce que je penserais. Il s'agissait de, savoir ce qu'il
ferait si son fils était chassé, et, ce qui était le moins apparent, ce
que ferait son fils s'il l'était lui-même, enfin quel parti prendre
s'ils l'étaient tous deux.

Au premier cas, mon avis fut qu'il tendit le dos à la disgrâce\,; qu'il
ne heurtât point le public, qui l'aimait lui et l'honorait, mais qui
éclaterait de joie d'être délivré de son fils\,; qu'il n'augmentât pas
le malaise que le roi prenait avec ceux qu'il jugeait mécontents, mais
qu'il prît sur lui de l'élargir, et sans abandonner son fils, se
réservant entier à le protéger en un autre temps, et glissant sur les
motifs de cette disgrâce, il se fit un mérite de la reconnaissance de
n'y être pas enveloppé, et persuadât le roi qu'il se trouvait bien
traité, favorisé même, d'être, en cette occurrence, conservé entier avec
les sceaux dans tous ses conseils, par conséquent dans sa confiance\,;
que cette conduite, à connaître le roi comme nous le connaissions, le
remettrait non seulement au large avec lui, mais lui plairait de façon à
espérer de le rapprocher comme avant que M\textsuperscript{me} de
Maintenon l'eût éloigné de lui, d'autant plus que-le fonds d'estime et
de goût était demeuré jusqu'à remarquer souvent la sécheresse dont le
chancelier payait la sienne, et jusqu'à s'en être plaint plus d'une
fois\,; qu'outre que cette voie était celle de maintenir sa
considération, c'était la seule encore qui lui prît faire espérer le
retour de son fils, soit après le roi par Monseigneur avec qui il était
bien et dont il demeurerait ainsi à portée, soit par le roi même, s'il
venait à se mécontenter du successeur de son fils, ou que les temps
changeassent à l'égard des personnes qui auraient procuré sa disgrâce,
toutes choses très possibles à espérer du cours du temps, des
révolutions des cours\,; de son âge et de sa santé, et auxquelles il
fallait renoncer absolument s'il se retirait par la disgrâce de son
fils, et consentir à survivre à sa fortune, et, au bien près, à voir ses
petits-enfants au même point où lui-même s'était trouvé en naissant.

{[}Quant{]} au second cas, il ne me parut pas vraisemblable. Je ne
voyais rien de personnel contre lui qui pût aller à lui ôter les sceaux,
ni aucun candidat qui en fût susceptible. Chassant Torcy ou Desmarets
pour faire place à Amelot, ni l'un ni l'autre n'étaient assez bien avec
le roi et M\textsuperscript{me} de Maintenon pour leur donner cette
consolation imaginaire, ni pour que le roi se pût résoudre à retenir
vis-à-vis de soi un homme qu'il aurait dépouillé et qui demeurerait
outré de l'être\,; que Voysin, tout nouveau, n'avait pas besoin de ce
surcroît\,; et que, dès que la pensée n'en était pas venue au roi pour
retenir et consoler son ami Chamillart, je ne voyais plus aucun péril à
craindre là-dessus. Mais pour couler à fond cette seconde matière,
quelque peu apparente qu'elle fût, mon avis était que son fils ne se
jetât pas volontairement lui et ses enfants dans le précipice, mais
qu'il demeurât et se conduisît comme je venais de le lui proposer à
lui-même en cas de chute de son fils.

Au troisième cas, où chassés tous deux, il s'agissait de savoir si le
chancelier retiendrait ou se démettrait de son office\,; même en rendant
volontairement les sceaux, si son fils était chassé, si à cette occasion
il eut pris le parti de la retraite, mon avis fut encore qu'il conservât
l'office de chancelier. Outre que cette sorte de démission a peu
d'exemples, et aucun depuis les derniers siècles, le possesseur n'en
peut être dépouillé que par jugement juridiquement prononcé pour crime.
Tant qu'il le conserve, en quelque exil qu'il soit, il demeure le second
officier de la couronne, le chef de la justice, et nécessairement en
considération assez pour être encore ménagé lui et sa famille. Il est
toujours regardé comme pouvant aisément revenir en première place. Rien
de si peu stable que les sceaux pour qui n'en a que la garde, dont
presque aucun n'est mort sans les avoir perdus\,; et les perdant, c'est
toujours une sorte de nouvelle violence de ne les pas rendre au
chancelier. D'ailleurs, quand cela n'arriverait pas de ce règne, il
était plus que moralement sûr que cela ne serait différé que jusqu'à
l'avènement de Monseigneur à la couronne, qui, l'aimant et l'estimant de
tous temps, serait bien aise de le rapprocher, pour avoir sous sa main
un chancelier et un ministre de son ancienne habitude et confiance\,; et
ces sortes de retours {[}sont{]} toujours si accompagnés de faveur, que
ce nouveau crédit pourrait remettre son fils en place\,; et enfin que la
démission ne le conduirait qu'à marquer son dépit, ne serait jamais
prise pour autre chose, et l'ensevelirait nécessairement dans une
retraite profonde et difficile pour un homme marié, puisqu'il n'y avait
plus moyen de se montrer sans cette robe, après en avoir été revêtu, ni
d'en espérer le retour par une vacance.

Toutefois c'était le goût et le voeu du chancelier qui, après m'avoir
écouté, me fit sur tous les trois points agités diverses réflexions et
difficultés, qui ne purent me déranger de l'avis que je rapporte sur
tous les trois. J'admirai la modestie, la défiance de soi-même, je dirai
jusqu'à l'humilité, d'un ancien ministre au plus haut degré de son état,
plein d'esprit, de lumière, d'expérience, vouloir consulter un homme de
mon âge, et avoir la docilité de l'en croire.

Je fus encore plus surpris de la chancelière, qui dans une grande piété
ne laissait pas d'aimer le monde et de craindre la solitude jusqu'à
l'avouer, et qui, avec un excellent sens, en était fort considérée, fort
instruite, et fort capable de donner les meilleurs conseils. Elle ne
consulta pas de moins bonne foi que son mari, et rie se récria que sur
la retraite assez grande pour être difficile à un homme marié. Elle ne
voulut y être comptée pour rien\,; et par ce dépouillement en faveur de
l'honneur, même du seul goût de son mari, acheva `de me donner idée de
la femme forte.

Nous délibérâmes de la sorte plus de deux bonnes heures tous trois, et
la résolution conforme à mon avis en fut la conclusion sur tous les
trois points. Qui nous eût dit alors que ce serait moi qui chasserais
leur fils sans retour, mais en conservant la charge au petit-fils\,? ce
sont de ces révolutions qui semblent incroyables, ajoutons tout pour le
prodige, du vivant du père, et sans perdre sa plus tendre amitié. C'est
ce qui se trouvera en son temps.

Tandis que je raisonnais des disgrâces et des retraites des autres, il
était temps, et plus, d'en venir à la mienne dans la pénible situation
où je me trouvais. Le maréchal de Boufflers, qui ne l'ignorait pas, ni à
quoi j'en étais avec le maréchal de Montrevel qui lui avait les
dernières obligations\,; avec ce droit sur lui et dans la brillante
posture où il se trouvait alors, il crut bien valoir Chamillart pour
finir ces disputes. Je lui donnai carte blanche, je l'instruisis, et
c'est ce qui m'arrêta. Montrevel, ravi de me voir destitué de
Chamillart, crut après pouvoir tout m'embler, il fit des compliments à
Boufflers, et finit par ne vouloir point s'en rapporter à lui ni à
personne, dont Boufflers demeura extrêmement piqué. Je n'étais pas en
temps favorable pour m'exposer à un jugement du roi, ainsi je laissai
faire Montrevel tout ce que bon lui sembla\,; mais je ne songeai plus à
aller en Guyenne, et me rabattis à la Ferté, où mon dessein était de
passer des années. Mais auparavant nous crûmes qu'il était sage de
prendre quelques mesures.

M\textsuperscript{me} de Saint-Simon n'était jamais entrée en rien
d'intime avec M\textsuperscript{me} la duchesse de Bourgogne, mais elle
en avait toujours été traitée sur un pied d'estime, d'amitié et de
distinction. Nous savions même qu'elle la voulait à la place de la
duchesse du Lude, si celle-ci âgée et goutteuse venait à manquer, et
nous n'en pouvions même douter. M\textsuperscript{me} de Saint-Simon eut
donc une conversation avec elle dans son cabinet, seule, un matin\,;
pour découvrir par elle la cause de la situation où je me trouvais, et
les moyens d'y remédier si cela était possible, avant que de prendre
notre parti prêt à exécuter.

Elle fut reçue personnellement avec toute la bonté et l'intérêt
possible, mais avec une froideur très marquée à mon égard\,; elle ne fut
pas même difficile à en rendre raison, et de dire à
M\textsuperscript{me} de Saint-Simon qu'il lui était beaucoup revenu que
j'avais été extrêmement contraire à Mgr le duc de Bourgogne pendant la
campagne de Flandre, et que je ne m'étais pas contraint de m'en
expliquer. La surprise de M\textsuperscript{me} de Saint-Simon fut
d'autant plus grande qu'elle avait su à mesure tout ce qui s'était passé
là-dessus par M\textsuperscript{me} de Nogaret, et même par M. de
Beauvilliers, et qu'il n'était pas possible que Mgr le duc de Bourgogne
ne lui eût dit lui-même combien il était content de moi là-dessus. Mais
la princesse était légère, en proie à chacun, et il s'était trouvé
d'honnêtes gens qui avaient détruit dans le cours de l'hiver tout ce qui
s'était passé dans celui de cette étrange campagne. Je reviendrai à ces
bons offices-là dans un moment.

M\textsuperscript{me} de Saint-Simon se récria, lui rappela ce que je
viens de dire\,; et pour lui faire une impression plus précise, la pria
de s'en informer particulièrement à M. de Beauvilliers, avec qui elle
avait été en si continuelle relation dans le cours de cette longue
campagne, et à M. le duc d'Orléans, dont elle était si fort à portée, et
avec lequel j'avais été en commerce de lettres continuelles pendant le
même temps, et si étroit avec lui toujours depuis son retour.

Ces réponses firent impression. La princesse s'ouvrit davantage à mesure
que M\textsuperscript{me} de Saint-Simon lui dit de faits forts et
précis là-dessus, et qu'elle lui fit entendre que la cabale de M. de
Vendôme, ne pouvant faire pis, pour se venger de ma liberté et de ma
force à parler et à agir contre elle, avait semé la fausseté contraire
de laquelle toute la cour avait été témoin\,; que Mgr le duc de
Bourgogne était bien persuadé de la vivacité de ma conduite à cet égard,
qui m'avait attiré de nombreux ennemis, et qu'il serait bien douloureux
qu'elle fût la seule qui ne la fût pas après avoir vu et su par
M\textsuperscript{me} de Nogaret, l'extrême intérêt que j'avais pris en
celui de Mgr le duc de Bourgogne. La même légèreté qui l'avait aliénée
la ramena aisément au souvenir de ce qu'on avait effacé de son esprit,
et les suites ont dû nous persuader que ces fausses impressions étaient
demeurées à leur tour effacées.

Elle dit ensuite à M\textsuperscript{me} de Saint-Simon que j'avais des
ennemis puissants, et en nombre, qui ne, perdaient point d'occasions de
me nuire\,; qu'on avait extrêmement grossi au roi mon attachement à ma
dignité, et parla de cette méchanceté de M. le Duc que j'ai rapportée
sûr les manteaux\,; qu'on m'accusait de blâmer sans mesure ce qu'il
faisait, et de parler mal des affaires\,; que M\textsuperscript{me} de
Saint-Simon était bien avec le roi, estimée et considérée, mais qu'il
avait conçu une grande opposition pour moi, que le temps seul et une
conduite fort sage et fort réservée pouvait diminuer\,; que l'on disait
que j'avais beaucoup plus d'esprit, de connaissances et de vues que
l'ordinaire des gens, que chacun me craignait et avait attention à moi,
qu'on me voyait lié à tous les gens en place, qu'on redoutait que j'y
arrivasse moi-même, et qu'on ne pouvait souffrir ma hauteur et ma
liberté à m'expliquer sur les gens et sur les choses d'une façon à
emporter la pièce, que ma réputation de probité rendait encore plus
pesante.

M\textsuperscript{me} de Saint-Simon la remercia fort d'avoir bien voulu
entrer ainsi en matière avec elle, et répliqua fort à propos que, n'y
ayant rien d'essentiel à reprendre dans l'essentiel de ma conduite ni
dans le courant de ma vie, on m'attaquait par des lieux communs qui, par
leur vague, pourvoient convenir à chacun de ceux qu'on voulait perdre\,;
que tous ces ennemis ne s'étaient montrés que depuis qu'ayant pensé,
sans y songer, aller ambassadeur à Rome, on s'était réveillé sur moi
pour me couper les ailes\,; que d'Antin et M\textsuperscript{me} la
Duchesse ne s'y étaient pas épargnés\,: le premier par la concurrence du
même emploi, qu'il avait vainement brigué\,; l'autre, en haine de ma
hauteur à son égard sur l'affaire de M\textsuperscript{me} de Lussan\,;
que les Lorrains, mes ennemis depuis l'affaire de M. le Grand et celle
de la princesse d'Harcourt, que j'ai racontée et qu'il ne m'avait pas
été possible d'éviter, ne cessaient de me nuire\,; que les envieux, si
communs dans les cours, se joignaient à eux\,; et sur l'esprit et le
reste parla en femme qui veut donner bonne opinion de son mari. Elle
s'étendit ensuite sur ce qui s'était passé sur ce pari célèbre de Lille
qui m'avait fait tant de mal, et s'étendit sur l'iniquité de se voir
tourner à crime d'avoir des vues justes et des amis qui devaient faire
honneur, et d'être si craint lorsqu'on ne pensait à rien, et qu'on ne
voulait mal à personne.

La conversation finit par toutes sortes de marques de bonté de
M\textsuperscript{me} la duchesse de Bourgogne, de peine de perdre
M\textsuperscript{me} de Saint-Simon pour du temps, et d'être attentive
à toutes les occasions, par elle et par M\textsuperscript{me} de
Maintenon, à me raccommoder avec le roi. Elle parla même si fortement à
Bloin pour nous faire donner un logement qu'il se détermina pour lui
plaire à y faire de son mieux, à ce qu'il dit au duc de Villeroy et à
d'autres de nos amis. M\textsuperscript{me} de Saint-Simon eut la
prudence de ne me dire que longtemps depuis tout l'éloignement du roi
pour moi que cette conversation lui avait appris, pour ne pas trop
fortifier mon dégoût extrême de la cour, que je voulais abandonner pour
toujours. Je fus sensible plus qu'à tout à la noirceur de la calomnie
sur Mgr le duc de Bourgogne, et pour cela seul plus affermi à m'éloigner
de scélérats si déclarés. Je ne pensai plus qu'à m'en aller à la Ferté.

Je me suis étendu sur cette conversation, parce que rien ne peint mieux
le roi et la cour que tout ce qui fut dit à M\textsuperscript{me} de
Saint-Simon par M\textsuperscript{me} la duchesse de Bourgogne. Cette
crainte et cette aversion si grande du roi pour l'esprit et pour les
connaissances au-dessus du commun, que faute de mieux, on m'en fit un
crime qui, en toute occasion, se renouvela auprès de lui, mais qui me
fit plus de mal que des choses qui eussent été véritablement mauvaises
et dangereuses. Jusqu'à la réputation de probité me nuisit auprès de
lui, par le tour qu'on y sut donner\,; et ceux qui le connaissaient bien
et qui me voulaient perdre sans avoir de quoi, n'y trouvèrent que des
louanges exagérées d'esprit et de connaissances, et de poids donné par
la probité à des discours pesants. L'amitié pour moi et la confiance des
principaux ministres et des seigneurs les plus distingués et lés plus
considérés, les plus avant dans la confiance du roi, devenus un autre
démérite auprès de lui, tellement que tout ce qui devait lui plaire
comme ce dernier article, et lui donner bonne opinion comme tous ces
autres, c'est ce qui fit son éloignement le plus grand, et qui encore,
en premier ordre, lui soufflait ce poison. MM. du Maine et d'Antin, les
deux hommes de sa cour qui voient le plus d'esprit, d'application et de
vues, et qui passaient pour tels\,: d'Antin, on a vu pourquoi\,; M. du
Maine était l'âme de la cabale de Vendôme et ne me pardonnait pas man
attachement pour Mgr le duc de Bourgogne. Lui et d'Antin avaient séduit
Bloin et Nyert dont le père, comme je l'ai raconté, devait sa fortune au
mien, qui me rendirent tous les mauvais offices qu'ils purent, et en
toutes les façons, sans l'avoir jamais mérité d'eux. M. et
M\textsuperscript{me} du Maine n'avaient pas oublié les vains efforts
qu'ils avaient prodigués pour m'attirer chez eux, et dès là me
craignirent pour leur rang. De là le crime auprès du roi d'être attaché
à ma dignité, de là la haine de M\textsuperscript{me} de Maintenon, qui
fut ma plus constante et ma plus dangereuse ennemie.

M\textsuperscript{me} la duchesse de Bourgogne, qui nous le voulut
cacher, coula, dans ce qu'elle dit à M\textsuperscript{me} de
Saint-Simon, qu'elle tâcherait, par elle-même et par
M\textsuperscript{me} de Maintenon, de profiter de toutes les occasions
de me raccommoder avec le roi. Elle savait mieux qu'elle ne disait, et
que M\textsuperscript{me} de Maintenon y était le plus grand obstacle.
Chamillart le trouva tel, lorsqu'au commencement du mariage de sa
dernière fille et de notre amitié, il me trouva mal avec le roi pour
avoir quitté le service, et m'y voulut raccommoder et me remettre des
voyages de Marly. Il en eut jusqu'à des disputes fortes, et souvent
redoublées, avec M\textsuperscript{me} de Maintenon, avec qui alors il
était dans l'entière intimité, et ce ne fut qu'avec beaucoup de temps et
de peine qu'il vint à bout, non de la changer à mon égard, mais
d'obtenir d'elle qu'elle ne s'opposerait plus à Marly, et qu'elle
cesserait de me nuire. Je l'ai su de Chamillart même, qui ne voulut
jamais s'en laisser entendre du vivant du roi, même depuis sa disgrâce,
de peur\,; à ce qu'il me dit depuis, de me dégoûter trop, et d'exposer
ma colère à me faire plus de mal encore avec elle. Je m'étais bien douté
qu'elle ne m'était pas favorable, je ne savais pourquoi au juste,
quoique je me défiasse de M. du Maine, qui toutefois ne se lassait
jamais de m'accabler de politesses, même recherchées\,; mais, pour la
haine, je ne la sus que lorsque, après la mort du roi, Chamillart me
demanda ce que j'avais fait à cette fée, pourquoi elle me haïssait tant,
et me conta ce que je viens de dire.

Pour M\textsuperscript{me} la duchesse, de Bourgogne, je fus redevable
des impressions dont M\textsuperscript{me} de Saint-Simon la fit revenir
à M. et à M\textsuperscript{me} d'O. On a vu (t. Ier, p.~362 et depuis)
quels ils étaient. Le mari avait conservé la confiance du roi, et ses
entrées privées, de l'éducation du comte de Toulouse. {[}On a vu{]} son
hypocrisie étudiée, la protection du duc de Beauvilliers, dupe achevée
par sa charité ignorante, son importance, une sorte de considération, et
le tout à l'épreuve de sa campagne de mer et de celle de terre dont j'ai
parlé. Il était créature de M\textsuperscript{me} de Maintenon, sa femme
encore davantage, et si commode à M\textsuperscript{me} la duchesse de
Bourgogne qu'elle l'avait réduite dans sa dépendance à force de services
de confiance. Ces gens-là avaient oublié. leur état, et le prodige de
leur fortune les avait aveuglés.

Le gouverneur du dernier fruit du plus scandaleux double adultère osa
imaginer de s'en faire un échelon pour se faire gouverneur de l'héritier
futur de la couronne. Dévoué à M. du Maine plus encore qu'au comte de
Toulouse, parce qu'il en espéra davantage, et protégé de
M\textsuperscript{me} de Maintenon, lui et sa femme, et tous deux tenant
aux plus intimes de la cour par les deux voies les plus opposées, ils
comptèrent s'assurer cette grande place en écartant ceux qui pouvaient y
atteindre\,; et j'ai su depuis très certainement que, m'ayant regardé
comme un compétiteur dangereux, et par le duc de Beauvilliers et par mes
autres amis considérables, et par moi-même, ils avaient travaillé à me
saper, et pour cela avaient persuadé cette horrible calomnie à
M\textsuperscript{me} la duchesse de Bourgogne. Jamais je n'avais pensé
à une placé qui ne devait être remplie que dans cinq ans\,; mais ces
champignons de fortune prenaient leurs mesures de loin. Ils en sont
néanmoins demeurés à celle qu'ils avaient faite, que leur ambition leur
rendit enfin amère\,; et tous deux ont vieilli et sont morts dans la
douleur et le mépris\,: le mari sans avoir pu dépasser la grand'croix de
Saint-Louis, et n'ayant plus d'administration chez le comte de
Toulouse\,; et la femme, devenue suivante de M\textsuperscript{me} de
Gondrin, dame du palais, sous sa conduite, avec elle, et remariée après
au comte de Toulouse, est morte abandonnée de tout le monde dans un
grenier de l'hôtel de Toulouse.

M\textsuperscript{me} des Ursins fit beaucoup de changements dans les
conseils, d'Espagne pour montrer des précautions et des suites de ses
découvertes. Le conseil du cabinet, autrement la junte, fut composé de
don Fr.~Ronquillo, quelle avait fait gouverneur du conseil de
Castille\,; des ducs de Veragua et de Medina-Sidonia\,: le premier
absolument dans sa dépendance, l'autre grand écuyer, chevalier du
Saint-Esprit, nullement à craindre, mais personnage du nom duquel elle
se voulut parer et fort attaché au roi, qui l'aimait\,; du comte de
Frigilliane, père du marquis d'Aguilar, que j'ai fait connaître (t. III,
p. 30), et qu'il fallait bien récompenser de s'être dévoué à elle, et en
sa personne, son fils d'avoir arrêté Flotte\,; du marquis de Bedmar,
homme doux, qui devait tout à la France, et à qui elle donna la guerre
qu'elle ôta au duc de Saint-Jean. Amelot en était toujours, qui à vrai
dire leur laissait la broutille ou les choses résolues, et faisait tout,
ou seul, ou avec la princesse des Ursins. Cette nouvelle forme fût
encore un prétexte de le garder en Espagne quelque temps.

Lorsqu'il arriva enfin, les bruits et les frayeurs se renouvelèrent,
quoique les ministres ne se fussent pas oubliés à faciliter les délais
de son retour, et à les employer de leur mieux à se parer de ce qu'ils
en craignaient. Lui-même aussi put y donner lieu, peu assuré d'embler en
France une des places du ministère toutes remplies, et hors de portée,
par son état d'homme de robé, des grandes récompenses d'Espagne où il
avait si dignement servi. Il leur entra dans l'esprit, à lui et à
M\textsuperscript{me} des Ursins, de faire le mariage de sa fille avec
Chalais, fils du frère du premier mari de M\textsuperscript{me} des
Ursins, dont elle avait toujours aimé les proches et celui-ci qu'elle
avait fait venir auprès d'elle, et en faveur de ce mariage récompenser
Amelot d'une grandesse pour son gendre. La difficulté ne fut pas en
Espagne dont ils disposaient tous deux, et tout les persuadait avec
raison qu'ils n'y en trouveraient pas en France du côté du roi, qui par
toutes ses dépêches marquait tant de satisfaction d'Amelot qui méprisait
les dignités, et à qui ce consentement ne coûtait rien et tenait lieu
d'une grande récompense. Leur surprise ne fut pas médiocre lorsqu'ils y
en trouvèrent, et telle qu'ils ne purent la vaincre.

Pendant cette sorte de combat dont M\textsuperscript{me} des Ursins,
avertie peut-être en secret par M\textsuperscript{me} de Maintenon, se
tint fort à quartier, Amelot arriva à Paris et à la cour. Sa réception y
fut brillante, mais néanmoins sans voir le roi en particulier que
quelques instants. Il alla voir les ministres. Le chancelier, pour
début, lui dit\,: «\,Monsieur, nous n'avons, tous tant que nous sommes,
qu'à nous bien tenir\,; et vous désirer que quelqu'un tombe. Sûrement
vous auriez sa place\,; mais dépêchez-vous d'enfoncer la porte du
cabinet, car je vous avertis que, si vous vous laissez refroidir, vous
n'y reviendrez plus.\,» Il disait très vrai et en bon connaisseur.

Amelot parla au roi du mariage de sa fille et de la grandesse\,; il fut
civilement éconduit. Quelques jours après, il revint à la charge, et le
fut encore. Il en fut outré, et de n'avoir point eu d'audience
particulière sur les affaires d'Espagne. Il ne se put empêcher de
laisser voir son mécontentement, et cependant les ministres se
rassurèrent.

Amelot se crut perdu et n'oublia rien dans sa surprise pour en pénétrer
la cause. On n'avait pu l'attaquer sur la capacité, ni sur l'intégrité,
ni sur aucune partie de l'exercice de ses emplois, mais on persuada au
roi qu'il était janséniste. Dire et persuader en ce genre était même
chose, et presque toujours le mal était devenu incurable avant que celui
qui en était attaqué en eût la première notion\,: c'est ce qui arriva à
Amelot. À la fin, il apprit de quoi il s'agissait, et n'en fut guère en
peine, parce que jamais il n'avait donné lieu à ce soupçon. Mais quand
il voulut s'en purger, il trouva si bien toutes les portes fermées qu'il
en demeura perdu, et réduit au simple emploi de conseiller d'État, et
confondu avec les manteaux, après avoir régné en effet en Espagne, et
fait trembler ici longtemps tous les ministres. Il dit souvent depuis au
chancelier qu'il n'avait que trop senti la justesse de son avis. Je n'ai
point su qui lui enfonça ce poignard dans le sein, mais après tant de
violents orages, le calme revint à la cour, dès qu'on n'y craignit plus
Amelot.

Cette même fille, dont il s'était flatté de se défaire moyennant une
grandesse, épousa depuis M. de Tavannes, lieutenant général en
Bourgogne, frère de l'archevêque de Rouen, et nous verrons Chalais fait
grand, sans chausse-pied, et malgré le roi. Amelot ne laissa pourtant
pas à la fin de tirer parole du roi de la première charge de président à
mortier pour son fils, tant il parut honteux de ne rien faire pour lui.
Eh ce même temps, la reine d'Espagne accoucha d'un fils qui ne vécut
pas.

\hypertarget{chapitre-xx.}{%
\chapter{CHAPITRE XX.}\label{chapitre-xx.}}

1709

~

{\textsc{Cardinal de Médicis rend son chapeau\,; épouse une
Gonzague-Guastalla.}} {\textsc{- Mort de la duchesse de Créqui.}}
{\textsc{- Mort et caractère de Lamoignon, président à mortier.}}
{\textsc{- Mort de Ricousse et de Villeras.}} {\textsc{- Mort du fils
unique du duc d'Albe.}} {\textsc{- Listenais chevalier de la Toison
d'or.}} {\textsc{- Changements parmi les intendants.}} {\textsc{-
M\textsuperscript{me} de Mantoue à Vincennes\,; ses prétentions\,; ses
tentatives\,; voit le roi et Monseigneur en particulier\,; réduite à
l'état de simple particulière.}} {\textsc{- Désordres de cherté et de
pain.}} {\textsc{- Boufflers apaise deux tumultes et devient dépositaire
de l'autorité du roi à Paris.}} {\textsc{- Sa rare modestie}}

~

Le cardinal de Médicis, dont j'ai parlé à l'occasion du passage de
Philippe V à Naples et en Lombardie, pressé par le grand-duc, son frère,
remit son chapeau et conclut son mariage avec une Guastalla-Gonzague.
Ils prévoyaient ce qui leur est arrivé, le fils aîné du grand-duc était
mort sans enfants d'une soeur de M\textsuperscript{me} la dauphine de
Bavière. Il ne lui en restait plus qu'un, brouillé, comme lui, avec sa
femme Saxe-Lauenbourg soeur de la veuve du prince Louis de Bade, mère de
la duchesse d'Orléans depuis, et grand-mère de M. le duc de Chartres,
toutes deux dernières de cette brande et première maison d'Allemagne, où
depuis plusieurs années elle s'était retirée chez elle, comme avait fait
M\textsuperscript{me} la grande-duchesse en France. Le grand-duc, son
fils, et son frère étaient les seuls Médicis de la branche ducale. Celle
d'Ottaïano, leur aînée, mais séparée longtemps, avant l'oppression des
Florentins, était établie à Naples, toujours mal avec les grands-ducs.
Le père et le fils, hors d'espérance d'enfants, voulurent tenter que le
cardinal en eût qui n'avait aucuns ordres, mais qui avait cinquante ans.
Son mariage fut heureux mais stérile.

La duchesse de Créqui ne survécut pas longtemps le duc de La Trémoille
son gendre, si connue par sa beauté, par sa vertu, par la fameuse
affaire des Corses de la garde du pape qui tirèrent sur elle et sur M.
de Créqui, ambassadeur à Rome, et par avoir été danse d'honneur de la
reine. On disait d'elle que son mari la montait à la cour tous les
matins comme une horloge. Elle succéda à la duchesse de Richelieu, que
M\textsuperscript{me} de Maintenon fit passer par confiance à
M\textsuperscript{me} la Dauphine, à son mariage, et
M\textsuperscript{me} de Créqui fut dame d'honneur jusqu'à la mort de la
reine. Depuis qu'elle fut veuve, elle alla rarement à la cour, et mena
une vie très pieuse et très retirée. C'était une femme d'une grande
douceur, et qui conserva toujours beaucoup de considération. Elle était
Saint-Gelais, comme je l'ai expliqué ailleurs.

Lamoignon, président à mortier, après avoir été longtemps avocat
général, mourut en même temps. Il était fils aîné du premier président
Lamoignon, et frère du trop fameux Bâville, intendant de Languedoc. Mais
Bâville était à lui, où il avait tant qu'il pouvait force seigneurs de
la cour, quelques jours pendant les vacances, et toujours le célèbre P.
Bourdaloue. C'était un homme enivré de la cour, de la faveur du grand et
brillant monde, qui se voulait mêler de tous les mariages et de tous les
testaments, et a qui, comme à tout Lamoignon, il ne se fallait fier que
de bonne sorte. Il avait cédé sa charge à son fils, que le fils de
celui-là possède encore, qui, en tout, ont bien moins valu même que
celui dont il s'agit ici.

Ricousse mourut aussi, qui avait été envoyé en Bavière. C'était un homme
d'esprit, de valeur, de ressource, estimé, et qui avait beaucoup d'amis
qui lui faisaient grand honneur, et Villeras, sous-introducteur des
ambassadeurs, fort honnête homme et modeste, savant, qui leur plaisait à
tous, et dont on se servait à toutes les commissions délicates à leur
égard. Il s'était fait fort estimer, et voyait gens fort au-dessus de
son état, par un mérite digne d'être remarqué. Son père était secrétaire
du président de Mesmes, et mort chez lui, où Villeras logea aussi toute
sa vie.

Le duc d'Albe perdit son fils unique, qui avait sept ou huit ans. Il le
faisait appeler le connétable de Navarre, dignité héréditaire dans sa
maison, vaine et réduite au seul nom comme celles de connétable et
d'amirante de Castille\,; mais ces deux-ci ont la grandesse que l'autre
prétendait, et qu'elle n'a eue que de Philippe V, lorsqu'il envoya le
duc d'Albe ambassadeur en France. Tous les voeux et les dévotions
singulières que fit la duchesse d'Albe pour obtenir la guérison de son
fils surprirent fort ici, jusqu'à lui faire prendre des reliques en
poudre par la bouche et en lavement. Enfin il mourut, et son corps fut
renvoyé en Espagne, en habit de cordelier, autre dévotion espagnole. Ils
furent fort affligés, surtout la duchesse d'Albe, avec des éclats
étranges. Le roi leur envoya faire compliment, et les fils de France et
toute la cour y fut.

M\textsuperscript{me} de Mailly, qui n'avait pas donné grand'chose à
M\textsuperscript{me} de Listenais en mariage, fit en sorte, par
M\textsuperscript{me} de Maintenon et M\textsuperscript{me} la duchesse
de Bourgogne, de faire donner la Toison à Listenais son gendre, malgré
la belle équipée dont j'ai parlé et dont elle avait été la dupe. Son nom
était Beaufremont, gens de qualité distingués de Bourgogne, dont
plusieurs autrefois avaient eu ce même ordre. Leur père, dont j'ai parlé
ailleurs, ne l'avait point\,; et, bien qu'élevé auprès de Charles II,
avait suivi le sort de la Franche-Comté, où il avait beaucoup de biens,
et était mort en France, assez jeune, ayant un régiment de dragons.
Cette Toison parut assez sauvage, non pour la naissance, mais par toutes
autres raisons.

Je marquerai ici un changement qui se fit de quelques intendants, parce
que quelques-uns de ceux-là ont fait parler d'eux depuis. Bouville,
conseiller d'État, beau-frère de Desmarets, voulut revenir d'Orléans. Il
avait acquis à la porte de Vernon un petit lieu appelé Bisy-en-Bellevue,
qu'il avait bâti et accommodé en bourgeois qu'il était, et dont
Belle-Île, depuis son échange dont il sera parlé en son lieu, a fait.
Une habitation digne en tout d'un fils de France. Ribeyre, conseiller
d'État estimé, obtint que La Bourdonnaie, son gendre, vint de Bordeaux à
Orléans\,; et on envoya à Bordeaux Courson, fils de Bâville, qui, dans
le manége des blés dont j'ai parlé, se fit presque assommer à Rouen, et
à diverses reprises, où il n'osait plus se montrer, et où ce qu'il fit
depuis à Bordeaux fait soupçonner qu'il ne s'oublia pas. Il avait la
dureté et la hauteur de son père, mais il n'en avait que cela\,;
ignorant, paresseux, brutal à l'excès. Il causa tant de désordres, qu'il
fallut y envoyer M, de Luxembourg, gouverneur de la province, ami de
Voysin, et l'y tenir longtemps, qui fit évader Courson, qui sans lui eût
été assommé. Richebourg le releva à Rouen, où il réussit fort mal, et se
fit enfin révoquer. Le fils de Mansart le fut aussi de Moulins. C'était
un débauché qui ne savait et ne faisait rien, et qui pour vivre à l'abri
de ses créanciers se fit gendarme. On envoya à sa place Turgot, gendre
de Pelletier de Sousy, que son crédit avait mis et soutenu longtemps à
Metz, mais si lourde bête qu'il l'en fallut ôter, et pour contenter son
beau-père, lui donner de petites intendances, d'où à la fin il fut
révoqué. Son, fils ne lui a pas ressemblé. Il est devenu conseiller
d'État, après avoir montré onze ans son intégrité et sa capacité dans la
place de prévôt des marchands, où il a fait de belles et de bonnes
choses, et où il a été fort regretté. On rappela aussi de Caen le fils
de Foucault, conseiller d'État, qui lui avait succédé dans cette
intendance, où il fit toutes les folies et toutes les sottises
imaginables. Il s'appelait Magny, et fit bien des sortes de personnages
dans la suite, et un enfin qui le bannit du royaume, et dont il sera
parlé en son temps. On voit ainsi un échantillon des intendants mis en
place d'insulter et de ruiner les provinces, sans esprit, sans aucun
sens, sans capacité, et moins encore d'expérience, mis et maintenus par
crédit. Labriffe, fils du feu procureur général, alla réparer les
désordres de Magny, et est mort longues années depuis conseiller d'État
et intendant de Bourgogne, extrêmement considéré pour sa capacité, sa
bonté, et son intégrité. Phélypeaux, conseiller d'État et frère du
chancelier, attaqué d'apoplexie, quitta l'intendance de Paris, que le
chancelier fit donner à Bignon, intendant des finances, fils de sa
soeur, avec parole de la première place de conseiller d'État, quoique
ses deux frères le fussent déjà, et de vendre alors sa charge
d'intendant des finances à Bercy, gendre de Desmarets. C'est ce Bignon
dont la femme était l'amie la plus intime de M\textsuperscript{lle}
Choin, et lui aussi.

Il faut ajouter ici que l'abbé Languet eut aussi une petite abbaye,
premier pas de sa fortune, et bien bas et petit compagnon en ce
temps-là, que j'ai vu, aumônier de M\textsuperscript{me} la duchesse de
Bourgogne, attendre souvent et longtemps dans l'antichambre de ses
dames, et y faire force courbettes. C'est cet archevêque de Sens qui a
tant fait parler de lui depuis par ses violences, ses calomnies, ses
fausses citations, ses tronquements de passages, et les gros ouvrages
adoptés et donnés sous son nom en faveur de la bulle \emph{Unigenitus},
qui a fait sa fortune, mais l'a laissé inconsolable que tant et de si
étranges personnages qu'il a faits, même sa \emph{Marie-Alacoque}, ne
lui aient pas procuré le chapeau qu'il a brigué toute sa vie, et qu'il a
cru tenir plus d'une fois. Ce personnage se retrouvera dans la suite.

M\textsuperscript{me} de Mantoue, ennuyée de son couvent de
Pont-à-Mousson, peu amusée de quelques tours à Lunéville sous la tutelle
de sa soeur de Vaudemont et la grandeur de la souveraine du pays, se
flatta qu'il était temps de la venir faire elle-même sur le théâtre de
Paris et de la cour, dont elle tirait de grosses pensions. Sa mère ne le
désirait pas moins qu'elle. Elle comptait sur son crédit auprès de
M\textsuperscript{me} de Maintenon, sur l'amitié si marquée de
M\textsuperscript{me} de Maintenon pour M\textsuperscript{me} de
Dangeau, dont le fils avait épousé la fille unique de
M\textsuperscript{me} de Pompadour, sa soeur, depuis le mariage de sa
fille, et qui, outre leur union, serait intéressée à la relever, et sur
la facilité si ordinaire en ce pays-ci pour les prétentions et les
chimères. Elle ne comptait pas moins sur l'appui de M. de Vaudemont et
de ses nièces, par conséquent sur Monseigneur. Le retour fut donc
résolu.

Sous prétexte du besoin de prendre l'air et du lait,
M\textsuperscript{me} d'Elboeuf obtint que sa fille s'établît à
Vincennes, et qu'on y meublât pour elle l'appartement qu'y occupait
autrefois Monsieur, quand la cour y était, et des chambres pour le
domestique dont ce château, depuis tant d'années entièrement vide, ne
manquait pas. Ce début d'un si grand air nourrit leurs espérances.

M\textsuperscript{me} de Mantoue arriva à Vincennes avec le dessein de
se former un rang pareil à celui des petites-filles de France,
c'est-à-dire de ne donner la main ni de fauteuil à qui que ce fût, ni
aucun pas de conduite. La maréchale de Bellefonds, depuis longues années
retirée dans ce château\,; dont son mari et ses enfants avaient été et
se trouvaient encore gouverneurs et capitaines, et qui vivait dans une
grande piété et une grande séparation du monde, y fut attrapée. Elle
alla voie la duchesse de Mantoue, et fut si étourdie de se voir
présenter un ployant, qu'elle se mit dessus, mais, quelques moments
après, revenue à soi, elle s'en alla et n'y remit pas le pied davantage.
M\textsuperscript{me} de Pompadour n'osa s'adresser à des femmes
titrées, mais y en mena d'autres tant qu'elle put, dont le concours
pourtant s'arrêta brusquement, et laissa M\textsuperscript{me} de
Mantoue livrée à son domestique nombreux d'abord, mais qui se raccourcit
bientôt faute de vivres.

Pendant tout cela, M\textsuperscript{me} d'Elboeuf négociait le
traitement de sa fille, et ne réussit à rien. M\textsuperscript{me} de
Maintenon, comme je l'ai quelquefois remarqué, avait des fantaisies et
des hauts et bas pour ses mieux aimées. M\textsuperscript{me} d'Elboeuf
ne se rencontra pas alors dans la bonne veine, et par une merveille, le
roi, pour cette fois, ne, se rendit pas facile aux prétentions. M. de
Mantoue était mort et n'avait point de successeur. Ses États étaient et
demeurèrent occupés par l'empereur. Le souvenir du mariage fait malgré
ses défenses était encore présent, et celui de toutes les tentatives et
de tous les artificieux manéges de M. de Vaudemont pour les siennes. Il
ne voulut donner aucun pied à M\textsuperscript{me} de Mantoue à la
cour, pour éviter les importunités de ses prétentions, et il régla
qu'elle viendrait vêtue comme pour Marly le voir chez
M\textsuperscript{me} de Maintenon, où se trouverait aussi
M\textsuperscript{me} la duchesse de Bourgogne, et, la visite faite,
s'en retournerait tout court à Vincennes. Cela se passa de la sorte.
Elle arriva à heure précise avec M\textsuperscript{me} d'Elboeuf à
Versailles, entrèrent chez M\textsuperscript{me} de Maintenon, le roi y
étant déjà\,; {[}elle{]} n'y demeura que fort peu de moments, le roi
debout, et qui ne la baisa point, ce qui parut extraordinaire\,; se
retira par le grand cabinet à la suite de M\textsuperscript{me} la
duchesse de Bourgogne qui l'y embrassa, et où Mgr le duc de Bourgogne et
M. le duc de Berry se trouvèrent. On ne s'assit point\,; et en moins
d'un quart d'heure fut congédiée et s'en alla tout de suite avec sa mère
à Vincennes, sans avoir pu voir M\textsuperscript{me} de Maintenon en
particulier.

Quelques jours après, elles allèrent voir Monseigneur à Meudon, et
arrivèrent comme il sortait de dîner. Mgrs ses fils et
M\textsuperscript{me} la duchesse de Bourgogne étaient dans sa petite
galerie du château neuf avec lui. Il les y reçut sans les faire asseoir,
sans leur rien proposer à manger ni à boire, ni aucun jeu, ni
promenade\,; une demi-heure au plus termina une visite si sèche, et la
mère et la fille, qui rie revit Meudon de sa vie, s'en retournèrent à
Vincennes, fort déconcertées de ces deux réceptions.

La princesse de Montauban, qui s'était fort mise sous la protection de
M\textsuperscript{me} d'Elboeuf, se laissa persuader ensuite d'aller à
Vincennes. Ce fut la seule femme titrée qui y alla, apparemment pour y
en exciter d'autres, et pour faciliter à M\textsuperscript{me} de
Mantoue de baisser équivoquement d'un cran. Elle prit, comme par hasard,
un ployant qui se trouva derrière elle, sans affecter de place, et en
donna un à M\textsuperscript{me} de Montauban, mais cette gentillesse ne
tenta personne.

M\textsuperscript{me} d'Elboeuf, difficile à rebuter, tenta après le
siège à dos pour sa fille chez M\textsuperscript{me} la duchesse de
Bourgogne. Les femmes et les veuves de vrais souverains réels et
existants, dont les ministres sont reconnus et reçus dans les cours et
les assemblées de l'Europe pour les négociations et les traités, ont eu
constamment un siège à dos, une seule fois, au cercle de la reine, après
quoi jamais qu'un tabouret, et parmi tous les autres sans distinction et
sans différence des duchesses. La duchesse de Meckelbourg, soeur du
maréchal de Luxembourg, et d'autres souveraines avaient eu ce traitement
du règne du roi, et en dernier lieu M\textsuperscript{me} de
Meckelbourg, qui, après cette unique fois de siège à dos, n'eut plus
qu'un tabouret partout, allait au souper du roi, et, à qui pas une
duchesse ni princesse étrangère ne cédait\,; néanmoins
M\textsuperscript{me} de Mantoue n'y put atteindre, et
M\textsuperscript{me} d'Elboeuf en fut refusée jusqu'à quatre
différentes fois.

Elle et sa fille, outrées de se voir si loin de leurs projets, crurent
pourtant qu'il ne fallait pas bouder, pour ne se fermer pas la porte à
des retours favorables. La mère négocia pour sa fille une seconde visite
chez M\textsuperscript{me} de Maintenon, le roi l'accorda. Elles l'y
trouvèrent comme la première fois, et M\textsuperscript{me} la duchesse
de Bourgogne. Le singulier fut que le roi et elle s'assirent et
laissèrent la mère et la fille debout, sans qu'on leur donnât de
ployants, sans que le roi leur proposât de s'asseoir en aucune façon\,;
il lui dit quelques mots à diverses reprises et puis la congédia.

Elle passa dans le grand cabinet, où M\textsuperscript{me} la duchesse
de Bourgogne la fut trouver aussitôt, et un moment après, l'y laissa et
rentra dans la chambre. M\textsuperscript{me} de Mantoue trouva dans ce
cabinet des dames du palais et quelques autres de celles qui avaient la
liberté d'y entrer. Elle essaya de se les concilier par les politesses
et les amitiés les plus excessives, et repartit de là pour son
Vincennes.

Ces dégoûts étaient grands pour des projets si hauts.
M\textsuperscript{me} d'Elboeuf avait eu la folie de parler de M. le duc
de Berry comme d'un parti sortable à peine pour sa fille, et je pense
que cela eut quelque part au refus du siège à dos.

Éconduite à la cour, où M\textsuperscript{me} de Mantoue ne remit plus
le pied de sa vie, elle voulut du moins dominer dans Paris, et s'y
former un rang à son gré. Elle parut d'abord aux spectacles avec sa
mère, toutes deux réduites à s'y faire suivre par M\textsuperscript{me}
de Pompadour, qui que ce soit n'ayant voulu tâter de leur compagnie, où
elles firent vider une loge à de petites bourgeoises, dont le petit état
couvrit l'affront et empêcha le monde de crier.

La première aventure qui lui arriva, outre celles des fiacres, fut à la
seconde porte du Palais-Royal, avec M. et M\textsuperscript{me} de
Montbazon, qui étaient seuls ensemble dans leur carrosse à deux chevaux,
que celui de M\textsuperscript{me} de Mantoue voulut faire reculer avec
hauteur. Sur la résistance, M\textsuperscript{me} d'Elboeuf, qui était
avec sa fille, envoya un gentilhomme dire à M. de Montbazon que c'était
M\textsuperscript{me} de Mantoue qui le priait de reculer. M. de
Montbazon répondit que, s'il était seul, il le ferait avec grand
plaisir, mais qu'il était avec M\textsuperscript{me} de Montbazon, et
qu'il ne savait pas que M\textsuperscript{me} de Mantoue eût aucun droit
sur elle. Un moment après, le même gentilhomme revint lui dire que
M\textsuperscript{me} de Mantoue ne cédait qu'à l'électeur de Bavière
qui était lors à Paris, car je raconte ceci tout de suite pour n'avoir
plus à revenir là-dessus, et qu'il vit donc ce qu'il voulait faire. M.
de Montbazon répondit sagement que c'était à sa maîtresse à voir si elle
voulait livrer combat, parce qu'il n'était pas résolu à reculer, qu'il
avait beaucoup de respect pour M\textsuperscript{me}s d'Elboeuf et de
Mantoue, mais nulle disposition à leur céder aucun rang. Là-dessus,
chamaillis entre les cochers et quelques injures\,;
M\textsuperscript{me} d'Elboeuf, la tête à la portière, criant qu'on fît
reculer, et M. de Montbazon qui allait mettre pied à terre et donner
cent coups à quiconque oserait approcher. Enfin, à la faveur de la
largeur de la route, et aux dépens des petites boutiques le long des
murs, les deux carrosses passèrent en se frôlant, et finirent la
ridicule aventure.

Au partir de là, M. et M\textsuperscript{me} de Montbazon allèrent se
consulter à l'hôtel de Bouillon, qui, en pareil cas, avait autrefois
appris à vivre à M\textsuperscript{me} d'Hanovre, comme je l'ai raconté,
et de là à Versailles, où M. de Bouillon rendit compte au roi le
lendemain matin de ce qui était arrivé à son gendre et à sa fille.
Plusieurs ducs l'appuyèrent. Tout Versailles et tout Paris se leva
contre M\textsuperscript{me} de Mantoue et M\textsuperscript{me}
d'Elboeuf, qui avait fort crié qu'elle demanderait justice au roi.

Comme on était dans l'attente de ce qui en arriverait,
M\textsuperscript{me} de Mantoue entra chez M\textsuperscript{me} de
Lislebonne comme M\textsuperscript{me} la grande-duchesse en allait
sortir. Les gens de M\textsuperscript{me} de Mantoue voulurent faire
ranger ceux de M\textsuperscript{me} la grande-duchesse, et parmi ce
débat, M\textsuperscript{me} de Mantoue descendit de carrosse, trouva
vis-à-vis d'elle M\textsuperscript{me} la grande-duchesse prête à monter
dans le sien, qui se retirait de la bagarre, et à qui
M\textsuperscript{me} de Mantoue essaya de gagner le dessus. Cette
insolence était complète. Jamais duc de Mantoue n'avait rien disputé au
grand-duc, et d'une petite-fille de France à M\textsuperscript{me} de
Mantoue la distance était encore tout autre\,; aussi fut-elle bien
relevée, et contribua-t-elle fort à la réduction de tant de folies à
raison. M\textsuperscript{me} de Mantoue ne fit pas la moindre civilité
à M\textsuperscript{me} la grande-duchesse sur ce qu'il se passait\,;
mais vingt-quatre heures après, elle eut ordre d'aller demander pardon à
M\textsuperscript{me} la grande-duchesse, qui, amie de
M\textsuperscript{me} de Lislebonne, passa la chose doucement.

M\textsuperscript{me} d'Elboeuf fit écrire sa fille sur l'aventure de M.
de Montbazon à Torcy, comme ministre des affaires étrangères. Elle n'eut
point de réponse. Elle écrivit une seconde fois longuement et fort
hautement. Torcy en rendit compte une seconde fois, porta la lettre au
conseil\,; elle y fut moquée et trouvée très ridicule. Torcy eut ordre
de lui conseiller d'abandonner cette affaire, dont elle ne tirerait
aucune raison, et de ne se pas commettre à s'en faire de nouvelles. La
mortification fut si publique et si sensible, qu'elle corrigea enfin
M\textsuperscript{me} de Mantoue de tout hasarder, et la persuada enfin
d'abandonner ses projets pour éviter de nouveaux dépôts. Elle comprit
qu'ils ne se pouvaient soutenir destitués des protections dont elle
s'était flattée, et qu'elle et sa mère étaient trop faibles pour en
faire réussir aucun.

Elle se résolut donc à renoncer à la cour, où on ne voulait point
d'elle, et à des prétentions qui la renfermaient chez elle dans la
solitude et l'ennui. Elle prit maison à Paris, envoya complimenter
toutes les dames un peu considérables, dans l'espérance de les engager à
la première visite. Voyant que la tentative ne réussissait pas, elle fit
répandre tant qu'elle put qu'elle ne savait sur quoi fonder qu'on lui
croyait des prétentions chimériques, qu'elle désirait qu'on fût persuadé
qu'elle ne vouloir pas vivre autrement que si elle était encore fille,
qu'elle était offensée qu'on s'imaginât autre chose, qu'elle comptait
être si attentive à toutes sortes de devoirs et de politesse qu'on ne
pourrait s'empêcher de l'aimer, et de vouloir vivre avec elle. Telle fut
son amende honorable au public, après tant de tentatives, inutiles de
force ou d'adresse.

Les choses ainsi préparées, elle la fit en personne\,: elle se mit à
faire des visites sans plus en attendre de premières, et dans un seul
carrosse à deux chevaux comme tout le monde. Elle accabla de civilités
et de caresses les dames qu'elle trouva, et redoubla même une seconde
visite à quelques-unes avant d'en avoir reçu d'elles. La duchesse de
Lauzun fut de ce dernier nombre, qui, bien sûre de son fait, la fut voir
ensuite. Elle fut reçue avec des remerciements infinis, eut un fauteuil,
la main sans équivoque\,; et en sortant fut conduite par
M\textsuperscript{me} de Mantoue à travers trois pièces entières, sans
qu'il fût possible de l'en empêcher, et au degré par sa soeur bâtarde
qui lui servait de dame d'honneur, et quelques demoiselles. Elle en usa
ainsi avec toutes les femmes titrées\,; et pour les autres elle les
reçut sans affectation sur rien, avec une grande politesse, leur
laissant les fauteuils à l'abandon, et les conduisant honnêtement.

Une conduite si différente de ses premiers essais lui réconcilia bientôt
le monde. Elle acheva de se l'attirer par un grand jeu de lansquenet
fort à la mode alors, qu'elle tint avec beaucoup d'égards, et assez de
dignité pour qu'il ne s'y passât rien de mal à propos. Ainsi fondit tout
à coup en un brelan public ce grand rang de souveraine, dont le modèle
le plus juste en avait été choisi sur celui des petites-filles de
France, et sans prétendre leur céder, comme on l'a vu, à l'égard de
M\textsuperscript{me} la grande-duchesse\,; et à tous les projets de
figurer grandement à la cour, succédèrent les soins de se faire une
bonne maison dans Paris. La chute fut grande et amère, et de plus,
souvent accompagnée d'embarras de subsistances dans un temps où celles
des armées absorbaient tout, et Desmarets, ne se mettant pas fort en
peine de ses besoins depuis qu'elle lui eut dit, un peu imprudemment,
qu'ils pouvaient juger qu'ils étaient grands, puisqu'elle venait
elle-même les lui demander.

La duchesse de Lesdiguières, à ce spectacle, se remercia de nouveau, et
s'applaudit de plus en plus d'avoir résisté aux persécutions du duc de
Mantoue, aux empressements extrêmes de M. le Prince, et {[}à{]} tout ce
que le roi voulut bien faire de démarches pour la faire consentir à
l'épouser, quoiqu'il soit à croire que, mariée de sa main et par
obéissance, et n'ayant pas dans la tête les chimères que l'autre étala
d'abord, elle eût été tout autrement traitée. M\textsuperscript{me} de
Mantoue ne vit et n'ouït parler d'aucune princesse du sang, ni Madame,
ni M\textsuperscript{me} la duchesse d'Orléans.

La cherté de toutes choses, et du pain sur toutes, avait causé de
fréquentes émotions dans toutes les différentes parties du royaume.
Paris s'en était souvent senti\,; et quoiqu'on eût fait demeurer près
d'une moitié plus que l'ordinaire du régiment des gardes, pour la garde
des marchés et des lieux suspects, cette précaution n'avait pas empêché
force désordres, en plusieurs desquels Argenson courut risque de la vie.

Monseigneur, venant et retournant de l'Opéra, avait été plus d'une fois
assailli par la populace et par des femmes en grand nombre, criant
\emph{du pain\,!} jusque-là qu'il en avait eu peur au milieu de ses
gardes, qui ne les osaient dissiper de peur de pis. Il s'en était tiré
en faisant jeter de l'argent et promettant merveilles\,; mais comme
elles ne suivirent pas, il n'osait plus venir à Paris.

Le roi en entendit lui-même d'assez fortes, de ses fenêtres, du peuple
de Versailles qui criait dans les rues. Les discours étoient hardis et
fréquents, et les plaintes vives et fort peu mesurées contre le
gouvernement, et même contre sa personne, par les rues et par les
places, jusqu'à s'exhorter les uns les autres à n'être plus si
endurants, et qu'il ne leur pouvait arriver pis que ce qu'ils
souffraient, et de mourir de faim.

Pour amuser ce peuple, on employa les fainéants et les pauvres à raser
une assez grosse butte de terre qui était demeurée sur le boulevard
entre les portes Saint-Denis et Saint-Martin\,; et on y distribuait par
ordre de mauvais pain aux travailleurs pour tout salaire, et en petite
quantité à chacun.

Il arriva que le mardi matin, 20 août, le pain manqua sur un grand
nombre. Une femme entre autres cria fort haut, ce qui en excita
d'autres. Les archers préposés à cette distribution menacèrent la
femme\,: elle n'en cria que plus fort\,; les archers la saisirent et la
mirent indiscrètement à un carcan voisin. En un moment tout l'atelier
accourut, arracha le carcan, courut les rues, pilla les boulangers et
les pâtissiers. De main en main les boutiques se fermèrent. Le désordre
grossit et gagna les rues de proche en pioche sans faire mal à personne,
mais criant \emph{du pain\,!} et en prenant partout.

Le maréchal de Boufflers, qui ne pensait à rien moins, était allé ce
matin-là chez Bérenger son notaire, dans ce voisinage-là. Surpris de
l'effroi qu'il y trouva, et en apprenant la cause, il voulut aller
lui-même tâcher de l'apaiser, malgré tout ce que le duc de Grammont,
qu'il trouva chez le même notaire, pût lui dire pour l'en détourner, et
qui, l'y voyant résolu, alla avec lui. À cent pas de chez ce notaire,
ils rencontrèrent le maréchal d'Huxelles dans son carrosse, qu'ils
arrêtèrent pour lui demander des nouvelles, parce qu'il venait du côté
de l'émotion. Il leur dit que ce n'était plus rien, les voulut empêcher
de passer outre, et pour lui gagna pays en homme qui n'aimait pas le
bruit et être fourré parmi ce désordre. Le maréchal et son beau-père
continuèrent d'aller, trouvant à mesure qu'ils avançaient une grande
épouvante, et qu'on leur criait des fenêtres de retourner, et qu'ils se
feraient assommer.

Arrivés au haut de la rue Saint-Denis, la foule et le tumulte firent
juger au maréchal de Boufflers qu'il était temps de mettre pied à terre.
Il s'avança ainsi à pied avec le duc de Grammont parmi ce peuple infini
et furieux, à qui le maréchal demanda ce que c'était, pourquoi tout ce
bruit, promettant du pain, et leur parlant de son mieux avec douceur et
fermeté, leur remontrant que ce n'était pas là comme il en fallait
demander. Il fut écouté, il y eut des cris à plusieurs reprises de
\emph{vive M. le maréchal de Boufflers}, qui s'avançait toujours parmi
la foule et lui parlait de son mieux. Il marcha ainsi avec le duc de
Grammont le long de la rue aux Ours, et dans les rues voisines jusqu'au
plus fort de cette espèce de sédition. Le peuple le pria de représenter
au roi sa misère et de lui obtenir du pain. Il le promit, et sur sa
parole, tout s'apaisa et se dissipa, avec des remercîments et de
nouvelles acclamations de \emph{vive M. le maréchal de Boufflers\,!} Ce
fut un véritable service.

Argenson y marchait avec des détachements des régiments des gardes
françaises et suisses, et sans le maréchal il y aurait eu du sang
répandu qui aurait peut-être porté les choses fort loin. On faisait même
déjà monter à cheval les mousquetaires.

À peine le maréchal était-il rentré chez lui à la place Royale avec son
beau-père, qu'il fut averti que la sédition était encore bien plus
grande au faubourg Saint-Antoine. Il y courut aussitôt avec le duc de
Grammont, et l'apaisa comme il avait fait l'autre. Il revint après chez
lui manger un morceau, et s'en alla à Versailles. Il ne voulut que sa
chaise de poste, un laquais derrière et personne avec lui à cheval
jusqu'au cours, affectant de traverser tout Paris de la sorte. À peine
fut-il sorti de la place Royale, que le peuple des rues et les gens de
boutiques se mirent à crier qu'il eût pitié d'eux, qu'il leur fit donner
du pain\,; et toujours\,: \emph{Vive M. le maréchal de Boufflers\,!} Il
fut conduit ainsi jusqu'au quai du Louvre.

En arrivant à Versailles, il alla droit chez M\textsuperscript{me} de
Maintenon où il la trouva avec le roi, tous deux bien en peine. Il
rendit compte de ce qui l'amenait et reçut de grands remercîments. Le
roi lui offrit le commandement de Paris, troupes, bourgeoisie, police,
etc., et le pressa de l'accepter\,; mais le généreux maréchal préféra à
cet honneur le rétablissement des choses dans leur ordre naturel. Il dit
au roi que Paris avait un gouverneur auquel il ne déroberait pas les
fonctions qui lui appartenaient, qu'il était honteux qu'il ne lui en
restât pas une et que le lieutenant de police et le prévôt des marchands
les eussent toutes emblées et partagées, jusque sur les troupes, et
engagea le roi dans ces moments de crainte de les rendre au duc de
Tresmes, qui les avait si bien perdues, ainsi que ses derniers
prédécesseurs, qu'il lui fallut expédier une patente nouvelle pour lui
rendre son autorité.

Il fut donc enjoint aux troupes et aux bourgeois de ne recevoir d'ordres
que du gouverneur, et de lui obéir en tout et partout à d'Argenson,
lieutenant de police, et à Bignon, prévôt des marchands, de lui rendre
compte de tout et lui être soumis en tout, ainsi que tous les différents
corps de la ville.

Le duc de Tresmes fut envoyé à Paris y exercer ce pouvoir, mais avec
ordre de ne rien faire sans le maréchal de Boufflers, à l'obéissance
duquel Argenson, Bignon, la bourgeoisie et les troupes furent aussi
soumis, mais par des ordres verbaux\,; et le maréchal fut aussi renvoyé
demeurer à Paris. Sa modestie lui donna une nouvelle gloire. Il renvoya
tout au duc de Tresmes, au nom et par l'ordre duquel tout se fit, et
chez qui il allait pour les délibérations qu'il ne voulut presque jamais
souffrir chez lui. Maître et tuteur en effet du duc de Tresmes, et le
vrai commandant, il s'en disait au plus l'aide de camp, et en usait de
même.

Aussitôt après on pourvut bien soigneusement au pain, Paris fut rempli
de patrouilles, peut-être un peu trop, mais qui réussirent si bien qu'on
n'entendit pas depuis le moindre bruit. Le duc de Tresmes et le maréchal
de Boufflers qui lui laissait jusqu'au scrupule l'honneur et l'apparence
de tout, allaient de temps en temps rendre compte au roi eux-mêmes, mais
sans découcher de Paris, puis rarement, jusqu'à ce qu'il ne fut plus
question de rien.

La considération de Boufflers, rehaussée de la modestie la plus simple,
était alors à son comble\,: maître dans Paris, modérateur des affaires
de la guerre, influant sur toutes les affaires à la cour. Mais la durée
de ce brillant ne fut pas longue et finit par ce qui le devait rendre et
plus solide et plus durable. On verra bientôt Voysin et Tresmes
affranchis de sa tutelle, Voysin devenir le maître et l'instrument de
tout, Argenson et. Bignon reprendre toutes les usurpations de leurs
places, et celle de gouverneur de Paris anéantie comme elle était
auparavant ces mouvements de Paris.

\hypertarget{chapitre-xxi.}{%
\chapter{CHAPITRE XXI.}\label{chapitre-xxi.}}

1709

~

{\textsc{Campagne d'Espagne.}} {\textsc{- Faute de Besons, à qui le roi
ne permet pas d'accepter la Toison.}} {\textsc{- Campagne de
Roussillon.}} {\textsc{- Campagne de Savoie.}} {\textsc{- Campagne de
Flandre.}} {\textsc{- Artagnan s'empare de Warneton.}} {\textsc{-
Tournai assiégé, Surville dedans.}} {\textsc{- La ville rendue.}}
{\textsc{- Voyage bizarre de Ravignan à la cour.}} {\textsc{- Citadelle
de Tournai rendue\,; la garnison prisonnière.}} {\textsc{- Mesgrigny se
donne aux ennemis et en conserve le gouvernement.}} {\textsc{- Surville
perdu pour toujours.}} {\textsc{- Calomnie sur Chamillart.}} {\textsc{-
Digne conduite de Beauvau, évêque de Tournai.}} {\textsc{- Boufflers
s'offre d'aller seconder Villars sans commandement\,; remercié, puis
accepté.}} {\textsc{- Conduite des deux maréchaux ensemble.}} {\textsc{-
Roi Jacques d'Angleterre.}} {\textsc{- Mons fort mal pourvu.}}
{\textsc{- Électeur de Bavière à Compiègne.}} {\textsc{- Campagne
d'Allemagne.}} {\textsc{- Projet sur la Franche-Comté.}} {\textsc{-
Conspiration dans cette province découverte.}} {\textsc{- Mercy défait
par du Bourg\,; sa cassette, etc., prise.}} {\textsc{- Du Bourg
chevalier de l'ordre.}} {\textsc{- Cassette de Mercy.}} {\textsc{-
Voyage plus que suspect de Vaudémont et de M\textsuperscript{lle} de
Lislebonne.}} {\textsc{- Procédures, etc., et suites.}} {\textsc{-
Courte réflexion sur la conduite de nos rois et de la maison de
Lorraine.}} {\textsc{- Pièce importante de la cassette de Mercy.}}

~

Besons, sorti enfin de ses quartiers, avait reçu quatre différents
contre-ordres. Ces incertitudes n'affermirent pas un homme naturellement
timide, et qui mourait toujours de peur de déplaire et de ne réussir
pas\,; aussi manqua-t-il la plus belle occasion du monde de défaire les
ennemis au passage de la Sègre. Il fut pressé d'en profiter, il le
voulut, puis il n'osa\,; la fin de tout cela fut qu'il ramena ses
troupes en France. L'armée de l'archiduc qui fut au moment d'être
perdue, la Sègre à moitié passée par ces contretemps, en sut profiter.
Notre cour en blâma fort Besons d'avoir été si exact à ses ordres
quoique très précis. Celle d'Espagne, outrée sur le reçu de ses
officiers généraux, prit un parti d'éclat. Philippe V partit brusquement
pour son armée, mais il marcha à trop petites journées. La reine
l'accompagna les trois premières, et retourna régente à Madrid. Besons
paya de respects\,; d'obéissance et de raisons, laissant faire le roi,
mais lui représentant les inconvénients qui se vérifièrent exactement
tous par l'expérience.

Le roi d'Espagne, qui avait été fort approuvé de notre cour d'avoir pris
le commandement lui-même, s'adoucit sur Besons jusqu'à lui vouloir
donner la Toison dans la vue des besoins\,; mais le roi ne voulut pas
lui permettre de l'accepter. De Bay qui la portait valait moins que lui,
s'il se peut, pour la naissance, et on la vit donner depuis encore plus
bassement.

Le roi d'Espagne, fâché de se voir hors de portée de rétablir les
choses, et de réparer ce qui avait été manqué, quitta l'armée au bout de
trois semaines, et retourna à Madrid plus vite qu'il n'en était venu.
Besons mit ordre à la subsistance et aux quartiers des vingt-six
bataillons qu'il devait laisser en Espagne sous Asfeld, et repassa les
Pyrénées avec le reste de ses troupes.

Telle fut la dernière campagne des Français en Espagne, puisque celles
qui y étaient restées rentrèrent en France avant l'ouverture de la
campagne suivante, et mirent ainsi d'accord les deux cabales après tant
de bruit pour et contre retour. Il fut funeste à l'Espagne et peu utile
à la France, fruit d'un genre de gouvernement tel que celui que nous
éprouvions depuis plusieurs années, et qui, sans un miracle tout à fait
étranger, eût perdu ce royaume sans aucunes ressources.

En Roussillon, l'objet est trop petit pour s'arrêter à des détails. Le
duc de Noailles, avec le peu qu'il y avait, eut affaire à moins encore.
Il y battit deux fois les ennemis, qu'il surprit dans des quartiers, et
ces légers succès retentirent fort à Versailles.

Berwick, sur la défensive, n'eut pas grand'chose à faire en Dauphiné. Le
duc de Savoie s'y remua tard et mollement. Il était fort mécontent de
l'empereur sur des fiefs de l'empire de son voisinage, que le feu
empereur lui avait promis, et que, celui-ci ne voulut pas lui donner.
D'autres discussions de quartiers et de subsistances de troupes
achevèrent de les brouiller, tellement que M. de Savoie ne se soucia
point de profiter des avantages solides qu'il s'était préparés dans la
campagne précédente pour celle-ci. Elle se passa en bagatelles, qui
auraient pu aisément devenir utiles, et avoir des suites heureuses, par
l'adresse du duc de Berwick, si le manque de vivres ne l'eut arrêté tout
court. Il ne laissa pas de battre Reybender, général des troupes de
Savoie, qui, avec trois mille hommes, voulut, le 28 août, attaquer
auprès de Briançon une maison appelée \emph{la Vachette}, que Dillon
avait retranchée. Dillon les fit attaquer de droite et de gauche par des
piquets et quelques compagnies de grenadiers, leur tua sept cents hommes
et rechassa le reste dans la montagne.

La Flandre, dès l'ouverture de la campagne, fut l'objet principal, pour
ne pas dire l'unique, de toute l'attention et de toutes les inquiétudes,
et le fut jusqu'à la fin de la campagne. Le prince Eugène et le duc de
Marlborough, joints ensemble, continuaient leurs vastes desseins et de
dédaigner de les cacher. Leurs amas prodigieux annonçaient des sièges.
Dirai-je que notre faiblesse les désirait, et que nous ne comptions sur
notre armée que pour les conserver\,?

Il est pourtant vrai qu'Artagnan, détaché avec huit bataillons de
l'armée et quatre de la garnison d'Ypres, commandés pour le joindre au
rendez-vous, enleva Warneton fort aisément, où les ennemis avaient mis
seize cents hommes avec quelques munitions dans le dessein de le
fortifier. Ces seize cents hommes se rendirent à discrétion, commandés
par un brigadier et quarante-cinq officiers\,; et que le maréchal de
Villars eut encore un autre petit avantage à un fourrage\,; mais
c'étaient des bagatelles.

L'orage se forma sur Tournai, comme je l'ai déjà dit, et où Surville
commandait, et Mesgrigny aussi, lieutenant général et gouverneur
particulier de la citadelle, avec les troupes dont j'ai fait mention. La
tranchée fut ouverte la nuit du 7 au 8 août. Le maréchal de Villars
laissa former ce siège et ne fit aucune contenance de s'y opposer,
content de subsister et de tenir force propos. Il faut dire aussi que le
pain lui était fourni plus régulièrement, que l'argent n'y arrivait que
peu à peu et par de très petites sommes, et que tout y était à craindre
de la désertion et du découragement. Surville {[}ne{]} tint que vingt
jours et battit la chamade le 28 juillet au soir. Il envoya le chevalier
de Rais au roi, qu'il trouva à Marly, et qui dit que la garnison n'était
que de quatre mille cinq cents hommes, réduite alors à trois mille
hommes pour entrer dans la citadelle, qu'il y avait des brèches de
trente toises aux trois attaques, que l'ouvrage à cornes des sept
fontaines avait été emporté avec le bastion voisin et le réduit de
l'ouvrage, et que l'assaut s'allait donner par les trois attaques à la
fois. On attendait mieux que cela d'un homme si fraîchement remis à flot
par la générosité du maréchal de Boufflers, et qui avait été témoin de
si près de sa défense dans Lille.

Le chevalier de Rais apprit qu'ils avaient toujours attaqué la citadelle
par un côté en même temps que la ville, que la capitulation portait
qu'ils ne la pourraient pas attaquer par la ville, et il assura qu'il y
avait dedans quantité de munitions de guerre, pour trois mois de
farines, quelques vaches et cinq cents moutons. Il y avait aussi six
cents invalides, dont la moitié peu en état de bien servir.

Le chevalier de Rais était arrivé à Marly le jeudi 1er août. Le mardi 6,
on y fut extrêmement surpris d'y voir Ravignan entrer, chez
M\textsuperscript{me} de Maintenon où était le roi, mené par Voysin, où
quelques moments après le maréchal de Boufflers fut appelé. Un envoi
aussi bizarre excita une grande curiosité. Le désir et le besoin
persuadaient qu'il pouvait être question de paix, d'autant qu'il
transpira assez promptement que, depuis la capitulation de la ville,
Surville était festoyé par les vainqueurs, et qu'ils ne devaient faire
aucune hostilité jusqu'au retour de Ravignan, fixé au 8 au soir. Enfin
on sut le mystère.

Les ennemis proposaient une suspension d'armes limitée à un temps
raisonnablement estimé que la citadelle pourrait se défendre, qui au
bout de ce temps convenu se rendrait sans être attaquée, et que
cependant les deux armées subsisteraient, à une certaine distance l'une
de l'autre et de la place, sans aucun acte d'hostilité. La proposition
parut aussi étrange que nouvelle\,; et on fut étonné que, Ravignan,
homme de sens et qui avait acquis de l'honneur dans Lille, où il avait
été fait maréchal de camp, se fût chargé de la venir faire. Une
suspension d'armes sans vues de paix, un temps marqué pour rendre une
place sans qu'elle fût attaquée, parurent des choses inouïes, un désir
des ennemis de ménager leur peine, leur argent, leurs fourrages, auquel
on ne crut pas devoir consentir, avec le mépris de notre armée qui, par
cette proposition, n'était pas estimée en état ni en volonté de rien
tenter pour le secours. Surville fut fort blâmé de l'avoir écouté, et
Ravignan de l'avoir apportée, qui fut renvoyé sur-le-champ avec le
refus.

On crut que la réputation de la place avait été le motif d'une
proposition si extraordinaire. Mesgrigny, le premier ingénieur après
Vauban, quoique inférieur en tout, avait bâti cette citadelle à plaisir,
et comme pour lui, parce qu'il en était gouverneur. C'était une des
places de toutes celles que le roi a faites des meilleures et des plus
régulièrement bâties, avec des souterrains excellents partout, et qui
surprenaient par leur hauteur et leur étendue\,; contre-minée sous tous
les ouvrages et jusque sous les courtines, ce qui bien manié allonge
fort un siège, déconcerte les assaillants qui ne savent où asseoir le
pied, et qui rebute fort le soldat. Rien n'était mieux fondé que la
réputation de cette place, rien ne lui fut si inutile que toutes ces
admirables précautions pour la conserver, ou pour la vendre du moins
chèrement.

Elle capitula le 2 septembre, sans avoir essuyé aucun coup de main. Cela
parut un prodige inconcevable. Un autre qui ne le fut pas moins, c'est
que Mesgrigny, qui avait quatre-vingts ans, et qui de tout le siège de
la ville et de la citadelle ne sortit presque point de sa chambre, n'eut
pas honte de déshonorer sa vieillesse en se donnant aux ennemis, qui
donnèrent le gouvernement de la ville au comte d'Albemarle, et
conservèrent celui de la citadelle à ce malheureux vieillard, qui avait
aidé le maréchal de Boufflers à la défense de Namur, et qui en avait été
fait lieutenant général.

Surville vint saluer le roi, et n'en fut pas mal reçu, autre surprise\,;
mais ce qu'une si molle défense lui devait coûter, et en un temps où il
était si important d'amuser longtemps les ennemis devant la place, si on
ne la pouvait sauver, il le reçut de son indiscrétion qui l'avait déjà
coulé à fond une fois. Il avait mangé plusieurs fois avec le prince
Eugène et le duc de Marlborough, entre les deux sièges et après la
dernière capitulation. On y parla du maréchal de Villars, qui prétendit
y avoir été maltraité, et que Surville, ou complaisant ou en pointe de
vin, ne l'avait pas ménagé. Surville aussi était blessé contre le
maréchal de n'avoir pas fait la moindre démonstration pour son secours,
en sorte que les plaintes furent vives de part et d'autre. Surville
pourtant, ne se sentant pas le plus fort, voulut capituler, mais il
trouva un homme aisé à prendre le montant, et qui, plein de sa fortune,
ne pardonnait point.

Outre ce point de Villars, on répandit que les deux généraux ennemis
parlèrent à Surville, et à table, des dernières conditions de paix qui
firent rompre Torcy à la Haye, et qu'ils dirent qu'on n'aurait jamais
osé proposer au roi de procurer, lui-même, par la force, la destitution
du roi d'Espagne, comme une chose qui était contre la nature et contre
toute bienséance, si un de ses principaux ministres, désignant
Chamillart, et dont le nom enfin leur échappa, ne leur en eût donné la
hardiesse, en écrivant au duc de Marlborough, qui en avait encore la
lettre, que dès qu'il ne s'agirait que du retour du roi d'Espagne, cet
article n'arrêterait pas la paix, et qu'il ne craignait pas, en
l'avançant de la sorte, d'être désavoué du roi.

Ce propos fit grand bruit et fut extrêmement relevé par les ennemis du
malheureux ex-ministre. Je lui demandai depuis si cela avait quelque
fondement. Il m'assura que longtemps avant de sortir de place, il ne
s'était plus mêlé de la paix, et que pour cette lettre rien n'était plus
faux ni plus absurde. Cela ne laissa pas d'exciter contre lui des
murmures désagréables. Pour Surville, il demeura perdu sans retour. Il
s'enterra chez lui, en Picardie, fort mal à son aise d'ailleurs, et on
ne le vit presque plus.

Beauvau, qui de Bayonne avait passé à l'évêché de Tournai, fit merveille
de sa personne pendant le siège et de sa bourse autant et plus qu'elle
se put étendre. Il offrit même à Surville de prendre l'argenterie des
églises. Il n'imita pas M. de Fréjus, il refusa nettement de chanter le
\emph{Te Deum}, dont il fut pressé avec toutes les caresses possibles,
encore plus de prêter serment, et partit le matin du jour du \emph{Te
Deum}, et avant l'heure de le chanter. Le roi le reçut très bien, et
l'entretint seul trois quarts d'heure. C'est le même qu'il fit
archevêque de Toulouse, qui passa après à Narbonne, et qui eut l'ordre
avec son frère à la grande promotion de M. le Duc, en 1724. Le rare est
qu'il fut beaucoup mieux traité sur les choses de la religion par le duc
de Marlborough que par le prince Eugène.

On avait été surpris qu'ils eussent préféré de s'attacher à ce grand
siège, à tâcher de pénétrer du côté de la mer. Villars, à la vérité,
s'était avantageusement posté à l'ouverture de la campagne pour les en
empêcher\,; mais il n'aurait pu parer les diverses façons de le tourner,
et au pis aller, s'ils eussent voulu, le forcer à un combat. Les uns
jugèrent que, plus soigneux de s'avancer solidement et commodément, par
les facilités que leur apportaient ces grandes conquêtes, que de se
hâter de pénétrer en se laissant des derrières contraignants, ils
avaient préféré les grands sièges pour se porter plus sûrement et plus
durablement en avant. D'autres, plus flatteurs et plus occupés de faire
leur cour que des raisonnements justes, prétendaient que les Hollandais,
qu'on s'opiniâtrait à se vouloir figurer désireux de la paix, s'étaient
opposés aux desseins du côté de la mer, et {[}avaient{]} emporté celui
de Tournai, pour amuser le temps de la campagne par quelque chose
d'utile et de spécieux, mais moins dangereux pour la France, écouler
ainsi l'été jusqu'au temps de remettre les négociations sur le tapis,
que le poids des dépenses pourrait rendre plus faciles de la part de
l'empereur et de l'Angleterre. On s'endormait ainsi à la cour sur ces
idées trompeuses\,; elle tâchait de les inspirer aux différentes parties
de l'État, moins soigneuse des affaires que de fermer les bouches par
persuasion ou par terreur. Le roi s'expliquait souvent sur ce qu'il
appelait les discoureurs\,; et on devenait coupable d'un crime sensible,
quelque borine intention qu'on eut en parlant, sitôt qu'on s'écartait un
peu de la fadeur de la \emph{Gazette de France}, et de celle des bas
courtisans.

Sur la fin du siège de la citadelle de Tournay, Boufflers sentit
l'étrange poids des affaires de Flandre\,; et s'inquiéta de ce qu'un
seul homme en était chargé, qui, mis hors de combat par maladie ou par
quelque autre accident, ne pourrait être remplacé à l'instant, et dans
des circonstances si pressantes et si critiques. Pénétré de ce danger,
il en parla au roi, lui dit qu'il voyait que tout se disposait à une
bataille, lui représenta le péril de son armée, si par un accident
arrivé à Villars elle tombait dans une anarchie dans des moments si
décisifs. Tout de suite, il s'offrit de l'aller seconder, d'oublier tout
pour lui obéir, n'être que son soulagement, et rien dans l'armée que par
lui, et à portée seulement de le suppléer en cas d'accident à sa
personne.

Pour comprendre la grandeur de ce trait, digne de ces Romains les plus
illustres des temps de la plus pure vertu de leur république, je
m'arrêterai ici un moment. Boufflers, au comble des honneurs, de la
gloire, de la confiance, n'avait qu'à demeurer en repos, à jouir d'un
état si radieux, avec une santé qui ne lui avait pas permis de commander
l'armée. Il était parvenu, avec réputation, à être chevalier de l'ordre
et de la Toison d'or, colonel, puis capitaine des gardes, et avait
justement sur le coeur d'avoir été forcé de quitter la première charge
pour l'autre. Maréchal de France, duc et pair, gouverneur de Flandre
{[}avec{]} la survivance pour son fils, maître et modérateur de Paris,
avec les entrées de premier gentilhomme de la chambre, la privante et la
confiance du roi et de M\textsuperscript{me} de Maintenon, et la tutelle
du ministre de la guerre, la gloire qu'il avait acquise forçait l'esprit
à applaudir à une si grande fortune\,; sa générosité, son
désintéressement, sa modestie, engageait les coeurs à s'y complaire très
bien avec Monseigneur et avec Mgr le duc de Bourgogne, il n'y avait
princes du sang, même bâtards ni ministres, ni seigneurs qui ne fussent
obligés de compter avec lui\,; et lui, au delà des grâces, des honneurs,
des récompenses et de toute espèce de lustre, il s'offrait d'aller
compter avec un homme avantageux, tout personnel, jaloux de tout, sans
principes, accoutumé à tout gain, à usurper la réputation d'autrui, à
faire siens les conseils et les actions heureuses, et à jeter aux autres
tous mauvais succès et ses propres fautes. Le comble est que Boufflers
ne l'ignorait pas, qu'il connaissait l'impudence de sa hardiesse, l'art
de ses discours, le faible pour lui du roi et de M\textsuperscript{me}
de Maintenon, et que c'était sous un tel homme, son cadet à la guerre de
si loin, maréchal de France près de dix ans après lui, et dans son
propre gouvernement où il venait de défendre Lille, qu'il allait se
mettre à sa merci pour le bien de l'État, et exposer une réputation si
grande, si, pure, si justement acquise, à la certitude de l'envie, et à
l'incertitude des succès, même dans la main d'un autre.

Boufflers vit tout cela, il le sentit dans toute son étendue, mais tout
disparut devant lui à la lueur du bien de l'État. Il pressa le roi\,; et
le roi qui n'en voyait pas tant, bien moins encore la magnanimité d'une
pareille offre, le loua, le remercia, et ne crut pas en avoir besoin,
sans en sentir le prix.

Dix ou douze jours après, Boufflers n'y pensant plus, le roi fit des
réflexions, l'envoya chercher et le fit entrer par les derrières. Ce fut
pour lui dire qu'il lui ferait plaisir d'aller en son armée de Flandre,
en la manière qu'il le lui avait offert. Le maréchal, qui pour la
première fois de sa vie se trouvait attaqué d'une goutte douloureuse, et
qui avait eu peine à se traîner jusque dans le cabinet du roi, lui
réitéra tout ce qu'il lui avait dit la première fois sur la conduite
qu'il se proposait de garder religieusement avec Villars, prit ses
derniers ordres, s'en alla à Paris et partit le lendemain lundi, 2
septembre, pour aller trouver le maréchal de Villars, c'est-à-dire le
jour même que la citadelle de Tournai se rendit. On fut vingt-quatre
heures à le savoir parti sans en deviner la cause. L'affolement de la
paix était à un point qu'on crut qu'il était allé moins pour la négocier
que pour la conclure.

La surprise ne fut pas moins grande à l'armée, où il fut annoncé par un
courrier, dépêché exprès douze ou quinze heures avant son arrivée. La
même contagion saisit aussi l'armée, elle n'imagina que la paix.

Villars le reçut avec un air de joie et de respect, le pourvut de
chevaux et de domestiques, et lui communiqua d'abord tous ses projets.
Boufflers fut avec peine tiré de sa voiture, tant la goutte s'était
augmentée, qui néanmoins ne le tint pas longtemps dans sa chambre.
Villars voulut recevoir le mot de lui, au moins qu'il le donnât. Après
bien des compliments, ils le firent donner par le lieutenant général de
jour, à qui de concert ils expliquèrent l'ordre à donner à l'armée, et
depuis Villars donna toujours le mot et l'ordre, et Boufflers ne fit
plus la façon de vouloir les recevoir de lui. Le concert et
l'intelligence fut parfait entre eux\,: l'un avec des manières de
confiance et des égards toujours poussés au respect\,; l'autre sans
cesse soigneux d'admirer, de tout faire valoir, de tout déférer, et,
s'il avait quelque avis à ajouter, ou quelque observation à présenter,
c'était toujours avec les ménagements d'un subalterne honoré de la
confiance de son supérieur\,; du reste appliqué à éviter et à refuser
les hommages de l'armée, qui se portaient tous vers lui, à ne se mêler
immédiatement de rien, à ne se charger de quoi que ce fût, et à n'être
rien qu'auprès du maréchal de Villars, et encore tête à tête, et avec
toutes les mesures qui viennent d'être rapportées, dont il ne se
départit jamais.

De cette conduite réciproque, personne ne put juger de ce que Villars
pensa de se voir tomber tout à coup un tel second, qu'il n'avait point
demandé, s'il en fut peiné, s'il s'en trouva contraint, si dans
l'angoisse des affaires il fut bien aise d'être doublé, si sa vanité
satisfaite de conserver le généralat dans son entier, en présence d'un
maître à tous égards, la lui rendit agréable\,; en un mot rien ne s'en
put démêler.

Quoi qu'il en fût, ces deux généraux n'en firent qu'un seul. Boufflers,
fidèle à sa résolution, en garde contre l'air de censeur, donna dans
tout ce que Villars voulut, sans y former la moindre résistance, et avec
une bonne grâce qui dut l'élargir.

L'armée ennemie marcha vers Mons incontinent après la prise de la
citadelle de Tournai. Villars rappela tous les corps qu'il avait
détachés\,; et le roi d'Angleterre, qui sous l'incognito et le nom de
chevalier de Saint-Georges, faisait la campagne, volontaire comme
l'année précédente, accourut avec un reste de fièvre et sans consulter
ses forces. Il avait été obligé de s'éloigner un peu de l'armée par une
fièvre violente\,; mais il ne voulut pas consulter sa santé ni sa
faiblesse en des moments si précieux à la guerre.

Il y avait dans Mons peu de troupes et peu de vivres. L'électeur de
Bavière en sortit, s'arrêta peu à Maubeuge, et s'en alla à Compiègne.

La garnison de la citadelle de Tournai, quoique prisonnière de guerre,
fut conduite à Condé. Les ennemis lui laissèrent ses armes et son
bagage, et firent à Surville la galanterie de deux pièces de canon. Elle
était encore de trois mille hommes, et destinée pour échange de leurs
prisonniers faits à Warneton et ailleurs. Surville et Ravignan eurent
leur liberté, mais à condition que, si nous faisions des prisonniers de
leur grade, on leur en rendrait deux sans échange.

Ce qui termina de bonne heure la campagne du Rhin est trop important
pour ne pas couper celle de Flandre, afin de rapporter cet événement
dans son ordre.

Rien de plus insipide que cette campagne jusqu'à la mi-août. Les armées,
séparées par le Rhin, se contentaient de subsister. Harcourt laissa
Saint-Frémont à Haguenau garder nos lignes de Lauterbourg, et passa le
Rhin, les premiers jours d'août, sur un pont qu'il dressa à Altenheim,
pour faire subsister ses troupes aux dépens de l'ennemi, qui s'était
toujours tenu tranquille jusqu'alors derrière ses lignes de Dourlach, et
qui se contenta, sur le passage du duc d'Harcourt, de garnir les gorges
des montagnes pour l'empêcher de pénétrer. Le duc d'Hanovre, celui qui
fut fait électeur et qui a succédé à la reine Anne à la couronne
d'Angleterre, père de celui qui règne aujourd'hui, devait commander
l'armée impériale. Il n'y arriva que vers le 15 août. Il fit aussitôt
passer le Rhin à son armée qu'il mena camper auprès de Landau, sur quoi
M. d'Harcourt passa le Rhin sur le pont de Strasbourg, et se mit
derrière ses lignes.

Il se mûrissait cependant un dessein vaste, conçu ou pour le moins
nourri en Lorraine, comme la suite de la découverte ne permit pas d'en
douter, qui n'allait à rien moins qu'à porter -l'État par terre par le
côté le moins soupçonné.

M\textsuperscript{me} de Lislebonne avait une belle et grande terre à
l'extrémité de la Franche-Comté. Dans cette terre se tramait par le
bailli, par des curés et par les officiers de M\textsuperscript{me} de
Lislebonne, une conspiration qui, sous ces chefs, se répandit dans la
province, et y entraîna beaucoup de gens principaux des trois ordres,
gagna des membres du parlement de Besançon, avait pris ses mesures pour
égorger la garnison de cette place, s'en rendre maître, en faire autant
de quelques autres, et faire révolter la province en faveur de
l'empereur, comme étant, un fief et un domaine ancien de l'empire. Le
voisinage si proche de la Suisse et du Rhin, qui se traversait aisément
en de petits bateaux qu'on appelle des \emph{védelins}, facilitait le
commerce entre les Impériaux et les conspirateurs\,; et les gens de
M\textsuperscript{me} de Lislebonne faisaient toutes les allées et
venues.

Un perruquier, dont le grand-père avait servi utilement à la seconde
conquête de la Franche-Comté, fut sondé, puis admis dans le complot. Il
en avertit Le Guerchois, qui de l'intendance d'Alençon avait passé à
celle de Besançon, mon ami très particulier, comme on l'a vu ailleurs,
et de qui j'ai su ce que je rapporte. Le Guerchois l'écouta, et lui
ordonna de continuer avec les conspirateurs pour être en état de savoir
et de l'avertir, ce qu'il exécuta avec beaucoup d'esprit, de sens et
d'adresse.

Par cette voie, Le Guerchois sut qu'il y avait dans la conspiration de
trois sortes de gens\,: les uns, en petit nombre, voyaient les officiers
principaux que l'empereur y employait, venus exprès et cachés aux bords
du Rhin, de l'autre côté, et ceux qui les voyaient par les védelins
savaient tout et menaient véritablement l'affaire\,; les autres,
instruits par les premiers, mais avec réserve et précaution,
s'employaient à engager tout ce qu'ils pouvaient de gens dans cette
affaire, distribuaient les libelles et les commissions de l'empereur,
ils étaient l'âme de l'intrigue et les conducteurs dans l'intérieur de
la province\,; les derniers enfin étaient des gens qui, par désespoir
des impôts et de la domination française, s'étaient laissé gagner, et
qui étaient en très grand nombre.

Le Guerchois voulut encore davantage, et y fut également bien servi par
le perruquier. Il s'insinua si avant auprès du bailli de
M\textsuperscript{me} de Lislebonne et du curé de la paroisse où
demeurait ce bailli, qu'ils l'abouchèrent delà le Rhin avec un général
de l'empereur, et de chez eux avec les principaux chefs de leur
intelligence et de toute l'affaire dans la province. Il apprit d'eux
qu'un gros corps de troupes de l'empereur devait tenter, à force de
diligence, d'entrer en Franche-Comté, et tout risquer pour y pénétrer
s'il rencontrait des troupes françaises qui s'y opposassent.

Instruit de la sorte, Le Guerchois, qui en avait déjà communiqué au
comte de Grammont, lieutenant général, qui, quoique de la province, y
commandait et était fort fidèle, crut qu'il n'y avait point de temps à
perdre, et dépêchèrent un courrier au duc d'Harcourt et un autre au roi,
sans qu'on s'en aperçût à Besançon, où ils prirent doucement et sagement
leurs mesures.

Les choses en étaient là, lorsqu'un gros détachement de l'armée de
l'empereur se mit à remonter le Rhin par l'autre côté, pour joindre un
autre corps arrivé en même temps de Hongrie et mené par Mercy, qui donna
jalousie au duc d'Harcourt qu'ils ne voulussent faire le siège
d'Huningue, tandis que le gros de l'armée impériale, sous le duc
d'Hanovre s'approchait des lignes de Lauterbourg, et faisait contenance
de les vouloir attaquer.

Harcourt avait laissé le comte du Bourg dans la haute Alsace, avec dix
escadrons et quelques bataillons, qui cependant étaient inquiétés par le
duc d'Hanovre, dont le grand projet était l'exécution du dessein sur la
Franche-Comté, mais avec celui de tomber sur les lignes de Lauterbourg,
si d'Harcourt les dégarnissait trop en faveur du secours de la haute
Alsace. Parmi ces manéges de guerre, Harcourt, profitant du long détour
que les Impériaux détachés de leur armée ne pouvaient éviter pour tomber
par le haut Rhin où ils en voulaient, et averti par le courrier de
Franche-Comté, se tint en apparente inquiétude sur ses lignes\,; et dès
qu'il vit le détachement impérial déterminé, par ses marches forcées
dont il était bien informé, il envoya huit escadrons et cinq ou six
bataillons à du Bourg, avec ordre de combattre les ennemis, fort ou
faible, sitôt qu'il pourrait les joindre.

Pendant ces mesures, Mercy, avec ce qu'il avait amené de Hongrie,
traversa le Rhin à Rhinfelds, et un coin du territoire des Suisses avec
l'air de le violer, tandis que le détachement impérial se préparait à
jeter un pont à Neubourg, pour y passer aussi le Rhin, à peu près
vis-à-vis d'Huningue, et Mercy parut près de Brisach, résolu de
pénétrer, s'il pouvait, même sans attendre le détachement de l'armée
impériale qui le venait joindre parce pont de Neubourg.

Harcourt, exactement informé, détacha encore deux régiments de dragons
pour joindre du Bourg à tire-d'aile, et lui réitérer l'ordre de
combattre fort ou faible. Ces deux régiments de dragons arrivèrent tout
à propos, le jour devenait grand, et du Bourg faisait ses dispositions
pour attaquer Mercy, qu'il venait d'atteindre. Avec ce petit renfort, il
les attaqua vigoureusement, et quoique inférieur de quelque nombre, il
les enfonça\,; et en une heure et demie, il les défit d'une manière si
complète, que les Impériaux se sauvèrent de vitesse à grand'peine. Le
combat fut sanglant. On leur prit leurs canons, leurs équipages, presque
tous les bateaux de leur pont et beaucoup de drapeaux et d'étendards, le
carrosse de Mercy et sa cassette, qui se sauva à Bâle, et qui dut son
salut à la vitesse de son cheval, après avoir soutenu jusqu'au bout,
quoique blessé dangereusement. C'est le même Mercy qui commanda en 1734
l'armée impériale en Italie, et qui y fut tué à la bataille de Parme. Le
comte Bruner fut tué à ce combat d'Alsace, et quantité de leurs troupes,
dont on fit deux mille cinq cents prisonniers. On crut qu'il y avait eu
quinze cents tués, et plus de mille noyés dans le Rhin.

Du Bourg n'envoya rien au roi, mais, aussitôt après le combat il fit
partir d'Anlezy, de la maison de Damas, l'un des deux maréchaux de camp
qu'il avait avec lui, vers le duc d'Harcourt, qui, dans l'instant qu'il
le reçut, le fit repartir pour en porter la nouvelle au roi. Il arriva à
Versailles le soir du dernier août. Le roi l'avait su la veille par
Monseigneur, à qui M\textsuperscript{me} la Duchesse venait de montrer
une lettre de Dijon de M. le Duc, à qui du Bourg avait écrit un mot par
un officier du régiment de Charolais qui s'était trouvé à l'action, où
Saint-Aulaire, colonel de ce régiment, avait été tué, et qui venait de
la part du corps le demander à M. le Duc pour le major, le
lieutenant-colonel ne s'y étant pas trouvé.

Deux heures après que Mercy fut entré dans Bâle, il envoya un trompette
savoir ce qu'était devenu un officier lorrain, et prier, s'il était
prisonnier, de le lui vouloir renvoyer sur sa parole. Il était
prisonnier, et du Bourg, galamment, le lui renvoya sans réflexion sur
cet empressement. Le lendemain, il reçut un courrier de Le Guerchois,
qui lui mandait de prendre garde sur toutes choses à ce Lorrain, s'il
était pris, et le félicitait de sa victoire, qui sauvait la
Franche-Comté, et par conséquent la France, d'un embarras auquel il
serait resté peu de remèdes. Il n'était plus temps. Le Lorrain était en
sûreté\,; et la cassette de Mercy envoyée à Harcourt et par lui au roi,
ne causa que plus de regrets à l'indiscrète générosité de du Bourg, qui
demeura encore quelque temps sur le haut Rhin, qu'il n'eut pas de peine
à nettoyer des restes échappés d'une défaite complète, qui avaient
repassé ce fleuve comme ils avaient pu\,; et la campagne s'acheva avec
la même tranquillité qu'elle avait commencé.

M. d'Harcourt s'était avancé au fort Louis, sur ce que M. d'Hanovre
avait enfin fait repasser le Rhin à son armée, voyant qu'on n'avait
point pris le change qu'il avait essayé de donner, et marchait vers le
haut pour envoyer des renforts à Mercy. Mais il rebroussa dès qu'il eut
appris sa défaite\,; et M. d'Harcourt retourna vers ses lignes, où il ne
fut plus question que de subsister de part et d'autre jusqu'à la
séparation des armées.

Du Bourg fut aussitôt après sa victoire nommé chevalier de l'ordre\,;
d'Anlezy eut un cordon rouge\,; Quoadt, l'autre maréchal de camp de ce
petit corps de du Bourg, trois mille livres de pension\,; et Fontaine,
qui avait apporté les drapeaux et les étendards à Harcourt, qui l'avait
envoyé au roi, fut fait brigadier.

La cassette de Mercy découvrit bien moins de choses qu'elle n'apprit
qu'il y avait bien des mystères cachés. Elle manifesta la conspiration
dans la Franche-Comté, mais avec une grande réserve de noms, tout le
dessein d'y pénétrer par ses troupes et de s'y établir\,; et sans
fournir de preuves positives contre M. de Lorraine, elle ne laissa pas
douter qu'il n'y fût entré bien avant, et qu'il n'eût fomenté ce projet
de toutes ses forces. Sur quoi on peut voir dans les Pièces ce qui le
regarde dans le voyage de Torcy à la Haye.

Dès les premiers jours de mai, M. de Vaudemont, sous prétexte des eaux
de Plombières, était parti de Paris avec sa chère nièce,
M\textsuperscript{lle} de Lislebonne, pour se rendre en Lorraine, et
avaient été toujours depuis beaucoup plus assidus à Lunéville qu'à
Plombières, ni même à Commercy. Ils y étaient encore lors de ce combat,
et il fallait plus que de la grossièreté pour ne s'apercevoir pas, au
moins après cela, de la cause d'un voyage d'une si singulière longueur
fait si à propos et si fort en cadence. Ils séjournèrent encore un mois
après en Lorraine\,; et pour que la chose fût complète, ils en partirent
pour arriver à Marly dans le milieu d'un voyage. Ils en furent quittes
pour l'étonnement de tout le monde, mais muet, tant ils s'étaient rendus
redoutables. Il est vrai pourtant que le roi les reçut avec beaucoup de
froid et de sérieux.

Cependant Le Guerchois commença des procédures juridiques. Le bailli,
les officiers, quantité de fermiers de M\textsuperscript{me} de
Lislebonne, et le curé de sa principale paroisse, s'enfuirent et n'ont
pas reparu depuis\,; beaucoup de ses vassaux disparurent aussi. Les
preuves contre tous ces gens-là se trouvèrent complètes\,; il furent
contumacés et sentenciés. Un de ses meuniers, plus hardi, envoyé dans le
pays par les autres aux nouvelles, y fut pris et pendu avec plusieurs
autres. Quantité d'autres un peu distingués prirent le large à temps.

Tel fut le succès d'un complot si dangereux, parvenu jusqu'au point de
l'exécution, sans qu'on osât parler des plus grands et des plus
véritables coupables\,; ce qui, faute de preuves parfaites, s'étendit
jusqu'à des membres du parlement de Besançon, lequel on ne voulut pas
effaroucher. On se souviendra ici de ce qui a été rapporté des trahisons
de Vaudemont et de ses nièces, qui, au fait de tout à notre cour, ne
laissaient rien ignorer à Vienne par le canal de M. de Lorraine\,;
beaucoup d'autres gens, et quelques-uns distingués, s'absentèrent aussi.

Tel fut le succès des pratiques si dangereuses que la maison de Lorraine
n'a cessé de brasser contre la France et contre ses rois, depuis
François Ier jusqu'à la fin de Louis XIV, qui n'ont tous cessé de leur
prodiguer biens, honneurs, charges, faveur et rangs\,; et qui se sont
montrés sans cesse aussi infatigables à dissimuler, et à lui pardonner
ses crimes, qu'elle à en commettre toutes les fois qu'elle l'a pu, et de
montrer son éternel regret d'avoir manqué le grand coup de la Ligue, et
de n'avoir pu exterminer les Bourbons et leur arracher la couronne pour
se la mettre sur la tête\,: sentiment tellement inné en elle que les
moins capables d'entreprise et les plus comblés ne peuvent s'empêcher de
le laisser échapper, témoin ce qui est rapporté de M. le Grand (t. VI,
p.~2).

Il se trouva dans la cassette de Mercy un mémoire instructif du prince
Eugène à ce général, dont plusieurs endroits étaient d'une obscurité
mystérieuse difficile à pénétrer. On y lut entre autres choses qu'il
fallait tout tenter pour remettre la France hors d'état à jamais
d'inquiéter l'Europe, et de plus sortir de ses limites, où il fallait la
rappeler, et, si on n'y pouvait enfin réussir par les armes, on serait
obligé d'avoir recours aux grands et derniers remèdes. Cela, avec
d'autres choses qu'on tint fort secrètes, donna beaucoup à penser au roi
et à ses ministres\,; il parut même qu'ils étaient fort fâchés que ceci
eût échappé à leur silence. Il était trop vrai pour courir après, mais
on étouffa ce trait autant qu'on le put.

L'exécution a été si familière à la maison d'Autriche dans tous les
temps jusqu'à ceux-ci, témoin la reine d'Espagne, fille de Monsieur, et
le prince électoral de Bavière, désigné héritier de la monarchie
d'Espagne du consentement de toute l'Europe, que je ne sais pourquoi on
fut si secret sur cette cassette dont presque tous les mystères ne
purent être bien développés.

\hypertarget{chapitre-xxii.}{%
\chapter{CHAPITRE XXII.}\label{chapitre-xxii.}}

1709

~

{\textsc{Reprise de la campagne de Flandre.}} {\textsc{- Artificieux
colloque des ennemis.}} {\textsc{- Bataille de Malplaquet.}} {\textsc{-
Fautes et inutilité de la bataille.}} {\textsc{- Belle retraite du
maréchal de Boufflers, fort inférieure à celle d'Altenheim.}} {\textsc{-
Mons assiégé.}} {\textsc{- Misère de l'armée française.}} {\textsc{-
Lettres pitoyables de Boufflers.}} {\textsc{- Nangis dépêché au roi.}}
{\textsc{- Villars pair.}} {\textsc{- Harcourt pair.}} {\textsc{-
Artagnan maréchal de France.}} {\textsc{- Famille, fortune et caractère
d'Artagnan.}} {\textsc{- Artagnan prend le nom de sa maison.}}
{\textsc{- Féroce éclat de M. le Duc.}} {\textsc{- Dégoûts et chute du
maréchal de Boufflers.}} {\textsc{- Défaite et ruine du roi de Suède par
le czar à Pultava.}}

~

Tournai pris, les ennemis repassèrent l'Escaut dans la nuit du 3 au 4
septembre, et la Haine le 5, au-dessus de Mons, gagnant la Trouille avec
beaucoup de diligence pour le passer aussi. Notre {[}armée{]} avec les
deux maréchaux marcha le 4 septembre\,; elle arriva le 6 au matin à
Quiévrain, d'où Ravignan fut dépêché au roi pour lui rendre compte de
l'état et de la disposition des choses. Les divers corps détachés y
rejoignirent l'armée\,; elle quitta ce camp de Quiévrain la nuit du 8 au
9, précédée d'un gros détachement commandé par le chevalier de
Luxembourg. La marche se passa sans inquiétude quoique par un terrain
fort coupé, et {[}l'armée{]} prit à neuf heures du matin le camp de
Malplaquet et de Tesnières, la droite et la gauche appuyées sur deux
bois\,; des haies et des bois assez étendus devant le centre, qui y
laissaient deux plaines par leurs coupures. Villars en occupa les
hauteurs, y établit son canon, mit son infanterie aux lisières des bois
coupés par ces deux plaines à la demi portée de son canon, et ordonna
quelques retranchements pour la couvrir.

Marlborough et le prince Eugène marchaient de leur côté, et dans la
crainte que Villars ne les gagnât de la main et ne les embarrassât pour
le siège de Mons qu'ils avaient résolu, ils avaient fait un très gros
détachement avec lequel le prince héréditaire de Hesse, depuis roi de
Suède, devança leur armée pour observer la nôtre. Il arriva à vue du
camp de Malplaquet en même temps qu'elle y entrait, dont il fut averti
plus tôt qu'il ne l'eût été par trois coups de canon que la fanfaronnade
de Villars fit tirer comme pour un appel au prince Eugène et au duc de
Marlborough dont il voyait toute l'armée assez proche, et dont il douta
encore moins lorsqu'il aperçut les colonnes du prince de Hesse qui
détacha même quelques gens pour escarmouches, pour mieux découvrir notre
armée et le terrain qu'elle occupait\,; il fit presque en même temps
avancer des colonnes d'infanterie vers notre droite, ce qui fit juger
qu'il voulait engager l'action\,; mais il se contenta de faire avancer
du canon pour contenir Villars en respect et en attention, et persuader
que toute leur armée était là. Sa crainte cependant était extrême d'être
lui-même attaqué, et il paya tellement d'effronterie par la hardiesse de
sa contenance, qu'on n'osa le tâter. Le canon tira de part et d'autre
avec un médiocre effet depuis deux heures après midi jusqu'à six que les
ennemis se retirèrent un peu de portée, mais demeurant en présence\,: la
nuit fut tranquille. Le lendemain 10, les escarmouches recommencèrent\,;
le canon tira presque tout le jour sans faire grand mal, sinon que
Coetquen, allant d'un lieu à un autre, eut une jambe emportée\,; ce fut
par le courrier qui en vint à sa famille qu'on sut les armées en
présence.

Marlborough et le prince Eugène, avertis de l'état périlleux où se
trouvait le prince de Hesse, qui était perdu s'il eût été attaqué, comme
Villars en fut souvent pressé, qui ne le voulut jamais, forcèrent leur
marche pour arriver à lui, et le joignirent dans le milieu de la matinée
du même jour 10. Leur premier soin fut de venir examiner la position de
notre armée, et celle que la leur pouvait prendre pour le faire avec
plus de loisir et de succès, et attendre leur arrière-garde\,; ils se
servirent d'une ruse qui leur réussit pleinement.

Ils firent approcher de nos retranchements, que notre infanterie
perfectionnait vers le centre, quelques officiers qui avaient l'air de
subalternes, avec ordre de tacher à lier quelque conversation avec nos
gardes avancées, et de passer outre sur parole. Il y a lieu de croire
qu'ils ne choisirent pas ces officiers au hasard par l'adresse dont ils
s'en acquittèrent. Ils s'avancèrent à pied au bord de nos
retranchements, excitèrent la curiosité de quelques-uns de nos
subalternes, causèrent avec eux, demandèrent à parler à des capitaines
et à des commandants de corps, firent sortir le commandant d'un
bataillon de la brigade de Charost, lui dirent qu'un gros d'officiers
qu'on voyait un peu dans l'éloignement était Cadogan, qui voudrait bien
dire un mot à un officier général, s'il y en avait là quelqu'un qui
voulût bien s'avancer un peu, et permettre qu'on se rapprochât de lui
sur parole.

Ces colloques duraient déjà depuis assez longtemps, lorsque Albergotti
passa par là, visitant les retranchements, qui demanda ce que c'était,
comme le marquis de Charost, qui venait d'en être averti, commençait à
faire retirer ces officiers ennemis et à remmener les nôtres. Albergotti
ne fut pas si difficile. Il manda à Cadogan qu'il était là, lui marqua
une certaine distance pour s'y avancer tous deux, et s'y achemina suivi
de peu d'officiers. Cadogan vint\,: c'était le confident de Marlborough,
et, au désintéressement près, le Puységur de leur armée\,; il prolongea
les compliments et les verbiages, qui durèrent assez longtemps.
Albergotti l'écouta avec sa glace accoutumée, lui dit que si le maréchal
de Villars se fût rencontré là, il l'adroit volontiers entretenu sur la
paix, et lui aurait témoigné qu'elle n'était pas si difficile à faire.
Cela servit d'objet à la conversation demandée, et de prétexte à
l'allonger. La troupe d'officiers grossit peu à peu autour d'eux. Le
propos de paix courut en un moment par les retranchements, et dans peu
d'autres par toute notre armée. Villars, à qui Albergotti n'avait rien
mandé, trouva fort mauvais cette espèce de conférence sans sa
permission, s'avança vers où elle se tenait et manda à Albergotti de la
finir. Elle se termina de la sorte par des désirs respectifs de la paix,
et des compliments qui ne signifiaient rien. On se retira lentement. Les
officiers ennemis s'opiniâtrèrent si longtemps à demeurer auprès des
retranchements, sous prétexte d'embrassades et de compliments à ceux des
nôtres dont ils s'étaient accostés sans les connaître, qu'il en fallut
venir à diverses reprises aux menaces de tirer sur eux, et même à tirer
quelques coups en l'air pour les faire retirer.

Pendant tous ces manèges, un très petit nombre de ce qu'ils avaient
d'officiers plus expérimentés, et de leurs meilleurs officiers généraux
à cheval, petit pour ne rien montrer et ne donner point de soupçon, et
un peu plus grand nombre d'ingénieurs et de dessinateurs à pied,
profitait de ces ridicules colloques pour bien examiner tout, jeter sur
le papier de principaux traits du terrain, prendre tout ce qu'ils purent
de remarquable, désigner les endroits à placer leur canon, se bien
mettre dans la tête le plan de leur disposition, et considérer avec
justesse tout ce qui pourrait leur être avantageux ou nuisible, dont ils
ne surent que trop bien profiter. On sut après cet artifice par les
prisonniers.

Albergotti s'excusa avec l'esprit et cet air de négligence qui ne lui
manquaient jamais. Villars le craignait à la cour, où il avait de
puissants appuis\,; Boufflers l'aimait et ne se portait point pour
général de l'armée\,; ils en avaient besoin pour le lendemain, au delà
duquel on voyait bien que la bataille ne se pouvait différer. Ainsi
Villars se contenta de tomber vaguement sur la sottise des subalternes
qui avaient donné la première occasion à ce parlementage, et on ne
songea plus qu'à se disposer à bien recevoir l'ennemi.

La nuit se passa avec la même tranquillité que la, précédente\,; un gros
brouillard la continua jusque vers six heures du matin. Les députés des
États généraux à l'armée avaient eu grand'peine à consentir à une
action. Contents de leurs avantages, ils les voulaient pousser par les
sièges, et s'avancer ainsi solidement sans rien mettre au hasard. Ce ne
fut que le 10, veille de la bataille et jour de ces artificieux
colloques, que le prince Eugène acheva de les persuader. Lui et
Marlborough prirent toutes leurs mesures dans cette même journée, en
sorte qu'ils se trouvèrent en état d'attaquer le 11 au matin l'armée du
roi.

On a vu ci-devant qu'elle avait sa droite et sa gauche appuyées à deux
bois, qu'elle en avait un au centre qui partageait une plaine dont il
faisait deux petites, ou deux grandes trouées. Maintenant il faut
remarquer que vis-à-vis ce centre et derrière le bois et les deux
trottées, il y avait une petite plaine et un bois au bout que nous ne
tenions point, propre à dérober aux ennemis les mouvements de notre
centre, mais bien plus à cacher dedans des troupes fort près de notre
centre, et à les avoir très brusquement sur les bras sans pouvoir s'en
apercevoir. Villars ne mit pas ses lignes droites, mais un peu
recourbées en croissant, c'est-à-dire les pointes des deux ailes bien
plus avancées que le centre, par conséquent moins difficiles à
envelopper et à enfoncer que dans la disposition droite et ordinaire. Le
même maréchal, jugeant sa gauche plus jalouse que sa droite, voulut s'y
mettre, et le maréchal de Boufflers se chargea de la droite.

Sur les sept heures du matin que le brouillard fut dissipé, on aperçut
les colonnes des ennemis marcher et se déployer, et pendant quelque
canonnade, les deux ailes de notre armée furent vigoureusement attaquées
par l'infanterie des ennemis. Ils avaient eu la précaution de tenir leur
cavalerie éloignée et presque en colonnes, pour ne la pas exposer à
notre artillerie, tandis que la nôtre, qui barrait les deux trouées pour
soutenir notre infanterie, était fouettée par leur canon à demi-portée,
et y perdit beaucoup sans utilité six heures durant, avec cette
inégalité que notre canon ne pouvait tirer que sur de l'infanterie
éloignée et qui fut bientôt aux prises avec la nôtre, ce qui fit cesser
notre artillerie sur elle.

L'attaque cependant se poussait vertement à notre gauche. Les ennemis
profitèrent de tous les avantages d'avoir bien reconnu notre terrain, et
ne se rebutèrent point des difficultés qu'ils y rencontrèrent à tacher
de rompre les pointes de nos ailes et d'en culbuter les courbures. Ils
jugèrent bien que l'attaque faite à tous les deux à la fois attirerait
toute l'attention du maréchal de Villars, et qu'ayant une plaine
vis-à-vis de son centre, c'est-à-dire les deux trouées qui ont été
expliquées, et la petite plaine au delà, il dégarnirait le centre au
besoin, dans la pensée qu'il aurait toujours loisir d'y voir former
l'orage, et d'y pourvoir à temps. C'est ce qui fit le malheur de la
journée.

Les ennemis repoussés de notre gauche y portèrent leurs plus grandes
forces d'infanterie et la percèrent. Alors Villars, voyant ses troupes
ébranlées et du terrain perdu, envoya chercher presque toute
l'infanterie du centre, où il ne laissa que les brigades des gardes
françaises et suisses, et celle de Charost, sans qu'avec ce renfort il
pût rétablir cette gauche sur laquelle les ennemis continuèrent de
gagner force terrain.

Attentifs en même temps à ce qu'ils avaient compté qui arriverait au
centre, ils firent sortir de ce bois qui était au bout de la petite
plaine, qui était vis-à-vis des deux trouées et de notre centre,
beaucoup d'infanterie dont ils l'avaient farcie sans que nous l'eussions
pu apercevoir, laquelle fondit sur ces brigades des gardes françaises et
suisses, et sur celle de Charost où le marquis de Charost fut tué
d'abord, de la résistance desquelles on ne paria pas bien, et qui furent
culbutées presque aussitôt qu'attaquées par un e grande supériorité de
nombre.

Malgré le désordre de notre gauche on y combattait toujours, et elle
vendait son terrain chèrement lorsque le maréchal de Villars y reçut une
grande blessure au genou, Albergotti une autre qui les mirent hors de
combat, et Chemerault tué, tout cela à cette gauche dont la défaite,
déjà bien avancée alors, ne tint presque plus depuis, malgré les efforts
et les exemples du roi Jacques d'Angleterre.

À la droite, le combat fut très vif\,; le maréchal de Boufflers, après
avoir vaillamment repoussé l'infanterie qui l'avait attaqué, avait
renversé la cavalerie qui était venue la soutenir, et gagné un grand
terrain\,; il traita de même d'autre cavalerie qui s'était présentée
devant lui, et jusqu'à trois fois de suite avec le même succès, lorsque,
tout occupé de pousser sa victoire, il apprit la défaite du centré et le
désastre de la gauche, déjà toute ployée par la droite des ennemis, la
retraite de la personne de Villars hors du combat par sa blessure, et
que le poids de tout portant désormais sur lui seul, c'était à lui à
tirer {[}l'armée{]} des précipices où Villars l'avait engagée.

Outré alors de se voir la victoire, qu'il tenait déjà, arrachée de la
main, et par des mains françaises, frappé des affres du péril où se
trouvait l'État par celui où il voyait l'armée, il se mit à inspirer
l'audace aux divisions de son aile par de courts propos en passant\,;
et, s'abandonnant à son courage, il leur donna l'exemple de cette
témérité permise aux affaires désespérées, qui leur fait quelquefois
changer de face, et il chargea en personne si démesurément à la tête de
tant d'escadrons et de bataillons, que cela put passer pour incroyable.
Ses troupes, animées par la vue des prodiges depuis si longtemps
inconnus d'un général si prodigue de soi, l'imitèrent à l'envi\,; mais
parmi tant d'efforts, Boufflers, craignant de perdre inutilement ce qui
lui restait en gagnant un terrain qui ne lui servirait qu'à le séparer
de plus loin du reste de l'armée, chercha à le gagner en biaisant pour
se rapprocher sur le centre, où il trouva les ennemis pris en flanc par
un seul régiment sorti de ceux des autres, les avait obligés à se
rejeter dans le bois\,; et que notre cavalerie, profitant de ce moment,
avait passé les retranchements pour les suivre et les pousser de plus en
plus\,; mais cette cavalerie rencontra un si grand feu d'artillerie de
ce bois, qu'elle fut contrainte de se retirer où elle était auparavant,
sous ce feu croisé qui fit un grand fracas dans ces troupes. Par ce feu
les ennemis nous éloignèrent toujours, et entretenant toujours le combat
de la droite à notre égard, profitèrent de ces mouvements pour achever
d'enfoncer notre centre. Ce fut là qu'on dit encore plus de mal des
régiments des gardes et de celui du roi qui s'y était porté, et qui en
un instant laissèrent emporter les retranchements du centre.

Les ennemis s'en trouvant maîtres s'y arrêtèrent, n'osant exposer leur
infanterie à cette cavalerie qui avait soutenu un si furieux feu avec
tant d'intrépidité\,; mais ils envoyèrent chercher leur cavalerie qui
n'avait presque pas combattu, avec leur infanterie, contre notre droite,
et avec cette cavalerie fraîche arrivée à toutes jambes, firent passer
par les intervalles de nos lignes une vingtaine d'escadrons. La nôtre
attendit trop à charger cette cavalerie qui grossissait à tous moments,
et la chargea enfin mollement et tourna aussitôt\,; c'était la
gendarmerie\,: la cavalerie qui la soutenait ne fit pas mieux, tant la
valeur et ses efforts ont leurs bornes. Quelques instants après parurent
les mousquetaires et Coettenfao à la tête des troupes rouges de la
maison du roi qui arrêtèrent cette cavalerie victorieuse et
l'enfoncèrent, mais qui rencontrant plusieurs lignes formées les unes
derrière les autres, à la faveur desquelles cette cavalerie poussée se
rallia, il fallut s'arrêter\,; alors arrivèrent les quatre compagnies
des gardes du corps qui enfoncèrent toutes ces lignes de cavalerie
ennemie l'une après l'autre.

Le salut de celle-ci fut une chose bien bizarre\,: elle trouva derrière
toutes ses lignes renversées l'une sur l'autre nos retranchements
qu'elle avait passés\,; cela la contint par la difficulté de les
repasser, et donna le temps au prince de Hesse et au prince d'Auvergne
de l'arrêter et de la rallier sous la protection du feu de leur
infanterie, restée à nos retranchements qu'elle avait gagnés. Alors les
escadrons de la maison du roi se trouvèrent rompus par tant et de si
vives charges, et sans être soutenus d'aucunes troupes, et perdirent du
terrain dont la cavalerie ennemie, qui se rétablissait et grossissait à
chaque instant, se saisit {[}de telle sorte{]}, que de battue elle
devint victorieuse. Cette reprise de combat dura longtemps et fut
disputée têtes de chevaux contre têtes de chevaux, tant qu'à la fin il
fallut céder au grand nombre et lui abandonner le champ de bataille.

Ce fut le dernier vrai combat de cette fatale journée\,; notre gauche
était déjà retirée sous les ordres d'Artagnan qui en avait rassemblé les
débris et qui les présenta si à propos et si fermement aux ennemis qu'il
les empêcha de troubler le commencement de leur retraite.

Dans ce fâcheux état., Boufflers, ne pouvant plus rien exécuter avec une
armée dispersée, une infanterie accablée, tout son terrain perdu, ne
songea plus qu'à éviter le désordre et à faire une belle et honorable
retraite. L'infanterie de la droite et de la gauche avait eu le temps de
s'y disposer pendant ce long combat de la cavalerie. À trois heures
après midi, toute notre cavalerie passa les défilés en grand ordre,
derrière lesquels elle se mit en bataille sans avoir été pressée\,; à
quatre heures le maréchal de Boufflers mit toute l'armée sur quatre
colonnes, deux d'infanterie de chaque côté le long des bois, deux de
cavalerie dans la plaine au milieu des deux autres. Elle se retira ainsi
lentement, Boufflers, à l'arrière-garde de tout, sans que les ennemis
donnassent la moindre inquiétude pendant toute la marche, qui dura
jusqu'à la nuit, et sans perdre cent traîneurs\,; tout le canon fut
retiré, excepté quelques pièces\,; et de bagage, il n'en put être
question, parce qu'il avait été renvoyé lorsqu'on s'était mis en marche
pour aller chercher les ennemis. L'armée ainsi ensemble arriva au
ruisseau de la Rouelle et campa derrière, entre Valenciennes et le
Quesnoy, où elle séjourna longtemps. Les blessés se retirèrent en ces
deux places et à Maubeuge et à Cambrai.

Les ennemis passèrent la nuit sur le champ de bataille et sur vingt-cinq
mille morts, et marchèrent vers Mons le lendemain au soir. Ils avouèrent
franchement qu'en hommes tués et blessés, en officiers généraux et
particuliers, en drapeaux et en étendards, ils avaient plus perdu que
nous. Il leur en coûta en effet sept lieutenants généraux, cinq autres
généraux, environ dix-huit cents officiers tués ou blessés, et plus de
quinze mille hommes tués ou hors de combat. Ils avouèrent aussi tout
haut combien ils avaient été surpris de la valeur de la plupart de nos
troupes, surtout de la cavalerie, et leurs chefs principaux rie
dissimulèrent pas qu'elle les aurait battus si elle avait été bien
conduite. Ils n'avaient pas douté, à la seule disposition de notre
armée, qu'elle la serait mal, puisque du lieu où commença le combat de
cavalerie, nos officiers virent leur camp tendu.

En effet, avec plus d'art et de mesure, on pouvait soutenir nos
retranchements\,; mais le terrain coupé qui était au delà, et la hauteur
que tenaient les ennemis, ne pouvaient laisser espérer de les déposter
après les avoir repoussés. Ce fut sans doute ce qui leur persuada
l'attaque, dans la pensée d'obtenir la victoire s'ils emportaient le
champ de bataille\,; et, s'ils étaient repoussés, de n'y pouvoir perdre
que des hommes et rien de plus, desquels ils ont bien plus que nous, et
des recrues tant qu'ils veulent.

L'idée du maréchal de Villars est demeurée fort difficile à comprendre.
Pourquoi de si loin marcher aux ennemis pour s'en laisser attaquer
exprès, ayant pu aisément les attaquer lui-même deux jours durant avant
d'être attaqué, au moins un grand jour et demi pour parler avec la
précision la plus exacte\,? Si on oppose qu'il ignorait que ce qu'il
prit pour toute leur armée n'était qu'un gros corps avancé, on peut
répondre qu'il fallait être mieux informé en chose si capitale, et qu'on
l'est quand on veut s'y adonner et bien payer. D'ailleurs, s'avançant
sur ce qu'il voyait, quand l'armée y eût été tout entière, il n'aurait
fait que ce pour quoi il avait marché à elle, gagnait la hauteur sur
elle, et mettait derrière lui ce bois funeste de vis-à-vis son centre
qui acheva la perte de la bataille, et ce bois encore de son centre avec
ses deux trouées, qui, en partageant en deux son champ de bataille,
coupa son armée, donna lieu de la battre en détail, et rendit inutile la
constante victoire de sa droite. Il paraît donc constant qu'il ne
pouvait jamais gagner la bataille dans un terrain si désavantageux.

Si on examine la disposition qu'il en fit, elle ne se trouvera pas plus
savante que le choix de ce bizarre terrain. Une forme de croissant qui,
comme on l'a dit, présente deux pointes difficiles à défendre, aisées à
envelopper\,; un centre tout aussitôt dégarni qu'on ne peut sauver,
faute énorme, et dont le souvenir d'Hochstedt eût au moins dû
préserver\,; un grand corps de cavalerie posté sous le feu des batteries
ennemies, sans aucun fruit à en pouvoir attendre\,; enfin nulle
nécessité de combattre après avoir laissé tranquillement prendre
Tournai\,; et pour Mons, en tenant d'abord les ennemis de plus près, on
eût aisément choisi un lieu plus avantageux\,; mieux encore {[}eût
valu{]} laisser former le siège, et se poster à temps, de manière à les
attaquer affaiblis, tant par le siège même que par la garde de leurs
tranchées et de leurs postes. Enfin il parut que de tous les moments et
de tous les terrains à choisir pendant toute cette campagne, le temps et
le terrain ne le pouvaient être plus mal pour combattre. Ce jugement fut
celui des deux armées\,; on verra qu'il ne fut pas celui du roi et de
M\textsuperscript{me} de Maintenon.

Les ennemis eurent en cette bataille cent soixante-deux bataillons,
trois cents escadrons, cent vingt pièces de canon, c'est-à-dire
quarante-deux bataillons, quarante escadrons, et quarante-deux pièces de
canon plus que l'armée du roi, qui y perdit dix mille hommes tués et
blessés, Chemerault et Pallavicin, lieutenants généraux, et le marquis
de Charost. Il était fils aîné du duc de Charost, dans la plus haute
piété et qui eût moins réussi à la cour qu'à la guerre. Il n'avait point
d'enfants de la fille de Brûlart, premier président du parlement de
Bourgogne, qui longues années depuis est devenue seconde femme du duc de
Luynes, aussi sans enfants, et dame d'honneur de la reine, après la
maréchale de Boufflers.

Chemerault était excellent officier général, fort dans le grand monde,
et honnête homme, quoique dans la liaison la plus intime de M. de
Vendôme. Il ne laissa point d'enfants de la fille de
M\textsuperscript{me} de Moreuil, qui avait été longtemps dame d'honneur
de M\textsuperscript{me} la Duchesse, dont le mari était un boiteux fort
plaisant et fort singulier, bâtard de cette grande maison de Moreuil,
éteinte il y a longtemps, et toute sa vie à M. le Prince et à M. le Duc,
fort mêlé dans le monde.

Pallavicin, aussi très bon officier général, était ce transfuge
piémontais de foi très douteuse, d'aventure fort ignorée, dont le
maréchal de Villeroy avait fait son favori, et le seul homme peut-être
capable d'estimer et de se fier à celui-là. Il n'était point marié.

Il y périt bien d'autres gens, mais moins connus que ceux-là.
Courcillon, fils unique de Dangeau, dont j'ai parlé ailleurs, y eut une
jambe emportée. Le prince de Lambesc, fils unique du comte de Brionne,
fils aîné de M. le Grand, y fut pris et renvoyé incontinent après sur
parole.

Les deux armées furent aussi également persuadées que le sort des armes
était décidé longtemps avant que le maréchal de Villars fût blessé,
quoiqu'il n'ait rien oublié pour que {[}sa blessure{]} fût cause de tout
le désastre. On soupçonna aussi que l'aile du maréchal de Boufflers, qui
fut toujours victorieuse, eût peut-être rétabli l'affaire, s'il eût
d'abord poussé sa pointe avec moins de précautions. Mais très
certainement on crut qu'il aurait remporté l'honneur de la journée, si
le dégarnissement du centre, par la défaite de la gauche, ne l'eût forcé
d'aller à leur secours.

Mais si la victoire lui fut arrachée des mains de la façon qui vient
d'être racontée, personne ne lui put ôter l'honneur de la plus belle
retraite qui ait été faite depuis celle d'Altenheim qui a immortalisé M.
le maréchal de Lorges, et qui eut supérieurement à celle-ci le
découragement de l'armée par la mort de M. de Turenne, la division des
chefs, l'armée ennemie sans cesse sur les bras, et le Rhin à passer
devant eux et malgré eux, et les équipages à sauver. Mais ces grandes
différences ne sauraient ternir la gloire de celle-ci, qui, dans un
genre à la vérité très inférieur pour les difficultés, fut également
sage, savante, ferme, et dans le meilleur et le plus grand ordre qu'il
est possible.

L'armée conserva sous lui un air d'audace et un désir d'en revenir aux
mains qui pensa être suivi de l'effet, mais qui se trouva arrêté court
par misère. Les ennemis ouvrirent la tranchée le 23 septembre devant
Mons\,; Boufflers et son armée petillaient de leur faire lever ce grand
siège. Quand ce vint aux dispositions, point de pain et peu de paye\,;
le prêt avait manqué souvent et n'était pas mieux rétabli\,; les
subalternes, réduits au pain de munition, s'éclaircissaient tous les
jours\,; les officiers particuliers mouraient de faim avec leurs
équipages\,; les officiers supérieurs et les officiers généraux étaient
sans paye et sans appointements, dès la campagne précédente\,; le pain
et la viande avaient manqué souvent des six et sept jours de suite\,; le
soldat et le cavalier, réduit aux herbes et aux racines, n'en pouvait
plus\,; nulle espérance de mieux pour cette lin de campagne, nécessité
par conséquent de laisser échapper les occasions de sauver Mons, et de
ne penser plus qu'à la subsistance, la moins fâcheuse qu'on pourrait,
jusqu'à la séparation des armées.

Aussitôt après la bataille, Boufflers dépêcha un courrier au roi pour
lui en rendre compte. Sa lettre fut juste, nette, concise, modeste, mais
pleine des louanges de Villars qui était au Quesnoy hors d'état de
s'appliquer à rien. Le lendemain, Boufflers en écrivit une plus étendue,
en laquelle tout ce qu'il avait vu faire aux troupes et son attachement
pour le roi l'égarèrent trop loin. Il songea tant à consoler le roi et à
louer la nation, qu'on eût dit qu'il annonçait une victoire et qu'il
présageait des conquêtes.

Nangis, duquel j'ai parlé plus d'une fois, était maréchal de camp dans
cette armée\,; Villars l'aimait, et le voulut avoir à la gauche sous sa
main\,; il le choisit aussi pour aller rendre compte au roi du détail et
du succès de la bataille. Le maréchal comptait sur son amitié\,; il
avait fort contribué à l'avancer\,; il sentait l'importance d'envoyer un
homme affidé et qui avait ses appuis à la cour. Nangis, avec moins
d'esprit que le plus commun des hommes, mais rompu au monde et à la cour
dès sa première jeunesse, eut assez de sens pour craindre de se trouver
entre les deux maréchaux, malgré toute leur intelligence. Villars le
pressa, il fut à Boufflers pour se faire décharger de la commission,
mais il suffisait à Boufflers que Nangis fût du choix de Villars pour
vouloir qu'il se soumît à son désir\,; il le chargea d'une lettre par
laquelle il marqua toute la répugnance du courrier qui ne partait que
par obéissance.

Le premier courrier avait porté toute la disgrâce de la nouvelle dont il
était chargé\,; on était d'ailleurs si malheureusement accoutumé aux
déroutes et à leurs funestes suites, qu'une bataille perdue comme
celle-ci la fut sembla une demi-victoire. Les charmes de l'heureux
Nangis rassérénèrent l'horizon de la cour, où il ne faut pas croire
qu'au nombre, au babil et à l'usurpation du pouvoir des dames, sa
présence fût inutile à rendre le malheur plus supportable.

Nangis rendit bon compte, mais concis, ne se piqua point de parler de ce
qu'il n'avait point vu, évita par là force questions embarrassantes, et
se tira d'affaires sans s'en être fait avec personne. Il exalta Villars
tarit qu'il put, et fit bouclier de sa blessure\,; c'était pour cela
qu'il était venu, et il y fut appuyé par la lettre qu'il apporta du
maréchal de Boufflers qui enchérit sur la première jusqu'à
l'enthousiasme sur les louanges de Villars, sur la valeur de la nation,
et sur les flatteries d'espérances pour consoler le roi.

Cette lettre, qui fut rendue publique, parut si outrée qu'elle fit un
tort extrême au maréchal de Boufflers. D'Antin, ami intime de Villars,
en saisit tout le ridicule pour l'obscurcir auprès du roi. Ses fines
railleries prirent avec lui jusqu'aux airs de mépris, et le monde,
indigné d'une lettre si démesurée, en oublia presque Lille, et ce
sentiment héroïque qui l'avait porté à l'aide de Villars. Tel fut
l'écueil qui froissa ce colosse de vertu à l'aide des envieux et des
fripons, et qui donna lieu à une raison plus cachée, qui se verra
bientôt, de réduire cette espèce de dictateur à la condition commune des
autres citoyens.

Le fortuné Villars, enrichi à la guerre où tous les autres se ruinent,
maréchal de France pour une bataille qu'il crut perdue, lors même que
d'autres que lui l'eurent gagnée\,; chevalier de l'ordre parce que le
roi s'avisa de le donner à tous les maréchaux de France\,; duc vérifié
pour un simple voyage en Languedoc où il se mit de niveau avec un
brigand en traitant sans fruit d'égal avec lui, fut fait pair pour la
bataille de Malplaquet dont on vient de voir les fautes et le triste
succès\,; le cri public sur sa naissance et sur la récompense durent le
mortifier.

Harcourt en frémit de rage\,; il sut des bords du Rhin crier si haut au
roi et à M\textsuperscript{me} de Maintenon, qu'il emporta d'emblée la
pairie, mais avec le dépit de l'occasion et de n'être pair qu'après
Villars, qui, en naissance et en toutes choses, était si loin de lui, et
fait duc vérifié si longtemps après lui.

Artagnan reçut en même temps le bâton de maréchal de France\,; il avait
pour lui M. du Maine, M\textsuperscript{me} de Maintenon, surtout les
valets intérieurs. Le public ni l'armée ne lui furent pas favorables,
que ses airs d'aisance et de s'y être attendu depuis longtemps
achevèrent de révolter. Le dépit et le murmure de cette prostitution de
la première dignité de l'État, et du premier office militaire, éclata si
haut malgré la politique et la crainte, que le roi en fut assez peiné
pour s'arrêter tout court, en sorte que ces dernières récompenses au
delà desquelles, chacune en leur genre, il n'est rien de plus, furent
les seules qui suivirent la perte de la bataille, où tant de gens de
tout grade s'étaient si fort signalés.

Artagnan avait paru dans le monde sous ce nom, d'une terre qui était
dans sa branche, mais dont il n'était pas l'aîné. Son père était
lieutenant de roi de Bayonne, où il mourut. Il avait épousé une sœur du
maréchal de Gassion plus de dix ans avant qu'il fût maréchal de France,
et que sa fortune n'était pas commencée. On ne connaissait point alors
l'ordre du tableau\footnote{\emph{L'ordre du tableau, établi par
  Louvois}, réglait l'avancement d'après le temps de service.
  Saint-Simon revient plusieurs fois sur l'ordre du tableau, et
  spécialement quand il parle du gouvernement de Louis XIV après la mort
  de ce prince. «\,Au moyen de cette règle, dit-il, il fut établi que,
  quel qu'on pût être, tout ce qui servait demeurait, quant au service
  et aux grades, dans une égalité entière. De là tous les seigneurs dans
  la foule des officiers de toute espèce\,; de là cette confusion que le
  roi désirait.\,»}, et il se formait de grands hommes qui allaient
vite. Artagnan fut mis dans le régiment des gardes qu'avait le maréchal
de Grammont, gouverneur de Bayonne, Navarre, etc. Il passa par tous les
grades de ce régiment, presque toujours dans l'état-major. Il en fut
longtemps major, et ce fut par les détails de cet emploi qu'il sut
plaire au roi. Lui et Artagnan mort capitaine de la première compagnie
des mousquetaires et chevalier de l'ordre en 1724, étaient enfants des
deux frères. Une soeur de leur père avait épousé M. de Castelmore, dont
le nom était Baatz, dont elle eut deux fils. L'aîné mourut, en 1712, à
plus de cent ans, gouverneur de Navarreins\,; le cadet trouva le nom
d'Artagnan plus à son gré et l'a porté toute sa vie. Il se fit estimer à
la guerre et à la cour, où il entra si avant dans les bonnes grâces du
roi, qu'il y a toute apparence qu'il eût fait une fortune considérable,
s'il n'eût pas été tué devant Maestricht en 1673. Ce fut à cause de lui
que celui dont il s'agit ici prit le nom d'Artagnan, que ce capitaine
des mousquetaires avait fait connaître, et que le roi aima toujours,
jusqu'à avoir voulu qu'Artagnan, mort chevalier de l'ordre, passât de
capitaine aux gardes qu'il avait été longtemps à la sous-lieutenance des
mousquetaires gris, dont il fut capitaine après Maupertuis.

Pour revenir au nôtre, il se poussa ténébreusement à la cour par
l'intrigue, et rendait compte de beaucoup de choses au roi par les
derrières, par des lettres et par les valets intérieurs, de presque tous
lesquels il se fit ami. Il sut gagner par les mêmes voies
M\textsuperscript{me} de Maintenon et M. du Maine, en sorte que, souple
sous ses colonels, ils ne laissaient pas de le ménager beaucoup. Il fut
inspecteur, puis directeur d'infanterie, des détails de laquelle il sut
amuser le roi, armures, habillements, exercices nouveaux, toutes ces
choses qui firent sa fortune et ne le firent pas aimer dans le régiment
des gardes, dans l'infanterie, ni même à la cour, où il vécut toujours
assez obscurément. Toutefois bon officier et entendu, mais avec qui on
ne vivait pas en confiance. Devenu maréchal de France, il prit le nom de
maréchal de Montesquiou, qui est le nom de leur maison.

Là-dessus M. le Duc entra en furie, vomit tout ce qu'il est possible de
plus violent et de plus injurieux, dit qu'il était bien insolent de
prendre le nom d'un traître qui avait assassiné son cinquième aïeul, et
publia que partout où il le rencontrerait, il lui ferait un affront et
une insulte publique.

Antoine de Montesquiou et qui en portait le nom, lieutenant des gens
d'armes du duc d'Anjou depuis Henri III, tua, de sang-froid et
par-derrière, le prince de Condé, chef des huguenots, et frère
d'Antoine, roi de Navarre père d'Henri IV, à la bataille de Jarnac, en
1569, comme ce prince venait d'être pris, la jambe cassée, assis à terre
et appuyé contre un arbre. Cette branche, distinguée des autres
Montesquiou par le nom de Sainte-Colombe, prétend avoir dans ses
archives l'ordre du duc d'Anjou pour tuer le prince de Condé. Le crime
n'en est ni moins honteux ni moins noir\,; mais ce prince de Condé était
le cinquième aïeul de M. le duc, et le Montesquiou qui le tua était issu
de germain du grand-père du maréchal\,: c'était là porter le
ressentiment bien loin.

M. le Duc crut se rendre par là redoutable\,: il n'avait pas besoin pour
cela d'une si étrange férocité\,; celle qu'il montrait chaque jour le
faisait fuir assez, sans qu'il prît soin de s'écarter encore plus tout
le monde, qui en cria autant qu'il en eut peur. Quelque étrange abus
qu'il fît de sa qualité de prince du sang, le maréchal de Montesquiou ne
s'en émut pas, se contint en respect, mais garda le nom de Montesquiou,
et dit que des insultes et des affronts, il n'en connaissait que les
faits et point les personnes dont ils venaient, et que des propos qu'il
ne pouvait croire vrais ne l'empêcheraient point d'aller et de se
présenter partout sous le nom de sa maison. On peut juger dans quel
redoublement de furie un propos si ferme et soutenu de rie point changer
de nom mit M. le Duc, à qui le maréchal ne fit rien dire. Il vint à
Paris et à la cour après la campagne, et il alla en effet librement
partout. Il ne rencontra M. le Duc nulle part, qui avait eu loisir de
faire ses réflexions, ou peut-être plus grand que lui les lui avait fait
faire. Le maréchal demeura fort peu à la cour et à Paris, et fut renvoyé
en Flandre. Pendant l'hiver M. le Duc mourut, et aucun prince de la
maison de Condé n'embrassa cette querelle qui finit avec lui, et dont
avec ce cour si immense en rancune, il n'avait pu éviter qu'on ne prit
la liberté de se moquer.

Le siège de Mons se continuait, et la misère extrême de l'armée du roi,
qui manquait de tout, la réduisait à le laisser faire avec tranquillité.
Boufflers ne pouvait songer qu'à la subsistance de plus en plus
difficile, et sentait avec une indignation secrète un homme tel que
Villars égalé à lui pour avoir perdu une importante bataille lorsqu'il
n'avait tenu qu'à lui de battre les ennemis en détail et de les mettre
hors de portée de songer à Mons, ni à aucun autre siège, et que lui
avait sauvé l'État en sauvant l'armée des fautes de Villars.

Celui-ci, moins attentif à sa blessure, qui allait bien, qu'au comble
d'honneur où une faveur inespérable venait de le porter des bords du
précipice, et de voir au secours de sa blessure Maréchal, premier
chirurgien du roi, et qui ne découchait jamais des lieux où était le
roi, dépêché vers lui avec ordre d'y demeurer jusqu'à ce qu'il pût être
ramené en France, et à profiter d'un état si radieux, tomba par ses
émissaires sur le maréchal de Boufflers qui, content d'avoir sauvé la
France, se reposait sur sa propre générosité, la vérité, la notoriété
publique, et content de l'avoir fait aux dépens de tout, glissait avec
son accoutumée grandeur d'âme sur des bagatelles que Villars entreprit
de censurer et de réformer, toujours avec l'air d'un blessé qui ne songe
qu'à guérir.

Le grand nombre de ces contradictions fit sentir à Boufflers une
conduite si différente de l'ordinaire, qu'il y soupçonna du dessein.
Cela l'aigrit, mais non pas jusqu'à rien montrer, ni le porter à changer
en rien à l'égard de l'autre qu'il avait comblé d'éloges et d'égards, et
les choses continuèrent quelque temps à se passer ainsi en entreprises
d'une part, et à supporter de l'autre avec impatience, mais sans en rien
témoigner. Son exactitude, qui lui faisait mettre dans la balance
jusqu'aux minuties, surtout quand il s'agissait de préférence et de
récompenses, lui fit perdre beaucoup de temps à proposer au roi les
sujets qui méritaient d'avoir part aux vacances des emplois. Il en avait
promis la liste plus d'une fois qu'il remettait toujours. Enfin il
l'envoya par un courrier quinze jours après l'action\,; mais il fut bien
étonné que le soir même du départ de ce courrier, il en reçût un de
Voysin qui lui apporta la disposition générale et entière de tout ce qui
vaquait, faite et expédiée, sans avoir eu le moindre avis que le roi
songeât à la faire avant d'avoir reçu celle qu'il devait proposer et qui
ne faisait que partir. Ce trait fut le premier salaire du service qu'il
venait de rendre, tel que le roi avait dit plus d'une fois, même en
public, que c'était Dieu assurément qui lui avait inspiré de l'envoyer à
l'armée, où tout était perdu sans lui. Il eut encore le dégoût que
personne dans l'armée n'ignorât ce qui lui arrivait, et qu'il était
peut-être le premier général d'armée sur qui un mépris aussi marqué fût
tombé.

La vérité qu'il faut observer avec exactitude m'engage à l'aveu dés
ténèbres où je suis demeuré, non sur l'occasion de la chute de
Boufflers, qui ne s'en releva de sa vie, mais sur l'ordre des occasions.
Il y en eut trois qui le perdirent, et, ce qui est étrange, par
l'avantage qu'on saisit d'un aussi futile fondement que celui de ces
lettres, dont le ridicule montrait à la vérité le peu d'esprit, mais le
montrait par le côté le plus respectable de couvrir les fautes de
Villars au lieu d'en profiter, de vouloir encourager contre l'abattement
dont il avait vu de si tristes effets, et surtout soutenir et consoler
le roi par les motifs si purs d'attachement et de reconnaissance.

Villars et Voysin, d'accord sans se concerter à se délivrer d'un tuteur,
l'un à la tête des armées, l'autre sur toutes les affaires de son
département de\,:la guerre, pesèrent sur tout ce qu'ils purent\,; l'un
fournit, l'autre fit valoir\,; les fripons intérieurs ajoutèrent tout ce
qu'ils purent contre une vertu qui avait pénétré les cabinets, et qu'ils
craignaient jusque dans leur asile. Plus que tout, la grandeur d'un
service au-dessus de toute récompense a presque dans tous les temps et
en tout pays porté par terre ceux qui l'ont rendu\,; l'envie se réunit
contre un homme qui ne peut être égalé, et pour l'autorité sans
contrepoids duquel tout crie, tout applaudit, tout en parle comme d'un
droit justement acquis, et on a vu peu de monarques dont l'équité l'ait
emporté sur l'amour-propre, et pour qui la vue d'un sujet, assez grand
pour être arrivé au-dessus des effets de la reconnaissance qu'il a
méritée par sa vertu, n'ait été pesante et même odieuse.

On souffre le poids des grandes actions, parce qu'on ose se flatter
qu'on n'est pas au-dessous d'en faire de semblables\,; ainsi M. le
Prince, M. de Turenne et d'autres pareils ont été supportés, ceux-là
mêmes sans peine, parce qu'il semblait que leurs exploits derniers
n'étaient qu'une manière d'éponge passée sur ceux par lesquels ils
avaient si puissamment travaillé à la ruine de l'État\,; que les uns
n'étaient qu'une compensation des autres, et qu'il ne leur était dû que
des lauriers\,; mais le poids des services les plus importants dont,
l'âme est la seule vertu, dont la grandeur passe toute récompense, quand
celui qui les a rendus est si comblé, qu'en les rendant il n'a pu se
proposer que l'honneur de les rendre, cette impuissance de retour
devient un poids qui tourne, sinon à crime, comme il n'y en a que trop
d'exemples, au moins à dégoût, à aversion, parce que rien ne blesse tant
la, superbe des rois par tous les endroits les plus sensibles, et c'est
ce qui arriva au maréchal de Boufflers, et il n'en fallait pas
davantage. Mais il est vrai qu'il y eut une autre cause qui lui fit
encore plus de mal. Toutes sont certaines et je ne suis en obscurité que
sur la date de cette dernière cause.

Il est certain que le dépit de se voir Villars et même Harcourt lui être
égalés par la pairie, d'une si grande distance de la manière d'y être
portés à celle dont lui-même y était arrivé, et dans la circonstance où
cela se trouvait, tourna la tête au maréchal, et y fit entrer ce qu'il
n'avait jamais imaginé jusqu'alors, et ce qu'il eût rejeté avec
indignation si quelqu'un le lui avait proposé comme un motif d'aller en
Flandre. L'épée de connétable lui vint dans l'esprit\,; il ne se crut
pas au-dessous d'elle après ce qu'il venait de faire quand il vit
Villars et Harcourt, pairs comme lui. La fonction qu'il avait exercée à
Paris jusqu'à son départ pour la Flandre, cette direction de Voysin et
des affaires de la guerre qu'il avait eue jusqu'à ce même départ lui
parurent des détachements des fonctions de ce premier office de la
couronne et des degrés pour y monter. Il ne vit point de maréchaux de
France en situation de le lui disputer, ni même de lui être en moindre
obstacle. De prince du sang que cela pût obscurcir, il n'y en avait
aucun\,; M. du plaine s'en était mis dès longtemps hors de portée\,; M.
le duc d'Orléans, par la grandeur de sa naissance et par ce qu'il venait
d'éprouver, ne pouvait oser même se montrer blessé de le voir à la main
d'un autre. Comme on se flatte toujours, ce qu'il achevait de faire lui
paraissait devoir pleinement rassurer sur le danger de faire revivre en
sa faveur un si puissant office. L'abus qu'en avaient fait ceux qui en
avaient été revêtus, et qui ne pouvait même être reproché aux quatre
derniers, ne pouvait être craint en lui après les preuves qu'il avait
faites, et ces preuves mêmes jointes à la grande récompense que Villars
venait de recevoir pour avoir perdu l'État, si lui-même ne l'eût sauvé,
étaient des motifs assez grands pour l'emporter sur ceux de rendre un
sujet trop grand et trop puissant, qui avaient fait, depuis près de cent
ans, disparaître les connétables. Cela, c'est ce qui est certain et
moi-même je ne puis en douter\,; mais ce que j'ignore, c'est le temps
qu'il hasarda cette insinuation\,: savoir si de l'armée et à la chaude
il la fit à M\textsuperscript{me} de Maintenon ou au roi même\,; savoir
s'il attendit son retour\,: c'est ce que je n'ai pu approfondir\,; mais
pour l'avoir faite et appuyée, et je crois à plus d'une reprise, c'est
ce qui n'est pas douteux, et c'est ce qui acheva de le couler à fond.

Mons rendu, les ennemis séparèrent leur armée\,; Boufflers en fit autant
et revint à la cour\,; il y fut reçu moins bien qu'un général ordinaire
sous qui il ne s'est rien passé. Nul particulier avec le roi, pas même
un mot en passant de Flandre\,; silence, fuite, éloignement, quelques
paroles indifférentes par-ci par-là et rien de plus. Le poids du dernier
service, celui des derniers mécontentements, formèrent comme un mur
entre le roi et lui, qui demeura impénétrable. M\textsuperscript{me} de
Maintenon, avec qui il fut toujours aussi bien qu'il y avait toujours
été, essaya vainement de le consoler\,; Monseigneur même, et Mgr le duc
de Bourgogne ne dédaignèrent pas d'y travailler\,; mais trop vertueux
pour envisager l'âge et la mort du roi comme une ressource, puisqu'il
était si plaint et si bien traité de ses deux nécessaires successeurs,
et trop entêté pour revenir sur soi-même, il eut bien le courage de
paraître le même à l'extérieur et de ne rien changer à sa vie ordinaire
pour la cour, mais un ver rongeur le mina peu à peu, sans avoir pu se
faire à la différence qu'il éprouvait ni au refus de ce qu'il croyait
mériter. Souvent il s'en est ouvert à moi sans faiblesse et sans sortir
des bornes étroites de sa vertu\,; mais le poignard dans le coeur, dont
le temps ni les réflexions ne purent émousser la pointe. Il ne fit plus
que languir depuis sans toutefois être arrêté au lit ou dans sa chambre,
et ne passa pas deux ans, Villars arriva triomphant\,; le roi voulut
qu'il vînt et demeurât à Versailles pour que Maréchal ne perdît pas de
vue sa blessure, et il lui prêta le bel appartement de M. le prince de
Conti qui était dans la galerie basse de l'aile neuve, parce qu'il
n'avait qu'un fort petit logement tout au haut du château, où il eût été
difficile de le porter. Quel contraste, quelle différence de services,
de mérite, d'état, de vertu, de situation, entre ces deux hommes\,! quel
fonds inépuisable de réflexions\,!

Cette année en fournit encore de plus grandes par le changement qui
arriva dans le Nord, l'abaissement, pour ne pas dire l'anéantissement de
la Suède qui avait si souvent fait trembler le Nord, et plus d'une fois
l'empire et la maison d'Autriche\,; et l'élévation formidable depuis
d'une autre puissance jusqu'alors inconnue, excepté le nom, et qui
n'avait jamais influé hors de chez elle et de ses plus proches voisins.
Ce fut l'effet de l'étrange parti que prit le roi de Suède, qui, enivré
de ses exploits et du désir de détrôner le czar comme il avait fait le
roi de Pologne, séduit par les funestes conseils de Piper, son unique
ministre, que l'argent des alliés contre la France avait corrompu, pour
se délivrer d'un prince qui s'était rendu si formidable, et avec lequel
ils avaient tous été forcés plus d'une fois à compter, il s'engagea à
poursuivre le czar, qui, en fuyant devant lui avec art, anima son
courage et son espérance, l'engagea dans des pays qu'il avait fait
dévaster, ruina son armée par toutes sortes de besoins, de famine, de
misères, le força ensuite de désespoir à un combat désavantageux, où
toute son armée périt sans aucune retraite, et où lui-même, fort blessé,
n'en trouva qu'à Bender, chez les Turcs, où il arriva à grand'peine et à
travers mille périls, lui troisième ou quatrième.

\hypertarget{chapitre-xxiii.}{%
\chapter{CHAPITRE XXIII.}\label{chapitre-xxiii.}}

1709

~

{\textsc{Électeur de Bavière à Paris, incognito, voit le roi et
Monseigneur.}} {\textsc{- Ses prétentions de rang surprenantes.}}
{\textsc{- Dire l'électeur au lieu de M. l'électeur.}} {\textsc{- Courte
réflexion.}} {\textsc{- Mort du cardinal Portocarrero\,; son humble
sépulture.}} {\textsc{- Mort, fortune et caractère de Godet, évêque de
Chartres.}} {\textsc{- M. de Chartres se choisit un successeur\,; son
caractère et sa vertu.}} {\textsc{- Bissy, évêque de Meaux, et La
Chétardie, curé de Saint-Sulpice, succèdent à M. de Chartres auprès de
M\textsuperscript{me} de Maintenon.}} {\textsc{- Caractère de La
Chétardie.}} {\textsc{- Mort de Crécy, frère de Verjus\,; son
caractère.}} {\textsc{- Mort, naissance et caractère de Marivaux.}}
{\textsc{- Mort et caractère de M\textsuperscript{me} de Moussy.}}
{\textsc{- Naissance de son mari.}} {\textsc{- Mort de la duchesse de
Luxembourg.}} {\textsc{- Disputes sur la grâce.}} {\textsc{- Jésuites.}}
{\textsc{- Molinisme.}} {\textsc{- Jansénisme.}} {\textsc{- Congrégation
fameuse de Auxiliis.}} {\textsc{- Port-Royal.}} {\textsc{- Formulaire.}}
{\textsc{- Affaire des quatre évêques.}} {\textsc{- Paix de Clément
IX.}} {\textsc{- Casuistes.}} {\textsc{- Lettres provinciales.}}
{\textsc{- Disputes sur les pratiques idolâtriques des Indes et les
cérémonies de la Chine.}} {\textsc{- Beau jeu du P. Tellier.}}
{\textsc{- Bulle Vineam Domini Sabaoth.}} {\textsc{- Projet du P.
Tellier.}} {\textsc{- Port-Royal des Champs refuse de souscrire à
l'acceptation de la bulle Vineam Domini Sabaoth, sans explication.}}
{\textsc{- Port-Royal des Champs privé des sacrements.}} {\textsc{-
Port-Royal des Champs innocent à Rome, criminel à Paris.}} {\textsc{-
Destruction militaire de Port-Royal des Champs.}} {\textsc{- Cardinal de
Noailles sans repos depuis cette époque jusqu'à sa mort.}}

~

L'électeur de Bavière, peu à peu exclu du commandement des armées,
brouillé avec Villars à qui on avait voulu le donner, languissant dans
les places de Flandre qui se raccourcissaient tous les jours, et
quelquefois à Compiègne où il était venu de Mons sur la fin du siège de
Tournai, avait jusqu'alors inutilement insisté pour obtenir la
permission de venir saluer le roi sous le même incognito, et sans
prétendre plus qu'avait fait son frère l'électeur de Cologne.

Le roi n'aimait point à avoir des compliments à faire, ni à se
contraindre pour faire les honneurs de sa cour, quoiqu'il s'en acquittât
avec une grâce et une majesté qui le relevaient encore. Peut-être
craignait-il encore plus les reproches tacites de la présence d'un
prince qui avait tout perdu par sa fidélité à ses engagements, et qui,
n'ayant plus ni États ni subsistance, était encore assez mal payé, par
les malheurs qui accablaient la France, de ce que le roi s'était obligé
de lui donner. Néanmoins il pressa tant, et il assura si fort qu'il
n'embarrasserait en rien, à l'exemple de son frère, qu'il n'y eut pas
moyen de le refuser.

Il vint donc sous un autre nom descendre chez Monasterol son envoyé, où
tout ce qu'il avait vu de gens de la cour à l'armée s'empressèrent de
l'aller voir, et d'Antin eut ordre du roi de lui faire les honneurs avec
une assiduité légère qui ne se préjudiciât point à l'entier incognito.

Il demeura quatre ou cinq jours à Paris, parmi le jeu, les spectacles,
les curiosités de la ville, et les soupers avec des dames de compagnie
facile et médiocre\,; après quoi d'Antin le mena dîner chez Torcy, à
Marly, où le roi était, et où le ministre des affaires étrangères lui
donna un grand repas avec compagnie choisie, et le conduisit après dans
le cabinet du roi. Torcy y demeura fort peu en tiers\,; l'électeur resta
seul avec le roi une heure et demie, ensuite le roi le mena dans le
salon. Toutes les dames y étaient sous les armes\,; il y avait un grand
lansquenet établi\,; le roi le présenta, sous le nom qu'il avait pris, à
Monseigneur, à Mgr et à M\textsuperscript{me} la duchesse de Bourgogne,
et aux dames\,: il ajouta que c'était un de ses amis qui l'était venu
voir et à qui il voulait montrer sa maison. Il se retira un moment
après\,; ces princes et les dames prirent soin d'entretenir l'électeur
debout, qui parut gai et très poli, mais avec un air de hauteur et de
liberté de maître du salon, parlant aux uns, s'informant des autres, qui
ne seyait pas mal à un prince malheureux. Une demi-heure après le roi le
vint prendre pour la promenade, monta dans un chariot à deux, traîné par
quatre porteurs, et lui commanda d'y monter aussi, ce qu'il ne se fit
pas répéter\,; entretint le roi et les courtisans qui marchaient autour
du chariot, d'un air aisé et familier, pourtant respectueux avec le roi,
et loua extrêmement les jardins. La promenade dura une heure et demie\,;
le roi le remena dans le salon où se trouva la même compagnie qu'il
avait laissée\,; le roi l'y laissa aussitôt, et lui, au bout d'un quart
d'heure, prit congé, et s'en alla avec d'Antin et quelques courtisans à
Paris, à l'Opéra, souper après et jouer chez d'Antin.

Il le mena deux jours après dîner chez lui à Versailles, lui fit voir le
château et les jardins, lui donna à souper et à coucher, et le mena le
lendemain au rendez-vous de chasse à Marly, où le roi et les daines
l'attendaient, après laquelle il s'alla rafraîchir chez Torcy, lit un
tour dans le salon, et s'en retourna à Paris. Deux jours après la même
chose se répéta, et il acheva de voir à Versailles ce qu'il n'avait pu
voir la première fois.

Monseigneur, étant allé de Marly à Meudon, y voulut donner à dîner à
l'électeur\,; mais la surprise fut grande de la prétention qu'il forma
d'y avoir la main. Elle était en tout sens également nouvelle et
insoutenable\,; jamais électeur n'en avait imaginé une semblable sur
l'héritier de la couronne, et celui-ci de plus était incognito, et hors
d'état par là de pouvoir prétendre quoi que ce fût non seulement avec
Monseigneur, mais avec personne. Il avait l'exemple de son frère auquel
il cédait partout comme plus ancien électeur que lui\,; il avait proposé
le premier de le suivre, et promis de ne faire aucun embarras, il
n'était venu qu'à cette condition. Nonobstant tout cela, il y eut des
pourparlers qui aboutirent à quelque chose de fort ridicule. Il fut à
Meudon\,; Monseigneur le reçut en dehors, ils n'entrèrent point dans la
maison à cause de la main\,; il se trouva une calèche dans laquelle ils
montèrent tous deux en même temps par chacun un côté. Ils se promenèrent
beaucoup\,; au sortir de la calèche, l'électeur prit congé et s'en alla
à Paris, et de manière que Monseigneur ne le vit, ni en arrivant ni en
partant, descendre ni monter en carrosse. De cette façon, quoique
Monseigneur fût à droite dans la calèche, la main fut couverte par
monter en même temps par différent côté, et par cette affectation de
n'entrer pas dans la maison, et ne la voir que par les dehors.

Cette tolérance, colorée du prétexte des malheurs d'un allié si proche,
parut une faiblesse qui scandalisa étrangement la cour. Une prétention
si nouvelle, si fort inouïe, et quand elle aurait eu un fondement, qui
lui manquait par l'incognito et l'exemple de l'électeur de Cologne, fut
le fruit du mépris où le roi laissa si volontiers tomber les premières
dignités de son royaume, d'où sa couronne même se sentit, et reçut en
cette occasion une flétrissure marquée. On se contente de renvoyer {[}à
ce qui a déjà été dit plus haut{]}, sans répéter ce qui s'y trouve
là-dessus.

L'électeur ne vit personne autre que le roi chez lui\,; Torcy
l'introduisit encore une fois dans son cabinet un matin, à Versailles,
par le petit degré. La conversation fut courte et d'affaires\,; il
retourna aussitôt à Paris, et peu de jours après à Compiègne. M. le duc
d'Orléans lui avait donné un grand souper à Saint-Cloud, dont il sera
parlé ailleurs.

Il ne faut pas oublier, parmi les entreprises et les prétentions de
l'électeur de Bavière, un changement de langage fort remarquable de
Monasterol son envoyé, et de toute sa petite cour, en parlant de lui.
Jusqu'alors ils avaient suivi l'usage de tous les temps dans notre
langue de dire \emph{M. l'électeur}, et je ne sais que le pape,
l'empereur et les rois qu'on nomme de leur dignité, parce que
\emph{monseigneur} ni \emph{monsieur} ne sont pour eux d'aucun usage. Ce
fut apparemment pour y égaler leur maître en tant qu'il fut en eux,
qu'ils supprimèrent le monsieur, et ne dirent plus que
\emph{l'électeur\,;} cette gangrène passa aisément aux François, se
communiqua à la cour, et changea l'usage ancien. \emph{M. l'électeur
fut} une façon de parler vieillie et abolie, et, sans aucune réflexion
\emph{, l'électeur} tout court s'introduisit, tellement que depuis on ne
dit plus que \emph{l'électeur de Bavière, l'électeur de Saxe, l'électeur
de Mayence}, ainsi des autres, comme on dit simplement \emph{le roi
d'Angleterre, le roi de Suède} et des autres rois.

Ainsi tout passe, tout s'élève, tout s'avilit, tout se détruit, tout
devient chaos, et il se peut dire et prouver, {[}pour{]} qui voudrait
descendre dans le détail, que le roi dans la plus grande prospérité de
ses affaires, et plus encore depuis leur décadence, n'a été pour le rang
et la supériorité pratique et reconnue de tous les autres rois et de
tous les souverains non rois, qu'un fort petit roi, en comparaison de ce
qu'ont été à leur égard à tous, et sans difficulté aucune, nos rois
Philippe de Valois, Jean, Charles V, Charles VI, que je choisis parmi
les autres, comme ayant régné dans les temps les plus malheureux et les
plus affaiblis de la monarchie.

Le fameux cardinal Portocarrero, duquel j'ai parlé tant de fois, mourut
en ce temps-ci, après s'être longtemps survécu, et laissa
M\textsuperscript{me} des Ursins plus puissante que jamais, délivrée
d'un fantôme qui depuis longtemps ne l'embarrassait plus, mais qui
intérieurement l'incommodait toujours. Ce cardinal, depuis qu'il ne fut
plus de rien, s'était tourné entièrement à la plus exacte piété, et
mourut d'une manière grande et édifiante à Madrid, qui est du diocèse de
Tolède. Il voulut être enterré dans le tournant d'un bas-côté de son
église de Tolède, devant l'entrée de la chapelle appelée des
Nouveaux-Rois, qui est elle-même une magnifique église, quia son
chapitre et son service particulier. Il défendit que sa sépulture fût
élevée ni ornée en aucune sorte, mais qu'on pût passer et marcher
dessus, et il ordonna pour toute épitaphe qu'on y gravât uniquement ces
paroles\,: \emph{Hic jacet cinis, pelvis et nihil}. Il a été exactement
obéi. Je l'ai vu à Tolède, où il est en grande vénération. Il n'y a ni
armes, ni quoi que ce soit sur sa tombe, toute plate et unie au pavé,
que ces seules paroles. On a seulement mis à la muraille, auprès de la
porte de cette chapelle des Nouveaux-Rois, ses armes, ses qualités, le
jour de sa mort, le lieu de sa sépulture, et qu'on s'y est conformé à sa
volonté.

L'évêque de Chartres mourut aussi consommé de travaux et d'étude, sans
être encore vieux. C'était fort peu de chose pour la naissance, et
néanmoins avec des alliances proches qui lui faisaient honneur. Il
s'appelait Godet, et il était frère de Françoise Godet, femme d'un riche
partisan nommé J. Gravé, dont ici fille épousa Ch. des Monstiers, comte
de Mérinville, fils aîné du lieutenant général de Provence, reçu
chevalier de l'ordre à la promotion de 1661, avec M. le prince de Conti
et quelques autres, par le duc d'Arpajon, chargé de la commission du
roi, et père de l'évêque de Chartres dont je parlerai bientôt.

Ce même Godet, évêque de Chartres, était cousin germain d'autre
Françoise Godet, femme d'Antoine de Brouilly, marquis de Piennes,
gouverneur de Pignerol et chevalier de l'ordre aussi en 1661, desquels
la duchesse d'Aumont et la marquise de Châtillon étaient filles.

M. de Chartres, Godet, des premiers élèves de Saint-Sulpice, fut
peut-être celui qui lit le plus d'honneur et de bien à ce séminaire, qui
est depuis devenu une manière de congrégation et une pépinière
d'évêques. C'était un grand homme de bien, d'honneur, de vertu,
théologien profond, esprit sage, juste, net, savant d'ailleurs, et qui
était propre aux affaires, sans pédanterie pour lui, et sachant vivre et
se conduire avec le grand monde, sans s'y jeter et sans en être
embarrassé. Ses talents et le crédit naissant de ce séminaire, ennemi du
jansénisme, le fit connaître. M\textsuperscript{me} de Maintenon venait
d'établir à Noisy ce qu'elle transporta depuis à Saint-Cyr, qui est du
diocèse de Chartres. L'abbé Godet avait été porté à cet évêché après la
mort du frère et de l'oncle des deux maréchaux de Villeroy, et y
paraissait déjà un grand évêque, tout appliqué à son ministère.
L'établissement de Saint-Cyr lui donna une relation nécessaire avec
M\textsuperscript{me} de Maintenon. Ce fut avec lui et par lui que tous
les changements de forme en ces commencements, et les règlements ensuite
se firent. M\textsuperscript{me} de Maintenon le goûta au point qu'elle
le fit le supérieur et le directeur immédiat de Saint-Cyr, son directeur
à elle-même\,; et pour en dire le vrai, le dépositaire de son cour et de
son âme, pour qui elle n'eut jamais depuis rien de caché\,; elle
l'approcha du roi tant qu'elle put, pour contrebalancer le P. de La
Chaise et les jésuites, qu'elle n'aimait pas, dans la distribution des
bénéfices, et elle l'avança jusqu'à ce point, qu'il devint le confident
de leur mariage. Il en parlait et en écrivait librement au roi, le
félicitant souvent d'avoir une épouse si accomplie. Je n'en ai pas vu
les lettres, mais son neveu et son successeur qui les a vues, et qui en
a encore des copies, parce que dans quelques-unes il s'agissait aussi
d'affaires, me l'a dit bien des fois, longues années depuis leur mort à
tous.

Un homme, parvenu à ce point de confiance et de familiarité devient un
personnage. Aussi le fut-il toute sa vie, devant qui le clergé rampait,
et avec qui les ministres étaient à «\,plaît-il, maître\,?» et il en
prit mal au chancelier de Pontchartrain d'avoir osé, quoiqu'il eût
raison, lui tenir tête, dont il ne s'est jamais relevé, comme je l'ai
rapporté ailleurs.

On a vu aussi en son lieu toute la part qu'il eut dans l'affaire de
M\textsuperscript{me} Guyon et de l'archevêque de Cambrai, avec quelle
adresse il s'y conduisit dans sa naissance, avec quelle force dans ses
suites, et avec combien d'union avec M. de Meaux et le cardinal de
Noailles.

Avec tant de crédit qu'il a eu toute sa vie sans lacune, jamais homme
plus simple, plus modeste, moins précieux\,; qui le fit moins sentir à
personne. Il logeait à Paris dans un petit appartement fort court au
séminaire de Saint-Sulpice, où il était parmi eux comme l'un d'eux, et
partout l'homme le plus doux et le plus accessible, quoique accablé
d'occupations. Il n'était que peu à Paris, et jamais que par nécessité
d'affaires, souvent à Saint-Cyr, et ne couchait jamais à Versailles\,;
il y faisait rarement sa cour, mais voyait le roi chez
M\textsuperscript{me} de Maintenon ou chez lui par les derrières, jamais
à Fontainebleau, et comme jamais à Marly, hors de quelque nécessité
pressante, et pour le moment précis\,; assidu dans son diocèse, à ses
visites tous les ans et à toutes ses fonctions, et au gouvernement de
son diocèse, comme s'il n'eut pas eu d'autres soins, et celui-là passait
devant tous il connaissait aussi tous ses curés, tous ses prêtres et
tout ce qui se passait dans son diocèse si exactement et par lui-même,
qu'il semblait qu'il n'avait que quelques paroisses à conduire, et son
gouvernement entrait dans tous les détails avec une charité pleine
d'égards, de douceur et de sagesse. Sa dépense, ses meubles, sa table,
tout était frugal, et tout le reste pour les pauvres.

Parmi tant d'affaires particulières du diocèse, et générales de tout ce
qui arrivait dans l'Église de France sur la doctrine et la discipline,
les lettres longues et journalières qu'il recevait et qu'il répondait à
M\textsuperscript{me} de Maintenon quand il n'était pas à Saint-Cyr, et
quelquefois au roi, il ne laissait pas d'écrire des ouvrages de
doctrine, et ce surcroît de travail le consuma.

L'impression lui coûtait, les voyages, ses visites\,; il n'avait ni le
temps ni le goût de songer à ses affaires temporelles. Elles se
trouvèrent si courtes qu'il demanda au roi une abbaye, et lui dit
franchement ses besoins. Ce détail, qui n'a jamais été su, son neveu me
l'a conté bien des années après. L'abbaye ne venait point\,; il en
reparla à M\textsuperscript{me} de Maintenon. Enfin le roi lui dit que,
dans la réputation où il était, une abbaye la ternirait et ferait parler
le monde\,; qu'après y avoir bien pensé, il aimait mieux lui donner une
pension de vingt mille livres qui ne se saurait point, qui n'aurait
point de bulles, et qui le soulagerait davantage. Il la lui fit expédier
et payer en secret jusqu'à sa mort, en sorte qu'elle a toujours été
ignorée. Cette petite anecdote montre combien il lui était cher.

Avec tant de qualités, ce prélat n'a pas laissé de ruiner le clergé de
France, et d'ouvrir par là une large porte à tout ce qui a coulé d'une
source si empoisonnée. Sa petite naissance, ou plutôt vile et obscure,
l'éloigna de la bonne comme par nature, et comme par une seconde nature
puisée à Saint-Sulpice, non seulement il prit en haine le jansénisme,
mais tout ce qui en put être soupçonné, particuliers, corps, écoles\,;
et avec une intention droite, mais aveuglée, il ne fut pas moins ardent,
ni moins aisément prévenu, ni moins capable de revenir là-dessus par
zèle, que les jésuites, par intérêt et par ambition, quoiqu'il les
connût et qu'il ne les aimât pas. Je me suis étendu (t. II, p.~422) sur
la plaie qu'il fit à l'Église de France par l'introduction dans
l'épiscopat de gens de rien, ignorants, ardents, sans éducation, dont
l'abus a si fort grossi depuis par le P. Tellier, et que la même raison
de naissance et d'autres qui se retrouveront peut-être ailleurs ont plus
que jamais suivi sous le règne du cardinal Fleury.

M. de Chartres, dont les infirmités augmentaient tous les jours, mais
qui n'en relâchait rien de ses travaux, se résolut à se donner un
coadjuteur qui, élevé de sa main et dans son esprit, fût un autre
lui-même pour le gouvernement de son diocèse. Il choisit l'abbé de
Mérinville, son petit-neveu, dans l'élection duquel il crut que la chair
et le sang cédaient toute la part à l'esprit. Il n'avait pourtant pas
encore vingt-six ans, et il en faut vingt-sept pour être sacré\,; le
proposer et le faire agréer fut pour lui la même chose, mais s'il n'eut
pas la satisfaction de le sacrer, il eut au moins celle de pouvoir
compter ne s'être pas trompé.

Son petit-neveu, le voyant au lit de la mort, lui témoigna ce qu'il
pensait de la différence d'être longtemps formé sous lui, ou de se voir
évêque en plein à son âge, et le pria instamment et avec larmes de le
décharger de ce fardeau\,; l'oncle l'écouta, ne répondit point, et
demeura longtemps recueilli\,; il le rappela ensuite, et lui dit
qu'après y avoir bien pensé devant Dieu, il persistait à croire qu'il
ferait bien et que c'était sa volonté qu'il fût évêque de Chartres.

Il mourut fort peu de temps après dans Chartres, fort saintement,
laissant un regret universel dans son diocèse. Le coadjuteur pressa
M\textsuperscript{me} de Maintenon, et par lettres, et dès qu'il put la
voir, de faire nommer un autre évêque. Sa jeunesse et ses instances ne
purent la persuader. Il fallut malgré lui demeurer évêque, et il fut
sacré dès qu'il eut vingt-sept ans avec la même supériorité et direction
de Saint-Cyr, qu'avait son oncle. Il a paru que Dieu a béni ce choix\,;
il en a fait un des plus saints et des plus sages évêques de France, des
plus assidus et appliqués en son diocèse, d'où il ne sort presque
jamais, et qui, sans avoir la science ni le monde de son oncle, fait
aimer et respecter la vertu et craindre le vice sans le poursuivre,
sinon dans les cas de nécessité et avec charité. Il fait craindre aussi
la cour par sa liberté à dire la vérité, et avec toute l'apparente
saleté et grossièreté des séminaires, il ne laisse pas d'avoir de
l'adresse et de la délicatesse dans le gouvernement\,; il vit
austèrement, tout à ses fonctions et ses visites, est à peine nourri et
vêtu, donne tout aux pauvres, et n'a jamais voulu demander d'abbayes ni
recevoir celles qui lui ont été données.

La mort de M. de Chartres mit deux hommes sur le chandelier qu'il avait
fort recommandés à M\textsuperscript{me} de Maintenon\,: Bissy, évêque
de Meaux, auparavant de Toul, bientôt après cardinal, qui succéda à
toute sa confiance pour les affaires de l'Église, dont il sut faire sa
fortune et bien pis, et La Chétardie, curé de Saint-Sulpice, fort saint
prêtre, mais le plus imbécile et le plus ignorant des hommes.

Ce dernier succéda à la confiance personnelle de M\textsuperscript{me}
de Maintenon, il fut son confesseur, son directeur, et par là le fut un
peu aussi de Saint-Cyr. Ce qui est étonnant à n'en pas revenir à qui a
connu le personnage, c'est que fort tôt après, M\textsuperscript{me} de
Maintenon avec tout son esprit n'eut plus de secret pour lui comme elle
n'en avait point pour feu M. de Chartres, et qu'elle lui écrivait sans
cesse pour le consulter, même sur les affaires, ou pour les lui
mander\,; et, ce qui n'est pas moins inconcevable, c'est que ce bonhomme
qui, non content des soins de sa vaste cure, était encore supérieur de
la Visitation-Sainte-Marie de Chaillot, y portait très souvent les
lettres de M\textsuperscript{me} de Maintenon, et les lisait à la
grille, même devant de jeunes religieuses. Une soeur de
M\textsuperscript{me} de Saint-Simon, religieuse en cette maison, dont
elle a été depuis souvent supérieure, et qui a infiniment d'esprit, et
d'esprit de gouvernement, avec toute la sainteté de son état, et toutes
les grâces du monde, pâmait quelquefois de stupeur des secrets qu'elle
entendait là avec d'autres religieuses, par lesquelles après mille
choses se savaient, sans que personne pût comprendre par où ces mystères
avaient pu transpirer, et sans que, tant que ce curé a vécu, ce qui fut
encore quelques années, M\textsuperscript{me} de Maintenon l'ait su et
s'en soit pu déprendre.

Il influait très gauchement à tout, gâtait force affaires, en protégeait
de fort misérables, n'avait pas les premières notions de rien, et tout
simplement se targuait de son crédit et se faisait une petite cour. Pour
le Bissy, on lui verra incontinent prendre le plus grand vol.

Crécy mourut fort vieux\,; il était frère du P. Verjus, jésuite, ami
intime du P. de La Chaise, qui avait fort contribué à sa fortune.
C'était un petit homme accort, doux, poli, respectueux, adroit, qui
avait passé sa vie dans les emplois étrangers, et qui en avait pris
toutes les manières, jusqu'au langage très longtemps à Ratisbonne, puis
dans plusieurs petites cours d'Allemagne\,; enfin, second ambassadeur
plénipotentiaire au traité de paix de Ryswick. Il avait beaucoup
d'insinuation, l'art de redire cent fois la même chose, toujours en
différentes façons, et une patience qui, à force de ne se rebuter point,
réussissait très souvent. Personne ne savait plus à fond que lui les
usages, les lois et le droit de l'empire et de l'Allemagne, et
{[}savait{]} fort bien l'histoire\,; il était estimé et considéré dans
les pays étrangers, et y avait fort bien servi. Il était fort vieux, et
homme de très peu.

Marivaux, lieutenant général, mourut aussi. Son nom était de L'Ile, de
la seigneurie de l'Ile, qu'ils possédaient en la châtellenie de
Pontoise, dès l'an 1069, qu'Adam Ier, seigneur de l'Ile, signa avec les
officiers de la couronne, en cette année, la charte de confirmation que
Philippe Ier fit à Pontoise de la fondation de Saint-Martin, lors
Saint-Germain de Pontoise. Ce même Adam de L'Ile fit bâtir la forteresse
et le bourg appelé de son nom l'Ile-Adam, qu'Isabelle, héritière de
l'aîné, et femme de Jean, seigneur de Luzarches, de Jouy, etc., laissa à
sa fille, veuve du seigneur de Joigny, laquelle vendit l'Ile-Adam en
1364, à Pierre de Villiers, seigneur de Macy, souverain maître d'hôtel,
c'est-à-dire grand maître de France, bisaïeul du célèbre Philippe de
Villiers, dernier grand maître de Rhodes et premier grand maître de
Malte, ou il mourut en 1534.

Marivaux dont je parle descendait en droite ligne de cet Adam Ier qui
bâtit l'Ile-Adam, qui des Villiers passa aux Montmorency, de là tomba
dans la branche de Condé de la maison royale, enfin à M. le prince de
Conti. Le grand-père de Marivaux était frère cadet du capitaine des
gardes d'Henri III, si connu par son duel derrière les Chartreux, contre
le seigneur de Marolles, ligueur\,; et de celui qui fût chevalier du
Saint-Esprit en qui n'eurent point de postérité masculine. Ce grand-père
de Marivaux avait épousé une Balzac\,; tua de sa main, à la bataille
d'Ivry (1590), le général de la cavalerie espagnole, fut gouverneur de
Corbeil, la Bassée, la Capelle, et d'Amiens. Son fils se maria mal et ne
figura point. Celui dont je parle, sans protection et avec peu de bien,
épousa une fille de Guénégaud, trésorier de l'épargne, et servit toute
sa vie avec réputation de valeur et de capacité.

Il savait et avait beaucoup d'esprit, une fort belle figure, de la
finesse et de la plaisanterie dans l'esprit, et la langue fort libre,
qui le faisait craindre. Il me prit en amitié à l'armée, et je
m'accommodais fort de lui\,; personne n'était de meilleure compagnie\,;
les secrétaires d'État de la guerre ni leurs commis ne l'aimaient pas,
et lui ne s'en contraignait guère.

Il pensa se noyer à un retour d'armée en traversant la Marne\,; le bac
enfonça\,; cette aventure fit du bruit\,; le roi lui demanda comment il
s'était sauvé\,: il avait été en effet longtemps rejeté par des bords
escarpés, sur lesquels il s'était trouvé des gens peu empressés de le
secourir. Il dit au roi que, désespérant de leur charité, il s'était
avisé de s'écrier qu'il était le neveu de M. l'intendant, et qu'à ce nom
il avait été secouru sur-le-champ et là-dessus fit une parenthèse au roi
sur le pouvoir des intendants, qui divertit extrêmement l'assistance,
mais qui ne plut pas tant au roi, et qui ne servit pas à son avancement.

Il mourut vieux, et a laissé un fils capitaine de gendarmerie qu'on dit
aussi avoir beaucoup d'esprit. Marivaux eut des amis et conserva toute
sa vie beaucoup de considération. Sa soeur, qui avait aussi beaucoup
d'esprit, et qui était la femme du monde la plus haute, avait épousé
Cauvisson, un des lieutenants généraux de Languedoc.
M\textsuperscript{me} de Nogaret, dame du palais de
M\textsuperscript{me} la duchesse de Bourgogne, était veuve sans enfants
de son fils, de laquelle j'ai parlé plus d'une fois\,; le nom des
Cauvisson est Louvet, gens nouveaux et de fort peu de chose.

M\textsuperscript{me} de Moussy, soeur du feu premier président Harlay,
grande dévote de profession, avec tous les apanages de ce métier, et
tout aussi composée que lui, mourut sans enfants. Elle avait toujours
vécu avec son frère et son neveu dans une grande amitié, et presque
toujours logée avec eux. Elle déshérita pourtant son neveu sans cause
aucune de brouillerie, qui fut bien étonné de trouver un testament qui
donnait tout aux hôpitaux. Elle était veuve du dernier Bouteiller\,;
c'est du dernier de cette grande maison dans laquelle le comté de Senlis
avait été longtemps, et à qui le nom de Bouteiller, ou de Bouteiller de
Senlis, était demeuré pour avoir eu plusieurs fois ce grand office alors
de grand bouteiller de France, dont on trouve la signature, c'est-à-dire
le sceau et la présence citée dans les anciennes chartes de nos rois
avec le dapifer, qui est le grand maître, ou comme ils disaient, le
souverain maître d'hôtel, le grand chambellan, le connétable, qui
n'était dans les premiers temps que le grand écuyer, et le chancelier le
dernier de tous\,; plus anciennement encore le premier était le sénéchal
\footnote{Le sénéchal fut jusqu'en 1191 le premier des grands officiers
  de la couronne. Philippe Auguste supprima cette charge comme donnant
  trop de puissance à celui qui en était revêtu.}, monté en maire du
palais, et descendu en grand maître, car le plus ou le moins de
puissance fit ces trois noms du même office.

M. de Luxembourg, pendant son séjour à Rouen, y perdit sa femme. J'ai
dit ailleurs qui elle était, et quelle aussi, par l'éclat que cela fit,
qui fut toujours caché pour le seul mari avec qui elle avait l'art et le
soin de vivre comme la femme la plus tendrement attachée à tous ses
devoirs. Il en fut aussi tellement affligé, que ce contraste avec la vie
qu'elle n'avait point cessé de mener fit le plus scandaleux ridicule.
Abeille, qui avait été secrétaire du maréchal de Luxembourg\,; et que
son esprit et son petit collet avait mêlé dans les meilleures et les
plus brillantes compagnies, et mis dans les académies, était un homme
d'honneur et de vertu, qui par reconnaissance et par attachement était
demeuré chez M. de Luxembourg. Il ne put souffrir une scène si publique,
et il apprit à M. de Luxembourg tout ce que lui-même avait été
jusqu'alors le premier à lui cacher. Le pauvre homme fut étrangement
surpris et très subitement consolé.

Cet automne fut la dernière saison qui vit debout le fameux monastère de
Port-Royal des Champs, en butte depuis si longtemps aux jésuites, et
leur victime à la fin. Je ne m'étendrai point sur l'origine, les
progrès, les suites, les événements d'une dispute et d'une querelle si
connue, ainsi que les deux partis moliniste et janséniste, dont les
écrits dogmatiques et historiques feraient seuls une bibliothèque
nombreuse, et dont les ressorts se sont déployés pendant tant d'années à
Rome et en notre cour. Je me contenterai d'un précis fort court, qui
suffira pour l'intelligence du puissant intérêt qui a tant remué de
prodigieuses machines, parce qu'on n'en peut supprimer les faits qui
doivent tenir place dans ce qui s'est passé de ce temps.

L'ineffable et l'incompréhensible mystère de la grâce, aussi peu à
portée de notre intelligence et de notre explication que celui de la
Trinité, est devenu une pierre d'achoppement dans l'Église, depuis que
le système de saint Augustin sur ce mystère a trouvé presque aussitôt
qu'il a paru des contradicteurs dans les prêtres de Marseille. Saint
Thomas l'a soutenu ainsi que les plus éclairés personnages\,; l'Église
l'a adopté dans ses conciles généraux, et en particulier l'Église de
Rome et les papes.

De si vénérables décisions et si conformes à la condamnation faite et
réitérée par les mêmes autorités, de la doctrine des pélagiens et des
demi-pélagiens, n'a pu empêcher une continuité de sectateurs de la
doctrine opposée qui, n'osant se présenter de front, ont pris diverses
sortes de formes, pour se cacher à la manière des demi-ariens autrefois.

Dans les derniers temps, les jésuites, maîtres des cours par le
confessionnal de presque tous les rois et de tous les souverains
catholiques, de presque tout le public par l'instruction de la jeunesse,
par leurs talents et leur art\,; nécessaires à Rome pour en insinuer les
prétentions sur le temporel des souverains, et la monarchie sur le
spirituel, à l'anéantissement de l'épiscopat et des conciles généraux,
devenus redoutables par leur puissance et par leurs richesses toutes
employées à leurs desseins, autorisés par leur savoir de tout genre, et
par une insinuation de, toute espèce, aimables par une facilité et un
tour qui ne s'était point encore rencontré dans le tribunal de la
pénitence, et protégés par Rome, comme des gens dévoués par un quatrième
voeu au pape, particulier à leur société, et plus propres que nuls
autres à étendre son suprême domaine, recommandables d'ailleurs par la
dureté d'une vie toute consacrée à l'étude, à la défense de l'Église
contre les hérétiques, et la sainteté de leur établissement et de leurs
premiers pères\,; terribles enfin par la politique la plus raffinée, la
plus profonde, la plus supérieure à toute autre considération que leur
domination, soutenue par un gouvernement dont la monarchie, l'autorité,
les degrés, les ressorts, le secret, l'uniformité dans les vues, et la
multiplicité dans les moyens sont l'âme\,; les jésuites, dis-je, après
divers essais, et surtout après avoir subjugué les écoles de delà les
monts, et tant qu'ils avaient pu, énervé celles de deçà partout,
hasardèrent, par un livre de leur P. Molina, une doctrine sur la grâce
tout à fait opposée au système de saint Augustin, de saint Thomas, de
tous les Pères, des conciles généraux, des papes et de l'Église de Rome,
qui, prête plusieurs fois à l'anathématiser, a toujours différé à le
faire. L'Église de France surtout se souleva contre ces agréables
nouveautés qui faisaient tant de conquêtes par la facilité du salut et
l'orgueil de l'esprit humain.

Les jésuites, embarrassés d'une défensive difficile, trouvèrent moyen de
semer la discorde dans les écoles de France, et par mille tours de
souplesse, de politique et de force ouverte, enfin par l'appui de la
cour, de changer la face des choses, d'inventer une hérésie qui n'avait
ni auteur ni sectateur, de l'attribuer à un livre de Cornelius
Jansenius\,; évêque d'Ypres, mort dans le sein de l'Église et en
vénération\,; de se rendre accusateurs de défendeurs qu'ils étaient, et
leurs adversaires d'accusateurs, défendeurs. De là est venu le nom de
moliniste et de janséniste, qui distingue les deux partis.

Grands et longs débats à Rome sur cette idéale hérésie, enfantée ou
plutôt inventée par les jésuites, pour faire perdre terre aux
adversaires de Molina\,; discussion devant une congrégation formée
exprès sous le nom \emph{De auxiliis}, tenu un grand nombre de séances
devant Clément VIII (Aldobrandin), et Paul V (Borghèse) qui, ayant enfin
formé un décret d'anathème contre la doctrine de Molina, n'osa le
publier, et se contenta de ne pas condamner cette doctrine sans oser
l'approuver, en les consolant par tout ce qui les put flatter sur cette
hérésie idéale, soutenue de personne, et dont ils surent si bien
profiter.

Plusieurs saints et savants personnages s'étaient les uns après les
autres retirés à l'abbaye de Port-Royal des Champs. Les uns y
écrivirent, les autres y rassemblèrent de la jeunesse qu'ils
instruisirent aux sciences et à la piété. Les plus beaux ouvrages de
morale, et qui ont le plus éclairé dans la science et la pratique de la
religion, sont sortis de leurs mains, et ont été trouvés tels par tout
le monde.

Ces messieurs eurent des amis et des liaisons\,; ils entrèrent dans la
querelle contre le molinisme. C'en fut assez pour ajouter à la jalousie
que les jésuites avaient conçue de cette école naissante, une haine
irréconciliable, d'où naquit la persécution des jansénistes, de la
Sorbonne, de M. Arnauld, considéré comme le maître de tous, et la
dissipation des solitaires de Port-Royal\,; de là l'introduction d'un
formulaire, chose si souvent fatale et si souvent proscrite dans
l'Église, par lequel la nouvelle hérésie, inventée et soutenue de
personne, fut non seulement proscrite, ce qui aurait été accepté de tout
le monde sans difficulté, mais fut déclarée contenue dans le livre
intitulé \emph{Augustines} composé par Cornelius Jansenius, évêque
d'Ypres, et ce formulaire\footnote{Le formulaire, ou déclaration par
  laquelle on condamnait les cinq propositions en affirmant qu'elles
  étaient contenues dans le livre de Cornelius Jansenius, avait été
  rédigé dès 1656. Mais les résistances que la signature du formulaire
  avait rencontrées décidèrent Louis XIV à venir au parlement le 29
  avril 1664 pour faire enregistrer une déclaration qui imposait cette
  signature à tous les ecclésiastiques et aux membres des communautés
  religieuses d'hommes et de femmes.} proposé à jurer pour la croyance
intérieure et littérale de son contenu.

Le droit, c'est-à-dire la proscription des cinq propositions hérétiques,
que personne ne soutenait, ne fit aucune difficulté\,: le fait,
c'est-à-dire qu'elles étaient contenues dans ce livre de Jansenius, en
fit beaucoup. Jamais on ne put en extraire aucune\,: on se sauva par
soutenir qu'elles s'y trouvaient éparses, sans pouvoir encore citer où
ni comment. Jurer sur son Dieu et sur son âme de croire ce qu'on ne
croit point fondé en chose de fait, qu'on ne peut montrer ce qu'on
propose de croire, parut un crime à tout ce qu'il y avait de gens
droits. Un très grand soulèvement éclata donc dès que ce formulaire
parut.

Mais ce qui en sembla encore plus insupportable, c'est que, pour
détruire Port-Royal, qu'on jugeait bien qui ne se résoudrait jamais à ce
serment, on le proposa à signer aux religieuses par tout le royaume. Or
proposer de jurer qu'un fait est contenu dans un livre qu'on n'a point
lu, dans un livre même qu'on n'a pu lire, parce qu'il est en latin et
qu'on ignore cette langue, c'est une violence qui n'eut jamais
d'exemple, et qui remplit les, provinces d'exilés, et les prisons et les
monastères de captifs.

La cour ne ménagea rien en faveur des jésuites, qui lui firent oublier
la ligue et ses suites, et accroire que les jansénistes étaient une
secte d'indépendants, qui n'en voulaient pas moins à l'autorité royale,
qu'ils se montraient réfractaires à celle du pape, que les jésuites
appelaient l'Église, qui avait approuvé, puis prescrit la signature du
formulaire. La distinction du fait d'avec le droit, soufferte quelque
temps, fut enfin proscrite, comme une rébellion contre l'Église, encore
que non seulement elle n'eût point parlé, mais qu'elle n'ait jamais
exigé la croyance des faits qu'elle a décidés par ses conciles
généraux\,; et les plus reconnus pour oecuméniques, de plusieurs
desquels, décidés de la sorte, on doute et on dispute encore sans être,
pour cela, ni répréhensible ni repris. Les bénéfices attachés à la
protection des jésuites, dont le confesseur du roi était distributeur\,;
le crédit ou l'inconsidération, et pis encore\,; qu'éprouvaient les
prélats à proportion que la cour et les jésuites étoient contents ou
mécontents, échauffèrent la persécution jusqu'à la privation des
sacrements, même à la mort.

De tels excès réveillèrent enfin quelques évêques, qui écrivirent au
pape, et qui s'exposèrent à la déposition à laquelle on commençait à
travailler lorsqu'un plus grand nombre de leurs confrères vinrent à leur
secours, et soutinrent la même cause.

Alors Rome et la cour craignirent un schisme. D'autres évêques
s'interposèrent, et avec eux le cardinal d'Estrées, évêque-duc de Laon
alors, et cardinal quatre ou cinq ans après. La négociation réussit par
ce que l'on nomma \emph{la paix} de Clément IX (Rospigliosi)\footnote{\emph{La
  paix} de Clément IX est de l'année 1668. Elle suspendit pour quelque
  temps les querelles du jansénisme.}, qui déclara authentiquement que
le saint-siège ne prétendait et n'avait jamais prétendu que la signature
du formulaire obligeât à croire que les cinq propositions condamnées
fussent implicitement ni explicitement dans le livre de Jansenius, mais
seulement de les tenir et de lés condamner comme hérétiques en quelque
livre et en quelque endroit qu'elles se pussent trouver. Cette paix
rendit la liberté et les sacrements aux personnes qui en avaient été
privées, et les places aux docteurs et autres qui en avaient été
chassés.

Je n'en dirai pas davantage, parce que ce peu que j'ai expliqué suffira
pour faire entendre ce qui doit être rapporté présentement et dans la
suite, et je continuerai à me servir des mots jansénisme et de
jansénistes, de molinisme et de molinistes pour abréger.

Les jésuites et leurs plus affidés furent outrés de cette paix que tous
leurs efforts ici et à Rome n'avaient pu empêcher. Ils avaient su
habilement donner le change {[}sur le jansénisme{]} et sur le molinisme,
et de défendeurs devenir agresseurs. Les jansénistes, tout en se
défendant sur les cinq propositions qu'ils condamnaient et que personne
n'avait jamais soutenues, et sur le formulaire quant au fait, n'avaient
point quitté prise sur la doctrine de Molina, ni sur les excès qui
s'ensuivaient de cette morale, que le fameux Pascal rendit également
palpables, existants dans la doctrine et la pratique des casuistes
jésuites, et ridicules, dans ces ingénieuses lettres au provincial, si
connues sous le nom de \emph{Lettres provinciales}. L'aigreur et la
haine continuèrent, et la guerre se perpétua par les écrits, et les
jésuites se fortifièrent de plus en plus dans les cours pour accabler et
pour écarter leurs adversaires ou les suspects de toutes les places de
l'Église et des écoles.

Vinrent longtemps après les disputes des jésuites avec les autres
missionnaires des Indes, surtout à la Chine, sur les cérémonies que les
uns prétendaient purement politiques, les autres idolâtriques, dont j'ai
parlé (t. II, p.~417) à l'occasion du changement de confesseur de
M\textsuperscript{me} la duchesse de Bourgogne, et depuis encore à
l'occasion du choix du P. Tellier pour confesseur du roi, engagé fort
avant dans cette dispute, qui en écrivit, dont le livre fut mis à
l'index, sauvé de pis à toute peine, et lui contraint de sortir de Rome
et de se retirer en France.

La querelle s'échauffait et bâtait mal pour les jésuites\,; le P.
Tellier y prenait une double part. C'était, comme je l'ai dit, un homme
ardent et dont la divinité était son molinisme et l'autorité de sa
compagnie. Il se vit beau jeu\,: un roi très ignorant en ces matières,
et qui n'avait jamais écouté là-dessus que les jésuites et les leurs,
suprêmement\,; plein de son autorité, et qui s'était laissé persuader
que les jansénistes en étaient ennemis, qui voulait se sauver, et qui,
ne sachant point la religion, s'était flatté toute sa vie de faire
pénitence sur le dos d'autrui, et se repaissait de la faire sur celui
des huguenots et des jansénistes qu'il croyait peu différents, et
presque également hérétiques\,; un roi environné de gens aussi ignorants
que lui et dans les mêmes préjugés, comme M\textsuperscript{me} de
Maintenon et MM. de Beauvilliers et de Chevreuse par Saint-Sulpice et
feu M. de Chartres, ou par des courtisans et des valets principaux qui
n'en savaient pas davantage, ou qui ne pensaient qu'à leur fortune\,; un
clergé détruit de longue main, en dernier lieu par M. de Chartres, qui
avait farci l'épiscopat d'ignorants, de gens inconnus et de bas lieu qui
tenaient le pape une divinité, et qui avaient horreur des maximes de
l'Église de France, parce que toute antiquité leur était inconnue, et
qu'étant gens de rien, ils ne savaient ce que c'était que l'État\,; un
parlement débellé et tremblant, de longue main accoutumé à la,
servitude, et le peu de ceux qui par leurs places ou leur capacité
auraient pu parler, dévoués comme le premier président Pelletier, ou
affamés de grâces.

Il restait encore quelques personnes à craindre pour les jésuites,
c'est-à-dire pour leurs entreprises, comme les cardinaux d'Estrées,
Janson et Noailles, et le chancelier\,: ce dernier était, comme je l'ai
dit ailleurs, éreinté, et le P. Tellier ne l'ignorait pas\,; Estrées
était vieux et courtisan, Janson aussi, et de plus fort tombé de
santé\,; Noailles n'avait rien de tout cela, il était de plus dans la
liaison la plus grande avec M\textsuperscript{me} de Maintenon, puissant
à la cour par le goût du roi, par sa famille, par sa réputation soutenue
de sa vie et de sa conduite, archevêque de Paris, et en vénération dans
son diocèse et dans le clergé, à la tête duquel il se trouvait par tout
le royaume\,; celui-là était capitalement en butte aux jésuites par sa
doctrine, non suspecte, mais qui n'était pas la leur, et pour avoir été
mis à Chatons, puis à Paris, sans leur participation, et promu de même à
la pourpre\,; ils savaient que les jansénistes n'étaient pas contents de
lui, parce qu'il n'avait pas voulu s'en laisser dominer ni donner dans
toutes leurs vues, et que lui était encore moins content d'eux depuis la
découverte du véritable auteur du fameux \emph{cas de conscience} dont
j'ai parlé. Le P. Tellier, bien ancré auprès du roi, résolut de
commettre le cardinal de Noailles avec le roi d'un côté, avec les
jansénistes de l'autre, et d'achever en même temps l'ouvrage auquel ils
travaillaient depuis tant d'années, par la destruction entière de
Port-Royal des Champs.

Le P. de La Chaise s'était contenté, depuis que la paix de Clément IX
avait rétabli ces religieuses, de les empêcher de recevoir aucune fille
à profession, pour faire périr la maison par extinction, sans y
commettre d'autre violence\,; on a vu (t. IV, p.~122), par ce qui a été
rapporté que le roi dit à Maréchal, sur le voyage qu'il lui avait permis
et même ordonné d'y faire, qu'il se repentait de les avoir laissé
pousser trop loin, et qu'au fond il les regardait comme de très saintes
filles. Le nouveau confesseur vint à bout en peu de temps de changer ces
idées.

Il réveilla ensuite une constitution faite à Rome, depuis trois ou
quatre ans, à la poursuite des molinistes toujours attentifs à revenir,
à donner le change, et ardents à chercher les moyens de troubler la paix
de Clément IX. Rome, qui les ménageait comme les athlètes des
prétentions ultramontaines, auxquelles elle a tant sacrifié de nations,
n'osa tout refuser, mais ne voulut pas aussi aller de front contre
l'autorité de Clément IX\,; elle donna donc une constitution ambiguë
contre le jansénisme, mais en effleurant, et faite avec assez d'adresse
pour que ceux qui étaient attachés à cette paix pussent, sans la
blesser, recevoir cette constitution, d'ailleurs parfaitement inutile\,;
les molinistes furent affligés de n'avoir pu obtenir qu'un si faible
instrument, qui en effet ne faisait que condamner les cinq propositions
déjà proscrites et dont personne n'avait jamais pris la défense, et qui
d'ailleurs ne prescrivait rien de nouveau\,; mais comme dans les
disputes longues, et dans lesquelles la puissance séculière prend parti
jusqu'à la persécution, les esprits s'échauffent, et de part et d'autre
passent les bornes, il était arrivé que quelques jansénistes avaient
soutenu secrètement une, plusieurs, et même les cinq propositions
hérétiques, mais en grand secret\,; ce mystère avait été révélé dans les
papiers saisis dans l'abbaye de Saint-Thierry, dont il a été parlé à
propos de l'affaire que cette recherche fit à l'archevêque de Remis\,;
tout le parti janséniste se récria contre, renouvela sa soumission de
cœur et d'esprit à la condamnation de toutes les cinq propositions, que
sans ménagement il dit être cinq hérésies, et contre l'injustice de lui
attribuer celle de quelques tètes brûlées qu'il désavouait entièrement,
et avec qui il rompait de tout commerce et de société. Ces particuliers
mêmes, qui soutenaient l'erreur condamnée, étaient on ne peut pas ni
plus rares ni en plus petit nombre, et là-dessus, les uns criant à
l'injustice, les autres au péril de l'Église, le bruit se renouvela, qui
donna lieu à la constitution dont il vient d'être parlé.

Faute de mieux, le P. Tellier résolut d'en faire usage, dans l'espérance
d'en tirer parti au moins contre Port-Royal, plus délicat là-dessus que
personne d'entre les jansénistes, et d'y embarrasser le cardinal de
Noailles, à qui le roi ordonna de faire signer cette constitution\,;
comme elle n'altérait point dans le fond la paix de Clément IX, il n'osa
contredire, et se mit à faire signer les plus faciles à conduire, et des
uns aux autres gagner les moins aisés.

Cette conduite lui réussit si bien que Gif même signa. C'est une abbaye
de filles à cinq ou six lieues de Versailles qui a toujours été
considérée comme la soeur cadette de Port-Royal des Champs, en tout
genre, par amis et ennemis, et deux maisons qui en tout temps avaient
conservé l'union entre elles la plus intime.

Avec cette signature, le cardinal de Noailles se crut fort, et se
persuada que Port-Royal ne ferait point de difficultés. Il y fut trompé.
Ces filles, tant de fois et si cruellement traitées, en garde contre des
signatures captieuses qu'on leur avait si souvent présentées, dans une
solitude qui était sans cesse épiée, et qu'on ne pouvait aborder sans
péril d'exil et quelquefois de prison, par conséquent destituées de
conseils de confiance, ne purent être amenées à une nouvelle signature.
Aucune de celles qu'on leur montra ne les toucha, non pas même celle de
Gif. En vain le cardinal les exhorta, leur expliqua ce qu'on leur
demandait, qui ne blessait en rien la paix de Clément IX, ni les vérités
auxquelles elles étaient attachées\,; rien ne put rassurer la frayeur de
ces âmes saintes et timorées. Elles ne purent comprendre qu'une
signature nouvelle ne renfermât pas quelque venin et quelque surprise,
et leur courage ne put être ébranlé par la considération de tout ce dont
leur refus les menaçait.

C'était là ce qu'avaient espéré les jésuites, d'engager le cardinal de
Noailles et de parvenir enfin à détruire une maison qu'ils détestaient,
et dont ils n'avaient cessé depuis tant d'années de machiner la dernière
ruine. Ils mouraient de peur que les religieuses qui restaient ne
survécussent le roi, qu'après lui ils ne pussent continuer d'avoir le
crédit de les empêcher de recevoir des filles à profession, et que cette
maison ennemie subsistât, et se relevât, qui était toujours regardée
comme le centre, le chef-lieu et le ralliement du parti janséniste, dès
qu'on oserait y aborder.

Le cardinal, qui prévit un orage, mais non le destructif qui ne se
pouvait imaginer, pressa ces filles à plusieurs reprises, par d'autres
et par lui-même\,; il y alla plusieurs fois, toujours inutilement. Le
roi le pressait vivement, poussé de même par son confesseur, tant
qu'enfin le cardinal lâcha pied, procéda et leur ôta les sacrements.

Alors le P. Tellier les noircit auprès du roi de toutes les anciennes
couleurs qu'il renouvela, les fit passer dans son esprit pour des
révoltées, qui seules dans l'Église refusaient une signature trouvée
partout orthodoxe, et lui persuada qu'il ne serait jamais en repos sur
ces questions tant que ce monastère, fameux par ses rébellions contre
les deux puissances, subsisterait\,; enfin que sa conscience était pour
le moins aussi engagée que son autorité à une destruction si nécessaire,
et qui n'avait tardé que trop d'années. Le bon père piqua et tourna si
bien\,:le roi que les fers furent mis au feu pour la destruction.

Port-Royal de Paris n'était qu'un hospice de celui des Champs. Celui-ci
fut en entier transporté à Paris pendant plusieurs années, pendant
lesquelles on entretint les bâtiments du monastère des Champs, lequel ne
fut plus qu'une ferme. Ensuite, les religieuses, qu'on avait pris soin
de diviser dans les diverses persécutions qui leur furent suscitées,
furent séparées en deux monastères. Celles qui firent tout ce qu'on
voulut formèrent la maison de Paris, les autres celle des Champs, qui
n'eurent pas de plus grandes ennemies que celles de Paris, à qui tous
les biens presque furent adjugés dans l'espérance de faire tomber les
Champs par famine, mais qui se soutint par le travail, l'économie et les
aumônes.

Lorsqu'il fut question de la destruction, Voysin, encore conseiller
d'État, mais homme sûr et à tout faire pour la fortune, fut commis pour
les prétentions sur les Champs, où on peut juger de l'équité qui y fut
gardée.

Mais ce qui surprit étrangement, c'est que les religieuses des Champs se
mirent en règle et se pourvurent à Rome, où elles furent écoutées. Comme
la bulle ou la constitution \emph{Vineam Domini Sabaoth} n'y avait
jamais été accordée pour détruire la paix de Clément IX, on n'y trouva
point mauvais les difficultés de ces filles à la signer sans
l'explication qu'elles offraient d'ajouter en signant\,: sans préjudice
de la paix de Clément IX à laquelle elles adhéraient. Ce qui était leur
crime en France, digne d'éradication et des dernières peines
personnelles, parut fort innocent à Rome. Elles se soumettaient à la
bulle, et dans le même esprit qu'elle avait été donnée\,; on n'y en
voulait pas d'avantage.

Cela fit changer de batterie aux jésuites, parce que cela affichait le
criminel usage qu'ils voulaient faire de cette bulle\,; et qu'ils ne
savaient comment réussir dès que Rome, sur qui ils avaient compté, leur
devenait plus que suspecte. Ils craignirent encore les longueurs des
procédures à Paris, à Lyon, à Rome, des commissaires \emph{in partibus}.
C'était un noeud gordien qu'il leur parut plus facile de couper que de
dénouer.

On agit donc sur le principe qu'il n'y avait qu'un Port-Royal, que ce
n'était que par tolérance qu'on en avait fait deux de la même abbaye\,;
qu'il convenait remettre les choses sur l'ancien pied\,; qu'entre les
deux il convenait mieux de conserver celui de Paris que l'autre, qui
avait à peine de quoi subsister, situé en lieu malsain, uniquement
peuplé de quelques vieilles opiniâtres, qui depuis tant d'années avaient
défense de recevoir personne à profession.

Il fut donc rendu un arrêt du conseil, en vertu duquel la nuit du 28 au
29 octobre, l'abbaye de Port-Royal des Champs se trouva secrètement
investie par les détachements des régiments des gardes françaises et
suisses, et vers le milieu de la matinée du 29, d'Argenson arriva dans
l'abbaye avec des escouades du guet et d'archers. Il se fit ouvrir les
portes, fit assembler toute la communauté au chapitre, montra une lettre
de cachet\,; et sans leur donner plus d'un quart d'heure, l'enleva tout
entière. Il avait amené force carrosses attelés, avec une femme d'âge
dans chacun\,; il y distribua les religieuses suivant les lieux de leur
destination, qui étaient différents monastères à dix, à vingt, à trente,
à quarante, et jusqu'à cinquante lieues du leur, et les fit partir de la
sorte, chaque carrosse accompagné de quelques archers à cheval, comme on
enlève des créatures publiques d'un mauvais lieu. Je passe sous silence
tout ce qui accompagna une scène si touchante et, si étrangement
nouvelle. Il y en a des livres entiers.

Après leur départ, Argenson visita la maison des greniers jusqu'aux
caves, se saisit de tout ce qu'il jugea à propos, qu'il emporta\,; mit à
part tout ce qu'il crut devoir appartenir à Port-Royal de Paris, et le
peu qu'il ne crut pas pouvoir refuser aux religieuses enlevées, et s'en
retourna rendre compte au roi et au P. Tellier de son heureuse
expédition.

Les divers traitements que ces religieuses reçurent dans leurs diverses
prisons pour les, forcer à signer sans restriction, est la matière
d'autres ouvrages, qui, malgré la vigilance des oppresseurs, furent
bientôt entre les mains de tout le monde, dont l'indignation publique
éclata à tel point que la cour et les jésuites même en furent
embarrassés.

Mais le P. Tellier n'était pas homme à s'arrêter en si beau chemin. Il
faut achever cette matière de suite, quoique le reste en appartient aux
premiers mois de l'année suivante. Ce ne furent qu'arrêts sur arrêts du
conseil, et lettres de cachet sur lettres de cachet. Il fut enjoint aux
familles qui avaient des parents enterrés à Port-Royal des Champs de les
faire exhumer et porter ailleurs\,; et on jeta dans le cimetière d'une
paroisse voisine tous les autres comme on put, avec l'indécence qui se
peut imaginer. Ensuite on procéda à raser la maison, l'église et tous
les bâtiments comme on fait les maisons des assassins des rois, en sorte
qu'enfin il n'y resta pas pierre sur pierre. Tous les matériaux furent
vendus, et on laboura et sema la place\,; à la vérité ce ne fut pas de
sel, c'est toute la grâce qu'elle reçut. Le scandale en fut grand jusque
dans Rome. Je me borne à ce simple et court récit d'une expédition si
militaire et si odieuse.

Le cardinal de Noailles en sentit l'énormité après qu'il se fut mis hors
d'état de parer un coup qui avait passé sa prévoyance, et qui en effet
ne se pouvait imaginer. Il n'en fut pas mieux avec les molinistes, mais
beaucoup plus mal avec les jansénistes, ainsi que les jésuites se
l'étaient bien proposé\,; et depuis cette funeste époque, il ne porta
quasi plus santé, je veux dire qu'il fut presque incontinent attaqué, et
peu à peu poussé, sans relâche aux dernières extrémités jusqu'à la fin
de sa vie.

\hypertarget{chapitre-xxiv.}{%
\chapter{CHAPITRE XXIV.}\label{chapitre-xxiv.}}

1709

~

{\textsc{Chamillart et ses filles à la Ferté.}} {\textsc{- Chamillart
achète Courcelles, où je mène la duchesse de Lorges.}} {\textsc{- Voyage
à la Flèche\,: aventure.}} {\textsc{- Étrange sermon de la Toussaint.}}
{\textsc{- Résolution et raisons de retraite.}} {\textsc{- Retour à
Paris.}} {\textsc{- Sage piège dressé à Pontchartrain.}} {\textsc{-
Triste situation de M. le duc d'Orléans.}} {\textsc{- Passage à
Versailles, où le chancelier me force d'accepter une chambre chez lui au
château.}} {\textsc{- Concours et conspirations d'amis.}} {\textsc{-
Bontés et désirs de Mgr le {[}duc{]} et de M\textsuperscript{me} la
duchesse de Bourgogne sur M\textsuperscript{me} de Saint-Simon pour
succéder à la duchesse du Lude.}} {\textsc{- Parti que je prends seul,
et ses motifs, de faire demander par Maréchal une audience au roi.}}
{\textsc{- Maréchale de Villars\,; son accortise.}} {\textsc{- Visite du
roi du maréchal, puis à la maréchale de Villars.}} {\textsc{-
Contretemps de Vendôme.}} {\textsc{- Je me propose de faire rompre M. le
duc d'Orléans avec M\textsuperscript{me} d'Argenton, et au maréchal de
Besons de m'y aider.}} {\textsc{- Caractère de Besons.}} {\textsc{-
Maréchal m'obtient une audience du roi.}}

~

Les différentes choses que j'ai racontées avaient retardé mon départ
jusque dans les commencements de septembre.

Les filles de Chamillart vinrent à {[}la Ferté{]}, lui-même aussi au
retour de ses courses pour aller voir des terres à acheter, voyage où,
pour être hors de Paris, les avis et les propos menaçants de
M\textsuperscript{me} de Maintenon l'avaient forcé, qui le voulait tenir
au loin, dans le dépit de la nombreuse et bonne compagnie qui ne
l'abandonnait point, et plus encore dans l'appréhension que lui donnait
le goût du roi pour lui. J'essayai de l'amuser par tout ce que la
campagne me put fournir, et de le recevoir bien mieux que s'il eût été
encore en place et en faveur. Après dix ou douze jours il s'en alla à
Paris conclure le marché de Courcelles. Ses filles le suivirent bientôt
après, excepté la duchesse de Lorges qui demeura avec nous et d'autre
compagnie. Son père et sa famille ne tardèrent pas à s'en aller à
Courcelles, et bientôt après j'y menai ma belle-soeur. Ce n'est pas
qu'on ne fit tout ce que l'on put pour me dissuader Ce voyage, qui en
effet était peu politique, mais je ne crus pas y devoir asservir
l'amitié.

Je demeurai trois semaines\,; j'y passais les matinées avec Chamillart,
qui m'y parla à coeur ouvert de bien des choses, et qui m'y en montra de
bien curieuses du temps de son ministère. Quand j'aurais ignoré
jusqu'alors les variations si fréquentes de l'esprit, de l'estime, de
l'amitié de M\textsuperscript{me} de Maintenon, sans autre cause que son
naturel changeant, je l'aurais vu là à découvert, ainsi que les
événements produits de cette cause qui ont si souvent gâté les
meilleures affaires, et perdu tant d'autres par le peu de suite et la
succession des différentes fantaisies. Le reste du jour s'y passait en
amusements et en promenades\,; Chamillart toujours doux, serein, sans
humeur, sans distraction, mais presque jamais seul, comme un homme qui
se craint et qui cherche à remplir le vide où il se trouve\,; la
conversation bonne, mais réservée sur les nouvelles, et changeant alors
la conversation adroitement\,; le voisinage assidu chez lui et bien
reçu, et sa famille cherchant à l'amuser et à se dissiper elle-même.

J'y fus témoin de deux aventures que je ne puis m'empêcher de rapporter.
Ce magnifique collège de la Flèche n'est qu'à deux lieues de
Courcelles\,; nous l'allâmes voir. Les jésuites firent de leur mieux
pour faire la meilleure réception qu'ils purent. Chauvelin, intendant de
la province, s'y trouva pour y ajouter tout ce qu'il put. C'est celui
qui devint après conseiller d'État, cousin de Chauvelin, qui longtemps
depuis eut les sceaux et bien mieux encore. Tessé avait donné pour rien
une de ses filles à La Varenne, qui était seigneur de la Flèche\,; elle
était veuve et y demeurait. Chamillart crut de la politesse de l'aller
voir et me le proposa\,; je crus lui devoir dire qu'elle était fille de
Tessé, parce que ce maréchal avait contribué à sa chute, et qu'il
n'avait pas gardé de mesures avec lui dans les derniers temps. Cela
n'arrêta pas Chamillart\,; je ne lui en dis pas aussi davantage\,; nous
y allâmes. La maison se trouva si dégarnie de domestiqués et si peu en
ordre, que nous demeurâmes tous deux seuls près d'un quart d'heure, dans
une antichambre. Il y avait une grande et vieille cheminée, sur laquelle
on lisait en fort grosses lettres ces deux vers latins\,:

«\,Quum fueris felix, multos numerabis amicos\footnote{Saint-Simon,
  citant de mémoire, a altéré le premier vers du distique si connu\,:
  \emph{Donec eris} felix, multos numerabis amicos\ldots{}}\,;

Tempora si fuerint nubila, solus eris.\,»

Je l'aperçus, et me gardai bien d'en faire aucun semblant\,; mais le
long temps que nous restâmes là donna loisir à Chamillart de tout
considérer et de la lire. Je le vis faire, et je m'écartai pour ne lui
pas montrer que je m'en apercevais, ni donner lieu de parler sur cette
morale.

L'autre aventure fut plus pesante. La paroisse de Courcelles est petite,
éloignée, et par un fort mauvais chemin. Contents d'y avoir été à la
grand'messe le jour de la Toussaint, nous allâmes à vêpres à une abbaye
de filles qui n'est qu'à demi-lieue, qui s'appelle la
Fontaine-Saint-Martin. Nous vîmes l'abbesse à la grille, les dames
entrèrent dans la maison. Chamillart et moi avions envie d'éviter un
mauvais sermon, mais l'abbesse nous dit que l'évêque du Mans, qui avait
su que nous devions aller ce jour-là chez elle, avait prié les jésuites
d'y envoyer leur meilleur prédicateur, qui serait mortifié, et ces
pères, si nous ne l'entendions point. Il fallut donc s'y résoudre.

Dès les premières périodes je frémis. Le sujet fut de la différence de
la béatitude des saints d'avec le bonheur le plus complet dont on puisse
jouir ici-bas\,; de l'éternelle solidité de l'une, de l'instabilité
continuelle de l'autre\,; des peines inséparables des plus grandes
fortunes\,; des dangers de la jouissance de la prospérité, des regrets
et des douleurs de sa perte. Le jésuite s'étendit sur cette peinture
qu'il rendit vive et démonstrative. S'il s'en fût tenu aux termes
généraux, cette indiscrétion eût pu passer à la faveur du jour qu'on
solennisait\,; mais après avoir bien déployé son sujet, il en vint à une
description particulière si propre à Chamillart qu'il n'y eut personne
de l'auditoire qui n'en perdit toute contenance. Il ne parla jamais
d'autre fortune, ni d'autre bonheur que celui de la faveur et de la
confiance d'un grand roi, que du maniement de ses affaires, que du
gouvernement de son État\,; il entra dans le détail des fautes qui s'y
peuvent faire ou qu'on impute aux malheureux succès, il ne ménagea aucun
trait parlant. Il vint après à la disgrâce, au dénuement, au vide, au
déchaînement. Il débita qu'un prince comptait au ministre chassé, comme
une grâce sans prix, la bonté de ne lui pas faire rendre un compte
rigoureux de son administration. Enfin il termina son discours par une
exhortation, à ceux qui se trouvaient réduits en cet état, d'en faire un
saint usage pour acquérir dans le ciel une plus haute fortune qui ne
doit jamais finir. S'il avait adressé la parole à Chamillart, il
n'aurait pas été plus manifeste qu'il avait entrepris de le prêcher tout
seul\,; rien de tout son discours n'était propre qu'à lui. Il n'y eut
personne qui n'en sortît confondu.

Chamillart seul ne parut point embarrassé. Après vêpres nous retournâmes
à la grille. Il loua le prédicateur, lui fit accueil après lorsqu'il
vint saluer la compagnie, le félicita du sermon\,; une collation vint
fort à propos pour donner lieu de parler d'autre chose. Nous retournâmes
à Courcelles, où nous nous déchargeâmes le coeur les uns aux autres de
cette scandaleuse indiscrétion où le jésuite apparemment avait cru faire
merveilles. Peu de jours après je retournai à la Ferté, après un mois
d'absence.

La compagnie en était partie, et nous eûmes alors le temps,
M\textsuperscript{me} de Saint-Simon et moi, de raisonner sur le parti
que je voulais prendre. Je trouvais que l'abandon de la cour était le
seul qui me convint. On ne me reprochait quoi que ce soit, je ne me
sentais en faute sur rien, je n'avais donc pas matière à aucune
justification, ni à aucune excuse, ni à espérer en y réussissant de me
remettre à flot\,: on me trouvait trop d'esprit et d'instruction, détour
que la connaissance de la faiblesse du roi à cet égard avait fait
prendre pour me perdre auprès de lui, lors de l'ambassade de Rome, et
dont on s'était si longtemps bien trouvé, qu'on le renouvelait plus que
jamais. Les amis considérables que j'avais à la cour, en seigneurs
principaux, en ministres, en dames considérables, était une autre
matière qui me tournait à mal. On craignait qu'ils ne me portassent, que
je susse en faire usage pour arriver\,; on ne voulait pas que j'eusse
des ailes, et pour la première fois que pareille chose soit arrivée dans
une cour, on me fit un crime auprès du roi de l'estime, de l'amitié, de
la confiance des personnes pour lesquelles il en avait lui-même, et qu'à
ce titre il avait élevées. Comment se disculper d'avoir de l'esprit et
des connaissances, puisqu'on en avait persuadé le roi à mauvais dessein
et avec succès\,? Comment lui faire entendre une ruse dont l'explication
ne pouvait lui être faite, parce qu'elle ne roulait que sur sa
faiblesse\,? Comment s'excuser sur l'usage de tant d'esprit prétendu,
puisque jamais je n'avais été ni attaqué là-dessus, ni eu occasion d'en
profiter\,? Enfin, comment se laver d'avoir des amis qui me faisaient
honneur par leur réputation, leur mérite, leurs places, et la part
qu'ils avaient dans les affaires, et dans, l'estime et la confiance du
roi, et dont l'amitié eût tenu lieu de mérite auprès de lui, à tout
autre qu'à moi\,?

Le rare est qu'on ne relevait point celle qui était entre M. le duc
d'Orléans et moi, quoique si publique et si peu ménagée, et lui si mal
auprès du roi. Rien ne montrait davantage le ressort qui faisait agir.
On ne craignait pas l'usage que je pourrais faire de celle-ci, on
redoutait celui que je pourrais tirer des autres. Mais de tout cela nul
moyen d'en revenir auprès du roi, qu'on avait prévenu là-dessus comme
sur des choses très dangereuses, et sur lesquelles il ne se pouvait rien
alléguer.

C'était l'effet de la jalousie d'une part, du dépit de l'autre, de ceux
que je n'avais pas ménagés pendant la campagne de Lille, et qui
s'étaient aperçus que j'avais vu trop clair dans leurs desseins. Ils en
craignaient les retours dans un temps ou dans un autre, et ils n'avaient
rien épargné pour me mettre hors de combat pour toujours.

Les affaires de rang que j'avais soutenues, l'impatience des usurpations
sur lesquelles je ne m'étais pas contraint, les fripons de toute espèce
sur lesquels je m'étais quelquefois expliqué un peu librement, peu de
commerce toute ma vie avec la jeunesse, dont la dissipation, le futile,
la débauche de quelques-uns, ne m'allaient point, tout cela ensemble
faisait un groupe et un cri sous lequel je succombais, et dont ces amis
qu'on relevait si fort étaient trop faibles pour me défendre.

Le pari de Lille fut un autre sujet qui avait mis à mon égard le doigt
sur la lettre, à la cabale de Vendôme, qui en prit occasion de répandre
et de persuader au roi que je blâmais le gouvernement, que j'en étais
ennemi, et tout ce qui se put broder là-dessus pour l'aigrir. Comment
encore s'aller excuser sur cet article, et quoique Vendôme fût en
disgrâce, comment aller montrer au roi ce projet contre son petit-fils,
où trempaient tant de gens si considérables, et lors encore si
considérés et si bien traités, et dont il s'en trouvait qui, en tout
genre, lui tenaient de si près\,?

Je trouvais donc le mal sans remède, par cela même qu'il était sans
consistance sur laquelle les remèdes pussent agir, et je ne me trouvais
pas disposé à avaler continuellement des dégoûts, en demeurant à la
cour, et à une basse servitude que je n'avais jamais pratiquée, et pour
laquelle je ne me sentais point fait, pour arriver à quoi que ce fût de
mieux, à plus forte raison, à pure perte.

M\textsuperscript{me} de Saint-Simon, sans se compter elle-même pour
rien, me représentait doucement les suites dangereuses du parti que je
voulais prendre\,: l'amortissement du dépit, l'ennui d'une vie
désoccupée, la stérilité de la promenade et des livres pour un homme de
mon état, dont l'esprit avait besoin de pâture, et était de tout temps
accoutumé à penser et à faire, les regrets que leur inutilité
appesantirait, le long temps qu'ils pouvaient durer à mon âge,
l'embarras et le chagrin qui accompagneraient l'entrée de mes enfants
dans le monde et dans le service, les besoins continuels de la cour pour
la conservation de son propre patrimoine, et les inconvénients ruineux
d'en être maltraité\,; enfin la considération des changements qui
pouvaient arriver et que devait amener la disproportion des âges.

Nous en étions là-dessus, toutefois mon parti pris de passer quatre mois
d'hiver à Paris et huit à la Ferté, sans voir la cour qu'en passant ou
par pure nécessité d'affaires, et de laisser liberté à
M\textsuperscript{me} de Saint-Simon sur moins de séjour, à la campagne,
lorsque nous apprîmes la mort de celui qui, depuis plus de trente ans,
conduisait toutes nos affaires avec toute l'affection, la capacité et la
réputation qui se pouvait désirer, laquelle arriva en trois jours à
Ruffec, où il était allé pour les affaires de cette terre en revenant de
celles de Guyenne. Ce malheur pressa notre retour\,;
M\textsuperscript{me} de Saint-Simon me proposa d'aller de la Ferté
coucher à Pontchartrain. Elle avait ajusté le voyage pendant un Marly,
et aux jours que le chancelier était chez lui, qu'elle avait instruit de
ce qui se pas soit entre nous, et qui m'attendait. Je donnai dans le
piège sans m'en douter, et nous arrivâmes à Pontchartrain le 19
décembre.

Dès le lendemain, le chancelier me prit dans le cabinet de sa femme avec
elle et la mienne, où, porte bien fermée, il me demanda où j'en étais
depuis que nous ne nous étions vus, et si les réflexions n'étaient point
venues à mon secours. Je m'expliquai au long avec lui sur ce que je
viens de rapporter. Il me laissa tout dire\,; ensuite il reprit toutes
mes raisons, et avec l'esprit et l'adresse qui lui étaient si naturels,
il essaya de retourner toutes mes raisons. Il vint ensuite à la censure,
mais avec une grâce et une amitié touchante. Il me montra que les
ennemis dont je me plaignais étaient bien payés pour l'être\,; et pour
m'éloigner de bonne heure d'arriver en état de leur foire du mal,
puisque, dans une situation commune à mon âge, je les ménageais si peu
et publiquement peu, qu'il était vrai que je parlais peu et souvent
point du tout, mais que l'énergie de mes expressions, même ordinaires,
faisait peur, et que mon silence encore n'était guère moins éloquent en
beaucoup de rencontres\,; qu'il ne s'agissait de rien de marqué ni de
grossier à faire, mais de montrer à l'avenir, par une circonspection
exacte, que je n'étais pas incapable de réfléchir et de me corrigera. Il
me soutint que, n'y ayant rien de marqué que ce pari de Lille, qui
vieillirait et s'oublierait enfin, c'était une erreur de me croire sans
ressource, et une autre encore qu'un homme de ma sorte pût en manquer
avec de la patience et de l'application. Il appuya sur les mêmes raisons
que M\textsuperscript{me} de Saint-Simon n'avait fait que me présenter.
Il s'étendit en exemples vivants sur ce qu'aucun de ceux dont la fortune
pouvait avoir fait et faire encore envie, n'y était parvenu sans avoir
passé par des situations plus fâcheuses que celle où je me croyais\,;
qu'il ne s'agissait point de bassesses, pour s'en relever, mais de
conduite et de sagesse. De là il vint aux dégoûts présents par lesquels
il fallait passer, qu'il compara à ceux que je me préparais par une
retraite. Il me maintint qu'il y avait moins d'honneur et de courage à
réjouir mes ennemis en leur quittant la partie, et me mettant de leur
côté pour accomplir sur moi leurs désirs, qu'à leur résister et à faire
ce que je devais pour ramener la fortune\,; et il finit par la
considération de mon âge et de celui de ceux à qui j'avais affaire.

La chancelière se mit de la partie\,; je répondis, ils répliquèrent.
J'omets ce qu'ils alléguèrent sur ce que je pouvais faire et devenir,
que l'amitié et l'estime grossissaient. Enfin ils me dirent que ce que
j'aurais de plus journellement incommode à essuyer était de loger à la
ville, parce que, outre l'incommodité, cela entraînait mille
contretemps, et rompait le commerce et la société dont on tire
imperceptiblement tant d'avantages. Que de cela je ne pouvais m'en
prendre qu'à la disgrâce d'autrui, non à la mienne\,; que le roi avait
compté que le logement de M. le maréchal de Lorges me demeurerait\,; que
je l'avoir si bien cru moi-même que, depuis sept ans que je l'occupais,
je n'avais demandé aucun de ceux qui avaient vaqué\,; que ce n'était la
faute de personne si mon beau-frère, délogé de chez son beau-père,
reprenait le logement de son père, qui lui avait été donné à sa mort,
qu'il n'avait point habité par la promptitude de son mariage\,; qu'ainsi
ce n'était point là ce que je devais prendre comme un dégoût\,; puis,
revenant sur l'incommodité, ils m'offrirent ce qu'ils pouvaient, qui
était une grande et belle chambre et une garde-robe chez eux au château,
qui était le logement de leur frère, qui, par ses apoplexies, ne sortait
plus de sa maison de Paris. Ils me dirent que je pourrais me tenir là
dans la journée, si je n'y voulais pas coucher, M\textsuperscript{me} de
Saint-Simon avoir où s'habiller, et tous deux y voir nos amis. L'un et
l'autre m'en pressèrent jusqu'à m'embarrasser, et toujours
M\textsuperscript{me} de Saint-Simon en silence pendant toute cette
conversation, qui dura près de trois heures. Le chancelier la finit par
me prier de ne plus rien dire\,; mais de faire mes réflexions au moins
pour l'amour de lui, et que nous verrions après l'impression qu'elles
m'auraient faites.

Ils me parlèrent le lendemain sur M\textsuperscript{me} de Saint-Simon,
sans elle, pour me battre par la considération de la triste vie que ma
retraite lui ferait mener, et par celle de tous les usages dont elle me
pouvait être à la cour, où elle était indistinctement et unanimement
aimée, estimée, considérée, à commencer par le roi. Il était vrai encore
que M\textsuperscript{me} la duchesse de Bourgogne s'était plainte à
M\textsuperscript{me} de Lauzun, plusieurs fois, de sa longue absence
avec beaucoup d'amitié et d'intérêt, et que M. le duc d'Orléans l'avait
entretenue de la mienne souvent, à Marly, avec amertume, et cherchant
les moyens de me ramener, jusqu'à me faire presser par elle de prendre
le petit logement au château qu'avait d'Effiat, comme étant son premier
écuyer, et dont il pouvait disposer à ce titre, d'Effiat surtout n'y
venant presque jamais. Je n'avais pas la plus légère connaissance avec
Effiat, et je me gardai bien d'accepter ainsi son logement d'un air de
supériorité.

Tous ces entretiens me flattaient par l'amitié, m'importunaient par le
combat, mais ne vainquirent ni mon dégoût ni ma résolution. Ils me
jetèrent seulement dans un tiraillement qui, sans qu'il y parût, me mit
extrêmement mal à mon aise.

Je fus trois nuits à Pontchartrain\,; je m'y informai de la situation de
M. le duc d'Orléans. Le chancelier m'apprit qu'elle ne pouvait être plus
triste, dans un éloignement du roi fort marqué, celui de Monseigneur
incomparablement davantage, un embarras, un malaise qui se montrait à
découvert, une solitude entière, et jusque dans les lieux publics, où
personne ne s'approchait de lui, et où rarement il s'approchait de
personne sans demeurer seul bientôt après un abandon entier à
M\textsuperscript{me} d'Argenton et à la mauvaise compagnie de Paris, où
il était fort souvent\,; qu'elle avait fait les honneurs d'un repas
qu'il avait donné, depuis peu de jours, à Saint-Cloud, à l'électeur de
Bavière, qui avait fait grand bruit et fort irrité le roi\,; en un mot,
que jamais prince de ce rang {[}ne fut{]} si étrangement anéanti. Je
m'étais bien attendu à une partie de ces choses, mais non à un si cruel
état. Il augmenta encore mes réflexions.

Il fallut passer et s'arrêter à Versailles. Nous y fûmes tous dîner chez
le chancelier, le samedi 21 décembre, jour que le roi revenait de Marly.
La chancelière nous mena voir le logement qu'elle nous destinait. Les
empressements avaient été poussés là-dessus avec adresse, jusqu'à faire
sentir qu'ils se tiendraient offensés et méprisés du refus. Ils y
avaient ajouté l'offre tout aussi poussée de nous y faire servir un
morceau pour nous et pour nos amis. En un mot, tant fut procédé qu'ils
me forcèrent comme on force un cerf. Il fallut accepter\,; mais je
capitulai sur le manger, que je ne voulus pas souffrir. Il est
impossible d'exprimer l'amitié et la grâce avec laquelle tout cela se
passa de leur part. Leur fils était à Marly, que nous ne vîmes que le
soir à Versailles.

J'étais peu persuadé, touché néanmoins des raisons et plus encore de
l'amitié, mais froncé de nouveau, en me revoyant dans Versailles,
relégué au fond de la ville, avec cet asile au château, peu capable de
soutenir le dégoût et la messéance d'une situation que je ne voyais
aucun moyen sensible de changer.

Sur le soir, au retour de la cour, je me trouvai environné d'amis, qui,
comme de concert, accoururent autour de moi, hommes et femmes,
Chevreuse, Beauvilliers, Lévi, Saint-Géran, Nogaret, Boufflers, Villeroy
et d'autres encore, qui me représentèrent toutes les mêmes
considérations en diverses façons qui m'avaient été faites, et qui
formèrent comme une conjuration contre ce que j'avais résolu, dont
quelques-uns étaient informés, et dont les autres s'étaient doutés par
la longueur de mon absence. Ils se relayaient {[}les uns{]} les autres,
comme s'ils s'étaient entendus pour ne me laisser aucun repos.

M\textsuperscript{me} la duchesse de Bourgogne envoya chercher
M\textsuperscript{me} de Saint-Simon sitôt qu'elle fut arrivée, qui
l'accabla de bontés, dont aussi Mgr le duc de Bourgogne me combla. Outre
ce qu'elle avoir dit sur la place de dame d'honneur, après la duchesse
du Lude, je sus par Cheverny, ce même soir, que Mgr le duc de Bourgogne
s'en était ouvert à lui.

Surpris d'une réception si vive, et touché d'une amitié si constante de
tant de gens considérables dans un état de disgrâce, et de ne pouvoir
encore, en revenant à flot, devenir utile à pas un d'eux, les
réflexions, tout ensemble me terrassa. Je résolus, ce même soir, à
l'insu de qui que ce fût, de tenter chose qui me décidât pour toujours,
soit en me raccrochant à la cour avec quelque succès, soit en
l'abandonnant, qui me délivrât de la sorte de persécution que je
souffrais là-dessus.

Quelque peu susceptibles que les choses vagues et sans fondement fussent
d'un éclaircissement avec le roi, dont les plus dangereuses, comme
l'esprit, ne se pouvaient traiter, et les plus aisées à détruire étaient
d'une périlleuse délicatesse, comme le pari de Lille et ses suites, ce
fut néanmoins la dernière ressource que j'embrassai, fondé sur ce que
cette voie m'avait si bien réussi plus d'une fois, et dans la vérité
encore sur ce qu'il y avait à croire que le roi ne voudrait pas
m'entendre, ou que m'écoutant, et cela court et sec, deux choses qui
favorisaient le parti que je voulais prendre, et qui mettraient fin aux
obstacles de raison et d'amitié que j'en rencontrais.

J'allai chez Maréchal, dont on a vu ailleurs l'attachement pour moi, et
quel il était d'ailleurs. Il était un de ceux qui me pressaient le plus
de ne point quitter la partie, et il m'en avait écrit fortement à la
Ferté pour hâter mon retour. Je le trouvai. La conversation ne tarda pas
à se tourner sur ma situation, et sur l'embarras que, ne portant sur
rien de particulier, mais sur un amas de bagatelles vraies et fausses,
{[}elles{]} étaient grossies et empoisonnées de manière qu'elles me
coulaient à fond plus sûrement que des fautes réelles et bien marquées.
Après quelques raisonnements là-dessus, je lui dis tout d'un coup que
tout le malheur était d'avoir affaire à un maître inabordable, auquel,
si je pouvais lui parler à mon aise, j'étais sûr de faire évanouir
toutes les friponneries dont on s'était servi pour lui rendre ma
conduite désagréable, et tout de suite j'ajoutai qu'il me venait en
pensée de lui faire une proposition, sans toutefois lui rien demander
au-dessus de ses forces, parce que j'avais tout lieu de compter sur son
amitié\,; que la volonté ne lui manquerait pas, et que dans cette
persuasion, je désirais qu'il demeurât en sa liberté de me répondre et
de ne rien faire que ce qui lui conviendrait\,; que ma proposition était
qu'il prit son temps de dire au roi qu'il m'avait vu affligé au dernier
point de me sentir mal auprès de lui sans l'avoir en rien mérité\,; que
cette seule raison m'avait tenu quatre mois à la campagne, où je serais
encore sans la mort d'un homme très principal dans mes affaires, pour
lesquelles j'avais été forcé à revenir\,; que je ne pouvais avoir de
repos qu'en lui parlant avec franchise et loisir, et que je le suppliais
de vouloir m'écouter avec bonté et loisir quand il lui plairait.
J'ajoutai que, par le refus de l'audience, je verrais bien que je
n'avais plus à songer à rien\,; que si je l'obtenais, le succès me
découvrirait ce qui me pourrait rester d'espérance.

Maréchal pensa un moment, puis me regardant\,: «\,Je le ferai, me dit-il
avec feu\,; et en effet il n'y a que cela à frire. Vous lui avez déjà
parlé plusieurs fois, il en a toujours été content\,; il ne craindra
point ce que vous aurez à lui dire, par l'expérience qu'il en a déjà
eue. Je ne réponds pourtant pas qu'il le veuille s'il est bien déterminé
contre vous\,; mais laissez-moi faire et bien prendre mon temps.\,» Nous
convînmes qu'il m'écrirait à Pari, par un exprès sitôt qu'il aurait
parlé.

En le quittant, je fus dire au chancelier et à M\textsuperscript{me} de
Saint-Simon le dessein que j'avais conçu et entrepris, et leur déclarer
en même temps que c'était le fruit de leurs persécutions et de celles de
tous mes amis, duquel dépendrait le parti que je prendrais\,; mais que,
poussé à bout pour demeurer à la cour, je voulais tricher de pénétrer,
par cette dernière tentative, ce que j'y pouvais raisonnablement
espérer, et par le succès de cette épreuve, m'y attacher ou l'abandonner
pour toujours. Tous deux goûtèrent fort ce que j'avais imaginé, sans
pouvoir s'opposer à ma résolution, en conséquence. Le chancelier
craignit que le roi, n'ayant rien de marqué contre moi, ne voulût point
m'entendre, dégoûté par un amas de choses sans corps, adroitement
empoisonnées et portées jusqu'à lui\,; M\textsuperscript{me} de
Saint-Simon {[}craignit{]} bien davantage, persuadée qu'elle était par
l'éloignement profond du roi pour moi, qu'elle avait appris de
M\textsuperscript{me} la duchesse de Bourgogne, qu'elle m'avait
judicieusement caché. Cependant la conclusion fut d'attendre,
d'espérer\,; que rien n'était mieux que ce que j'avais fait, par
l'obscurité dans laquelle cette audience serait demandée\,; que ce
serait bon signe si elle était accordée\,; qu'en tout événement, on
serait sur ses pieds pour voir et consulter, ne voulant pas consentir à
la retraite, quand même l'audience serait refusée.

Ce soir même, tout tard, je montai chez M\textsuperscript{me} de
Saint-Géran, qui sortait de la grande opération de la fistule, et qui
m'avait envoyé prier, en arrivant, de ne pas me retirer sans l'aller
voir. La maréchale de Villars y vint. Jusqu'à la disgrâce de
Chamillart\,; nous avions logé, porte à porte.

C'était une femme qui, à travers les galanteries, s'était mise en
considération personnelle par les grâces et l'application avec
lesquelles elle tâchait d'émousser la jalousie de la fortune de son
mari. Elle n'avait rien oublié, ni lui aussi, pour se mettre bien avec
M\textsuperscript{me} de Saint-Simon et avec moi dans le temps le plus
radieux de leur vie, et où nous ne pouvions leur être de nul usage. Ils
avaient passé légèrement sur ma douleur peu contrainte de leur énorme
duché, dont jamais je ne leur avais fait le moindre compliment. Sur la
pairie, je m'étais aussi bien gardé de leur en faire faire, encore moins
de leur en écrire. L'accueil, au bout de quatre mois d'absence, fut
comme si nous ne nous étions pas quittés. Elle me pria à dîner avec
M\textsuperscript{me} de Saint-Simon pour le lendemain, et m'en pressa
de manière à ne m'en pouvoir défendre. Ils étaient lors en l'apogée de
la plus brillante faveur. Elle savait que le roi devait aller voir son
mari le lendemain, mais elle n'eut garde de me le dire. Elle me l'avoua
depuis, et son intention fut de nous donner occasion de lui faire notre
cour.

Je fus voir le lendemain matin la duchesse de Villeroy. Elle {[}et{]}
son mari me demandèrent où je dînais, et m'avertirent de la visite du
roi, de peur que, dans la surprise, il m'échappât quelque chose. Le duc
de Villeroy m'avait écrit la pairie de Villars à la Ferté, sans me
mander autre chose dans la même lettre. Ma réponse fut laconique\,: je
lui mandai que je le remerciais de sa nouvelle, que je le priais de
s'aller, en propres termes, et de me croire, etc. Ils en rirent
beaucoup\,; mais cette disposition qu'ils me connaissaient les engagea à
me donner l'avis.

Nous dînâmes en compagnie assez courte, et que nous reconnûmes aisément
avoir été choisie pour nous. Vers le fruit, on vint poster les gardes,
et le roi vint au sortir du sermon. La compagnie s'était grossie depuis
le dîner. Le roi la salua, puis vint au lit de repos sur lequel était le
maréchal de Villars, l'embrassa par deux fois avec des propos
obligeants, congédia le monde, et demeura deux heures là tête à tête.
Comme il sortait, le maréchal lui dit qu'il se méprenait de porte. Le
roi l'assura qu'il avait bien remarqué le chemin, et qu'il allait rendre
une visite à la maréchale dans son appartement. Il l'y trouva avec
quelques dames. Il y fut peu, mais avec cette galanterie majestueuse qui
lui était si naturelle. Il s'en alla de lis chez lui. Cette visite
excita un renouvellement d'envie et fit grand bruit dans le monde. Le
maréchal de Grammont, mort à Bayonne en 1678, est le dernier seigneur
qu'il ait visité dans une maladie, ce qui n'était pas rare autrefois. En
allant chez Villars, il dit, comme par manière d'excuse, que puisque le
maréchal de Villars ne pouvait venir chez lui, il fallait bien qu'il
l'allât trouver.

Le maréchal de Boufflers ne fut pas celui à qui cette visite fut la
moins sensible. Il se tint fort chez lui pendant qu'elle dura, et tout
le jour. Mais le hasard donna une rude mortification à un autre illustre
disgracié. Le duc de Vendôme, qui, depuis son exclusion de Marly et de
Meudon, faisait des courses rares d'Anet à Versailles, y arriva
justement dans ce temps-là. Il en usa en courtisan\,: il vint dans la
galerie où donnait l'appartement qu'occupait Villars attendre que le roi
en sortît, et y demeura une bonne heure confondu avec tout le monde. Le
roi, qui le vit en sortant, lui demanda à quelle heure il était parti
d'Anet. C'est tout ce qu'il en eut en tout le temps qu'il demeura à
Versailles, qui fut jusqu'au premier jour de l'an. Ce spectacle de
Vendôme ne laissa pas d'amuser assez de gens.

Tandis que je mettais les fers au feu pour moi-même, je ne perdais point
de vue la triste situation de M. le duc d'Orléans. Il était allé de
Marly à Paris, ainsi je ne l'avais point vu, et à Paris je ne le voyais
jamais. Frappé de la profondeur de sa chute, il ne se présenta à moi
qu'un seul moyen de le relever, terrible à la vérité, et même dangereux
à lui proposer vainement, très difficile à espérer de lui faire prendre,
mais qui, tel qu'il était, ne fut pas capable de m'épouvanter\,: c'était
de le séparer d'avec sa maîtresse pour ne la revoir jamais. J'en sentis
tout le poids et le péril, mais j'en sentis tellement la nécessité et le
fruit, que je résolus de l'entreprendre\,; mais je n'osai me charger
seul d'une entreprise si pleine d'écueils.

Je jetai les yeux sur Besons, le seul homme qui fût en état et qui pût
être en volonté de m'y aider, encore qu'il fût à peine de ma
connaissance. On a vu en plus d'un endroit ici quel il était, et ses
raisons de liaison et d'attachement pour M. le duc d'Orléans, qui avait
beaucoup de confiance en lui, et qui avait fort contribué à son
élévation.

Besons était un rustre, volontiers brutal, avec peu d'esprit, mais tout
tourné à son fait et à cheminer\,; avec assez de sens, mais une tête
faite pour un Rembrandt et un Van Dyck, avec de gros sourcils et une
grosse perruque qui lui en faisaient attribuer bien davantage\,;
excellent officier général, surtout de cavalerie, médiocre général
d'armée, qui, avec une valeur personnelle fine et tranquille, craignait
tous les dangers pour la besogne dont il était chargé. Il était droit,
franc, honnête homme, avait de la vertu, austère pour autrui, adoucie
pour soi, en homme qui sentait son peu de bien, d'alliance, de
naissance, qui avait beaucoup de famille qu'il aimait, et qu'il désirait
passionnément avancer et établir, à qui l'amitié de M. le duc d'Orléans
avait été fort utile, à qui, par toutes ces raisons, il ne pouvait être
que fort sensible que ce prince fût en état ou hors d'état d'en tirer
protection et parti, et à qui sûrement il eût fort pesé d'avoir la honte
de se retirer de chez lui, ou l'embarras d'y demeurer attaché, dans
l'état fâcheux où M. le duc d'Orléans s'allait précipitant sans
ressource.

Voilà ce qui me détermina à m'associer de lui, outre qu'il était le seul
dans la confiance de ce prince dont je pusse faire cet usage\,; ainsi,
sans consulter ni m'ouvrir de mon dessein à personne, trouvant Besons
dans le grand appartement pendant la messe du roi, le lendemain de la
visite de Sa Majesté au maréchal de Villars, je l'abordai, et, sans
autre façon, je le pris à part, et je lui parlai de l'état terrible
auquel M. le duc d'Orléans s'était mis. Le maréchal, qui n'ignorait pas
mon intimité avec ce prince, s'ouvrit d'abord avec moi, et me peignit sa
situation avec des couleurs plus vives et plus fâcheuses que n'avait
fait le chancelier. Il me dit que sa solitude était telle que ses gens
lui avaient avoué que, depuis un mois, il était le seul homme qui fût
entré chez lui, non seulement de gens de marque, mais le seul absolument
qui ne fût pas son domestique\,; qu'à Marly on le fuyait dans le salon
sans détour\,; que, s'il y abordait une compagnie, chacun désertait
d'autour de lui, en sorte qu'il demeurait seul un moment après, et avait
encore le dégoût de voir les mêmes gens se rassembler dans un autre coin
tout de suite\,; qu'à Meudon c'était encore pis, qu'à peine Monseigneur
y pouvait souffrir sa présence, et, contre sa manière, ne se
contraignait pas de le marquer\,; que chacun craignait d'être vu avec M.
le duc d'Orléans, et se faisait un mérite et un devoir de lui répondre à
peine\,; que pour lui, il était au désespoir de voir une chose si
funeste et si fort inouïe, et plus outré encore d'y voir si peu de
remède.

Alors je le regardai entre deux yeux, et lui dis que j'en savais bien
un, moi, et prompt et certain, mais unique, difficile, et hasardeux à
tenter\,; que ce que j'avais appris depuis peu de jours, après une
longue absence, m'avait tellement pénétré de douleur là-dessus que
j'avais conçu ce remède et le dessein de le tenter, mais que, ne l'osant
seul, j'avais cru pouvoir oser le lui proposer, comme au seul homme
capable, de m'y donner conseil et aide\,; qu'en un mot, s'il voulait me
seconder, lui et moi parlerions net au prince, et lui ferions ensemble e
la proposition de quitter M\textsuperscript{me} d'Argentan, la source de
ses fautes et de ses malheurs dont il pourrait faire celle de son
rétablissement auprès du roi, outré de son désordre, et avec le monde,
scandalisé à l'excès\,; qu'avec elle disparaîtraient tous ses torts aux
yeux d'un maître qui savait, par une longue et funeste expérience,
jusqu'où pouvait conduire l'aveuglement d'une forte passion\,; d'un père
sensible pour sa fille, d'un oncle qui avait eu de l'inclination pour
son neveu, d'un bienfaiteur qui serait ravi de trouver qu'il ne s'était
pas mépris\,; que le public suivrait la même impulsion, ainsi que les
personnes royales, tous si dépendants des mouvements du roi\,; qu'il n'y
avait que cette porté pour sortir et pour rentrer, qu'un plus long délai
confirmerait de plus en plus un éloignement, peut-être une aversion\,;
qu'en un mot, il examinât bien la chose, et qu'il vît s'il savait mieux,
ou s'il voudrait y concourir avec moi.

Le maréchal fut moins surpris de l'ouverture que saisi que c'était
l'unique ressource de M. le duc d'Orléans. Il l'approuva sur-le-champ,
quoiqu'il en sentit bien les difficultés, me promit d'être mon second\,;
mais comme il entrait en matière, nous vîmes passer d'Antin près de
nous. Nous nous regardâmes pensant tous deux la même chose, et nous
convînmes de nous quitter sur-le-champ, et de nous trouver tête à tête
chez moi, à Paris, l'après-dînée du jour de Noël, pour conférer de
toutes choses, et les mieux digérer ensemble, pour les conduire à une
prompte exécution.

Rempli de tant de pensées importantes, je m'en allai l'après-dînée à
Paris avec M\textsuperscript{me} de Saint-Simon, où je lui contai, et à
ma mère, le dessein que j'avais conçu sur M. le duc d'Orléans. Il leur
fit peur à toutes deux. Elles m'en dissuadèrent\,; elles me dirent que
jamais ce prince n'aurait la force de renvoyer sa maîtresse, ni celle de
lui cacher nos efforts\,; qu'elle était méchante, insolente, hardie au
dernier point, intimement liée à la duchesse de Ventadour, à la
princesse de Rohan, à toute cette dangereuse séquelle qui déjà me
haïssait à cause des Soubise et des Lislebonne, liée encore aux plus
méchantes femmes de Paris, et à un grand nombre de gens qui la
regardant, les uns comme leur gagne-pain, les autres comme une amie
commode, deviendraient furieux contre moi, me susciteraient de nouvelles
affaires par de nouvelles noirceurs, me brouilleraient avec M. le duc
d'Orléans\,; qu'en un mot, ce n'étaient point là mes affaires, ni de
bonnes affaires\,; que les miennes n'avaient pas besoin de supplément de
tracasseries, de méchancetés, d'ennemis, et que je ferais beaucoup mieux
de me tenir en repos, en évitant même avec sagesse un commerce trop
étroit avec M. le duc d'Orléans, de même que ce qui pourrait aussi
sentir l'abandon, dont ses courses continuelles me donneraient le moyen,
si je voulais bien m'en aider. Ce conseil me parut fort sage et me tenta
fort de le suivre.

Besons vint au rendez-vous chez moi le jour de Noël. D'entrée de
discours, je le trouvai refroidi, et comme je l'étais aussi beaucoup, au
lieu de l'échauffer et de le fortifier, je lui présentai les doutes et
les difficultés, que je lui avouai m'avoir touché, par les réflexions
que j'avais faites depuis que je ne l'avais vu. Il douta pareillement
que M. le duc d'Orléans pût être déterminé à quitter
M\textsuperscript{me} d'Argenton\,; que, ne la quittant pas, il pût nous
garder le secret avec elle\,; et me parut aussi persuadé de la fureur de
cette fille et de tout ce qui l'environnait, qui ne serait pas sans
danger. Ainsi, sans nous départir de nos vues, mais sans nous y tenir
entièrement attachés, nous convînmes de ne point parler expressément à
M. le duc d'Orléans de quitter sa maîtresse\,; mais que, s'il le donnait
beau dans la conversation à l'un de nous cieux, celui de nous deux qui
trouverait jour le saisirait pour pousser l'ouverture mesurément, selon
qu'il le jugerait à propos, et aurait pouvoir de citer l'autre, même de
découvrir au prince la résolution formée entre eux deux, pour en tirer
ce qu'il serait possible, mais avec une sage discrétion. Nous
raisonnâmes longtemps sur l'état auquel il s'était laissé tomber, nous
parlâmes des diableries et de l'affaire d'Espagne, dont le maréchal ne
savait pas plus que moi, et nous nous séparâmes de la sorte après être
convenus de nos faits.

Le pénultième jour de cette année, soupant seul avec
M\textsuperscript{me} de Saint-Simon, je reçus par un exprès un billet
de Maréchal, qui me mandait qu'il s'était acquitté de mes ordres, qu'il
n'avait pas été mal reçu, et que je parlerais quand je voudrais\,;
néanmoins qu'il était à propos que je le visse avant personne. Ce billet
nous donna une joie sensible à M\textsuperscript{me} de Saint-Simon et à
moi. Nous jugeâmes que c'était un grand pas fait que la sûreté d'une
audience\,; que la question serait de voir si elle ne serait ni
forlongée ni étranglée\,; le succès qu'on s'en pourrait promettre, que
je verrais bien dans l'audience même.

Nous résolûmes d'aller le lendemain à Versailles, pour marquer au roi de
l'impatience, et y demeurer sans le presser, en attendant qu'il voulût
m'écouter. Je voulus que M\textsuperscript{me} de Saint-Simon vînt avec
moi, pour avoir son conseil dans une conjoncture d'où dépendait
entièrement le genre de vie que nous devions embrasser désormais, chose
si critique pour nous et pour notre famille.

Arrivant à Versailles le dernier jour de l'an, j'allai chez Maréchal,
qui me dit qu'ayant trouvé la veille le roi plus seul et de meilleure
humeur qu'à l'ordinaire, il avait tourné pour lui parler, afin de faire
retirer d'auprès de son lit le peu de petits domestiques qui sont de
cette entrée, qui précède celle du grand chambellan et des premiers
gentilshommes de la chambre\,; que resté seul auprès du roi, il l'avait
voulu sonder, en lui parlant d'abord d'une petite affaire qui le
regardait\,; que le roi lui ayant favorablement répondu, il lui avait
dit que ce n'était pas tout, et qu'il en avait une autre à lui dire qui
lui tenait bien autrement au coeur\,; que le roi lui avait demandé d'un
air fort ouvert ce que c'était, et qu'il lui avait dit qu'il m'avait vu
profondément peiné de me croire mai avec lui\,; sur quoi il avait pris
occasion de me louer, et de lui vanter mon attachement pour lui, et mon
assiduité à la cour\,; que le roi, sans se refrogner, s'était cependant
refroidi, et avait répondu qu'il n'avait rien contre moi, et qu'il ne
savait pas pourquoi je me persuadais le contraire\,; que là-dessus, lui
Maréchal redoubla et demanda mon audience comme la chose du monde que je
désirais le plus, et qui lui ferait à lui le plaisir le plus sensible\,;
que le roi, pressé de la sorte, sans répondre sur l'audience, avait
reparti\,: «\,Mais que me veut-il dire\,? Il n'y a rien. Il est bien
vrai qu'il m'est revenu plusieurs bagatelles de lui, mais rien de
marqué. Dites-lui de demeurer en repos, et que je n'ai rien contre
lui.\,» Que là-dessus, lui Maréchal avait insisté de nouveau pour
l'audience, le priant de me donner cette satisfaction, sans laquelle je
n'en pouvais avoir, mais à son loisir, et point un jour plutôt qu'un
autre, pourvu que ce fût seul dans son cabinet\,; à quoi le roi avait
enfin répondu avec assez d'indifférence\,: «\,Eh bien, je le veux
bien\,; quand il voudra.\,» Maréchal m'assura qu'il avait bien senti de
l'éloignement dans le roi, mais nulle colère, et me dit qu'il espérait
que j'aurais une audience particulière et tranquille\,; que je lui
expliquasse bien tous mes faits une bonne fois, et que je rie craignisse
point d'être trop\,; long, puisqu'il était question d'un éclaircissement
sur des bagatelles grossies, dont le dépouillement demandait du
détail\,; qu'il me conseillait de lui parler avec franchise et liberté,
et de, mêler une sorte d'amitié dans mes respects\,; que du reste je me
présentasse devant lui avec assiduité, pour lui donner lieu de choisir
son temps de me parler.

La conversation finit par des remercîments proportionnés au service
qu'il me rendait, dont l'importance se devait mesurer sur ce que nul
autre de mes amis, ministres, seigneurs, personnages, gens en place,
n'était à portée de me rendre, chose bien étonnante, et néanmoins très
vraie, et qui marquait bien la défiance du roi pour tout le monde, dont
ses valets seuls étaient exceptés.

Maréchal me demanda un secret inviolable, excepté pour
M\textsuperscript{me} de Saint-Simon et le chancelier, que je lui tins
fidèlement. Il ne craignait pas qu'on sût que j'avais eu une audience,
puisque après l'avoir eue, ce serait une nouvelle qui ne se pourrait
cacher, mais bien qu'il me l'eût obtenue. Ainsi finit l'année 1709.

\hypertarget{note-i.-cardinal-de-la-rochefoucauld.}{%
\chapter{NOTE I. CARDINAL DE LA
ROCHEFOUCAULD.}\label{note-i.-cardinal-de-la-rochefoucauld.}}

Saint-Simon (p.~195 de ce volume) parle du cardinal de La Rochefoucauld,
qui avait laissé un grand souvenir à Sainte-Geneviève. Je trouve dans
les Mémoires inédits d'un contemporain, André d'Ormesson (fol.~234, v°),
quelques détails sur ce personnage\,:

«\,François, cardinal de La Rochefoucauld, fut, à treize ans, abbé de
Tournus, puis évêque de Clermont, maître de la chapelle du roi sous
Henri III, cardinal en 1607, sous Henri IV, grand aumônier de France
sous Louis XIII, en 1618, par la mort du cardinal du Perron\,; premier
ministre et chef du conseil en 1622, par la mort du cardinal de Retz\,;
se retira des affaires vers 1628, laissant l'autorité entière au
cardinal de Richelieu\,; fut fait abbé de Sainte-Geneviève par le décès
de l'évêque de Laon (Brichanteau-Nangis). Il y alla faire sa demeure,
fit les réformations presque dans tous les ordres religieux, qui étaient
fort dépravés, assisté de conseillers d'État propres à son dessein,
qu'il avoir choisis, le tout en vertu de brefs du pape et lettres
patentes du roi\,; remit en règle l'abbaye de Sainte-Geneviève, qui
était auparavant à la nomination du roi\,; transféra les
Haudriettes\footnote{Ces religieuses tiraient leur nom d'Étienne Haudri
  qui leur avait donné, au XIIIe siècle, la maison où ils s'établirent,
  et le revenu nécessaire pour leur communauté.} au faubourg
Saint-Honoré, et en fit le monastère de l'Assomption, près les
Capucins\,; a fait bâtir la maison des Incurables, et ne voulut pas
qu'on le sût. Il donnait tout son revenu aux pauvres et aux hôpitaux. Il
fit la réconciliation de l'année 1619, entre le roi Louis XIII et la
reine sa mère, du temps du duc de Luynes, favori\,; méprisa les
grandeurs du monde, foula aux pieds les richesses, les distribuant en
oeuvres pies et nourriture des pauvres. Il vécut dans une telle pureté,
tout le temps de sa vie, que dans Rome il était appelé le cardinal
vierge, et sa sainteté et dévotion tellement louées et estimées que,
nonobstant qu'il fût François, il fut nommé par Robert, cardinal,
Bellarmin, jésuite, e dis\,: autres cardinaux, pour être pape. Il vécut
toujours très sobrement, sans ornements, sans magnificence. Il était
commandeur de l'ordre du Saint-Esprit, et en portait la croix et le
cordon bleu. Ayant vécu saintement, il mourut encore plus saintement,
ayant gagné le jubilé, reçu tous les sacrements, l'esprit sain et
entier, et a fait, une très belle fin, telle que promettait une très
belle et très sainte vie.\,» Le cardinal de La Rochefoucauld mourut le
14 février 1645.

\hypertarget{note-ii.-origine-du-marquis-de-saumery.}{%
\chapter{NOTE II. ORIGINE DU MARQUIS DE
SAUMERY.}\label{note-ii.-origine-du-marquis-de-saumery.}}

Pages 204 et suiv.

M. le marquis de Saumery a adressé à M. le duc de Saint-Simon une
nouvelle note, avec prière de la faire insérer en réponse aux
allégations de Saint-Simon sur l'origine du marquis de Saumery, p.~204
de ce volume.

«\,Le marquis de Saumery, dont M. le duc de Saint-Simon a attaqué
l'origine, descendait de Gérault de Johanne et de Pausato, devenu
seigneur de Mauléon, vers 1275, époque à laquelle les vicomtes
héréditaires de Soule cédèrent leur souveraineté au roi d'Angleterre.
Dès ce temps, cette famille tenait un rang distingué parmi la noblesse
de Béarn. Son arrière-grand-père (le prétendu jardinier de Chambord),
Arnault de Johanne de La Carre, seigneur de Saumery, était fils de
François Arnault de Johanne, seigneur de Mauléon, capitaine d'une
compagnie d'arquebusiers à cheval, et de Gratiane Henriques de Lacarre,
de l'illustre maison navarraise de ce nom. Son grand-père, don
Henriques, baron de La Carre, était, comme ses ancêtres, maréchal
héréditaire de Navarre\,; il avait épousé doña Maria de Luna. Son père,
Pierre Henriques, baron de La Carre, s'était allié à
M\textsuperscript{lle} de Belsunce, dont il n'eut que deux filles.

«\,Arnault fut appelé dans le Blaisois par son grand-oncle Bernard de
Ruthie, grand aumônier de France, et par son oncle maternel, Menault
Henriques de La Carre, aumônier du roi, lequel lui légua, en 1533, sa
seigneurie de Saumery. Il servit d'abord sous les ordres de son père, et
le suivit durant les guerres que le roi de Navarre, depuis Henri IV, eut
à soutenir. Ce prince le prit en qualité de secrétaire de la chambre,
puis le nomma premier président de la chambre des comptes de Blois. Il
mourut conseiller d'État, et ne fut jamais rien à Chambord, dont le
gouvernement n'a été établi qu'au commencement du règne de Louis XIII,
et donné à son fils, François de Johanne de La Carre, seigneur de
Saumery, premier gentilhomme de la chambre de Gaston de France, frère du
roi, capitaine des chasses du comté de Blois, etc., etc.

«\,Enfin le frère du marquis de Saumery, si cruellement traité, Jacques
de Johanne de La Carre, marquis de Saumery, commença à servir dans les
mousquetaires\,; il devint mestre de camp du régiment d'Orléans, puis
maréchal de camp, grand bailli de Blois le 15 février 1650, et
conseiller d'État. Il avait épousé Catherine Charron de Nozieux, peur de
M\textsuperscript{me} Colbert.

«\,Quant à la familiarité dont M. de Saumery aurait usé envers à MM. de
Chevreuse et de Beauvilliers, elle semble parfaitement justifiée,
puisque ces deux seigneurs étaient ses cousins germains.

«\,Des documents de famille, l'estime dont fut toujours entourée
Marguerite de Montlezun, marquise de Saumery, ses liaisons avec les
femmes de la cour les plus vertueuses, attestent que l'auteur des
\emph{Mémoires} était aussi mal renseigné sur sa conduite que sur la
naissance de MM. de Saumery\footnote{Archives du château de Saumery,
  registres de l'état civil de Huillier, archives de Pampelune, lettres
  patentes d'Henri IV du 4 novembre 1598, histoire de Navarre, histoire
  des grands officiers, etc.}.\,»

\hypertarget{note-iii.-uxe9tat-de-lespagne-en-1709.}{%
\chapter{NOTE III. ÉTAT DE L'ESPAGNE EN
1709.}\label{note-iii.-uxe9tat-de-lespagne-en-1709.}}

Saint-Simon parle (p.~210 de ce volume) de l'effet que produisit en
Espagne la nouvelle qu'on y avait répandue du rappel prochain des
troupes françaises. Plusieurs lettres d'Amelot, ambassadeur de France,
donnent des détails importants sur ce fait, et en général sur la
situation de l'Espagne en 1709.

§ 1. AMELOT À LOUIS XIV\footnote{Bibl. imp. du Louvre, F 325, papiers de
  la famille de Noailles, t. XXVI, pièce 2, copie du temps.}.

À Madrid, le 6 mai 1709.

«\,Sire,

«\,J'ai reçu la dépêche dont Votre Majesté m'a honoré le 22 du mois
passé. Le roi d'Espagne a entendu avec plaisir le compte que je lui ai
rendu de ce que Votre\,; Majesté me mande au sujet de la sortie du
nonce. IL est certain, Sire, bien loin que les Espagnols aient
désapprouvé la résolution de Sa Majesté Catholique, qu'il a paru, au
contraire, qu'elle était applaudie, même par la plupart des religieux,
que la juridiction du nonce fatigue en bien des choses, et dont elle
tire beaucoup d'argent, par les petites grâces ou exemptions qu'ils
sollicitent auprès du ministre de Sa Sainteté, et par les procès qu'ils
ont continuellement les uns contre les autres. On a vu, même pendant le
voyage du nonce depuis Madrid jusqu'à la frontière de France, que fort
peu de prêtres et de moines se sont empressés à le voir dans les lieux
de son passage.

«\,Votre Majesté, Sire, aura entendu bien au long, par ma dernière
lettre, ce qui s'est passé ici pendant les trois derniers jours qui
l'ont précédée. La déclaration que le roi d'Espagne a faite à plusieurs
de ses ministres, s'étant répandue dans le public, a causé beaucoup de
rumeur, et a donné lieu à des discours fort extraordinaires parmi les
gens de toute espèce qui composent cette grande ville. On a rapporté ce
que Sa Majesté Catholique avait dit d'une manière bien différente de la
vérité, comme il arrive ordinairement lorsque les choses passent par
plusieurs bouches, et sont redites par des gens ou ignorants ou
malintentionnés. On a publié non seulement que Votre Majesté abandonnait
l'Espagne, mais encore que le roi votre petit-fils était sur le point
d'en sortir, et qu'il n'avait appelé ses principaux ministres que pour
leur en faire part. Il a cependant paru en général de l'attachement pour
le roi d'Espagne, et de l'amour pour Mgr le prince des Asturies, que les
Espagnols regardent comme Espagnol, et comme devant les gouverner un
jour à leur manière.

«\,L'ancienne haine contre la nation française s'est réveillée en cette
occasion, et on ne parlait pas moins que de couper la gorge aux Français
qui sont à Madrid, et de saccager leurs maisons. Les gens sensés ont
connu l'injustice de ces emportements, sachant les prodigieux efforts
que la France fait depuis huit ans pour conserver la monarchie d'Espagne
en son entier, et conserver Sa Majesté sur le trône\,; mais cela ne peut
empêcher le premier effet que fait dans le public une nouveauté de cette
nature.

«\,Le roi d'Espagne a cru, Sire, après la démarche qu'il a faite, qu'il
convenait de nommer des ministres pour les conférences de la paix.
Quoique Votre Majesté ne lui ait pas encore mandé qu'il en fût temps, je
n'ai pas jugé qu'il fût du bien de votre service de m'y opposer, outre
que mes représentations auraient peut-être été inutiles. Il m'a paru, au
contraire, que cette nomination pourrait peut-être calmer les esprits,
en faisant voir qu'il n'y avait encore rien de conclu, puisque Sa
Majesté Catholique nommait des ministres pour traiter la paix. Après
avoir cherché des sujets propres pour une pareille négociation, on a
trouvé tant de difficultés et d'inconvénients dans deux qu'on avait pu
envoyer de Madrid, que le roi d'Espagne s'est déterminé au duc d'Albe et
au comte de Bergheyck.

«\,Je crois, Sire, que ce choix sera plus agréable à Votre Majesté
qu'aucun autre, puisque ces deux ministres ont l'honneur d'être connus
de Votre Majesté\,; qu'ils se trouvent actuellement sur les lieux (ce
qui épargne de la dépense, de l'embarras et du temps, s'il était
question d'une prompte négociation), et que d'ailleurs Votre Majesté
prendra plus aisément les mesures qu'elle jugera convenables avec ces
deux ministres, qui sont au fait des affaires, et qui sont déjà occupés
par d'autres emplois, qu'avec deux qui viendraient d'Espagne remplis
d'idées et de maximes très opposées aux nôtres.

«\,Cette nomination de plénipotentiaires, que le roi a communiquée
aussitôt aux ministres du Despacho et au conseil d'État, a été très
approuvée\,; elle a même fort apaisé les bruits qui couraient et les
mauvais discours qui se tenaient dans les conversations et dans les
places. Sa Majesté Catholique fait travailler aux instructions par le
marquis de Mejorada, et elle en donnera une secrète de sa main, qui, je
crois, se réduira à ne jamais céder l'Espagne et les Indes, et à se
rapporter du reste à tout ce que Votre Majesté jugera de plus
convenable.

«\,Pour ce qui regarde, Sire, la nouvelle forme à donner au
gouvernement, l'idée qu'on a toujours eue quand on en a parlé d'avance,
même avec les Espagnols, et qui se renouvelle aujourd'hui, est de
charger plusieurs ministres de différents départements d'affaires,
indépendants les uns des autres, pour en rendre compte au roi d'Espagne
séparément ou dans un Despacho, selon qu'il sera jugé plus à propos. La
grande difficulté est de trouver des sujets dont ce prince puisse
espérer d'être bien servi, et c'est ce qui s'agite tous les jours entre
Leurs Majestés Catholiques, M\textsuperscript{me} la princesse des
Ursins et votre ambassadeur, sans avoir pu encore se fixer sur aucun des
nouveaux ministres, qu'il faut pourtant tirer du nombre de ceux qu'il y
a présentement. Les Espagnols de confiance auxquels on en parle n'y sont
pas moins embarrassés eux-mêmes.

«\,Je me confirme, au reste, Sire, de plus en plus dans l'opinion que ce
changement est nécessaire, de quelque manière que les choses tournent.
Si le roi d'Espagne demeure sur le trône, on a toujours dit, et il
convient qu'il établisse un gouvernement certain, composé de ministres
espagnols, et qu'on connaisse que Votre Majesté n'est entrée par son
ambassadeur dans le détail et la direction des affaires que par la
nécessité indispensable d'une guerre dont Votre Majesté supportait
presque tout le poids. Si, au contraire, Sa Majesté Catholique est
forcée d'abandonner l'Espagne, et qu'elle exécute la résolution qu'elle
a prise de se défendre jusqu'à l'extrémité avec ses seules forces, en
cas que Votre Majesté retire ses troupes, il y a beaucoup de raison
encore de mettre le ministère dès à présent sur un autre pied, ainsi que
Votre Majesté le jugera aisément, sans que je m'étende davantage pour le
prouver. Lorsque Votre Majesté jugera à propos de m'instruire du progrès
des négociations de M. Rouillé, je serai plus en état de représenter à
Votre Majesté ce que j'estimerai du bien de son service en ce pays-ci,
et d'agir en conformité.

«\,Sur l'article de mon congé, j'espère, Sire, que Votre Majesté ne
désapprouvera pas ce que j'ai eu l'honneur de lui proposer, et qu'elle
connaîtra que le système présent des affaires d'Espagne le demande ainsi
beaucoup plus que mes convenances particulières, tant par rapport à ma
santé que pour le reste.\,»

§ II. EXTRAIT D'UNE LETTRE d'AMELOT, AMBASSADEUR DE FRANCE EN ESPAGNE,
AU ROI LOUIS XIV\footnote{Bibl. imp. du Louvre, F 32S, t. XXVI, pièce
  12.}.

«\,Madrid, 27 mai 1709.

«\,Les choses, Sire, sont ici au même état que j'ai eu l'honneur de
l'expliquer à Votre Majesté par mes dernières lettres. Le roi d'Espagne
paraît plus résolu que jamais à ne point abandonner sa couronne, et à se
défendre avec ses seules forces jusqu'à l'extrémité, si Votre Majesté
retire ses troupes. Il songe en ce cas à M. l'électeur de Bavière pour
commander son armée sur la frontière de Catalogne, et il écrit à Votre
Majesté pour la prier de permettre qu'il prenne à son service les quatre
bataillons irlandais de Votre Majesté qui servent en ce pays-ci\,;
savoir\,: les deux de Berwick, un de Dillon et un de Bourck, et le
bataillon allemand de Reding, dont le colonel est Suisse. Comme je ne
sais point sur quel pied la paix se traite, ni quelles seront les
intentions de Votre Majesté sur le dessein du roi son petit-fils de se
maintenir en Espagne tant qu'il pourra, je ne puis ni ne dois lui
déconseiller des choses qui vont à son but\,; et quand je le ferais, ce
serait fort inutilement. Il me paraît donc absolument nécessaire que
Votre Majesté ait agréable de m'instruire au plus tôt, et plus
particulièrement de la conduite que je dois tenir, afin que je m'y
conforme sans abuser de la confiance que le roi d'Espagne a encore en
moi, ce que je sais bien que Votre Majesté ne m'ordonnera jamais.

«\,Plus je réfléchis sur l'état des choses, Sire, plus je suis persuadé
qu'il est du service de Votre Majesté de m'accorder le congé que je lui
ai demandé, sans attendre que Votre Majesté déclare au roi son
petit-fils les articles dont elle sera convenue pour la conclusion de la
paix. Il est certain que, dès le moment de cette déclaration faite, le
roi d'Espagne, persistant dans sa résolution, sera forcé de se mettre
entièrement entre les mains de ses ministres espagnols. Ceux-ci ne
manqueront pas de demander l'éloignement de votre ambassadeur, ne
croyant pas ou ne voulant pas croire qu'il puisse demeurer ici sans
avoir part à la confiance du roi leur maître, ce qui leur servirait de
motif et d'excuse pour ne pas s'efforcer de bien servir, et pour dire
qu'on ferait échouer leurs desseins par des avis et des insinuations
secrètes.

«\,Il faut cependant, Sire, que Votre Majesté ait, dans tous les cas, un
ministre à Madrid. Celui que vous honorez du caractère de votre
ambassadeur ne serait pas ici, dans une pareille conjoncture, aussi
agréablement qu'il convient à un représentant du premier ordre. Il ne
serait aussi nullement à propos qu'un homme tout neuf, qui ne
connaîtrait pas le terrain, et qui ne trouverait pas les mêmes accès que
ses prédécesseurs, fût choisi pour un emploi de cette nature. Ces deux
réflexions me font prendre la liberté de dire à Votre Majesté que
personne, à mon sens, ne serait plus propre à être chargé des ordres de
Votre Majesté en Espagne, dans cette conjoncture, que M. de Blécourt,
qui a été revêtu de la qualité de votre envoyé extraordinaire pendant
quelque temps, sans compter le séjour qu'il a fait avec M. le maréchal
d'Harcourt. Je ne connais point M. de Blécourt, mais je sais qu'il a ici
la réputation d'un honnête homme, que les Espagnols se sont bien
accommodés de lui, et que comme il connaît et le pays et les sujets, il
peut servir Votre Majesté plus utilement qu'un autre, sans donner trop
d'inquiétude à ceux qui auront le plus de part au gouvernement. Je
conçois, du reste, que l'envoi de M. de Blécourt ne serait que pour un
temps, et jusqu'à ce que le changement des affaires et le système de
cette cour donnassent lieu à Votre Majesté d'envoyer un ambassadeur.\,»

\hypertarget{note-iv.-muxe9moire-pour-le-marquis-de-bluxe9court-envoyuxe9-extraordinaire-du-roi-en-espagne.}{%
\chapter{NOTE IV. MÉMOIRE POUR LE MARQUIS DE BLÉCOURT, ENVOYÉ
EXTRAORDINAIRE DU ROI EN
ESPAGNE.}\label{note-iv.-muxe9moire-pour-le-marquis-de-bluxe9court-envoyuxe9-extraordinaire-du-roi-en-espagne.}}

Le\footnote{Bibl. imp. du Louvre, F 325, 1. XXVI, pièce 74.} successeur
d'Amelot, Blécourt, étant arrivé à Madrid le 23 août 1709, Amelot, avant
son départ, rédigea pour lui un mémoire important sur les relations de
la France et de l'Espagne, et sur la conduite que devait tenir
l'ambassadeur français à Madrid. Voici ce mémoire, qui se trouve dans
les papiers du maréchal de Noailles\,:

«\,Le roi ayant jugé à propos, dès le mois de juin dernier, de faire
revenir son ambassadeur de Madrid, et lui ayant ordonné en môme temps de
se retirer du maniement des, affaires du roi d'Espagne, à moins que Sa
Majesté Catholique ne le désirât autrement pour le bien de son service,
il semble que, dans la situation présente, M. de Blécourt doit donner
uniquement ses soins à entretenir la correspondance entre les deux
cours, à maintenir le roi et la reine d'Espagne dans les sentiments de
reconnaissance et d'attachement pour le roi leur grand-père, dont ils ne
se sont jamais écartés\,; à protéger la nation et le commerce de France,
et à faire payer régulièrement les troupes du roi pour y servir à la
solde de Sa Majesté Catholique.

«\,Il d'est pas besoin de s'étendre sur ces deux derniers points\,; il
suffit de remettre à M. de Blécourt, comme je le fais, les différents
décrets du roi d'Espagne qui ont été obtenus sur les matières qui se
sont {[}présentées{]}, et qui établissent en bien des choses de
nouvelles règles plus avantageuses à la marine de France, au commerce et
aux privilèges de la nation. Je remets en même temps à M. de Blécourt
des états bien détaillés de toutes les troupes françaises qui restent en
Espagne, de ce qu'il faut leur payer par mois, y compris les
états-majors, l'artillerie et tout le reste de ce qui en dépend, et j'y
joins un mémoire de ce qui a été payé à compte à Sa Majesté Catholique.

«\,Le premier point demande plus de réflexions. Leurs Majestés
Catholiques sont certainement très bien disposées\,; elles pensent sur
ce qui regarde la France comme il convient à leur sang et à leur
élévation\,; elles connaissent parfaitement l'intérêt qu'elles ont de
conserver l'union entre les cieux couronnes\,; elles sentent les
obligations infinies qu'elles ont au roi leur grand-père\,; mais comme
le système qui a duré pendant toute la guerre est sur le point de
changer, par la retraite des troupes de France, qui apparemment ne
demeureront pas encore longtemps en Espagne, et peut-être par la
conclusion prochaine d'une paix particulière de la France avec les
alliés, il y aura en ce cas plus de mesures à prendre qu'auparavant pour
détourner tout ce qui pourrait altérer la bonne intelligence, pour
dissiper et comme pour provenir les impressions sinistres et dangereuses
que bien des gens s'efforceront de donner à Leurs Majestés Catholiques
dans des conjonctures aussi délicates et aussi épineuses. C'est l'objet,
ce me semble, de la principale application de M. de Blécourt. Il n'y a
rien pour cela de plus convenable que de s'ouvrir avec franchise au roi
et à la reine d'Espagne, de les informer de tout ce qu'il apprendra des
vues et des intrigues des seigneurs et des ministres espagnols, et de
leur en faire voir les inconvénients.

«\,Si M\textsuperscript{me} la princesse des Ursins demeure à Madrid, il
n'y aura rien de mieux que d'agir de concert avec elle\,; de commencer
par lui donner part de tout, et de profiter de ses conseils et de
l'extrême confiance que Leurs Majestés Catholiques ont justement en
elle. Si M. de Blécourt ne connaît pas à fond M\textsuperscript{me} des
Ursins, il s'apercevra bientôt que rien n'est plus éloigné de la vérité
que les idées qu'on a voulu donner du génie et de la conduite de cette
dame. Il trouvera qu'on ne peut penser plus noblement qu'elle fait, agir
avec plus de désintéressement, ni se conduire en tout avec plus de zèle
pour le service du roi et plus d'attachement pour Leurs Majestés
Catholiques qu'elle a toujours fait.

«\,Si M\textsuperscript{me} la princesse des Ursins se retire, M. de
Blécourt sera certainement privé d'un grand secours et d'une grande
consolation. Il faudra en ce cas, comme je l'ai marqué ci-dessus, non
seulement qu'il s'explique avec franchise au roi et à la reine
d'Espagne, mais qu'il les supplie de lui prescrire les règles de sa
conduite, pour la leur rendre agréable\,; qu'il leur demande quelles
sont les personnes de leur cour avec qui il doit former ses liaisons, et
qui sont celles qu'il doit éviter\,; les consulter sur la manière dont
il devra parler sur des matières de l'importance de celles qui peuvent
se présenter tous les jours dans des temps aussi difficiles que ceux-ci,
et leur dire que c'est l'ordre qu'il a reçu du roi. Rien, à mon sens,
n'est plus propre à plaire à Leurs Majestés Catholiques, à gagner leur
confiance, et à les entretenir clans les sentiments qu'ils doivent au
roi leur grand-père. Je n'ai pas besoin de dire en cet endroit, à un
homme comme M. de Blécourt, que tout ceci ne s'entend qu'autant qu'il
n'y aura rien de contraire aux intentions du roi notre maître.

«\,J'ai dit en particulier mon sentiment à M. de Blécourt sur la plupart
des ministres et des seigneurs de cette cour\,; mais je ne puis
m'empêcher de remarquer ici que le duc de Veragua est un de ceux qui est
le plus dévoué au roi d'Espagne, sur qui l'on peut le plus compter et
avec qui on peut plus sûrement avoir des liaisons\footnote{Voy. Mémoires
  de Saint-Simon, t. III. p.~5.}. Le marquis de La Jamaïque son
fils\footnote{\emph{Ibidem}. t. VI, p.~305.} a beaucoup d'esprit et est
du même génie que son père. Ils sont haïs des autres grands, parce
qu'ils ont constamment été attachés au gouvernement.

«\,Le duc de Popoli est homme de bon sens, de bon esprit, d'un zèle à
toute épreuve, et Leurs Majestés Catholiques ont pour lui plus d'estime
et de confiance que pour aucun autre homme de son rang. M. de Blécourt
ne saurait mieux faire que de rechercher son amitié.

«\,Comme M. de Blécourt n'a pas connu sans doute les deux secrétaires du
Despacho\footnote{Secrétaires d'État.}, qui n'étaient pas en place de
son temps, il est bon de les peindre ici en peu de mots.

«\,Le marquis de Mejorada est homme d'esprit, de mérite et attaché au
roi son maître, mais il est entêté des anciens usages\,; il est
opiniâtre, abonde dans son sers, et n'approuve presque jamais rien
lorsqu'il ne l'a point pensé. Le roi et la reine d'Espagne le
connaissent tel qu'il est et ils ne laissent pas d'en faire cas, parce
qu'il a effectivement de très bonnes choses. Son département se réduit
aux affaires politiques et ecclésiastiques, et à celles de justice\,; ce
qui ne lui donne pas infiniment de travail dans un temps comme celui-ci.

«\,Don Joseph Grimaldo a du bon sens et de l'activité pour le travail.
Il est modeste et désintéressé. Il a été mis en place de mon temps et il
est plus au fait que personne de la nouvelle forme que l'on donné aux
affaires de guerre et de finance, qui avec le commerce forment son
département. La multiplicité des affaires dont il est chargé lui donne
des occasions plus fréquentes d'approcher du roi. Sa Majesté Catholique
s'est fort accoutumée à lui et fait passer par son canal presque toutes
les affaires secrètes et extraordinaires, qui seraient naturellement du
département de son collègue. Don Joseph de Grimaldo est fort aimé et u
des manières polies\,; il n'a jamais abusé de tout ce qu'on lui a confié
ni de l'estime qu'on lui a témoignée. C'est un homme à conserver. Il
sait qu'il m'a obligation de son avancement, et j'ai lieu de croire
qu'il ne changera pas de style et qu'il ne s'éloignera pas des lions
sentiments où je l'ai toujours vu pour maintenir une étroite union entre
les deux couronnes.

«\,J'ai informé en particulier M. de Blécourt des gens qui ont eu le
malheur de déplaire à Sa Majesté Catholique, et avec qui par conséquent
il ne convient pas d'avoir des liaisons. Il serait inutile de le répéter
ici.

«\,Il y a, au reste, trois idées dont il me semble qu'il est bon d'être
prévenu, pour s'en expliquer dans les occasions qui se présenteront de
s'entretenir avec les grands et les ministres d'Espagne. La première
regarde l'intérêt que les principaux seigneurs peuvent avoir à faire en
sorte que la monarchie sait réunie en son entier en la personne de
l'archiduc, supposant qu'elle ne pourrait se conserver entre les mains
de Philippe V qu'avec des démembrements très considérables. Outre que la
vanité de la nation serait flattée par cette réunion prétendue, les
grands y croiraient trouver en particulier leur avantage, par les
vice-royautés et les grands gouvernements de Naples, de Sicile, de
Flandre et de Milan, auxquels ils auraient espérance de parvenir. Il est
important de détruire le fondement d'une pareille tentation, qui
pourrait être dangereuse. Il n'y a pour cela qu'à leur faire faire
attention sur les articles préliminaires que les alliés ont proposés en
dernier lieu à la Haye, et qu'ils ont fait imprimer dans toutes les
langues. On voit dans ces articles qu'il y a des démembrements promis
aux Hollandais, au roi de Portugal et au duc de Savoie, et qu'on se
réserve encore le pouvoir de régler d'autres conventions entre
l'archiduc et les alliés\,; ce qu'on ne peut presque douter qui ne
regarde les États d'Italie, qu'on sait que l'empereur veut s'approprier.
Si l'on prend soin de faire faire là-dessus de sérieuses réflexions aux
Espagnols, ceux qui sont de bonne foi et non prévenus de passion ne
pourront s'empêcher de convenir qu'ils ne trouveront aucun avantage
particulier à avoir l'archiduc pour maître.

«\,La seconde idée, dont on peut faire usage avec gens de toute
condition, surtout avec les ecclésiastiques, c'est qu'il est visible que
la religion souffrirait beaucoup par un changement de domination. On ne
peut douter que les Anglais et les Hollandais, qui ne font la guerre que
pour leur commerce, ne se rendissent maîtres absolus de celui des Indes
et par conséquent des principaux ports de ces vastes royaumes, où ils ne
manqueraient pas d'introduire leur religion. Il faut s'attendre en même
temps qu'ils s'établiraient de la même manière à Cadix, à Bilbao, à
Mahon et peut-être dans d'autres ports d'Espagne, et que la cour de
Madrid ne pourrait plus s'y faire obéir que sous leur bon plaisir. On
sait ce qu'ils ont fait en Aragon et en Valence, pendant qu'ils en ont
été les maîtres\,; que la doctrine catholique y a été corrompue en bien
des endroits, et que l'on a trouvé sur un vaisseau anglais qui a été
pris, quatorze mille exemplaires du catéchisme de la liturgie anglicane,
que la reine Anne envoyait pour faire distribuer dans ces deux royaumes.

«\,La troisième idée consiste à faire connaître aux Espagnols qu'il leur
convient beaucoup plus par rapport à leur repos et à leur sûreté que le
roi Philippe V demeure sur le trône que d'y laisser monter l'archiduc.
On ne peut disconvenir, dans ce dernier cas, que, malgré l'usurpation
violente du prince autrichien, les droits du roi d'Espagne et du prince
des Asturies, juré et reconnu par les états, ne demeurent en leur
entier, surtout ceux du prince des Asturies, qui n'est pas en âge de
faire une renonciation. La France rétablira ses affaires après quelques
années de paix, comme les alliés le publient eux-mêmes\,; elle sera en
état de remettre sur pied de nouvelles et nombreuses armées, et dix ans
ne se passeront pas que Philippe V, ou en son nom ou en celui du prince
des Asturies, ne rentre en Espagne et n'en fasse la conquête. Ce royaume
deviendra alors le théâtre de la guerre, et Dieu sait à combien de
désolations et de nouveaux malheurs il se trouvera exposé, au lieu que
conservant leur roi légitime sur le trône, tout demeure tranquille, sans
trouble et sans fondement légitime de craindre de nouvelles révolutions.
Il semble que ce raisonnement peut frapper les Espagnols.\,»

\hypertarget{note-v.-arrestation-de-flotte-et-de-renaut.}{%
\chapter{NOTE V. ARRESTATION DE FLOTTE ET DE
RENAUT.}\label{note-v.-arrestation-de-flotte-et-de-renaut.}}

Saint-Simon parle (p.~300 et, suiv. de ce volume) de l'accusation que
l'on porta contre le duc d'Orléans à l'occasion des affaires d'Espagne,
ainsi que de l'arrestation de ses agents Flotte et Renaut. On trouve
dans la correspondance d'Amelot quelques passages relatifs à cette
accusation. Voici les principaux\,:

LETTRE D'AMELOT À LOUIS XIV\footnote{Papiers des Noailles, Bibl. imp. du
  Louvre, ms. F 325, lettre 41°. Copie du temps.}.

«\,29 juillet 1709.

«\,Sire,

«\,Je n'ai point été honoré des ordres de Votre Majesté par le dernier
ordinaire. J'ai rendu compte, il y a huit jours\footnote{Cette lettre ne
  se trouve pas dans les papiers des Noailles.}, à Votre Majesté de tout
ce qui regarde la détention du sieur Flotte. Il a été conduit au château
de Ségovie et a dit, sans être pressé, aux officiers qui l'ont approché,
le sujet de sa négociation avec le sieur Stanhope avec des circonstances
qu'il a crues propres à rendre sa conduite moins odieuse, ajoutant qu'il
faudrait faire une alliance entre le roi d'Espagne et Mgr le duc
d'Orléans, auquel Sa Majesté Catholique pourrait céder quelque partie de
sa monarchie. Je ne crois pas, Sire, devoir entrer dans de plus grands
détails à cet égard, sachant que le roi d'Espagne envoie des extraits à
Votre Majesté des déclarations volontaires du sieur Flotte.

«\,Don Bonifacio Manrique, gentilhomme biscaïen, lieutenant général des
armées de Sa Majesté Catholique, dont la conduite a été désagréable au
roi son maître, et qui ne servait plus depuis trois ans, a été arrêté à
Madrid, sur un grand mémoire écrit de sa main, qui a été trouvé dans les
papiers du sieur Flotte. Il promet par ce mémoire d'engager plusieurs
gens de distinction dans le projet et d'aller catéchiser (ce sont ses
termes) dans les provinces d'Andalousie et d'Estrémadure, où il a
beaucoup de connaissances.\,»

Amelot revient sur ce sujet dans une lettre adressée à Louis XIV le 19
août 1709\footnote{Papiers des Noailles, \emph{ibidem}., lettre 50.}\,:

«\,Votre Majesté me marque, par rapport à l'affaire du sieur Flotte, que
les circonstances en sont si fâcheuses de quelque côté qu'on les
regarde, que le seul parti qu'il y ait à prendre est celui de l'assoupir
au plus tôt et de finir les recherches dont la découverte ne peut
produire que de mauvais effets\,; que Votre Majesté demande au roi,
votre petit-fils, d'observer un secret que vous souhaiteriez pour ses
propres intérêts qu'il n'eût jamais laissé pénétrer\,; qu'il ne faut
plus songer qu'à faire cesser l'éclat que la résolution du roi d'Espagne
a causé et que j'y dois travailler pendant le temps qu'il me reste à
demeurer à Madrid.

«\,J'ai commencé à m'acquitter, Sire, des ordres de Votre Majesté en
informant le roi d'Espagne que je les avais reçus et en le pressant,
autant qu'il m'a été possible, de finir les recherches dont il s'agit,
remettant les sieurs Flotte et Renaut à la disposition de Votre Majesté.
Je lui ai représenté les raisons expliquées dans la dépêche de Votre
Majesté et celles que la connaissance de l'affaire offre naturellement à
l'esprit. Sa Majesté Catholique m'a répondu qu'elle souhaitait montrer
en tout sa déférence aux sentiments de Votre Majesté\,; qu'elle avait
envoyé ordre à Ségovie d'interroger encore une fois les sieurs Flotte et
Renaut, surtout le premier qui s'était contredit dans ses déclarations
sur plusieurs articles importants, et qu'aussitôt après qu'elle aurait
vu leurs réponses elle prendrait sa résolution, dont elle me ferait
part. Ce prince m'a dit que, pour ce qui est du secret, il avait été
gardé de sa part avec la dernière exactitude, et que, si le sieur Flotte
n'avait pas tenu les discours que l'on sait à un grand nombre
d'officiers et d'autres personnes, le véritable motif de sa prison
n'aurait jamais été pénétré. Je continuerai, Sire, à presser Sa Majesté
Catholique d'assoupir entièrement cette affaire., ainsi que Votre
Majesté me l'ordonne.\,»

La lettre d'Amelot à Louis XIV, en date du 26 août 1709, parle encore de
Flotte et de Renaut\,:

«\,Le roi d'Espagne a reçu avant-hier les dernières interrogations des
sieurs Flotte et de Renaut\,; elles sont différentes des premières en
bien des choses et presque conformes entre elles\,; ainsi il y a lieu de
croire qu'elles contiennent les faits dans leurs véritables
circonstances. Je presse Sa Majesté Catholique de finir au plus tôt
cette affaire, suivant l'avis de Votre Majesté. J'espère que cela n'ira
pas loin, et je pourrai peut-être en savoir quelque chose de plus
positif avant le départ de l'ordinaire.\,»

\hypertarget{note-vi.-procession-de-la-chuxe2sse-de-sainte-geneviuxe8ve.}{%
\chapter{NOTE VI. PROCESSION DE LA CHÂSSE DE SAINTE
GENEVIÈVE.}\label{note-vi.-procession-de-la-chuxe2sse-de-sainte-geneviuxe8ve.}}

Saint-Simon rappelle (p.~220 de ce volume) que la procession de sainte
Geneviève se faisait \emph{dans les plus pressantes nécessités\,;} mais
il ne donne pas de détails sur cette solennité. Les \emph{Mémoires
d'André d'Ormesson} (fol. 327 r° et v°) contiennent le récit d'une de
ces processions\,:

«\,\emph{L'ordre de la procession de madame sainte Geneviève, qui fut
faite le jour Saint-Barnabé} (13 \emph{juin} 1652).\,»

«\,La France étant en si piteux état, et menacée d'une ruine entière par
l'animosité des princes, qui demandaient l'éloignement du cardinal
Mazarin de la cour, et la reine y résistant de toute sa force, croyant y
aller de son honneur et de son autorité de le maintenir, lesdits
princes, pour l'y forcer, firent entrer les Espagnols, les ennemis du
roi, dans le royaume. Le duc de Nemours les alla quérir. Le duc de
Lorraine y entra avec son armée, ruina et fourragea tous les lieux par
où il passait, amena son armée dans la Brie, et, lui, entra et fut bien
reçu à Paris des princes, et encore du peuple ennemi du cardinal. Les
François se combattant ensemble dans le coeur du royaume, les Espagnols
prirent Gravelines, qui ne put être secouru, et ils étaient en train de
prendre encore Dunkerque. Le parlement donnait des arrêts contre
Mazarin, lequel empêchait le roi de rentrer dans Paris.

«\,Dans ce désordre, auquel il était difficile de remédier, le prévôt
des marchands demanda à messieurs de Notre-Dame, et ensuite aux
religieux et abbé de Sainte-Geneviève, la descente de sa châsse, pour
obtenir, par son intercession, la lin des ruines et misères de la guerre
civile, {[}puis{]} se présenta au parlement, qui donna le jour de la
cérémonie au 13 juin, fête de Saint-Barnabé. Voici l'ordre qui y fut
tenu\,:

«\,Les religieux de Sainte-Geneviève, ayant jeûné trois jours et fait
les prières ordonnées, descendirent la châsse ledit jour du mardi 13
juin, à une heure après minuit. Le lieutenant civil d'Aubray, le
lieutenant criminel, le lieutenant particulier et le procureur du roi la
prirent en leur garde. Les quatre mendiants\footnote{Les quatre ordres
  de religieux qui faisaient voeu de ne vivre que d'aumônes. Les noms de
  ces ordres sent indiqués dans la suite du récit.} marchaient les
premiers, savoir\,: les cordeliers, les jacobins, les augustins et les
carmes, et puis les sept paroisses filles de Notre-Dame, avec leurs
bannières\,; puis furent portées les châsses de saint Papan, saint
Magloire, saint Médéric, saint Landry, sainte Avoie, sainte Opportune et
autres reliquaires\,; puis la châsse de saint Marcel, évêque de Paris,
qui fut portée parles orfèvres. Celle de sainte Geneviève fut portée par
des bourgeois de Paris, auquel cet honneur appartient.

«\,A l'entour et à la suite d'icelle, étaient les officiers du Châtelet,
qui l'avaient en garde. Le clergé de Notre-Dame marchait à gauche, et
l'abbé de Sainte-Geneviève avait la droite, marchait les pieds nus,
comme tous les religieux de Sainte-Geneviève. Ceux qui portaient la
châsse de Sainte-Geneviève étaient aussi pieds nus. M. l'archevêque de
Paris était assis dans une chaire à cause de son indisposition, avait à
côté de lui ledit sieur abbé, et donnaient tous deux des bénédictions au
peuple. Le parlement suivait après, où étaient les présidents Le
Bailleul, de Nesmond, de Maisons, d'Irval et Le Coigneux. Le maréchal de
L'Hôpital, gouverneur de Paris, marchait entre les deux premiers
présidents\,; MM. de Vertamont, Villarceaux-Mangot, Laffemas et
Montmort, maîtres des requêtes, et puis les conseillers de la cour en
grand nombre\,; les gens du roi, MM. Talon, Fouquet et du Bignon, après
eux\,; la chambre des comptes à côté du parlement, en sorte que deux
présidents des comptes étaient à côté de deux présidents de la cour, et
ensuite tous de même.

«\,Par après marchait la cour des aides, au côté droit, MM. Amelot et
Dorieux présidents\,; le prévôt des marchands, M. Le Feron, conseiller
de la cour, avec sa robe mi-partie, avec les échevins et conseil de
ville, au côté gauche.

«\,L'on me dit que M. le duc d'Orléans\footnote{Gaston d'Orléans\,;
  frère de Louis XIII.} et M. le Prince\footnote{Louis de Bourbon, dit
  le grand Condé.} étaient ensemble vers le petit Châtelet. L'on ne vit
jamais tant de peuple\,; les fenêtres remplies de gens d'honneur\,; et
cette procession fut faite en grande dévotion et grand respect. La
châsse de monsieur saint Marcel était très belle et très riche\,; celle
de sainte Geneviève l'était encore plus, y ayant de grosses perles,
rubis et émeraudes en grande quantité, qui avaient été donnés par la
feue reine Marie de Médicis.

«\,Fait et écrit à Paris, l'après-dînée dudit jour Saint-Barnabé (13
juin 1652).\,»

\end{document}
